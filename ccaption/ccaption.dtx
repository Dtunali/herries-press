% \iffalse meta-comment
%
% ccaption.dtx
% Author: Peter Wilson (Herries Press)
% Maintainer: Will Robertson (will dot robertson at latex-project dot org)
% Copyright 1998 -- 2005 Peter R. Wilson
%
% This work may be distributed and/or modified under the
% conditions of the LaTeX Project Public License, either
% version 1.3c of this license or (at your option) any 
% later version.
% The latest version of the license is in
%    http://www.latex-project.org/lppl.txt
% and version 1.3c or later is part of all distributions of
% LaTeX version 2003/06/01 or later.
%
% This work has the LPPL maintenance status "maintained".
% The Current Maintainer of this work is Will Robertson.
%
% This work consists of the files listed in the README file.
%
% 
%<*driver>
\documentclass[twoside]{ltxdoc}
\usepackage{url}
  \usepackage[draft=false,
              plainpages=false,
              pdfpagelabels,
              bookmarksnumbered,
              hyperindex=false
             ]{hyperref}
\providecommand{\phantomsection}{}
\makeatletter
  \@mparswitchfalse
\makeatother
% a simpler version of chngpage,sty's adjustwidth
\newenvironment{addtomargins}[1]{
  \begin{list}{}{%
  \topsep 0pt
  \addtolength{\leftmargin}{#1}%
  \addtolength{\rightmargin}{#1}%
  \listparindent\parindent
  \itemindent\parindent
  \parsep\parskip
  }%
  \item[]}{\end{list}}
\usepackage{ccaption}
%%\usepackage[subfigure]{ccaption}
%%\usepackage{subfigure}
\EnableCrossrefs
\CodelineIndex
\renewcommand{\MakeUppercase}[1]{#1}
\pagestyle{headings}
\setcounter{StandardModuleDepth}{1}
%\setcounter{IndexColumns}{2}
\begin{document}
  \raggedbottom
  \DocInput{ccaption.dtx}
\end{document}
%</driver>
%
% \fi
%
% \CheckSum{1461}
%
% \DoNotIndex{\',\.,\@M,\@@input,\@addtoreset,\@arabic,\@badmath}
% \DoNotIndex{\@centercr,\@cite}
% \DoNotIndex{\@dotsep,\@empty,\@float,\@gobble,\@gobbletwo,\@ignoretrue}
% \DoNotIndex{\@input,\@ixpt,\@m}
% \DoNotIndex{\@minus,\@mkboth,\@ne,\@nil,\@nomath,\@plus,\@set@topoint}
% \DoNotIndex{\@tempboxa,\@tempcnta,\@tempdima,\@tempdimb}
% \DoNotIndex{\@tempswafalse,\@tempswatrue,\@viipt,\@viiipt,\@vipt}
% \DoNotIndex{\@vpt,\@warning,\@xiipt,\@xipt,\@xivpt,\@xpt,\@xviipt}
% \DoNotIndex{\@xxpt,\@xxvpt,\\,\ ,\addpenalty,\addtolength,\addvspace}
% \DoNotIndex{\advance,\Alph,\alph}
% \DoNotIndex{\arabic,\ast,\begin,\begingroup,\bfseries,\bgroup,\box}
% \DoNotIndex{\bullet}
% \DoNotIndex{\cdot,\cite,\CodelineIndex,\cr,\day,\DeclareOption}
% \DoNotIndex{\def,\DisableCrossrefs,\divide,\DocInput,\documentclass}
% \DoNotIndex{\DoNotIndex,\egroup,\ifdim,\else,\fi,\em,\endtrivlist}
% \DoNotIndex{\EnableCrossrefs,\end,\end@dblfloat,\end@float,\endgroup}
% \DoNotIndex{\endlist,\everycr,\everypar,\ExecuteOptions,\expandafter}
% \DoNotIndex{\fbox}
% \DoNotIndex{\filedate,\filename,\fileversion,\fontsize,\framebox,\gdef}
% \DoNotIndex{\global,\halign,\hangindent,\hbox,\hfil,\hfill,\hrule}
% \DoNotIndex{\hsize,\hskip,\hspace,\hss,\if@tempswa,\ifcase,\or,\fi,\fi}
% \DoNotIndex{\ifhmode,\ifvmode,\ifnum,\iftrue,\ifx,\fi,\fi,\fi,\fi,\fi}
% \DoNotIndex{\input}
% \DoNotIndex{\jobname,\kern,\leavevmode,\let,\leftmark}
% \DoNotIndex{\list,\llap,\long,\m@ne,\m@th,\mark,\markboth,\markright}
% \DoNotIndex{\month,\newcommand,\newcounter,\newenvironment}
% \DoNotIndex{\NeedsTeXFormat,\newdimen}
% \DoNotIndex{\newlength,\newpage,\nobreak,\noindent,\null,\number}
% \DoNotIndex{\numberline,\OldMakeindex,\OnlyDescription,\p@}
% \DoNotIndex{\pagestyle,\par,\paragraph,\paragraphmark,\parfillskip}
% \DoNotIndex{\penalty,\PrintChanges,\PrintIndex,\ProcessOptions}
% \DoNotIndex{\protect,\ProvidesClass,\raggedbottom,\raggedright}
% \DoNotIndex{\refstepcounter,\relax,\renewcommand,\reset@font}
% \DoNotIndex{\rightmargin,\rightmark,\rightskip,\rlap,\rmfamily,\roman}
% \DoNotIndex{\roman,\secdef,\selectfont,\setbox,\setcounter,\setlength}
% \DoNotIndex{\settowidth,\sfcode,\skip,\sloppy,\slshape,\space}
% \DoNotIndex{\symbol,\the,\trivlist,\typeout,\tw@,\undefined,\uppercase}
% \DoNotIndex{\usecounter,\usefont,\usepackage,\vfil,\vfill,\viiipt}
% \DoNotIndex{\viipt,\vipt,\vskip,\vspace}
% \DoNotIndex{\wd,\xiipt,\year,\z@}
%
% \changes{v11}{1997/09/30}{Output character table to class and package files only}
% \changes{v2.1}{1998/10/14}{Added legends}
% \changes{v2.2}{1998/12/19}{Added fixed captions and new floats}
% \changes{v2.2a}{1999/01/23}{Improved documentation}
% \changes{v2.3}{1999/01/24}{Added namedlegend command}
% \changes{v2.3a}{1999/08/06}{Fixed uppercasing bug. Changed UpperCase to MakeUppercase}
% \changes{v2.4}{1999/09/20}{Added subfigure v2.0 package support}
% \changes{v2.5}{2000/02/20}{Added subfigure v2.1 package support}
% \changes{v2.6}{2000/02/26}{Added bilingual captions}
% \changes{v2.6a}{2000/02/29}{Improved facilities for subfigure captions}
% \changes{v2.6b}{2000/02/29}{Facility to define new subfloat captions}
% \changes{v2.6c}{2000/02/29}{Fixed potential failures in empty argument test}
% \changes{v2.6d}{2001/01/02}{Fixed fragile failures in contsub... commands}
% \changes{v2.6d}{2001/01/02}{Fixed problem with continued captions in new floats with subfigure 2.1}
% \changes{v2.6e}{2001/01/12}{Fixed problem with new floats and hyperref}
% \changes{v2.7}{2001/02/24}{Subsumes caption2 functions}
% \changes{v3.0}{2001/03/03}{Major revision of new float code to match the tocloft package}
% \changes{v3.0a}{2001/08/03}{Fix for \cs{@tempa} with old amsmath package}
% \changes{v3.1}{2002/02/20}{Support for released subfigure v2.1 package}
% \changes{v3.1a}{2002/04/01}{Support for subfigure v2.1.2 package}
% \changes{v3.1b}{2002/10/18}{Fixed bug wrt brackets in argument to
%                             \cs{caption} with subfigure option}
% \changes{v3.1c}{2003/11/14}{Made \cs{label} work after \cs{contcaption}}
% \changes{v3.2}{2005/03/21}{Support for bilingual captions in longtable}
% \changes{v3.2a}{2005/03/29}{Fix bicaption labels}
%
% \def\dtxfile{ccaption.dtx}
% \def\fileversion{v2.6c} \def\filedate{2000/03/15}
% \def\fileversion{v2.6d} \def\filedate{2001/01/02}
% \def\fileversion{v2.6e} \def\filedate{2001/01/12}
% \def\fileversion{v2.7} \def\filedate{2001/02/24}
% \def\fileversion{v3.0} \def\filedate{2001/03/15}
% \def\fileversion{v3.0a} \def\filedate{2001/08/15}
% \def\fileversion{v3.1} \def\filedate{2002/02/20}
% \def\fileversion{v3.1a} \def\filedate{2002/04/01}
% \def\fileversion{v3.1b} \def\filedate{2002/10/18}
% \def\fileversion{v3.1c} \def\filedate{2003/11/14}
% \def\fileversion{v3.2} \def\filedate{2005/03/21}
% \def\fileversion{v3.2a} \def\filedate{2005/03/29}
% \newcommand*{\Lpack}[1]{\textsf {#1}}           ^^A typeset a package
% \newcommand*{\Lopt}[1]{\textsf {#1}}            ^^A typeset an option
% \newcommand*{\file}[1]{\texttt {#1}}            ^^A typeset a file
% \newcommand*{\Lcount}[1]{\textsl {\small#1}}    ^^A typeset a counter
% \newcommand*{\pstyle}[1]{\textsl {#1}}          ^^A typeset a pagestyle
% \newcommand*{\Lenv}[1]{\texttt {#1}}            ^^A typeset an environment
% \newcommand*{\pname}{ccaption}                    ^^A name of the package
%
% \title{The \Lpack{\pname} package\thanks{This
%        file (\texttt{\dtxfile}) has version number \fileversion, last revised
%        \filedate.}}
%
% \author{%
% Author: Peter Wilson, Herries Press\\
% Maintainer: Will Robertson\\
% \texttt{will dot robertson at latex-project dot org}
% }
% \date{\filedate}
% \maketitle
% \begin{abstract}
%  The \Lpack{\pname} package enables restyling of captions and
% provides for `continuation' captions,
%  unnumbered captions, bilingual captions, and
%  an `anonymous' caption (a legend) that can be used in any
%  environment. It also provides commands to define captions
%  that can be used outside float environments as well as 
%  a mechanism for creating new types of float environments and subfloats.
%
%  The package has been tested in conjunction with the 
% \Lpack{tocloft}
% \Lpack{rotating},
%  \Lpack{caption2}, \Lpack{sidecap}, \Lpack{subfigure}, \Lpack{endfloat},
% \Lpack{longtable}, \Lpack{xtab} 
% and \Lpack{hyperref}
%  packages.
% \end{abstract}
% \tableofcontents
% \listoftables
% \listoffigures
%
% \StopEventually{}
%
% 
%
% \section{Introduction}
%
% Some publishers require and some authors prefer captioning styles
% other than the one style provided by LaTeX. The \Lpack{\pname}
% package provides the tools to design your own captioning styles.
%
% Some publishers require that documents that include multi-part
% tables use a \textit{continuation caption} on all but the first
% part of the multi-part table. For the times where such a table
% is specified by the author as a set of tables, the 
% \Lpack{\pname} package
% provides a simple `continuation' caption command to meet this 
% requirement. It also provides
% a facility for an `anonymous' caption which can be used in any
% float environment. The package has been tested with the
% \Lpack{rotating}, \Lpack{caption2}, \Lpack{sidecap},
% \Lpack{subfigure} (v2.0 and the current version),
% \Lpack{endfloat}, \Lpack{longtable}, \Lpack{xtab} 
% and the \Lpack{hyperref} packages.
%
% Captions can be defined that are suitable for use in non-float
% environments, such as placing a picture in a minipage and captioning
% it just as though it had been put into a normal figure environment.
% Further, a mechanism is provided for defining new float environments.
%
%
% These facilities were originally developed in support of a suite
% for typesetting ISO international standard~\cite{PRW96i},
% but they are generally applicable.
% This manual is typeset according to the conventions of the
% LaTeX \textsc{docstrip} utility which enables the automatic
% extraction of the LaTeX macro source files~\cite{GOOSSENS94}.
%
%    Section~\ref{sec:usc} provides a short overview of the commands
% in the package and shows some examples of their use. This section
% also gives examples of how LaTeX's captioning style can be changed
% to a limited extent without the use of any package and provides 
% general information on floats. For a more comprehensive description
% of floats read Keith Reckdahl's excellent \textit{Using Imported
% Graphics in LaTeX2e}~\cite{EPSLATEX}, 
% although this was written before the advent
% of the \Lpack{ccaption} package.
% The implementation is given in Section~\ref{sec:code}.
%
% \section{The \Lpack{\pname} package} \label{sec:usc}
%
% \subsection{Options}
%
%    The package may take one or more options, depending on which other
% packages are used in conjunction with \Lpack{\pname}. The options are
% designed so that the package loading order does not matter.
% The current options are:
% \begin{itemize}
% \item[\Lopt{subfigure}] for use with the current version
%    of Steven Douglas Cochran's \Lpack{subfigure} package~\cite{SUBFIGURE}.
% \item[\Lopt{subfigure20}] when used together with the old
%    version~2.0 of the \Lpack{subfigure} package.
% \item[\Lopt{caption2}] when used together with Harald Axel Sommerfeldt's
%   \Lopt{caption} or \Lopt{caption2} package~\cite{CAPTION2}.
% \item[\Lopt{titles}] When new floats, and their corresponding
%     `List of\ldots', are defined, the list headings may be individually
%      configured. The \Lopt{titles} option disables the configuration
%      mechanism. This may be useful if, say, the \Lpack{fncychap} package
%      is used to redefine the appearance of chapter titles.
% \end{itemize}
%
% For example, if the package is being used
% with both \Lpack{subfigure} version~2.1 and \Lpack{caption} then it
% should be called as: \\
% |\usepackage[subfigure,caption2]{ccaption}|
%
% \subsection{Changing the caption style}
%
%    The discussion in \S\ref{sec:ltx} includes example methods for 
% changing the typeset appearance of captions without the benefit of
% any package. The \Lpack{caption2} and \Lpack{caption} packages 
% provides a set of predefined
% captioning styles, and the \Lopt{\pname} package 
% also provides an easy means
% of changing the style. 
%
% The style of subcaptions is controlled by the
% \Lpack{subfigure} package.
%
%     \emph{Note that if the \Lopt{caption2} option is used
% then it is assumed that the \Lpack{caption(2)} package is being used
% and the facilities described in this section are unavailable.}
%
% \DescribeMacro{\captiondelim}
% The default captioning style is to put a delimeter in the form
% of a colon between the caption
% number and the caption title. The command |\captiondelim{|\meta{delim}|}|
% can be used to change the delimeter. For example, to have an en-dash instead
% \verb*?\captiondelim{-- }? will do the trick. Notice that no space is
% put between the delimeter and the title unless it is specified in the
% \meta{delim} parameter. 
% The package initially specifies |\captiondelim{: }|
% to give the normal delimeter.
%
% \DescribeMacro{\captionnamefont}
% The \meta{font} specified by |\captionnamefont{|\meta{font}|}| is used
% for typesetting the caption name; that is, the first part of the caption
% upto and including the delimeter (e.g., the portion `Table 3:').
% \meta{font} can be any kind of font specification and/or command and/or 
% text. This first part of the caption is treated like: |{<font> Table 3; }|,
% so font declarations, not font text-style commands, are needed,
% like |\captionnamefont{\Large\sffamily}| to specify a large sans-serif font.
% The package initially specifies |\captionnamefont{}|
% to give the normal font.
% 
%
% \DescribeMacro{\captiontitlefont}
% Similarly, the \meta{font} specified by |\captiontitlefont{|\meta{font}|}|
% is used for typesetting the title text of a caption. For example,
% |\captiontitlefont{\itshape}| for an italic title text.
% The package initially specifies |\captiontitlefont{}|
% to give the normal font.
%
% \DescribeMacro{\captionstyle}
% By default the name and title of a caption are typeset as a block 
% (non-indented)
% paragraph. |\captionstyle{|\meta{style}|}| can be used to alter this.
% Sensible values for \meta{style} are: |\centering|, |\raggedleft| or
% |\raggedright| for styles corresponding to these declarations. 
% The |\centerlastline| style gives a block paragraph but with the last line 
% centered.
% The package initially specifies |\captionstyle{}|
% to give the normal block paragraph style.
%
% \DescribeMacro{\hangcaption}
% \DescribeMacro{\indentcaption}
% \DescribeMacro{\normalcaption}
% The command |\hangcaption| will cause captions to be typeset with the second
% and later lines of a multiline caption title indented by the width
% of the caption name. The command |\indentcaption{|\meta{length}|}|
% will indent title lines after the first by \meta{length}. These
% commands are independent of the |\captionstyle{...}|. Note that a
% caption will not be simultaneously hung and indented. The |\normalcaption|
% command undoes any previous |\hangcaption| or |\indentcaption| command.
% The package initially specifies |\normalcaption|
% to give the normal non-indented paragraph style.
%
% \DescribeMacro{\changecaptionwidth}
% \DescribeMacro{\normalcaptionwidth}
% \DescribeMacro{\captionwidth}
% Issuing the command |\changecaptionwidth| will cause the captions to
% be typeset within a total width \meta{length} as specified by
% |\captionwidth{|\meta{length}|}|. Issuing the command |\normalcaptionwidth|
% will cause captions to be typeset as normal full width captions.
% The package initially specifies |\normalcaptionwidth| and 
% |\captionwidth{\linewidth}|
% to give the normal width. If a caption is being set within the 
% side captioned environments from the \Lpack{sidecap} package~\cite{SIDECAP}
% then it must be a |\normalcaptionwidth| caption.
%
% \DescribeMacro{\precaption}
% \DescribeMacro{\postcaption}
%  The commands |\precaption{|\meta{pretext}|}| and 
% |\postcaption{|\meta{posttext}|}|
% specify \meta{pretext} and \meta{posttext} that will be processed at the
% start and end of a caption. For example \\
% |\precaption{\rule{\linewidth}{0.4pt}\par}| \\
% |\postcaption{\rule{\linewidth}{0.4pt}}| \\
%  will draw a horizontal line above and below the captions.
% The package initially specifies |\precaption{}| and |\postcaption{}|
% to give the normal appearance.
%
% 
%    If any of the above commands are used in a float, or other, environment
% their effect is limited to the environment. If they are used in the preamble
% or the main text, their effect persists until replaced by a similar
% command with a different parameter value. The commands do not affect the
% apperance of the title in any \textbf{List of\ldots}.
%
% \DescribeMacro{\\}
% The normal LaTeX command |\\[|\meta{length}|]| can be used within the
% caption text to start a new line. Remember that |\\| is a fragile 
% command, so if it is
% used within text that will be added to a \textbf{List of\ldots} 
% it must be protected.
% As examples: \\
% |\caption{Title with a \protect\\ new line in both the body and List of}| \\
% |\caption[List of entry with no new line]{Title with a \\ new line}| \\
% |\caption[List of entry with a \protect\\ new line]{Title text}|
%
% Effectively, a caption is typeset as though it were:
% \begin{verbatim}
% \precaption
% {\captionnamefont NAME NUMBER \captiondelim}
% {\captionstyle\captiontitlefont THE TITLE}
% \postcaption
% \end{verbatim}
% Replacing the above commands by their defaults leads to the simple
% format: \\
% |{NAME NUMBER: }{THE TITLE}|
%
% As well as using the styling commands to make simple changes to the
% captioning style more noticeable modifications can also be made.
% To change the captioning style so that the name and title are typeset in
% a sans font it is sufficient to do:
% \begin{verbatim}
% \captionnamefont{\sffamily}
% \captiontitlefont{\sffamily}
% \end{verbatim}
%
% \begin{table}
% \centering
% \captionnamefont{\sffamily}
% \captiondelim{}
% \captionstyle{\\}
% \captiontitlefont{\scshape}
% \setlength{\belowcaptionskip}{10pt}
% \caption{Redesigned table caption style} \label{tab:style}
% \begin{tabular}{lr} \hline
%  three & III \\
%  five  & V \\
%  eight & VIII \\ \hline
%  \end{tabular}
% \end{table}
%
% A more obvious change in styling is shown in table~\ref{tab:style},
% which was coded as:
% \begin{verbatim}
% \begin{table}
% \centering
% \captionnamefont{\sffamily}
% \captiondelim{}
% \captionstyle{\\}
% \captiontitlefont{\scshape}
% \setlength{\belowcaptionskip}{10pt}
% \caption{Redesigned table caption style} \label{tab:style}
% \begin{tabular}{lr} \hline
%  ...
% \end{table}
% \end{verbatim}
% This leads to the approximate caption format (processed within |\centering|): \\
% |{\sffamily NAME NUMBER}{\\ \scshape THE TITLE}| \\
% Note that the newline command (|\\|) cannot be put in the first part
% of the format (i.e., the |{\sffamily NAME NUMBER}|); it has to go into
% the second part, which is why it is specified via |\captionstyle{\\}|
% and not |\captiondelim{\\}|.
%
%    If a mixture of captioning styles will be used you may want to
% define a special caption command for each non-standard style. For
% example for the style of the caption in table~\ref{tab:style}:
% \begin{verbatim}
% \newcommand{\mycaption}[2][\@empty]{%
%   \captionnamefont{\sffamily\hfill}
%   \captiondelim{\hfill}
%   \captionstyle{\centerlastline\\}
%   \captiontitlefont{\scshape}
%   \setlength{\belowcaptionskip}{10pt}
%   \ifx\@empty#1 \caption{#2}\else \caption[#1]{#2}}
% \end{verbatim}
% \textbf{NOTE:}  Any code that involves the |@| sign must be either in
% a package (|.sty|) file or enclosed between a |\makeatletter| \ldots
% |\makeatother| pairing.
%
% The code for the table~\ref{tab:style} example can now be written as:
% \begin{verbatim}
% \begin{table}
% \centering
% \mycaption{Redesigned table caption style} \label{tab:style}
% \begin{tabular}{lr} \hline
%  ...
% \end{table}
% \end{verbatim}
% Note that in the code for |\mycaption| I have added two
% |\hfill| commands and |\centerlastline| compared with the original
% specification.
% It turned out that the original definitions
% worked for a single line caption but not for a multiline caption.
% The additional commands makes it work in both cases, forcing the
% name to be centered as well as the last line of a multiline title,
% thus giving a balanced appearence.
%
% 
%
%
% \subsection{Continuation captions and legends}
%
% \DescribeMacro{\contcaption}
%    The |\contcaption{|\meta{text}|}| command can be used to put 
% a `continuation' or `concluded'
% caption into a float environment. It neither increments the
% float number nor makes any entry into a float listing, but it
% does repeat the numbering of the previous |\caption| command.
% 
%
%   Table~\ref{tab:m} illustrates the use of the |\contcaption|
% command. The table was produced from the following code.
% \begin{verbatim}
%   \begin{table}
%   \centering
%   \caption{A multi-part table} \label{tab:m}
%   \begin{tabular}{lc} \hline
%    just a single line & 1 \\ \hline
%   \end{tabular}
%   \end{table}
%
%   \begin{table}
%   \centering
%   \contcaption{Continued}
%   \begin{tabular}{lc} \hline
%    just a single line & 2 \\ \hline
%   \end{tabular}
%   \end{table}
%
%   \begin{table}
%   \centering
%   \contcaption{Concluded}
%   \begin{tabular}{lc} \hline
%    just a single line & 3 \\ \hline
%   \end{tabular}
%   \end{table}
% \end{verbatim}
%
%   \begin{table}
%   \centering
%   \caption{A multi-part table} \label{tab:m}
%   \begin{tabular}{lc} \hline
%    just a single line & 1 \\ \hline
%   \end{tabular}
%   \end{table}
%
%   \begin{table}
%   \centering
%   \contcaption{Continued}
%   \begin{tabular}{lc} \hline
%    just a single line & 2 \\ \hline
%   \end{tabular}
%   \end{table}
%
%   \begin{table}
%   \centering
%   \contcaption{Concluded}
%   \begin{tabular}{lc} \hline
%    just a single line & 3 \\ \hline
%   \end{tabular}
%   \end{table}
%
% \DescribeMacro{\legend}
%  The |\legend{|\meta{text}|}| command is intended to be used to put an 
% anonymous 
%  caption into a float environment, but may be used anywhere.
%
%   \begin{table}
%   \centering
%   \caption{Another table} \label{tab:legend}
%   \begin{tabular}{lc} \hline
%    A legendary table & 5 \\
%    with two lines    & 6 \\ \hline
%   \end{tabular}
%   \legend{The legend}
%   \end{table}
%
%    For example, the following code was used to produce the two-line
% table~\ref{tab:legend}. The |\legend| command can be used within a float
% independently of any |\caption| command.
% \begin{verbatim}
%   \begin{table}
%   \centering
%   \caption{Another table} \label{tab:legend}
%   \begin{tabular}{lc} \hline
%    A legendary table & 5 \\
%    with two lines    & 6 \\ \hline
%   \end{tabular}
%   \legend{The legend}
%   \end{table}
% \end{verbatim}
%
%     \marginpar{\legend{Title legend}
%                This is a marginal note with a legend.}
%
%  Captioned floats are usually thought of in terms of the |table|
%  and |figure| environments. There can be other kinds of float.
%  As perhaps a more interesting example, the following code produces
%  the titled marginal note which should be displayed near here.
% \begin{verbatim}
%     \marginpar{\legend{Title legend}
%                This is a marginal note with a legend.}
% \end{verbatim}
%
%  You can even \legend{Legend in running text} use the |\legend|
%  command in running text, as has been done in this sentence, 
%  but I'm not sure why one might want to do that as LaTeX already
%  provides the |center| environment.
%
% If you want the legend text to be included in the \textbf{List of\ldots}
% use the |\addcontentsline| command in conjunction with the 
% |\legend|. For example:
% \begin{verbatim}
% \addcontentsline{lot}{table}{Titling text} % left justifified
% \addcontentsline{lot}{table}{\protect\numberline{}Titling text} % indented
% \end{verbatim}
% The first of these forms will align the first line of the legend text
% under the normal table numbers. The second form will align the first
% line of the legend text under the normal table titles. In either case,
% second and later lines of a multi-line text will be aligned under
% the normal title lines.
%
%   \begin{table}
%   \centering
%   \captiontitlefont{\sffamily}
%   \legend{Legendary table}
%   \addcontentsline{lot}{table}{Legendary table (toc 1)}
%   \addcontentsline{lot}{table}{\protect\numberline{}Legendary table (toc 2)}
%   \begin{tabular}{lc} \hline
%    An anonymous table & 5 \\
%    with two lines     & 6 \\ \hline
%   \end{tabular}
%   \end{table}
%
% As an example, the \textsf{Legendary table} is produced by the following code:
% \begin{verbatim}
%   \begin{table}
%   \centering
%   \captiontitlefont{\sffamily}
%   \legend{Legendary table}
%   \addcontentsline{lot}{table}{Legendary table (toc 1)}
%   \addcontentsline{lot}{table}{\protect\numberline{}Legendary table (toc 2)}
%   \begin{tabular}{lc} \hline
%    An anonymous table & 5 \\
%    with two lines     & 6 \\ \hline
%   \end{tabular}
%   \end{table}
% \end{verbatim}
% Look at the List of Tables to see how the two forms of |\addcontentsline|
% are typeset.
%
% \DescribeMacro{\abovelegendskip}
% \DescribeMacro{\belowlegendskip}
%    Correspondingly to the |\abovecaptionskip| and |\belowcaptionskip|
% commands associated with the |\caption| command, the spacing before 
% and after
% a legend is controlled by the |\abovelegendskip| and |\belowlegendskip|
% commands. If necessary, these can be modified via the |\setlength|
% command. By default these are defined to give a half baseline spacing
% before and after the legend.
%
% \DescribeMacro{\namedlegend}
% As a convenience, the |\namedlegend[|\meta{short-title}|]{|\meta{long-title}|}|
% command is like the |\caption| command except that it does not number
% the caption and, by default, puts no entry into a \textbf{List of\ldots} file. Like
% the |\caption| command, it picks up the name to be prepended to the
% title text from the float environment in which it is called (e.g.,
% it may use |\tablename| if called within a |table| environment). The
% following code is the source of the \textit{Named legendary table}.
% \begin{verbatim}
% \begin{table}
% \centering
% \captionnamefont{\sffamily}
% \captiontitlefont{\itshape}
% \namedlegend{Named legendary table}
% \begin{tabular}{lr} \hline
% seven & VII \\
% eight & VIII \\ \hline
% \end{tabular}
% \end{table}
% \end{verbatim}
%
% \begin{table}
% \centering
% \captionnamefont{\sffamily}
% \captiontitlefont{\itshape}
% \namedlegend{Named legendary table}
% \begin{tabular}{lr} \hline
% seven & VII \\
% eight & VIII \\ \hline
% \end{tabular}
% \end{table}
%
%
% \DescribeMacro{\fleg@type}
% The macro |\fleg@type{|\meta{name}|}|, where |type| is the name 
% of a float environment
% (e.g., |table|) is called by the |\namedlegend| macro. It is provided
% as a hook that defines the \meta{name} to be used as the name in
% |\namedlegend|. Two defaults are provided, namely:
% \begin{verbatim}
% \newcommand{\fleg@table}{\tablename}
% \newcommand{\fleg@figure}{\figurename}
% \end{verbatim}
% which may be altered via |\renewcommand| if desired (put between
% a |\makeatletter| and |\makeatother| pair if done in the document).
%
% \DescribeMacro{\flegtoc@type}
% The macro |\flegtoc@type{|\meta{title}|}|, where |type| is the name 
% of a float environment
% (e.g., |table|) is called by the |\namedlegend| macro. It is provided
% as a hook that can be used to add \meta{title} to the listof file.
% By default it is defined to do nothing, and can be changed via
% |\renewcommand|. For instance, it could be changed for tables as:
% \begin{verbatim}
% \makeatletter
% \renewcommand{\flegtoc@table}[1]{%
%   \addcontentsline{lot}{table}{#1}}
% \makeatother
% \end{verbatim}
%
%
% \DescribeMacro{\newfixedcaption}
% \DescribeMacro{\renewfixedcaption}
% \DescribeMacro{\providefixedcaption}
%  The |\legend| command produces a plain, unnumbered heading. It can also
% be useful sometimes to have named and numbered captions outside
% a floating environment, perhaps in a |minipage| if you want the table
% or picture to appear at a precise location in your document.
%
% The |\newfixedcaption[|\meta{capcommand}|]{|\meta{command}|}{|\meta{env}|}|
% command, and its friends, can be used to create a new captioning
% \meta{command} that may be used outside the float environment \meta{env}.
% Both the environment \meta{env} and a captioning command, 
% \meta{capcommand}, for that environment must have been defined before
% calling |\newfixedcaption|. Note that |\namedlegend| can be used
% as \meta{capcommand}.
%
% The |\renewfixedcaption| and |\providefixedcaption| commands take the same 
% arguments as |\newfixedcaption|; the three commands are analagous
% to those in the |\newcommand| family.
%
% For example, to define a new |\figcaption| command for captioning pictures
% outside the |figure| environment, do\\
% |\newfixedcaption{\figcaption}{figure}| \\
% The optional \meta{capcommand} argument is the name of the float
% captioning command that is being aliased. It defaults to |\caption|.
% As another example, where the optional argument is required, if you
% want to create a new continuation caption command for non-floating
% tables, say |\ctabcaption|, then do \\
% |\newfixedcaption[\contcaption]{\ctabcaption}{table}|
%
% Captioning commands created by |\newfixedcaption| will be named and
% numbered in the same style as the original \meta{capcommand}, can
% be given a |\label|, and will appear in the appropriate 
% \textbf{List of \ldots}. They can also be used within floating
% environments, but will not use the environment name as a guide to
% the caption name or entry into the \textbf{List of \ldots}. For
% example, using |\ctabcaption| in a |figure| environment will still
% produce a \textbf{Table\ldots} named caption.
%
%   Sometimes captions are required on the opposite page to a figure, and
% |\newfixedcaption| can be useful in this context. For example, if figure
% captions should be placed on an otherwise empty page immediately before
% the actual figure, then this can be accomplished by the following hack:
% \begin{verbatim}
% \newfixedcaption{\figcaption}{figure}
%  ...
% \afterpage{% fill current page then flush pending floats
%   \clearpage
%   \begin{midpage}  % vertically center the caption
%   \figcaption{The caption}  % the caption
%   \end{midpage}
%   \clearpage
%   \begin{figure}THE FIGURE, NO CAPTION HERE\end{figure}
%   \clearpage
% } % end of \afterpage
% \end{verbatim}
% Note that the \Lpack{afterpage} package is required, which is part of the
% required tools bundle. The \Lpack{midpage} package supplies the |midpage|
% environment, which can be simply defined as:
% \begin{verbatim}
% \newenvironment{midpage}{\vspace*{\fill}}{\vspace*{\fill}}
% \end{verbatim}
% The code might need adjusting to meet your particular requirements.
% The \Lpack{nextpage} package might also be useful in this context as it
% provides a |\cleartoevenpage| command which ensures that you get to the next
% even-numbered page (the |\cleardoublepage| gets you to the next odd-numbered
% page and |\clearpage| gets you to the next page which may be odd or even).
%
% \subsection{Bilingual captions}
%
%    Some documents require bilingual (or more) captions. The package
% provides a set of commands for bilingual captions. Extensions to the
% set, perhaps to support trilingual captioning, are left as an exercise
% for the document author.
%
% \DescribeMacro{\bitwonumcaption}
% \DescribeMacro{\bionenumcaption}
%  Bilingual captions can be typeset by the |\bitwonumcaption| 
% command. This
% takes 6 arguments as: \\
% |\bitwonumcaption[|\meta{label}|]{|\meta{short-1}|}{|\meta{long-1}|}{|\meta{NAME}|}{|\meta{short-2}|}{|\meta{long-2}|}|
%
% The first, optional argument \meta{label}, is the name of a label, if
% required.
% \meta{short-1} and \meta{long-1} are the short (i.e., equivalent
% to the optional argument
% to the |\caption| command) and long caption texts for
% the main language of the document. The value of the \meta{NAME} argument
% is used as the caption name for the second language caption, while
% \meta{short-2} and \meta{long-2} are the short and long caption texts
% for the second language. For example, if the main and secondary languages
% are English and German and a figure is being captioned: \\
% |\bitwonumcaption{Short}{Long}{Bild}{Kurz}{Lang}| \\
% 
% If the short title text(s) is not required, then leave the appropriate
% argument(s) either empty or as one or more spaces, like: \\
% |\bitwonumcaption[fig:bi1]{}{Long}{Bild}{  }{Lang}| \\
% Both language texts are entered into the appropriate List of \ldots,
% and both texts are numbered.
%
% Figure~\ref{fig:bi1} is an example of using the above code.
% \begin{figure}
% \centering
% EXAMPLE FIGURE WITH BITWONUMCAPTION
% \bitwonumcaption[fig:bi1]{}{Long}{Bild}{  }{Lang} 
% \end{figure}
%
%  The |\bionenumcaption| command takes the same arguments as |\bitwonumcaption|.
% The difference between the two commands is that |\bionenumcaption| does
% not number the second language text in the List of.
% Figure~\ref{fig:bi3} is an example of using |\bionenumcaption|.
% \begin{figure}
% \centering
% EXAMPLE FIGURE WITH BIONENUMCAPTION
% \bionenumcaption[fig:bi3]{}{Long English}{Bild}{  }{Lang Deutsch} 
% \end{figure}
%
%
% \DescribeMacro{\bicaption}
%  When bilingual captions are typeset via the |\bicaption| 
% command the second language text is not put into
% the List of \ldots. The command
% takes 5 arguments as: \\
% |\bicaption[|\meta{label}|]{|\meta{short-1}|}{|\meta{long-1}|}{|\meta{NAME}|}{|\meta{long-2}|}|
%
% The optional \meta{label} is for a label if required.
% \meta{short-1} and \meta{long-1} are the short and long caption texts for
% the main language of the document. The value of the \meta{NAME} argument
% is used as the caption name for the second language caption. The last
% argument, \meta{long-2}, is the caption text
% for the second language (which is not put into the List of). 
% For example, if the main and secondary languages
% are English and German: \\
% |\bicaption{Short}{Long}{Bild}{Langlauf}| \\
% If the short title text is not required, then leave the appropriate
% argument either empty or as one or more spaces.
%
% Figure~\ref{fig:bi2} is an example of using |\bicaption| and was 
% produced by the following code:
% \begin{verbatim}
% \begin{figure}
% \centering
% EXAMPLE FIGURE WITH A RULED BICAPTION
% \precaption{\rule{\linewidth}{0.4pt}\par}
% \midbicaption{\precaption{}\postcaption{\rule{\linewidth}{0.4pt}}}
% \bicaption[fig:bi2]{Short English}{Longingly}{Bild}{Langlauf}
% \end{figure}
% \end{verbatim}
%
% \begin{figure}
% \centering
% EXAMPLE FIGURE WITH A RULED BICAPTION
% \precaption{\rule{\linewidth}{0.4pt}\par}
% \midbicaption{\precaption{}\postcaption{\rule{\linewidth}{0.4pt}}}
% \bicaption[fig:bi2]{Short English}{Longingly}{Bild}{Langlauf}
% \end{figure}
%
%
% \DescribeMacro{\bicontcaption}
%  Bilingual continuation captions can be typeset via the |\bicontcaption| 
% command. In this case, neither language text is put into
% the List of \ldots. This command
% takes 3 arguments as: \\
% |\bicontcaption{|\meta{long-1}|}{|\meta{NAME}|}{|\meta{long-2}|}|
%
% \meta{long-1} is the caption text for
% the main language of the document. The value of the \meta{NAME} argument
% is used as the caption name for the second language caption. The last
% argument, \meta{long-2}, is the caption text
% for the second language.
% For example, if the main and secondary languages
% are again English and German: \\
% |\bicontcaption{Continued}{Bild}{Fortgefahren}|
%
% \DescribeMacro{\midbicaption}
% The bilingual captions are implemented by calling |\caption| twice,
% once for each language. The command |\midbicaption{|\meta{midtext}|}|,
% which is similar to the |\precaption| and |\postcaption| commands,
% is executed 
% just before calling the second |\caption|. Among other things,
% this can be used to
% modify the style of the second caption with respect to the first.
% For example, if there is normally a line above and below normal
% captions, it is probably undesirable to have a double line in the
% middle of a bilingual caption. So, for bilingual captions the
% following may be done within the float before the caption:
% \begin{verbatim}
% \precaption{\rule{\linewidth}{0.4pt}\par}
% \postcaption{}
% \midbicaption{\precaption{}\postcaption{\rule{\linewidth}{0.4pt}}}
% \end{verbatim}
% This sets a line before the first of the two captions, then the
% |\midbicaption{...}| nulls the pre-caption line and adds a post-caption
% line for the second caption. The package initially specifies
% |\midbicaption{}|.
%
% \subsubsection{Bilingual captions with \Lpack{longtable}}
%
%    If \Lpack{ccaption} and \Lpack{longtable} are both being used,
% the \Lpack{longtable} package must be loaded before the \Lpack{ccaption}
% package as the \Lpack{ccaption} package makes some changes to
% \Lpack{lontable}'s code.
%
%    Captions in a |longtable| work slightly differently than in
% other floats. This necessitates special versions of the bilingual
% caption commands for use in a |longtable|. These are similar
% to the commands described earlier but they do not take the optional
% \meta{label} argument. If you need a \cs{label} put it in the
% second argument (\meta{long-1}).
%
% \DescribeMacro{\longbitwonumcaption}
% This corresponds to the \cs{bitwonumcaption} command and takes 5 arguments as: \\
% |\longbitwonumcaption|\marg{short-1}\marg{long-1}\marg{NAME}\marg{short-2}\marg{long-2}. \\
% Both captions are numbered and both are put into the List of Tables (LoT).
%
% \DescribeMacro{\longbionenumcaption}
% This corresponds to the \cs{bionenumcaption} command and takes 5 arguments as: \\
% |\longbionenumcaption|\marg{short-1}\marg{long-1}\marg{NAME}\marg{short-2}\marg{long-2}. \\
% Both captions are numbered. Both are put into the ToC but only
% first is numbered there.
%
% \DescribeMacro{\longbicaption}
% This corresponds to the \cs{bicaption} command and takes 4 arguments as: \\
% |\longbicaption|\marg{short-1}\marg{long-1}\marg{NAME}\marg{long-2}. \\
% The first caption is numbered and put into the ToC. The second is neither
% numbered nor put into the ToC.
%
% \DescribeMacro{\midbicaption}
% This is not used by the \cs{longbi...} caption commands; the style of both
% captions is the same. The spacing after a longtable caption, though, 
% is controlled
% by the value of \cs{belowcaptionskip}.
%
%
% \subsection{Use with the \Lpack{subfigure} package}
%
%     The \Lpack{subfigure} package enables the captioning of sub-figures
% within a larger figure, and similarly for tables. If a figure that
% includes sub-figures is itself continued then it may be desireable to
% continue the captioning of the sub-figures. For example, if Figure~3
% has three sub-figures, say A, B and C, and Figure~3 is continued then
% the sub-figures in the continuation should be D, E, etc.
%
% \DescribeMacro{\contsubtop}
% \DescribeMacro{\contsubbottom}
%  The command |\contsubtop[|\meta{list-entry}|][|\meta{subcaption}|]{|\meta{text}|}|
% will continue the sub-caption numbering scheme across (continued) floats,
% putting the \meta{subcaption} at the top of the \meta{text}. If both
% optional arguments are supplied, \meta{list-entry} will be the entry in the
% List of\ldots and \meta{subcaption} will be used as the text for the
% subcaption.
% The |\contsubbottom| command is similar but puts the \meta{subcaption}
% at the bottom of the \meta{text}. In either case, the main caption can 
% be at the top or bottom of the float.
%
% \DescribeMacro{\subconcluded}
% The |\subconcluded| command is used to indicate that the continued 
% (sub) float has been concluded and the numbering
% scheme is reinitialized. The command should be placed immediately
% before the end of the last continued environment.
%
% \DescribeMacro{\subtop}
% \DescribeMacro{\subbottom}
% The command |\subtop[|\meta{list-entry}|][|\meta{subcaption}|]{|\meta{text}|}| is in addition
% to the \Lpack{subfigure} package commands |\subfigure| and |\subtable|.
% It puts the \meta{subcaption} at the top of the \meta{text}, and similarly
% |\subbottom[|\meta{subcaption}|]{|\meta{text}|}| puts \meta{subcaption}
% at the bottom of the \meta{text}.
%
%    For example:
% \begin{verbatim}
% \begin{figure}
% \subbottom{...} % captioned as (a) below
% \subbottom{...} % captioned as (b) below
% \caption{...}
% \end{figure}
% \begin{figure}
% \contsubtop{...} % captioned as (c) above
% \contsubtop{...} % captioned as (d) above
% \contcaption{Concluded}
% \subconcluded
% \end{figure}
% ...
% \begin{table}
% \caption{...}
% \subtop{...}    % captioned as (a) above
% \subbottom{...} % captioned as (b) below
% \end{table}
% \end{verbatim}
%
%    Depending on the age of your LaTeX distribution, you may find that
% you have either version~2.0 or a later version of the \Lpack{subfigure}
% package.  If you have version~2.0, 
% then call the \Lpack{\pname} package
% as: \\
% |\usepackage[subfigure20]{ccaption}|, otherwise as: \\
% |\usepackage[subfigure]{ccaption}|.
%
%    Version~2.1 of the \Lpack{subfigure} package uses many package options
% some of which had been provided as commands in version~2.0. 
% The \Lpack{\pname} commands just described apply to the current version. They
% also apply to version~2.0 except that the |\...top| and |\...bottom|
% commands do \emph{not} take the first optional argument as:
% |\...top[|\meta{subcaption}|]{|\meta{text}|}|.
%
%     Both versions of \Lpack{subfigure} provide the commands
% |\subfigure| and |\subtable| which may be used with the \Lpack{\pname}
% package (which also provides matching |\contsubfigure| and |\contsubtable|
% commands) but I recommend using the generic |\...top| and |\...bottom|
% commands instead. One reason being that the generic commands can be used
% for subcaptions in new kinds of floats, whereas the specific |\...figure|
% and |\...table| commands cannot.
%    In the current version of \Lpack{subfigure} the placement (top or bottom) of
% the subcaptions and the expected placement of the main caption are set
% by package options. Using these options in conjunction with the 
% \Lpack{\pname} package may cause unexpected results, which is another
% reason for using the generic subcaption commands.
%
%
% \subsection{Use with the \Lpack{endfloat} package}
%
%    The \Lpack{endfloat} package~\cite{ENDFLOAT} has the capability
% of putting all floats at the end of the printed document and inserting
% comments in the main text that a float should be placed about \emph{there}.% There is a slight problem if continuation captions are used in conjunction
% with the package, as \Lpack{endfloat} effectively numbers each float
% whether or not it is captioned, and thus will increment the numbering for
% and continued float.
%
%    One way of getting \Lpack{endfloat} and \Lpack{\pname} continued
% captions to cooperate is to put the following in the document preamble
% (modifying or extending it to suit):
% \begin{verbatim}
% \newcommand{\contendfloat}{}
% \renewcommand{\tableplace}{%
%    \begin{center}
%    [\tablename~\theposttbl\ \contendfloat\ about here.]
%    \end{center}
% \newenvironment{conttable}{%
%    \addtocounter{posttbl}{-1}%
%    \def\contendfloat{(continued)}}{}
% \renewcommand{\figureplace}{%
%    \begin{center}
%    [\figurename~\thepostfig\ \contendfloat\ about here.]
%    \end{center}
% \newenvironment{contfigure}{%
%    \addtocounter{postfig}{-1}%
%    \def\contendfloat{(continued)}}{}
%  \end{verbatim}
% and then, for a table, in the document:
% \begin{verbatim}
% ...
% \begin{table}
% \caption{...}
% ...
% \end{table}
% ...
% \begin{conttable}
% \begin{table}
% \contcaption{Continued}
% ...
% \end{table}
% \end{conttable}
% \end{verbatim}
% and similarly for any continued figures.
%
%
%
% \subsection{New float environments}
%
% The commands in the previous sections have been tested with 
% the \Lpack{caption2} and 
% \Lpack{rotating} packages. They will most likely fail if used with
% the \Lpack{float} package because of the way this package redefines
% the basic |\caption| command.
%
% The \Lpack{float} package, developed by Anselm Lingnau~\cite{LINGNAU95},
%  provides a simple scheme for creating new kinds
% of floats with a variety of captioning styles. Unfortunately the package
% does not effectively seperate the float creation aspects and the captioning
% styles.  I have therefore included in the \Lpack{\pname} package a
% poor man's version of some aspects of the float creation elements that are
% in \Lpack{float}. Both the commands and their coding differ from those
% in the \Lpack{float} package.
%
% \DescribeMacro{\newfloatlist}
% The command 
% |\newfloatlist[|\meta{within}|]{|\meta{fenv}|}{|\meta{ext}|}{|\meta{listname}|}{|\meta{capname}|}|
% creates both a new kind of floating environment called \meta{fenv}
% and a new kind of `List of' for \meta{fenv}; the title of this
% new listing is \meta{listname}. A caption within
% the environment will be written out to a file with extension \meta{ext}.
% The caption, if present, will start with \meta{capname}. For example, if this
% command had been used to create the |figure| environment for the \Lpack{article}
% class it would have been used as (remembering that LaTeX uses 
% |\listfigurename| to store the `List of Figures' text 
% and |\figurename| to store the `Figure' text): \\
% |\newfloatlist{figure}{lof}{\listfigurename}{\figurename}| \\
% and the command |\listoffigure| (generated by |\newfloatlist|)
% would typeset the List of Figures.
%
% The optional \meta{within} argument can be used if you want the captions to be 
% numbered within a particular document division, as figures are within the
% \Lpack{book} and \Lpack{report} classes with the numbering starting afresh with
% each new chapter. Creating the figure environment for either of these classes
% would have used: \\
% |\newfloatlist[chapter]{figure}{lof}{\listfigurename}{\figurename}|
%
% The captioning style for floats defined with |\newfloatlist| is the same as
% for figures and tables in the standard classes.
%
%    The |\newfloatlist| command generates several new commands that you can
% use for styling the new listing, similar to the facilities given by
% the \Lpack{tocloft} package~\cite{TOCLOFT}; for more detailed information
% you may wish to read the \Lpack{tocloft} documentation. 
% For ease of explanation, assume that
% the command was called as |\newfloatlist{X}{Z}{flist}{fcap}|, so that |X|
% corresponds to the name of the new environment \meta{fenv} and |Z|
% corresponds to the file extension \meta{ext}. The following float environment
% and commands
% are then created.
%
% \DescribeEnv{X}
% The new float environment is called |X|, and can be used as either
% |\begin{X}| or |\begin{X*}|, with the matching |\end{X}| or |\end{X*}|.
%
% \DescribeMacro{\listofX}
% |\listofX| is similar to |\listoffigures|, etc., in that it typesets the new
% listing, and heads the list with the value of |flist|.
%
% \DescribeMacro{Zdepth}
% The |Zdepth| counter is analogous to the standard |tocdepth| counter
% in that it specifies that entries in the new listing should not be
% typeset if their numbering level is greater than |Zdepth|. The
% default definition is |\setcounter{Zdepth}{1}|. To have a subfloat
% of |Z| appear in the listing do |\setcounter{Zdepth}{2}|.
%
% \DescribeMacro{\cftmarkZ}
% This macro sets the appearance of the running heads on the new listing pages.
% The default definition gives the same appearance as for the LoF or LoT.
%
% \DescribeMacro{\cftbeforeZtitleskip}
% \DescribeMacro{\cftafterZtitleskip}
%  The lengths |\cftbeforeZtitleskip| and |\cftafterZtitleskip| control
% the vertical spacing before and after the title of the new listing. By
% default they are set to give the normal spacing, but you can change them
% with |\setlength| if you wish.
%
% \DescribeMacro{\cftZtitlefont}
% \DescribeMacro{\cftafterZtitle}
% The code for typesetting the title of the new listing looks 
% roughly like this
% \begin{verbatim}
% \vspace*{\cftbeforeZtitleskip}
% {\cftZtitlefont flist}{\cftafterZtitle}
% \cftmarkZ
% \vskip \cftafterZtitleskip
% \end{verbatim}
% The default definition of |\cftZtitlefont| is for a bold font. If,
% for example, you would prefer the title to be in a large italic font
% and set flushright you could: \\
% |\renewcommand{\cftZtitlefont}{\hfill\Large\itshape}| \\
% By default |\cftafterZtitle| is defined to do nothing; you can
% change it to serve your own purposes. For example: \\
% |\renewcommand{\cftafterZtitle}{\thispagestyle{empty}}| \\
% will set the page style of the first page of the new listing to 
% be |empty|.
%
% \DescribeMacro{\newfloatentry}
% The command 
% |\newfloatentry[|\meta{within}|]{|\meta{counter}|}{|\meta{ext}|}{|\meta{level-1}|}|
% is used internally by |\newfloatlist| to generate a new counter for the new
% float environment and to generate the typesetting code for entries in the
% new listing. The required \meta{counter} argument is the name for a new
% counter. If the optional \meta{within} argument is used, the counter
% \meta{counter} will be reset each time the counter \meta{within} is changed.
% These first two arguments have the same effect as calling 
% |\newcounter{|\meta{counter}|}[|\meta{within}|]|. The \meta{ext} argument
% is the extension for the file holding the entries, and \meta{level-1} is
% one less than the `level' of the entry. Continuing the figure example, \\
% |\newfloatlist[chapter]{figure}{lof}{\listfigurename}{\figurename}| \\
% will internally call \\
% |\newfloatentry[chapter]{figure}{lof}{0}|.
%
%    |\newfloatentry| generates a set of commands in addition to those directly
% generated by |\newfloatlist|. Assuming, as above, that we had \\
% |\newfloatlist{X}{Z}{flist}{fcap}|
% then we will also have \\
% |\newfloatentry{X}{Z}{0}|. This generates the following.
%
% \DescribeMacro{X}
% The counter |X| matches the environment |X|. This counter is used for numbering
% captions. Remember that it will be reset according to the \meta{within} argument.
%
% \DescribeMacro{\theX}
%    The command |\theX| prints the value of the |X| counter. It is initially
% defined so that it prints arabic numerals. If the optional \meta{within}
% argument is used, |\theX| is defined as \\
% |\renewcommand{\theX}{\thewithin.\arabic{X}}| otherwise as \\
% |\renewcommand{\theX}{\arabic{X}}|.
%
% \DescribeMacro{\cftbeforeXskip}
% This length controls the vertical space above a caption entry in the listing.
% It can be changed by using |\setlength|.
%
% \DescribeMacro{\cftXindent}
% \DescribeMacro{\cftXnumwidth}
% The indentation of a caption entry in the listing from the left margin is
% given by the length |\cftXindent|, and the space for the caption number is
% set by the length |\cftXnumwidth|. These may be changed via |\setlength|
% or by |\setnewfloatindents|. The default values for these depend
% on the value of the \meta{level-1} argument. A value of zero for
% this sets the defaults to the figure and table values. A value of one
% sets them to the defaults for subfigure and subtable values.
%
%    The code for typesetting a caption entry is roughly like:
% \begin{verbatim}
% {\cftXfont {\cftXpresnum SNUM\cftXaftersnum\hfil} \cftXaftersnumb TITLE}%
%            {\cftXleader}{\cftXpagefont PAGE}\cftXafterpnum\par
% \end{verbatim}
% where |SNUM| is the caption number, |TITLE| is the caption text, and |PAGE|
% is the page number. The other commands are described below.
%
% \DescribeMacro{\cftXfont}
% This controls the appearance of the number and title. By default it is defined
% to use the normal font but it can be changed with
% |\renewcommand|.
%
% \DescribeMacro{\cftXpresnum}
% \DescribeMacro{\cftXaftersnum}
% \DescribeMacro{\cftXaftersnumnb}
%  The caption number is typeset in a box of width |\cftXnumwidth|. Within the
% box, |\cftXpresnum| is first called, then the number is typeset, then
% |\cftXaftersnum| is called and finally there is a |\hfil| to make the box 
% contents flush left. After the number box is typeset |\cftXaftersnumb| is
% called and then the caption text is typeset. By default these three macros
% are defined to do nothing, but |\renewcommand| can be used to make them do
% something interesting.
%
% \DescribeMacro{\cftXleader}
% \DescribeMacro{\cftXdotsep}
% |\cftXleader| defines the leader between the text and the page number; it can be
% changed by |\renewcommand|. By default it produces a dotted leader with 
% |\cftXdotsep| space between the dots. It default definition is \\
% |\newcommand{\cftXdotsep}{4.5}| which gives a 4.5mu (math units) seperation.
% In spite of it appearing to be a length, changes to |\cftXdotsep| must be
% made by |\renewcommand|.
%
% \DescribeMacro{\cftXpagefont}
% \DescribeMacro{\cftXafterpnum}
% |\cftXpagefont| specifies the font to be used for typesetting the page number.
% By default it is set to the normal font. Finally, |\cftXafterpnum| is called
% after setting the page number; by default is does nothing. Both these commands
% can be changed by |\renewcommand|.
%
%    Note that |\newfloatlist| effectively generates all the above commands. Their
% defaults are set so that the typesetting mimics that for figure and table captions.
% It is probable that you can ignore all of them, but if you do want to change 
% something the \Lpack{tocloft} documentation provides many examples.
%
% \DescribeMacro{\setnewfloatindents}
% The command
% |\setnewfloatindents{|\meta{fenv}|}{|\meta{indent}|}{|\meta{numwidth}|}|
% sets the \meta{fenv}'s entry indent to the length \meta{indent}
% and its numwidth to the length \meta{numwidth}. The \meta{fenv}
% argument is the full name of the (sub)float.
% 
%    As a fuller example of |\newfloatlist|, suppose you wanted both 
% figures (which come with the
% standard classes), and diagrams. You could then do something like the following.
% \begin{verbatim}
% \usepackage{ccaption}
% ...
% \newcommand{\diagramname}{Diagram}
% \newcommand{\listdiagramname}{List of Diagrams}
% \newfloatlist{diagram}{dgm}{\listdiagramname}{\diagramname}
% \newfixedcaption{\fdiagcaption}{diagram}
% \begin{document}
% ...
% \listoffigures
% \listfofdiagram
% ...
% \begin{diagram}
% \caption{A diagram} \label{diag1}
% ...
% \end{diagram}
% As diagram~\ref{diag1} shows ...
% \begin{minipage}{.9\textwidth}
% \fdiagcaption{Another diagram} \label{diag2}
% ...
% \end{minipage}
%
% In contrast to diagram~\ref{diag1}, diagram~\ref{diag2} provides ...
% \end{verbatim}
%
%    As a word of warning, if you mix both floats and fixed environments with the
% same kind of caption you have to ensure that they get printed in the correct
% order in the final document. If you do not do this, then the |\list...| of
% captions will come out in the wrong order (the lists are ordered according the
% page number in the typeset document, \emph{not} your source input order).
%
% \DescribeMacro{\newsubfloat}
% The |\newsubfloat{|\meta{fenv}|}| command, which is only of use with
% the \Lpack{subfigure} package and the \Lopt{subfigure20} or
% \Lopt{subfigure} option,
% creates
% subcaptions (|\subtop| and |\subbottom|, together with their continued
% forms) for use within the float 
% environment \meta{fenv} previously
% defined via |\newfloatlist[...]{|\meta{fenv}|}{...}|.
%
%    The |\newsubfloat| macro internally calls the |\newfloatentry| command
% and assuming our usual 
% |\newfloatlist{X}{Z}{flist}{fcap}|
% then |\newsubfloat{X}| calls \\
% |\newfloatentry[X]{subX}{Z}{1}| \\
% so there is a further set of |\cftsubX...| commands generated for adjusting
% the typesetting of the subcaption entries. Note that the full name of the
% entry in the listing is `sub\meta{fenv}', not just simply `\meta{fenv}'.
%
% \DescribeMacro{\newfloatpagesoff}
% The |\newfloatpagesoff{|\meta{fenv}|}| command will turn off page numbering for
% list entries for \meta{fenv}. This is probably most likely to be used
% for switching off page numbers for subfloat entries, in which case
% it should be called as |\newfloatpagesoff{sub|\meta{fenv}|}|.
% 
% \DescribeMacro{\newfloatpageson}
% The |\newfloatpageson{|\meta{fenv}|}| command reverses the effect of a
% corresponding |\newfloatpagesoff{|\meta{fenv}|}|.
% 
%
% \DescribeMacro{\newfloatenv}
% \DescribeMacro{\listfloats}
% \textbf{NOTE:} These two macros were in version 2.7 of the package but were 
% replaced in version 3.0 by the functionally extended 
% |\newfloatlist| and |\listofX| commands, respectively.
%
%     There is a limit to the number of List of\ldots listings that (La)TeX
% can handle. Each kind of listing requires a |\jobname.ext| file and
% the TeX program has an upper limit on the number of files it can
% handle. In the most limited circumstance LaTeX requires three files
% --- the \file{log}, \file{aux} and \file{dvi} files. Further files are
% required for things like a ToC (\file{toc}) or an index (\file{idx}). 
% If you try and create
% too many new listings LaTeX will respond with the error message:
% \begin{center}
% \texttt{No room for a new write}
% \end{center}
% If you get such a message the only recourse is to redesign your document.
%
% \section{How LaTeX makes captions} \label{sec:ltx}
%
% This section provides an overview of how LaTeX creates captions and
% gives some examples of how to change the captioning style without
% having to use any package. 
% The section need not be looked at more than once unless you like 
% reading LaTeX code
% or you want to make changes to LaTeX's style of captioning.
%
% The LaTeX kernel provides tools to help in the definition of captions,
% but it is the particular class that decides on their format.
%
% \DescribeMacro{\caption}
% The kernel (in \file{ltfloat.dtx}) defines the caption command via \\
% |\def\caption{\refstepcounter\@captype \@dblarg{\@caption\@captype}}|
%
% \DescribeMacro{\@captype}
% |\@captype| is defined by the code that creates a new float environment
% and is set to the environment's name (see the code for |\@xfloat|
% in \file{ltfloat.dtx}). For a |figure| environment,
% there is an equivalent to \\
% |\def\@captype{figure}|.
%
% \DescribeMacro{\@caption}
% The kernel also provides the 
% |\@caption{|\meta{type}|}[|\meta{short-title}|]{|\meta{full-title}|}|
% command as:
% \begin{verbatim}
% \long\def\@caption#1[#2]#3{%
%   \par
%   \addcontentsline{\csname ext@#1\endcsname}{#1}%    <----
%     {\protect\numberline{\csname the#1\endcsname}{\ignorespaces #2}}%
%   \begingroup
%      \@parboxrestore
%      \if@minipage
%        \@setminipage
%      \fi
%      \normalsize
%      \@makecaption{\csname fnum@#1\endcsname}{\ignorespaces #3}\par % <----
%   \endgroup}
% \end{verbatim}
% where \meta{type} is the name of the environment in which the caption
% will be used.
% Putting these three commands together results in the user's view of the caption
% command as |\caption[|\meta{short-title}|]{|\meta{full-title}|}|.
%
% It is the responsibilty of the class (or package) which defines floats
% to provide definitions for |\ext@type|, |\fnum@type| and |\@makecaption|
% which appear in the definition of |\@caption| (in the lines marked
% |<----| above).
%
% \DescribeMacro{\ext@type}
% This macro holds the name of the extension for a `List of\ldots' file.
% For example for the |figure| float environment there is the
% definition equivalent to \\
% |\newcommand{\ext@figure}{lof}|.
%
% \DescribeMacro{\fnum@type}
% This macro is responsible for typesetting the caption number. For example,
% for the |figure| environment there is the definition equivalent to \\
% |\newcommand{\fnum@figure}{\figurename~\thefigure}|.
%
% \DescribeMacro{\@makecaption}
% The |\@makecaption{|\meta{number}|}{|\meta{text}|}|, where \meta{number}
% is a string such as `Table~5.3' and \meta{text} is the caption text,
% performs the typesetting of the caption, and
% is defined in the standard classes (in \file{classes.dtx}) as the
% equivalent of:
% \begin{verbatim}
% \newcommand{\@makecaption}[2]{%
%   \vskip\abovecaptionskip       %  <- 1
%   \sbox\@tempboxa{#1: #2}%      %  <- 2
%   \ifdim \wd\@tempboxa >\hsize
%     #1: #2\par                  %  <- 3
%   \else
%     \global \@minipagefalse
%     \hb@xt@\hsize{\hfil\box\@tempboxa\hfil}%
%   \fi
%   \vskip\belowcaptionskip}      %  <- 4
% \end{verbatim}
%
% \DescribeMacro{\abovecaptionskip}
% \DescribeMacro{\belowcaptionskip}
%  Vertical space is added before and after a caption (lines marked 1 and 4
% in the code for |\@makecaption| above) and the amount of space is given
% by the lengths |\abovecaptionskip| and |\belowcaptionskip|. The
% standard classes set these to 10pt and 0pt respectively. If you want
% to change the space before or after a caption, use |\setlength| to change
% the values. In figures, the caption is usually placed below the
% illustration. The actual space between the bottom of the illustration
% and the baseline of the first line of the caption
% is the |\abovecaptionskip| plus the |\parskip| plus the |\baselineskip|.
% If the illustration is in a |center| environment then additional space
% will be added by the |\end{center}|; it is usually better to use 
% the |\centering| command rather than the |center| environment.
%
% The actual typesetting of a caption is effectively performed by the code
% in lines marked 2 and 3 in the code for |\@makecaption|; note that
% these are where the colon that is typeset after the number is specified. 
% If you want to
% make complex changes to the default captioning style you may have to
% create your own version of |\@caption| using 
% |\renewcommand|. On the other hand, many such changes can be achieved
% by changing the definition of the 
% the appropriate |\fnum@type| command(s). For example, to make the 
% figure name and number bold: \\
% |\renewcommand{\fnum@figure}{\textbf{\figurename~\thefigure}}|
%
% REMEMBER: If you are doing anything involving commands that include
% the |@| character, and it's not in a class or package file, you have
% to do it within a |\makeatletter| and |\makeatother| pairing. So,
% if you modify the |\fnum@figure| command anywhere in your document
% it has to be done as:
% \begin{verbatim}
% \makeatletter
% \renewcommand{\fnum@figure}{......}
% \makeatother
% \end{verbatim}
%
% \makeatletter
% \renewcommand{\fnum@figure}{\textsc{\figurename~\thefigure}}
% \makeatother
% \begin{figure}
% \centering
% A THOUSAND WORDS\ldots
% \caption{A picture is worth a thousand words}\label{fig:sc}
% \end{figure}
%
% As an example, Figure~\ref{fig:sc} was created by the following code:
% \begin{verbatim}
% \makeatletter
% \renewcommand{\fnum@figure}{\textsc{\figurename~\thefigure}}
% \makeatother
% \begin{figure}
% \centering
% A THOUSAND WORDS\ldots
% \caption{A picture is worth a thousand words}\label{fig:sc}
% \end{figure}
% \end{verbatim}
%
% As another example, suppose that you needed to typeset the |\figurename|
% and its number in a bold font, replace the colon that normally appears
% after the number by a long dash, and typeset the actual title text in
% a sans-serif font, as is illustrated by the caption for 
% Figure~\ref{fig:sf}. The following code does this.
%
% \makeatletter
% \renewcommand{\fnum@figure}[1]{\textbf{\figurename~\thefigure} --- \sffamily}
% \makeatother
% \begin{figure}
%  \centering
%  ANOTHER THOUSAND WORDS\ldots
% \caption{A different kind of figure caption}\label{fig:sf}
% \end{figure}
%
% \begin{verbatim}
% \makeatletter
% \renewcommand{\fnum@figure}[1]{\textbf{\figurename~\thefigure} --- \sffamily}
% \makeatother
% \begin{figure}
%  \centering
%  ANOTHER THOUSAND WORDS\ldots
% \caption{A different kind of figure caption}\label{fig:sf}
% \end{figure}
% \end{verbatim}
% Perhaps a little description of how this works is in order.
% Doing a little bit of \TeX 's macro processing by hand, the typesetting
% lines in |\@makecaption| (lines 2 and 3) get instantiated like: \\
% |\fnum@figure{\figurename~\thefigure}: text| \\
% Redefining |\fnum@figure| to take one argument and then not using the
% value of the argument essentially gobbles up the colon. Using \\
% |\textbf{\figurename~\thefigure}| \\
% in the definition causes |\figurename| and the number to be typeset in
% a bold font. After this comes the long dash. Finally, putting |\sffamily|
% at the end of the redefinition causes any following text (i.e., the actual 
% title) to be typeset using the sans-serif font.
%
% If you do modify |\@makecaption|, then spaces in the definition may be
% important; also you must use the comment (\%) character in the same
% places as I have done above.
%
% You may also want to take a look at the \Lpack{caption2} package by
% Harald Axel Sommerfeldt which provides a ready-made set of differing
% captioning styles. This basically works by redefining the 
% |\@makecaption| command to provide some hooks. Of course the \Lpack{\pname}
% package provides the tools that you need to make most, if not all,
% of any likely caption styles.
%
% \subsection{Changing the numbering scheme}
%
% In the \Lpack{article} class and its derivatives, captions are numbered
% continuously throughout the document, while in the \Lpack{book} and
% \Lpack{report} classes, numbering starts anew in each chapter.
%
% If you want captions to be numbered anew with sections in the 
% \Lpack{article} class you can do this:
% \begin{verbatim}
% \makeatletter
% \@addtoreset{table}{section}
% \renewcommand{\thetable}{\thesection.\arabic{table}}
% \makeatletter
% \end{verbatim}
% and similarly for all the other float environments.
%
% If you are using the \Lpack{book} or \Lpack{report} class and you want
% the captions to be numbered consecutively throughout the document you
% can do this:
% \begin{verbatim}
% \makeatletter
% \@removefromreset{table}{chapter}
% \renewcommand{\thetable}{\arabic{table}}
% \makeatother
% \end{verbatim}
% and similarly for all the other float environments. Note that you
% will need the \Lpack{remreset} package\footnote{Available on CTAN in
% \texttt{tex-archive/macros/latex/contrib/supported/carlise}.}
% which provides the definition of |\@removefromreset|.
%
% You can play with other combinations of |\@addtoreset|, |\@removefromreset|,
% and |\renewcommand{\the...}{...}|
% to get the numbering scheme you want.
%
% \subsection{Captions with footnotes}
%
%    If you want to have a caption with a footnote, think long and hard
% as to whether this is really essential. It is not normally considered
% to be good typographic practice, and to rub the point in LaTeX does not
% make it necessarily easy to do. However, if you (or your publisher)
% insists, read on.
%
%    If it is present, the optional argument to |\caption| is put into
% the LoF/LoT as appropriate. If the argument is not present, then the
% text of the required argument is put into the LoF. In the first case,
% the optional argument is moving, and in the second case the required
% argument is moving. The |\footnote| command is fragile and must be
% |\protect|ed (i.e., |\protect\footnote{}|) if it is used in a moving 
% argument. If you don't want the footnote to appear in the LoF, use a
% footnoteless optional argument and a footnoted required argument.
%
%   You will probably be surprised if you just do, for example:
% \begin{verbatim}
% \begin{figure}
% ...
% \caption[For LoF]{For figure\footnote{The footnote}}
% \end{figure}
% \end{verbatim}
% because (a) the footnote number may be greater than you thought, and (b)
% the footnote text has vanished. This later is because LaTeX won't typeset
% footnotes from a float. To get an actual footnote within the float you
% have to use a minipage, like:
% \begin{verbatim}
% \begin{figure}
% \begin{minipage}{\linewidth}
% ...
% \caption[For LoF]{For figure\footnote{The footnote}}
% \end{minipage}
% \end{figure}
% \end{verbatim}
% Now you may find that you get two footnotes for the price of one. 
% Fortunately, if you use the \Lpack{\pname} package \emph{without} the
% \Lopt{caption2} option, this will not occur. 
%
%    When using a minipage as above, the footnote text is typeset at the
% bottom of the minipage (i.e., within the float). If you want the footnote
% text typeset at the bottom of the page, then you have to use the
% |\footnotemark| and |\footnotetext| commands like:
% \begin{verbatim}
% \begin{figure}
% ...
% \caption[For LoF]{For figure\footnotemark}
% \end{figure}
% \footnotetext{The footnote}
% \end{verbatim}
% This will typeset the argument of the |\footnotetext| command at the
% bottom of the page where you called the command. Of course, the figure
% might have floated to a later page, and then it's a matter of some
% manual fiddling to get everything on the same page, and possibly
% to get the footnote marks to match correctly with the footnote text.
%
% At this point, you are on your own.
%
% \section{Floats}
%
%     As far as LaTeX is concerned, a float is a box which certain 
% restrictions as to where it can be placed. 
%
% \subsection{Multiple floats}
%
%    You can effectively 
% put what you like inside a float box. Normally there is just a single
% picture or tabular in a float but you can put as many of these as will
% fit inside a float.
%
% \begin{figure}
% \centering
% \hspace*{\fill} {ILLUSTRATION 1} \hfill {ILLUSTRATION 2} \hspace*{\fill}
% \caption{Float with two illustrations} \label{fig:mult1}
% \end{figure}
%
%    Three typical cases of multiple figures/tables in a single
% float come to mind:
% \begin{itemize}
% \item Multiple illustrations/tabulars with a single caption.
% \item Multiple illustrations/tabulars each individually captioned.
% \item Multiple illustrations/tabulars with one main caption and
%       individual subcaptions.
% \end{itemize}
%
%    The \Lpack{subfigure} package is designed for the last of these cases;
% the others do not require a package.
%
%    Figure~\ref{fig:mult1} is an example of multiple illustrations 
% in a single float with a single caption.
%     This figure was produced by the following code.
% \begin{verbatim}
% \begin{figure}
% \centering
% \hspace*{\fill} {ILLUSTRATION 1} \hfill {ILLUSTRATION 2} \hspace*{\fill}
% \caption{Float with two illustrations} \label{fig:mult1}
% \end{figure}
% \end{verbatim}
% The |\hspace*{\fill}| and |\hfill| commands were used to space the two
% illustrations equally. Of course |\includegraphics| or |tabular|
% environments could just as well
% be used instead of the |{ILLUSTRATION N}| text.
%
%    The following code produces Figures~\ref{fig:mult2} and~\ref{fig:mult3}
% which are examples of two seperately captioned illustrations in one
% float.
% \begin{verbatim}
% \begin{figure}
% \centering
% \begin{minipage}{0.4\textwidth}
%   \centering
%   ILLUSTRATION 3
%   \caption{Illustration 3} \label{fig:mult2}
% \end{minipage} 
% \hfill
% \begin{minipage}{0.4\textwidth}
%   \centering
%   ILLUSTRATION 4
%   \caption{Illustration 4} \label{fig:mult3}
% \end{minipage} 
% \end{figure}
% \end{verbatim}
% In this case the illustrations (or graphics or tabulars) are put into
% seperate |minipage| environments within the float, and the captions
% are also put within the |minipage|s. Note that any required |\label|
% must also be inside the |minipage|. If you wished, you could add yet
% another caption after the end of the two |minipage|s.
%
% \begin{figure}
% \centering
% \begin{minipage}{0.4\textwidth}
%   \centering
%   ILLUSTRATION 3
%   \caption{Illustration 3} \label{fig:mult2}
% \end{minipage} 
% \hfill
% \begin{minipage}{0.4\textwidth}
%   \centering
%   ILLUSTRATION 4
%   \caption{Illustration 4} \label{fig:mult3}
% \end{minipage} 
% \end{figure}
%
%  Keith Reckdahl~\cite{EPSLATEX} provides more examples of this
% kind of thing.
%
%
%
%
% \subsection{Where LaTeX puts floats}
%
% The general format for a float environment is: \\
% |\begin{float}[|\meta{loc}|] ... \end{float}| or for double column floats: \\
% |\begin{float*}[|\meta{loc}|] ... \end{float*}| \\
% where the optional argument \meta{loc}, consisting of one or more characters,
% specifies a location where the float may be placed. Note that the 
% \Lpack{multicol} package only supports the starred floats and it will not 
% let you have a single column float. The possible \meta{loc} values are one
% or more of the following:
% \begin{itemize}
% \item[\texttt{b}] \textit{bottom}: at the bottom of a page. This does not apply
% to double column floats as they may only be placed at the top of a page.
% \item[\texttt{h}] \textit{here}: if possible exactly where the float environment
% is defined. It does not apply to double column floats.
% \item[\texttt{p}] \textit{page}: on a seperate page containing only 
% floats (no text).
% \item[\texttt{t}] \textit{top}: at the top of a page. 
% \item[\texttt{!}] make an extra effort to place the float at the earliest place
%  specified by the rest of the argument.
% \end{itemize}
% The default for \meta{loc} is |tbp|, so the float may be placed at the top, 
% or bottom, or on a float-only page; the default works well 95\% of the time.
%  Floats of the same kind are output in
% definition order, except that a double column float may be output before
% a later single column float of the same kind, or 
% \textit{vice-versa}\footnote{This little quirk
% is fixed by the \Lpack{fixltx2e} package, at least for tables and figures.
% The package is part of a normal LaTeX distribution.}. 
% A float is never put on
% an earlier page than its definition but may be put on the same or later page
% of its definition. If a float cannot be placed, all
% suceeding floats will be held up, and LaTeX can store no more than 16 held
% up floats. A float cannot
% be placed if it would cause an overfull page, or it otherwise cannot be fitted
% according the the float parameters.
% A |\clearpage| or |\cleardoublepage| or |\end{document}| flushes
% out all unprocessed floats, irrespective of the \meta{loc} and float
% parameters, putting them on float-only pages. 
%
% \DescribeMacro{\suppressfloats}
%    You can use the command |\suppressfloats[|\meta{pos}|]| to suppress floats
% at a given \meta{pos} on the current page. |\suppressfloats[t]| prevents
% any floats at the top of the page and |\suppressfloats[b]| prevents any
% floats at the bottom of the page. The simple |\suppressfloats| prevents
% both top and bottom floats.
%
%    The \Lpack{flafter} package, which should have come with your LaTeX
% distribution, provides a means of preventing floats from moving
% backwards from their definition position in the text. This can be useful to
% ensure, for example, that a float early in a |\section{}| is not typeset before
% the section heading.
%
% \begin{table}
% \begin{addtomargins}{-0.67in}
% ^^A \centering
% \captionnamefont{\small\sffamily}
% \captiontitlefont{\small\sffamily}
% \setlength{\belowcaptionskip}{10pt}
% \caption{Float placement parameters}\label{tab:fpp}
% \begin{tabular}{lp{0.5\textwidth}r} \hline
% Parameter & Controls & Default \\ \hline
% \multicolumn{3}{c}{Counters --- change with \cs{setcounter} } \\ \hline
% |topnumber|  & max number of floats at top of a page & 2 \\
% |bottomnumber| & max number of floats at bottom of a page & 1 \\
% |totalnumber| & max number of floats on a text page & 3 \\
% |dbltopnumber| & like |topnumber| for double column floats & 2 \\ \hline
% \multicolumn{3}{c}{Commands --- change with \cs{renewcommand} } \\ \hline
% |\topfraction| & max fraction of page reserved for top floats & 0.7 \\
% |\bottomfraction| & max fraction of page reserved for bottom floats & 0.3 \\
% |\textfraction| & min fraction of page that must have text & 0.2 \\
% |\dbltopfraction| & like |\topfraction| for double column floats & 0.7 \\
% |\floatpagefraction| & min fraction of a float page that must have float(s) & 0.5 \\
% |\dblfloatpagefraction| & like |\floatpagefraction| for double column floats & 0.5 \\ \hline
% \multicolumn{3}{c}{Text page lengths --- change with \cs{setlength} } \\ \hline
% |\floatsep| & vertical space between floats & 12pt \\
% |\textfloatsep| & vertical space between a top (bottom) float and suceeding (preceeding) text & 20pt  \\
% |\intextsep| & vertical space above and below an \texttt{h} float & 12pt \\
% |\dblfloatsep| & like |\floatsep| for double column floats & 12pt \\
% |\dbltextfloatsep| & like  |\textfloatsep| for double column floats & 20pt \\ \hline
% \multicolumn{3}{c}{Float page lengths --- change with \cs{setlength} } \\ \hline
% |\@fptop| & space at the top of the page & |0pt plus 1fil| \\
% |\@fpsep| & space between floats & |8pt plus 2fil| \\
% |\@fpbot| & space at the bottom of the page & |0pt plus 1fil| \\
% |\@dblfptop| & like |\@fptop| for double column floats & |0pt plus 1fil| \\
% |\@dblfpsep| & like |\@fpsep| for double column floats & |8pt plus 2fil| \\
% |\@dblfpbot| & like |\@fpbot| for double column floats & |0pt plus 1fil| \\ \hline
% \end{tabular}
% \end{addtomargins}
% \end{table}
%
% Table~\ref{tab:fpp} lists the various float parameters and typical 
% default values. All the lengths are rubber lengths, and the actual 
% defaults depend on both the class and its size option.
%
%    Given the displayed defaults, the height of a top float must be 
% less than 70\% of the textheight and there can be no more than 2 top floats
% on a text page. Similarly, the height of a bottom float must not
% exceed 30\% of the textheight and there can be no more than 1 bottom
% float on a text page. There can be no more than 3 floats (top, bottom and here)
% on the page. At least 20\% of a text page with floats must be text.
% On a float page (one that has no text, only floats) the sum of the heights
% of the floats must be at least 50\% of the textheight. The floats on a float
% page should be vertically centered.
%
%    It can be seen that with the defaults LaTeX might have trouble finding
% a place for a float. Consider what will happen if a float is a bottom float
% whose height is 40\% of the textheight and this is followed by a float whose
% height is 90\% of the textheight. The first is too large to actually go at the
% bottom of a text page but too small to go on a float page by itself. The second
% has to go on a float page but it is too large to share the float page with the
% first float. LaTeX is stuck!
%
%    At this point it is worthwhile to be precise about the effect of a
% one character \meta{loc} argument:
% \begin{itemize}
% \item[\texttt{[b]}] means: `put the float at the bottom of a page with some
%      text above it, and nowhere else'. The float must fit into the 
%      |\bottomfraction| space otherwise it and subsequent floats will be held up.
% \item[\texttt{[h]}] means: `put the float at this point and nowhere else'.
%      The float must fit into the space left on the page otherwise it and 
%      subsequent floats will be held up.
% \item[\texttt{[p]}] means: `put the float on a page that has no text but may
%      have other floats on it'. There must be at least `|\floatpagefraction|'
%      worth of floats to go on a float only page before the float will be
%      be output.
% \item[\texttt{[t]}] means: `put the float at the top of a page with some
%      text below it, and nowhere else'. The float must fit into the 
%      |\topfraction| space otherwise it and subsequent floats will be held up.
% \item[\texttt{[!...]}] means: `ignore the |\...fraction| values for this
%      float'.
% \end{itemize}
%
% You must try and pick a combination from these that will let LaTeX find
% a place to put your floats. However, you can 
% also can change the float parameters to make it easier to find places
% to put floats. Some examples are:
% \begin{itemize}
% \item Decrease |\textfraction| to get more `float' on a text page, but
% the sum of |\textfraction| and |\topfraction| and the sum of |\textfraction|
% and |\bottomfraction| should not exceed 1, otherwise the placement algorithm
% falls apart. A minimum value for |\textfraction| is about 0.10 --- a page
% with less than 10\% text looks better with no text at all, just floats.
%
% \item Both |\topfraction| and |\bottomfraction| can be increased, and it does
% not matter if their sum exceeds 1.0. A good typographic style is that floats
% are encouraged to go at the top of a page, and a better balance is achieved
% if the float space on a page is larger
% at the top than the bottom.
%
% \item Making |\floatpagefraction| too small might have the effect of a
% float page just having one small float. However, to make sure that a float
% page never has more than one float on it, do: \\
% |\renewcommand{\floatpagefraction}{0.01}| \\
% |\setlength{\@fpsep}{\textheight}|
%
% \item Setting |\@fptop| to |0pt|, |\@fpsep| to |8pt| and |\@fpbot|
% to |0pt plus 1fil| will force floats on a float page to start at the top
% of the page.
% \end{itemize}
% If you are experimenting, a reasonable starting position is:
% \begin{verbatim}
% \setcounter{topnumber}{3}
% \setcounter{bottomnumber}{2}
% \setcounter{totalnumber}{4}
% \renewcommand{\topfraction}{0.85}
% \renewcommand{\bottomfraction}{0.5}
% \renewcommand{\textfraction}{0.15}
% \renewcommand{\floatpagefraction}{0.7}
% \end{verbatim}
% and similarly for double column floats if you will have any.
%
%    One of LaTeX's little quirks is that on a text page, the `height' of a float
% is its actual height plus |\textfloatsep| or |\floatsep|, while on a float
% page the `height' is the actual height. This means that when using the default
% \meta{loc} of |[tbp]| at least one of the text page float fractions 
% (|\topfraction| and/or |\bottomfraction|) must be
% larger than the |\floatpagefraction| by an amount sufficient to take account
% of the maximum text page seperation value.
%
%
% \section{The package code} \label{sec:code}
%
%    Announce the name and version of the package, which requires
% LaTeX2e.
%    \begin{macrocode}
%<*usc>
\NeedsTeXFormat{LaTeX2e}
\ProvidesPackage{ccaption}[2005/03/29 v3.2a Extended captioning and new floats]

%    \end{macrocode}
%   In an attempt to avoid name clashes with other packages, all internal
% commands include the string |@cont|.
%
% Note (2001/08/03): Older versions of the \Lpack{amsmath} package did 
% odd things with
% |\@tempa|, |\@tempb| and |\@tempc|. I have replaced any original use
% of these by |\@conttempa|, etc.
%
% Do the options first.
%
% \begin{macro}{\if@contsubfigxx}
% \begin{macro}{\if@contsubfigxxi}
% \begin{macro}{\if@contsubfig}
% These three |\if...| are used to remember if the \Lopt{subfigure20} 
% or \Lopt{subfigure} option has been given. 
% \changes{v3.1}{2002/02/20}{Change to subfigure options, deprecate subfigure21}
%    \begin{macrocode}
\newif\if@contsubfigxx
  \@contsubfigxxfalse
\newif\if@contsubfigxxi
  \@contsubfigxxifalse
\newif\if@contsubfig
  \@contsubfigfalse
\DeclareOption{subfigure20}{\@contsubfigxxtrue\@contsubfigxxifalse\@contsubfigtrue}
\DeclareOption{subfigure21}{\@contsubfigxxfalse\@contsubfigxxitrue\@contsubfigtrue
  \PackageWarningNoLine{ccaption}{%
      The subfigure21 option is deprecated.\MessageBreak
      Try and use the subfigure option instead}}
\DeclareOption{subfigure}{\@contsubfigxxfalse\@contsubfigxxitrue\@contsubfigtrue}
%    \end{macrocode}
% \end{macro}
% \end{macro}
% \end{macro}
%
% \begin{macro}{\if@contcapoption}
% This |\if...| is used to remember if the \Lopt{caption2} 
% option has been given
%    \begin{macrocode}
\newif\if@contcapoption
  \@contcapoptionfalse
\DeclareOption{caption2}{\@contcapoptiontrue}
%    \end{macrocode}
% \end{macro}
%
% \begin{macro}{\if@conttitleopt}
% This |\if...| is used to remember if the \Lopt{titles} 
% option has been given
%    \begin{macrocode}
\newif\if@conttitleopt
  \@conttitleoptfalse
\DeclareOption{titles}{\@conttitleopttrue}
%    \end{macrocode}
% \end{macro}
%
% \begin{macro}{\ProcessOptions}
% Now process the options.
%    \begin{macrocode}

\ProcessOptions\relax

%    \end{macrocode}
% \end{macro}
%
% \subsection{Caption styling}
%
%    The caption styling\footnote{Thanks to Donald Arseneau and Arash
% Esbatil for their perceptive comments on early versions of the 
% styling code.}
% is accomplished by redefining the |\@makecaption|
% command. First, though, define and initialise the user-level commands.
% \changes{v2.7}{2001/01/27}{Added all the styling commands}
%
% The styling is only defined if the \Lopt{caption2} option 
% is \emph{not} given.
% But first we have to declare some new |\if| commands before testing 
% the option.
% \begin{macro}{\if@contcw}
% \begin{macro}{\if@conthang}
% \begin{macro}{\if@contindent}
% For use when checking caption width and captioning styles styles.
%    \begin{macrocode}
\newif\if@contcw
\newif\if@conthang
\newif\if@contindent

%    \end{macrocode}
% \end{macro}
% \end{macro}
% \end{macro}
%  Issue a warning if the \Lopt{caption2} option has been used.
%    \begin{macrocode}
\if@contcapoption
  \PackageWarningNoLine{ccaption)}%
    {You have used the caption2 option.\MessageBreak
     The ccaption styling commands\MessageBreak
     are unavailable to you}
\else

%    \end{macrocode}   
%
% \begin{macro}{\captiondelim}
% \begin{macro}{\@contdelim}
%  For the caption delimeter.
%    \begin{macrocode}
\newcommand{\captiondelim}[1]{\def\@contdelim{#1}}
\captiondelim{: }

%    \end{macrocode}
% \end{macro}
% \end{macro}
%
% \begin{macro}{\captionnamefont}
% \begin{macro}{\@contnfont}
% The font for the caption name.
%    \begin{macrocode}
\newcommand{\captionnamefont}[1]{\def\@contnfont{#1}}
\captionnamefont{}

%    \end{macrocode}
% \end{macro}
% \end{macro}
%
% \begin{macro}{\captiontitlefont}
% \begin{macro}{\@conttfont}
% The font for the caption title.
%    \begin{macrocode}
\newcommand{\captiontitlefont}[1]{\def\@conttfont{#1}}
\captiontitlefont{}

%    \end{macrocode}
% \end{macro}
% \end{macro}
%
% \begin{macro}{\flushleftright}
% \begin{macro}{\centerlastline}
% These are in addition to the |\centering|, |\raggedleft| and |\raggedright|
% declarations for paragraphing. |\flushleftright| sets the skips to TeX's
% normal (block) paragraphing values,
% while |\centerlastline| sets the skips to give a centered last line in
% a block paragraph.
%    \begin{macrocode}
\newcommand{\flushleftright}{%
  \leftskip\z@ \rightskip\z@
  \parfillskip=\z@ plus 1fil}
\newcommand{\centerlastline}{%
  \leftskip=\z@ plus 1fil
  \rightskip=\z@ plus -1fil
  \parfillskip=\z@ plus 2fil}

%    \end{macrocode}
% \end{macro}
% \end{macro}
%
% \begin{macro}{\captionstyle}
% \begin{macro}{\@contcstyle}
% The paragraphing style for the caption.
%    \begin{macrocode}
\newcommand{\captionstyle}[1]{\def\@contcstyle{#1}}
\captionstyle{}

%    \end{macrocode}
% \end{macro}
% \end{macro}
%
% \begin{macro}{\@contcwidth}
% \begin{macro}{\captionwidth}
% \begin{macro}{\changecaptionwidth}
% \begin{macro}{\normalcaptionwidth}
% The macros for dealing with the caption width.
%    \begin{macrocode}
\newlength{\@contcwidth}
\newcommand{\captionwidth}[1]{\setlength{\@contcwidth}{#1}}
\captionwidth{\linewidth}
\newcommand{\changecaptionwidth}{\@contcwtrue}
\newcommand{\normalcaptionwidth}{\@contcwfalse}
\normalcaptionwidth

%    \end{macrocode}
% \end{macro}
% \end{macro}
% \end{macro}
% \end{macro}
%
% \begin{macro}{\@contindw}
% \begin{macro}{\hangcaption}
% \begin{macro}{\indentcaption}
% \begin{macro}{\normalcaption}
% The macros for hanging and indented captions.
%    \begin{macrocode}
\newlength{\@contindw}
\newcommand{\hangcaption}{\@conthangtrue\@contindentfalse}
\newcommand{\indentcaption}[1]{\setlength{\@contindw}{#1}%
  \@conthangfalse\@contindenttrue}
\newcommand{\normalcaption}{\@conthangfalse\@contindentfalse}
\normalcaption

%    \end{macrocode}
% \end{macro}
% \end{macro}
% \end{macro}
% \end{macro}
%
% \begin{macro}{\precaption}
% \begin{macro}{\@contpre}
% \begin{macro}{\postcaption}
% \begin{macro}{\@contpost}
% \begin{macro}{\midbicaption}
% \begin{macro}{\@contmidbi}
% The macros for the pre- and post-caption text/commands, and
% for the mid-caption command for bilingual captions.
%    \begin{macrocode}
\newcommand{\precaption}[1]{\def\@contpre{#1}}
\precaption{}
\newcommand{\postcaption}[1]{\def\@contpost{#1}}
\postcaption{}
\newcommand{\midbicaption}[1]{\def\@contmidbi{#1}}
\midbicaption{}

%    \end{macrocode}
% \end{macro}
% \end{macro}
% \end{macro}
% \end{macro}
% \end{macro}
% \end{macro}
%
% \begin{macro}{\@makecaption}
%  This is a reimplementation of the kernel |\@makecaption| command.
% As well as including the caption typesetting commands it enables
% captions that include forced newlines (e.g., by |\\|).
%
% The first part is due to 
% Donald Arseneau\footnote{Email: \texttt{asnd@triumf.ca}} from postings
% to the CTT newsgroup and Email discussions. The |\topskip| strut is
% used whenever the caption is the first part of the float. This means,
% among other things, that if a caption comes at the
% top of a page, then the first line of the caption will be aligned with
% the normal first line of a page. The |\abovecaptionskip| is only used
% when there is something above the caption in the current float.
% \changes{v2.7}{2001/01/27}{Major surgery to \cs{@makecaption}}
% \changes{v3.0a}{2001/08/03}{Replaced \cs{@tempa} by \cs{@conttempa} to foil
%                old versions of amsmath package}
%    \begin{macrocode}
\long\def\@makecaption#1#2{\let\@conttempa\relax
  \ifdim\prevdepth>-99\p@ \vskip\abovecaptionskip
  \else \def\@conttempa{\vbox to\topskip{}}\fi
%    \end{macrocode}
% \begin{macro}{\@contfnote}
% \begin{macro}{\@contfmark}
% The caption title will be typeset twice, firstly to measure its width
% and secondly to actually typeset it. To avoid problems caused by
% a footnote in the caption getting processed twice, we temporarily
% disable the expected relevant commands.
% \changes{v3.1b}{2002/10/18}{Added \cs{label} to nulled macros in 
%                             \cs{@makecaption} to stop hyperref claiming
%                             multiple anchor points}
%    \begin{macrocode}
  \let\@contfnote\footnote \renewcommand{\footnote}[2][]{}
  \let\@contfmark\footnotemark \renewcommand{\footnotemark}[1][]{}
  \let\@contlabel\label \renewcommand{\label}[1]{}
%    \end{macrocode}
% \end{macro}
% \end{macro}
% Now measure the width of the total caption, not forgetting to take account
% of the font specifications, and then restore the footnoting. 
%    \begin{macrocode}
  \sbox\@tempboxa{\@contnfont #1\@contdelim \@conttfont #2}
  \let\footnote\@contfnote
  \let\footnotemark\@contfmark
  \let\label\@contlabel
%    \end{macrocode}
% If the caption is less than one
% line, then the whole caption needs to be centered on the page (otherwise
% the short caption may be typeset flushleft).
%    \begin{macrocode}
  \ifdim\wd\@tempboxa<\linewidth \centering \fi
  \if@contcw
%    \end{macrocode}
% For typesetting at anything other than the normal width, put the caption
% into a |\parbox| of the specified width. This must be centered.
%    \begin{macrocode}
    \centering
    \parbox{\@contcwidth}{%
  \fi
  \if@conthang
%    \end{macrocode}
% For a hanging caption we have to measure the width of the caption name,
% then typeset the whole caption in a hanging paragraph.
%    \begin{macrocode}
    \sbox\@tempboxa{\@contnfont #1\@contdelim}
    \@contpre%
    {\@contnfont #1\@contdelim}\@conttempa 
    {\@contcstyle\hangindent=\wd\@tempboxa\hangafter=\@ne\@conttfont #2\par}
  \else
    \if@contindent
%    \end{macrocode}
% An indented caption is similar, except the amount of indentation is
% kept in |\@contindw|.
%    \begin{macrocode}
      \@contpre%
      {\@contnfont #1\@contdelim}\@conttempa 
      {\@contcstyle\hangindent=\@contindw\hangafter=\@ne\@conttfont #2\par}
    \else
%    \end{macrocode}
% For the normal style, just typeset the caption.
%    \begin{macrocode}
      \@contpre%
      {\@contnfont #1\@contdelim}\@conttempa 
      {\@contcstyle\@conttfont #2\par}
    \fi
  \fi
%    \end{macrocode}
% Finish off the typesetting by processing the post-text,  and if not using
% the normal width then close off the |\parbox|, and lastly put in some
% vertical space. 
%    \begin{macrocode}
  \@contpost
  \if@contcw
    \par
    }  % end of parbox
  \fi
  \vskip\belowcaptionskip}

%    \end{macrocode}
% \end{macro}
%
%  This finishes off the non \Lopt{caption2} option.
%    \begin{macrocode}
\fi  % end of test (\if@contcapoption) on caption2 option

%    \end{macrocode}
%
%
%
%
% \subsection{Continuation captions and legends}
%
% \begin{macro}{\contcaption}
%    |\contcaption{|\meta{text}|}| is a user-level command. 
%    It is a simplified
%    version of the normal |\caption| command as it doesn't have to deal
%    too much with numbering or list of \dots entries.
% \changes{v3.1c}{2003/11/14}{Added number resetting to \cs{contcaption}}
%    \begin{macrocode}
\newcommand{\contcaption}{%
  \addtocounter{\@captype}{\m@ne}%
  \refstepcounter{\@captype}%
  \@contcaption\@captype}

%    \end{macrocode}
% \end{macro}
%
%
% \begin{macro}{\@contcaption}
%    This is the workhorse for the |\contcaption| command. In turn,
% it uses the |\@makecaption| command (defined in the usual classes)
% to do most of its work. It
% uses the number of the previous |\caption| command in the same
% type of float and its implementation includes much of the code
% used in the LaTeX |\@caption| command.
% 
%    \begin{macrocode}
\long\def\@contcaption#1#2{%
  \par
  \begingroup
     \@parboxrestore
     \if@minipage
       \@setminipage
     \fi
     \normalsize
     \@makecaption{\csname fnum@#1\endcsname}{\ignorespaces #2}\par
  \endgroup}

%    \end{macrocode}
% \end{macro}
%
%
% \begin{macro}{\abovelegendskip}
% \begin{macro}{\belowlegendskip}
%    These two lengths control the vertical spacing before and after a
% legend. We will give these values such that a legend will ocupy an
% integral number of lines.
%    \begin{macrocode}
\newlength{\abovelegendskip}
\setlength{\abovelegendskip}{0.5\baselineskip}
\newlength{\belowlegendskip}
\setlength{\belowlegendskip}{\abovelegendskip}

%    \end{macrocode}
% \end{macro}
% \end{macro}
%
% \begin{macro}{\legend}
%    The command is called as |\legend{|\meta{text}|}|. It is intended
% to be used in a float environment for an `anonymous' caption, but can be
% used anywhere.
%
%    The implementation is similar to the |\caption| command but we have
% to eliminate printing of a delimeter.
% \changes{v2.7}{2001/01/27}{Changed \cs{legend} to use \cs{@makecaption} instead of \cs{@makelegend}}
% \changes{v2.7}{2001/01/27}{Deleted \cs{@makelegend} and \cs{formatlegend}}
%    \begin{macrocode}
\newcommand{\legend}[1]{%
  \par
  \begingroup
     \@parboxrestore
     \if@minipage
       \@setminipage
     \fi
     \normalsize
     \captiondelim{\mbox{}}
     \@makecaption{}{\ignorespaces #1}\par
  \endgroup}

%    \end{macrocode}
% \end{macro}
%
%
% \begin{macro}{\namedlegend}
% |\namedlegend[|\meta{short-title}|]{|\meta{long-title}|}| is like the
% |\caption| command except that it does not number the caption.
%    \begin{macrocode}
\newcommand{\namedlegend}{\@dblarg{\@legend\@captype}}

%    \end{macrocode}
% \end{macro}
%
% \begin{macro}{\@legend}
% |\@legend{|\meta{type}|}[|\meta{short-title}|]{|\meta{long-title}|}|
% is the workhorse for the |\namedlegend| command. In turn, it calls
% |\@makelegend|. It requires two commands to have been defined, namely
% |\flegtoc@type| and |\fleg@type|. The command |\flegtoc@type{|\meta{text}|}|
% is responsible for writing a title text to the appropriate listof file.
% |\fleg@type| is responsible for typeseting the name of the legend.
% \changes{v2.7}{2001/01/27}{Changed \cs{@legend} to use \cs{@makecaption}
% instead of \cs{@makelegend}}
%    \begin{macrocode}
\long\def\@legend#1[#2]#3{%
  \par
  \csname flegtoc@#1\endcsname{#2}%
  \begingroup
    \@parboxrestore
    \if@minipage
      \@setminipage
    \fi
    \normalsize
    \@makecaption{\csname fleg@#1\endcsname}{\ignorespaces #3}\par
  \endgroup}

%    \end{macrocode}
% \end{macro}
%
% \begin{macro}{\flegtoc@table}
% \begin{macro}{\flegtoc@figure}
%  These macros write a |\namedlegend| title to the respective listof file.
% By default they do nothing.
%    \begin{macrocode}
\newcommand{\flegtoc@table}[1]{}
\newcommand{\flegtoc@figure}[1]{}

%    \end{macrocode}
% \end{macro}
% \end{macro}
%
% \begin{macro}{\fleg@table}
% \begin{macro}{\fleg@figure}
%  These macros typeset the name before the title of a |\namedlegend|. By
% default they are defined to mimic the normal captioning style.
% \changes{v2.7}{2001/01/27}{Deleted the : delimeter from \cs{fleg@table}
% and \cs{fleg@figure}}
%    \begin{macrocode}
\newcommand{\fleg@table}{\tablename}
\newcommand{\fleg@figure}{\figurename}

%    \end{macrocode}
% \end{macro}
% \end{macro}
%
%
% \subsection{Non-float captions}
%
% \begin{macro}{\newfixedcaption}
% \begin{macro}{\renewfixedcaption}
% \begin{macro}{\providefixedcaption}
% These commands are defined in terms of their |\...command| counterparts.\\
% Call as |\...fixedcaption[|\meta{capcommand}|]{|\meta{command}|}{|\meta{env}|}|
%    \begin{macrocode}
\newcommand{\newfixedcaption}[3][\caption]{%
  \newcommand{#2}{\def\@captype{#3}#1}}
\newcommand{\renewfixedcaption}[3][\caption]{%
  \renewcommand{#2}{\def\@captype{#3}#1}}
\newcommand{\providefixedcaption}[3][\caption]{%
  \providecommand{#2}{\def\@captype{#3}#1}}

%    \end{macrocode}
% \end{macro}
% \end{macro}
% \end{macro}
%
% \subsection{Bilingual captions}
%
%    The bilingual caption commands all use internal grouping so
% that any changes are kept local. This has the unfortunate side-effect
% that any |\label| command must be within the grouping otherwise the
% wrong number is picked up. To make the coding, if not necessarily the 
% use, of the commands simpler, I have not used the traditional style
% of square brackets for optional caption text arguments. Instead, empty
% `required' arguments are used as the implementation means. 
%
% \begin{macro}{\@if@contemptyarg}
%  For dealing with empty arguments. 
% |\@if@contemptyarg{|\meta{testarg}|}{|\meta{YES}|}{|\meta{NO}|}| checks
% if \meta{testarg} is empty (consists of zero or more spaces only). If
% it is empty then the \meta{YES} argument is processed otherwise the 
% \meta{NO} argument is processed. The implementation uses code suggested
% by Donald Arseneau (see section~\ref{sec:peril} for some background on this).
% \changes{v2.6c}{2000/03/15}{Rewrite of \cs{@if@contemptyarg}}
%    \begin{macrocode}
\begingroup
\catcode`\Q=3
\long\gdef\@if@contemptyarg#1{\@xif@contmt#1QQ\@secondoftwo\@firstoftwo\@nil}
\long\gdef\@xif@contmt#1#2Q#3#4#5\@nil{#4}
\endgroup

%    \end{macrocode}
% \end{macro}
%
% \begin{macro}{\bitwonumcaption}
% \changes{v2.6}{2000/02/26}{New \cs{bitwonumcaption} command}
%  The 6 arguments are: optional label, short and long in language 1, 
% name in
% language 2, and short and long in language 2. Both texts are put
% into the List of as numbered entries.
% \changes{v3.2a}{2005/03/29}{Fix empty check in \cs{bitwonumcaption}}
%    \begin{macrocode}
\newcommand{\bitwonumcaption}[6][\@empty]{%
  \begingroup
%    \end{macrocode}
% Check if the first language argument is vacuous, then call
% the normal |\caption| for language 1.
%    \begin{macrocode}
  \@if@contemptyarg{#2}{\caption{#3}}{\caption[#2]{#3}}
%    \end{macrocode}
% Do the optional labeling.
%    \begin{macrocode}
  \ifx\@empty#1\else
    \label{#1}
  \fi
%    \end{macrocode}
% Remove any extra spacing between the captions, and set the
% NAME for the second caption. Use a command to transfer
% the NAME to the renewell code to avoid circularity if
% for example, we are trying to redefine |\tablename| as
% |\tablename|. Decrement the caption counter.
%    \begin{macrocode}
  \setlength{\abovecaptionskip}{0pt}
  \setlength{\belowcaptionskip}{0pt}
  \edef\@conttempc{#4}
  \expandafter \renewcommand \csname \@captype name\endcsname{\@conttempc}
  \addtocounter{\@captype}{-1}
%    \end{macrocode}
% Now repeat for the second language caption.
% \changes{v2.7}{2001/01/27}{Added \cs{@contmidbi} to bilingual captions}
%    \begin{macrocode}
  \@contmidbi
  \@if@contemptyarg{#5}{\caption{#6}}{\caption[#5]{#6}}
  \endgroup}

%    \end{macrocode}
% \end{macro}
%
% \begin{macro}{\bionenumcaption}
% \changes{v2.6}{2000/02/26}{New \cs{bionenumcaption} command}
%  The 6 arguments are: optional labelling,
%  short and long in language 1, name in
% language 2, and short and long in language 2. Both texts are put
% into the List of, but only the first is numbered.
% \changes{v3.2a}{2005/03/29}{Fix empty check in \cs{bionenumcaption}}
%    \begin{macrocode}
\newcommand{\bionenumcaption}[6][\@empty]{%
  \begingroup
%    \end{macrocode}
% Check if the first language argument is vacuous, then call
% the normal |\caption| for language 1.
%    \begin{macrocode}
  \@if@contemptyarg{#2}{\caption{#3}}{\caption[#2]{#3}}
%    \end{macrocode}
% Do the optional labeling.
%    \begin{macrocode}
  \ifx\@empty#1\else
    \label{#1}
  \fi
%    \end{macrocode}
% Do the between captions code.
%    \begin{macrocode}
  \setlength{\abovecaptionskip}{0pt}
  \setlength{\belowcaptionskip}{0pt}
  \edef\@conttempc{#4}
  \expandafter \renewcommand \csname \@captype name\endcsname{\@conttempc}
%    \end{macrocode}
% Use a continuation caption for the second language, not forgetting
% to add the appropriate unnumbered text to the List.
%    \begin{macrocode}
  \@contmidbi
  \contcaption{#6}
  \@if@contemptyarg{#5}{%
    \addcontentsline{\csname ext@\@captype\endcsname}{\@captype}%
      {\protect\numberline{}{\ignorespaces #6}}}{%
    \addcontentsline{\csname ext@\@captype\endcsname}{\@captype}%
      {\protect\numberline{}{\ignorespaces #5}}}
  \endgroup}

%    \end{macrocode}
% \end{macro}
%
% \begin{macro}{\bicaption}
% \changes{v2.6}{2000/02/26}{New \cs{bicaption} command}
%  The 5 arguments are: optional labelling, 
% short and long in language 1, name in
% language 2, and long in language 2. 
% Only the first text is put into the List.
% \changes{v3.2a}{2005/03/29}{Fix empty check in \cs{bicaption}}
%    \begin{macrocode}
\newcommand{\bicaption}[5][\@empty]{%
  \begingroup
%    \end{macrocode}
% Check if the first language argument is vacuous, then call
% the normal |\caption| for language 1.
%    \begin{macrocode}
  \@if@contemptyarg{#2}{\caption{#3}}{\caption[#2]{#3}}
%    \end{macrocode}
% Do the optional labeling.
%    \begin{macrocode}
  \ifx\@empty#1\else
    \label{#1}
  \fi
%    \end{macrocode}
% Do the between captions code and 
% finally just use |\contcaption| for the
% second language.
%    \begin{macrocode}
  \setlength{\abovecaptionskip}{0pt}
  \setlength{\belowcaptionskip}{0pt}
  \edef\@conttempc{#4}
  \expandafter \renewcommand \csname \@captype name\endcsname{\@conttempc}
  \@contmidbi
  \contcaption{#5}
  \endgroup}

%    \end{macrocode}
% \end{macro}
%
% \begin{macro}{\bicontcaption}
% \changes{v2.6}{2000/02/26}{New \cs{bicontcaption} command}
%  The 3 arguments are long in language 1, name in
% language 2, and long in language 2.
%    \begin{macrocode}
\newcommand{\bicontcaption}[3]{%
  \begingroup
%    \end{macrocode}
% Call |\contcaption| for language 1.
%    \begin{macrocode}
  \contcaption{#1}
%    \end{macrocode}
% Do the between captions code and use |\contcaption| for the second
% language.
%    \begin{macrocode}
  \setlength{\abovecaptionskip}{0pt}
  \setlength{\belowcaptionskip}{0pt}
  \edef\@conttempc{#2}
  \expandafter \renewcommand \csname \@captype name\endcsname{\@conttempc}
  \@contmidbi
  \contcaption{#3}
  \endgroup}

%    \end{macrocode}
% \end{macro}
%
% \subsection{The code for the \Lpack{longtable} package}
%
% \begin{macro}{\LT@makecaption}
% This is defined in the \Lpack{longtable} package and sets a caption 
% essentially as a centered multicolumn entry in the table. To utilize
% \Lpack{ccaption}'s font settings it has to be modified.
% \changes{v3.2}{2005/03/21}{Added \cs{LT@makecaption}}
%    \begin{macrocode}
\providecommand*{\LT@makecaption}[3]{}
\renewcommand*{\LT@makecaption}[3]{%
  \LT@mcol\LT@cols c{\hb@xt@ \z@{\hss\parbox[t]\LTcapwidth{%
    \sbox\@tempboxa{#1{\@contnfont #2\@contdelim}\@conttfont #3}%
    \ifdim\wd\@tempboxa>\hsize
      #1{\@contnfont #2\@contdelim}\@conttfont #3%
    \else
      \hb@xt@ \hsize{\hfil\box\@tempboxa\hfil}%
    \fi
    \endgraf\vskip\belowcaptionskip}%
  \hss}}}

%    \end{macrocode}
% \end{macro}   
%
% \begin{macro}{\longbitwonumcaption}
% A version of \cs{bitwonumcaption} for use in a |longtable|.
% \changes{v3.2}{2005/03/21}{Added \cs{longbitwonumcaption}}
%    \begin{macrocode}
\newcommand*{\longbitwonumcaption}[5]{%
  \@if@contemptyarg{#1}{\caption{#2}}{\caption[#1]{#2}}%
  \global\let\@cont@oldtablename\tablename
  \gdef\tablename{#3}
  \\
  \@if@contemptyarg{#4}{\caption{#5}}{\caption[#4]{#5}}%
  \global\let\tablename\@cont@oldtablename}

%    \end{macrocode}
% \end{macro}
%
% \begin{macro}{\@cont@LT@nonumintoc}
% \begin{macro}{\@cont@oldLT@c@ption}
% We need a special version of \Lpack{longtable}'s \cs{LT@c@ption}
% that does not put a number in the ToC.
% \changes{v3.2}{2005/03/21}{Added \cs{@cont@LT@nonumintoc}}
%    \begin{macrocode}
\def\@cont@LT@nonumintoc#1[#2]#3{%
  \LT@makecaption#1\fnum@table{#3}%
  \def\@tempa{#2}%
  \ifx\@tempa\@empty\else
    {\let\\\space
      \addcontentsline{lot}{table}{\protect\numberline{}{#2}}}%
  \fi}
\let\@cont@oldLT@c@ption\LT@c@ption

%    \end{macrocode}
% \end{macro}
% \end{macro}
%
% \begin{macro}{\longbionenumcaption}
% A version of \cs{bionenumcaption} for use in a |longtable|.
% \changes{v3.2}{2005/03/21}{Added \cs{longbionenumcaption}}
%    \begin{macrocode}
\newcommand*{\longbionenumcaption}[5]{%
  \@if@contemptyarg{#1}{\caption{#2}}{\caption[#1]{#2}}%
  \global\let\@cont@oldtablename\tablename
  \gdef\tablename{#3}
  \global\let\LT@c@ption\@cont@LT@nonumintoc
  \\
  \@if@contemptyarg{#4}{\caption{#5}}{\caption[#4]{#5}}%
  \global\let\tablename\@cont@oldtablename
  \global\let\LT@c@ption\@cont@oldLT@c@ption}

%    \end{macrocode}
% \end{macro}
%
% \begin{macro}{\longbicaption}
% A version of \cs{bicaption} for use in a |longtable|.
% \changes{v3.2}{2005/03/21}{Added \cs{longbicaption}}
%    \begin{macrocode}
\newcommand*{\longbicaption}[4]{%
  \@if@contemptyarg{#1}{\caption{#2}}{\caption[#1]{#2}}%
  \\
  \caption*{{\normalfont\@contnfont #3\@contdelim} #4}}

%    \end{macrocode}
% \end{macro}
%
%  
%
% \subsection{The \Lopt{subfigure} options}
%
%
% \begin{macro}{\@contkeep}
% \begin{macro}{\@contset}
% \begin{macro}{\subconcluded}
% \begin{macro}{\subfigold@contcaption}
%   These are common to both \Lopt{subfigure} options. |\@contkeep| stores
% the current sub(figure/table) number in counter |@contsubnum| and 
% |\@contset| sets the sub(figure/table) number to the value of |@contsubnum|.
% |\subconcluded| sets the sub(figure/table) number to zero. The original
% definition of |\@contcaption| is kept in |\subfigold@contcaption|.
%    \begin{macrocode}
\if@contsubfig
  \newcounter{@contsubnum}
  \newcommand{\@contkeep}{\setcounter{@contsubnum}{\value{sub\@captype}}}
  \newcommand{\@contset}{\setcounter{sub\@captype}{\value{@contsubnum}}}
  \newcommand{\subconcluded}{\setcounter{sub\@captype}{0}}
  \let\subfigold@contcaption\@contcaption
%    \end{macrocode}
% \end{macro}
% \end{macro}
% \end{macro}
% \end{macro}
%
% \begin{macro}{\toclevel@subtable}
% \begin{macro}{\toclevel@subfigure}
%  These are needed if the \Lpack{hyperref} package is loaded
% as well as subfigures.
% \changes{v2.6e}{2001/01/12}{Added toclevel@subtable and @subfigure commands for hyperref}
%    \begin{macrocode}
  \providecommand{\toclevel@subtable}{1}
  \providecommand{\toclevel@subfigure}{1}
\fi
%    \end{macrocode}
% \end{macro}
% \end{macro}
%
% \begin{macro}{\if@contmaincaption}
% This is set TRUE after the (cont)caption in a float has been processed.
% (A |\newif| cannot be used within an |\if...\fi| construct.)
%    \begin{macrocode}
\newif\if@contmaincaption
  \@contmaincaptionfalse
%    \end{macrocode}
% \end{macro}
%
% \begin{macro}{\if@contbotsub}
% A flag indicating whether the subcaption is to be at the bottom or
% top of the subfigure/subtable; TRUE for the subcaption at the bottom.
%    \begin{macrocode}
\newif\if@contbotsub
  \@contbotsubtrue

%    \end{macrocode}
% \end{macro}
%
% \subsubsection{Option \Lopt{subfigure20}}
%
% \changes{v2.5}{2000/02/20}{Addition of subfigure20 section in code portion}
%     In order to eliminate an ordering dependency between the 
% \Lpack{subfigure} and \Lpack{\pname} packages, modifications to the
% original \Lpack{subfigure} code have to be done at the start of the
% document after all packages have been loaded.
% First for subfigure 2.0, if it is called for.
%    \begin{macrocode}
\AtBeginDocument{%
\if@contsubfigxx
%    \end{macrocode}
%
% \begin{macro}{\caption}
% \begin{macro}{\contcaption}
% \begin{macro}{\@float}
% \begin{macro}{\@dbflt}
%  These original commands are all modified to set the value of 
% |\if@contmaincaption|. The (cont)caption commands set it to TRUE and the
% float commands set it FALSE. Additionally, the |\@float| and |\@dbflt|
% commands are modified to zero the subfloat counter, if it is defined.
%    \begin{macrocode}
  \let\@contoldc\caption
  \renewcommand{\caption}{\@contmaincaptiontrue\@contoldc}
  \let\@contoldcont\contcaption
  \renewcommand{\contcaption}{\@contmaincaptiontrue\@contoldcont}
  \let\@contoldf\@float
  \renewcommand{\@float}[1]{\@contmaincaptionfalse
                \@ifundefined{c@sub#1}{}{\csname c@sub#1\endcsname = 0\relax}
                \@contoldf{#1}}
  \let\@contoldff\@dbflt
  \renewcommand{\@dbflt}[1]{\@contmaincaptionfalse
                \@ifundefined{c@sub#1}{}{\csname c@sub#1\endcsname = 0\relax}
                \@contoldff{#1}}

%    \end{macrocode}
% \end{macro}
% \end{macro}
% \end{macro}
% \end{macro}
%
% \begin{macro}{\@subfloat}
% This macro from \Lopt{subfigure} v2.0 is modified to enable subcaptions
% to be placed at either the top or bottom of the sub... (the original only
% placed them at the bottom). First, the subfigure/table is set in a box.
%    \begin{macrocode}
  \def\@subfloat#1[#2]#3{%
    \setbox\@tempboxa \hbox{#3}%
    \@tempdima=\wd\@tempboxa
    \if@contbotsub
%    \end{macrocode}
% The subcaption is to be put at the bottom, so typeset the figure, followed 
% by the caption, if any.
%    \begin{macrocode}
      \vtop{%
        \vbox{\vskip\subfigtopskip
              \box\@tempboxa}%
        \ifx \@empty#2\relax \else
          \vskip\subfigcapskip
          \@subcaption{#1}{#2}%
        \fi
        \vskip\subfigbottomskip}%
    \else
%    \end{macrocode}
% The subcaption is to be put at the top, so typeset the caption if any,
% followed by the figure.
%    \begin{macrocode}
      \vtop{%
        \ifx \@empty#2\relax \else
          \vskip\subfigcapskip
          \begingroup\@subcaption{#1}{#2}\endgroup%
        \fi
        \vbox{\vskip\subfigtopskip
              \box\@tempboxa}%
        \vskip\subfigbottomskip}%
    \fi
    \egroup}

%    \end{macrocode}
% \end{macro}
%
%
% \begin{macro}{\@subcaption}
% The original |\@subcaption| command produces unexpected results in the
% ToC (i.e., |numberline| appears instead of |\numberline| because of
% the original internal definition of |\protect|). I have also
% modified it so that when a top main caption is being used, it
% adds the subcaption to the ToC directly.
%
%    Sebastien Derriere found that there were problems when fragile commands
% were used within a continued subcaption. Steven Douglas Cochran kindly 
% provided a fix for this.
% \changes{v2.6d}{2001/01/02}{Applied SDC fix to @subcaption command}
% 
%    \begin{macrocode}
  \renewcommand{\@subcaption}[2]{%
    \begingroup
      \let\label\@gobble
      \let\protect\string      % SDC mod
      \if@contmaincaption
        \addcontentsline{\csname ext@#1\endcsname}{#1}%
          {\protect\numberline{\csname p@#1\endcsname\csname the#1\endcsname}%
          {\ignorespaces #2}}%
        \gdef\@subfigcaptionlist{}
      \else
        \xdef\@subfigcaptionlist{%
        \@subfigcaptionlist,%
%%        {\string\numberline {\@currentlabel}%   % SDC mod
        {\protect\numberline {\@currentlabel}%   % SDC mod
         \noexpand{\ignorespaces #2}}}%
      \fi
    \endgroup
    \@nameuse{@make#1caption}{\@nameuse{@the#1}}{#2}}

%    \end{macrocode}
% \end{macro}
%
% \begin{macro}{\subfigure}
% \begin{macro}{\subtable}
%  These are revised versions of the original commands. They are now
% aliases for |\subbottom| and |\subtop| respectively. In their original
% form they were both effectively aliases for |\subbottom| only.
%    \begin{macrocode}
  \let\subfigure\subbottom
  \let\subtable\subtop
%    \end{macrocode}
% \end{macro}
% \end{macro}
%    \begin{macrocode}
\fi
}
%    \end{macrocode}
%    The end of the |\AtBeginDocument| code for \Lopt{subfigure20}.
%
%
%  Do the remaining code for the \Lopt{subfigure20} option, if called for.
%    \begin{macrocode}
\if@contsubfigxx
%    \end{macrocode}
% \begin{macro}{\subbottom}
% \begin{macro}{\@contsubbody}
%  |\subbottom[|\meta{caption}|]{|\meta{text}|}| typesets a subcaption
% when the main caption is at the end of the float environment. The code
% is a slight modification of the original |\subfigure| command in that
% the bottom flag is added and set to true and the subcaption number is
% stored. The caption number must be locally advanced if the main caption
% has not yet been processed (i.e., is at the bottom of the float). As most
% of the code is common with |\subtop| it is placed into the |\@contsubbody|
% macro.
%    \begin{macrocode}
  \newcommand{\subbottom}{%
    \@contbotsubtrue
    \@contsubbody}

  \newcommand{\@contsubbody}{%
    \bgroup
    \if@contmaincaption\else
      \advance\csname c@\@captype\endcsname\@ne
    \fi
    \refstepcounter{sub\@captype}\@contkeep%
    \leavevmode
    \@ifnextchar [%
      {\@subfloat{sub\@captype}}
      {\@subfloat{sub\@captype}[\@empty]}}

%    \end{macrocode}
% \end{macro}
% \end{macro}
%
% \begin{macro}{\contsubbottom}
% \begin{macro}{\subbody@cont}
%  The continued version of |\subbottom|. It restores the kept subcaption 
% number before incrementing and keeping it. As most of the code is common
% with |\contsubtop| it is kept in the |\subbody@cont|.
%    \begin{macrocode}
  \newcommand{\contsubbottom}{%
    \@contbotsubtrue
    \subbody@cont}

  \newcommand{\subbody@cont}{%
    \bgroup
    \@contset
    \refstepcounter{sub\@captype}\@contkeep%
    \leavevmode
    \@ifnextchar [%
      {\@subfloat{sub\@captype}}
      {\@subfloat{sub\@captype}[\@empty]}}

%    \end{macrocode}
% \end{macro}
% \end{macro}
%
% \begin{macro}{\subtop}
%  |\subtop[|\meta{caption}|]{|\meta{text}|}| typesets a subcaption
% at the top of the subfigure/table. This is almost identical to |\subbottom|.
%    \begin{macrocode}
  \newcommand{\subtop}{%
    \@contbotsubfalse
    \@contsubbody}

%    \end{macrocode}
% \end{macro}
%
% \begin{macro}{\contsubtop}
%  The continued version of |\subtop|. 
%    \begin{macrocode}
  \newcommand{\contsubtop}{%
    \@contbotsubfalse
    \subbody@cont}

%    \end{macrocode}
% \end{macro}
%
% \begin{macro}{\@contcaption}
% The |\@contcaption| command must be modified to add the listed
% subcaptions (if any, and there should be none for top main captions)
% to the ToC. A simplified version of the \Lopt{subfigure} redefinition
% of |\@caption|.
%    \begin{macrocode}
  \long\def\@contcaption#1#2{%
    \subfigold@contcaption{#1}{#2}%
    \@for \@conttempa:=\@subfigcaptionlist \do {%
      \ifx\@empty\@conttempa\relax \else
        \addcontentsline
          {\@nameuse{ext@sub#1}}%
          {sub#1}%
          {\@conttempa}%
       \fi}%
    \gdef\@subfigcaptionlist{}}

%    \end{macrocode}
% \end{macro}
%
% \begin{macro}{\contsubtable}
% \begin{macro}{\contsubfigure}
%  Aliases for |\contsubtop| and |\contsubbottom|, respectively.
%    \begin{macrocode}
  \let\contsubtable\contsubtop
  \let\contsubfigure\contsubbottom

%    \end{macrocode}
% \end{macro}
% \end{macro}
%
% The end of the \Lopt{subfigure20} option code.
%    \begin{macrocode}
\fi

%    \end{macrocode}
%
%    This is the end of the version 2.0 code. 
%
% \subsubsection{Option \Lopt{subfigure21}}
%
%
% \begin{macro}{\caption}
% \begin{macro}{\contcaption}
% \begin{macro}{\@float}
% \begin{macro}{\@dbflt}
%  These original commands are all modified to set the value of 
% |\if@contmaincaption|. The (cont)caption commands set it to TRUE and the
% float commands set it FALSE. Additionally, the |\@float| and |\@dbflt|
% commands are modified to zero the subfloat counter, if it is defined.
% \changes{v3.1a}{2002/04/01}{Added changes to \cs{@float}, etc, to
% subfigure (21) option code}
%    \begin{macrocode}
\if@contsubfigxxi
  \let\@contoldc\caption
  \renewcommand{\caption}{\@contmaincaptiontrue\@contoldc}
  \let\@contoldcont\contcaption
  \renewcommand{\contcaption}{\@contmaincaptiontrue\@contoldcont}
  \let\@contoldf\@float
  \renewcommand{\@float}[1]{\@contmaincaptionfalse
                \@ifundefined{c@sub#1}{}{\csname c@sub#1\endcsname = 0\relax}
                \@contoldf{#1}}
  \let\@contoldff\@dbflt
  \renewcommand{\@dbflt}[1]{\@contmaincaptionfalse
                \@ifundefined{c@sub#1}{}{\csname c@sub#1\endcsname = 0\relax}
                \@contoldff{#1}}
\fi

%    \end{macrocode}
% \end{macro}
% \end{macro}
% \end{macro}
% \end{macro}
%
% \begin{macro}{\@contsubfloat}
%  This is a version of the \Lpack{subfigure} |\subfigure| command. 
% The revised version stores the subcounter. 
% \changes{v3.1}{2002/02/20}{Replaced \cs{do@contsubfig} by \cs{@contsubfloat}}
% \changes{v3.1a}{2002/04/01}{Added \cs{subfig@oldlabel} to \cs{@contsubfloat}}
%    \begin{macrocode}
\newcommand{\@contsubfloat}{%
  \bgroup
  \let\subfig@oldlabel=\label
  \let\label=\sub@label
  \refstepcounter{sub\@captype}\@contkeep%   % <- change here
  \@ifnextchar [%
    {\@@cont@subfloat}%
    {\@@cont@subfloat[\@empty]}}

%    \end{macrocode}
% \end{macro}
%
% \begin{macro}{\@@contsubfloat}
% This is a revised version of the \Lpack{subfigure} |\@subfigure| command
% (just the called macronames are changed).
%    \begin{macrocode}
\def\@@contsubfloat[#1]{%
  \@ifnextchar [%
    {\@@@contsubfloat{sub\@captype}[#1]}%
    {\@@@contsubfloat{sub\@captype}[\@empty #1][#1]}}

%    \end{macrocode}
% \end{macro}
%
% \begin{macro}{\@@@contsubfloat}
%    This is a modified version of the \Lpack{subfigure} |\@subfloat|
% command. Essentially the |\csname if#1topcap\endcsname| constructs are
% replaced by |\if@contbotsub|. This is actually only required for 
% user-defined floats where I haven't been able to work out if it is 
% possible to create new |\if#1...| commands within a command that has a
% a parameter |#1|.
%    \begin{macrocode}
\long\def\@@@contsubfloat#1[#2][#3]#4{%
  \@tempcnta=\@ne
  \ifsf@tight
    \if@minipage
      \@tempcnta=\z@
    \else
      \ifdim\lastskip=\z@
        \@tempcnta=\@ne
      \else
        \@tempcnta=\tw@
      \fi
    \fi
  \fi
  \if@contbotsub
    \def\subfig@top{\subfigtopskip}%
    \def\subfig@bottom{\subfigbottomskip}%
  \else
    \def\subfig@top{\subfigbottomskip}%
    \def\subfig@bottom{\subfigtopskip}%
  \fi
  \setbox\@tempboxa \hbox{#4}%
  \@tempdima=\wd\@tempboxa
  \vtop\bgroup
    \vbox\bgroup
    \ifcase\@tempcnta
      \@minipagefalse
    \or
      \vspace{\subfig@top}
    \or
      \ifdim \lastskip=\z@ \else
        \@tempskipb\subfig@top\relax\@xaddvskip
      \fi
    \fi
    \if@contbotsub
      \box\@tempboxa\egroup
      \ifx \@empty#3\relax \else
        \vskip\subfigcapskip
        \@subcaption{#1}{#2}{#3}%
      \fi
    \else
      \ifx\@empty#3\relax \else
        \@subcaption{#1}{#2}{#3}%
        \vskip\subfigcapskip
        \vskip\subfigcaptopadj
      \fi\egroup
      \box\@tempboxa
    \fi
    \vspace{\subfig@bottom}
  \egroup
\egroup}
  
%    \end{macrocode}
% \end{macro}
%
% \begin{macro}{\cont@subfig@oldcaption}
% Keep the definition of |\@caption|.
%    \begin{macrocode}
\let\cont@subfig@oldcaption\@caption

%    \end{macrocode}
% \end{macro}
%
% The remainder of the \Lopt{subfigure21} option code.
% \begin{macro}{\doxxi@contcaption}
%  This command redefines the |\@contcaption| command to flush out any
% pending subcaptions. The redefinition cannot be done within |\if...\fi|
% because of the internal |\if...| creation.  The code is simplified from 
% the \Lpack{subfigure} v2.1 redefinition of |\@caption|.
%    \begin{macrocode}
\newcommand{\doxxi@contcaption}{%
  \long\def\@contcaption##1##2{%
    \if@contbotsub
      \@listsubcaptions{##1}%
      \subfigold@contcaption{##1}{##2}
    \else
      \subfigold@contcaption{##1}{##2}
      \@listsubcaptions{##1}%
    \fi}
}

%    \end{macrocode}
% \end{macro}
% We can now call the rest of the \Lopt{subfigure21} code, if required.
%    \begin{macrocode}
%%%\if@contsubfigxxi

%    \end{macrocode}
%
% \begin{macro}{\subbottom}
% \begin{macro}{\@contsubbody}
%    |\subbottom[|\meta{list-entry}|][|\meta{subcaption}|]{|\meta{text}|}|
% typesets a subcaption below the \meta{text}. Most of the  work is
% performed by the |\@contsubbody| macro.
% \changes{v3.1a}{2002/04/01}{Added \cs{subfig@oldlabel} to \cs{@contsubbody}}
%    \begin{macrocode}
  \newcommand{\subbottom}{%
    \@contbotsubtrue
    \@contsubbody}

  \newcommand{\@contsubbody}{%
    \bgroup
    \let\subfig@oldlabel=\label
    \let\label=\sub@label
    \if@contmaincaption\else
      \advance\csname c@\@captype\endcsname\@ne
    \fi
    \refstepcounter{sub\@captype}\@contkeep%
    \leavevmode
    \@ifnextchar [%
      {\@@contsubfloat}%
      {\@@contsubfloat[\@empty]}}

%    \end{macrocode}
% \end{macro}
% \end{macro}
%
% \begin{macro}{\contsubbottom}
% \begin{macro}{\subbody@cont}
%     These are the continued versions of |\subbottom| and |\@contsubbody|.
% \changes{v3.1a}{2002/04/01}{Added \cs{subfig@oldlabel} to \cs{subbody@cont}}
% \begin{macrocode}
  \newcommand{\contsubbottom}{%
    \@contbotsubtrue
    \subbody@cont}

  \newcommand{\subbody@cont}{%
    \bgroup
    \let\subfig@oldlabel=\label
    \let\label=\sub@label
    \@contset
    \refstepcounter{sub\@captype}\@contkeep%
    \leavevmode
    \@ifnextchar [%
      {\@@contsubfloat}%
      {\@@contsubfloat[\@empty]}}

%    \end{macrocode}
% \end{macro}
% \end{macro}
%
% \begin{macro}{\subtop}
% \begin{macro}{\contsubtop}
%  These are similar to |\subbottom| and |\contsubbottom| except that they
% put the subcaption on top of the \meta{text}.
%    \begin{macrocode}
  \newcommand{\subtop}{%
    \@contbotsubfalse
    \@contsubbody}

  \newcommand{\contsubtop}{%
    \@contbotsubfalse
    \subbody@cont}

%    \end{macrocode}
% \end{macro}
% \end{macro}
%

% \begin{macro}{\contsubfigure}
%   This a simplified version of |\subfigure| in that the main caption counter
% is not incremented (we should be in a continued float), and the subcounter
% is restored before being incremented.
% \changes{v3.1a}{2002/04/01}{Added \cs{subfig@oldlabel} to \cs{contsubfigure}}
%    \begin{macrocode}
  \newcommand{\contsubfigure}{%
    \bgroup
    \let\subfig@oldlabel=\label
    \let\label=\sub@label
    \@contset
    \refstepcounter{sub\@captype}\@contkeep%
    \@ifnextchar [%
      {\@@contsubfloat}%
      {\@@contsubfloat[\@empty]}}

%    \end{macrocode}
% \end{macro}
%
% \begin{macro}{\@contsf}
% \begin{macro}{\@contst}
%  These are versions of the |\subfigure| and |\subtable| macros written
% using the \Lpack{ccaption} style.
%    \begin{macrocode}
\newcommand{\@contsf}{\@contbotsubtrue%
  \ifsubfiguretopcap\@contbotsubfalse\fi%
  \@contsubbody}
\newcommand{\@contst}{\@contbotsubtrue%
  \ifsubtabletopcap\@contbotsubfalse\fi%
  \@contsubbody}

%    \end{macrocode}
% \end{macro}
% \end{macro}
%
%
% Now these can be used if appropriate within the |\AtBeginDocument| code.
% But first call for the new version of |\@contcaption|.
% \changes{v3.1b}{2002/10/18}{Added braces to bracketed arg in 
%                 \cs{cont@subfig@oldcaption}, otherwise brackets
%                 in the \cs{caption} argument confuse things.}
%    \begin{macrocode}
\if@contsubfigxxi

  \doxxi@contcaption

  \AtBeginDocument{%
    \let\@subfloat\@@@contsubfloat
    \let\@subfigure\@@contsubfloat
    \let\subfigure\@contsf
    \let\subtable\@contst
    \let\contsubfigure\contsubbottom
    \let\contsubtable\contsubtop
    \long\def\@caption#1[#2]#3{%
      \cont@subfig@oldcaption{#1}[{#2}]{#3}}
}

%    \end{macrocode}
%
% The end of the \Lopt{subfigure21} option code.
%    \begin{macrocode}
\fi

%    \end{macrocode}
%
% \subsection{New floats}
%
% To define a float environment, say |fenv|, the following macros must be defined:
% \begin{itemize}
% \item |\fps@fenv| The default placement specifier (normally |tbp|). 
% \item |\ftype@fenv| The type number which is an integer and a power of 2.
% \item |\ext@fenv| The file extension for the contents list.
% \item |\c@fenv| A counter for the environment (for caption numbering).
% \item |\fnum@fenv| A macro to generate the caption `number'.
% \item |\l@fenv| A macro to produce an entry in a list of\ldots.
% \item |\flegtoc@fenv| A macro to write a |\namedlegend| title to a listof file.
% \item |\fleg@fenv| A macro to typeset the name of a |\namedlegend|.
% \item |\toclevel@fenv| Holding a bookmark level.
% \end{itemize}
% Note that the |\fleg...| macros are only required for the \Lpack{\pname}
%  package, and |\toclevel@fenv| is only required if the \Lpack{hyperref}
% package is being used. 
% The others are required for any new float, whether or not the
% \Lpack{\pname} package is being used.
%
%
%
% \begin{macro}{newflo@tctr}
% A counter for the type number of a new float. Normally
% figures are of type 1, tables type 2, and the next float type is then 4, and so
% on. 
%    \begin{macrocode}
\newcounter{newflo@tctr}
\@ifundefined{c@figure}{\setcounter{newflo@tctr}{1}}{
  \@ifundefined{c@table}{\setcounter{newflo@tctr}{2}}{
    \setcounter{newflo@tctr}{4}}}

%    \end{macrocode}
% \end{macro}
%
% \begin{macro}{\cftdot}
% \begin{macro}{\cftdotsep}
% \begin{macro}{\cftdotfill}
% \begin{macro}{\@cfttocstart}
% \begin{macro}{\@cfttocfinish}
% These macros are also provided by the \Lpack{tocloft} package, but
% we need them in any event.
%    \begin{macrocode}
\providecommand{\cftdot}{.}
\providecommand{\cftdotsep}{4.5}
\providecommand{\cftdotfill}[1]{%
  \leaders\hbox{$\m@th\mkern #1 mu \hbox{\cftdot}\mkern #1 mu$}\hfill}
\providecommand{\@cfttocstart}{%
  \@ifundefined{chapter}{}{%
    \if@twocolumn
      \@restonecoltrue\onecolumn
    \else
      \@restonecolfalse
    \fi}}
\providecommand{\@cfttocfinish}{%
  \@ifundefined{chapter}{}{\if@restonecol\twocolumn\fi}}

%    \end{macrocode}
% \end{macro}
% \end{macro}
% \end{macro}
% \end{macro}
% \end{macro}
%
% \begin{macro}{\newfloatentry}
% |\newfloatentry[|\meta{within}|]{|\meta{counter}|}{|\meta{ext}|}{|\meta{level-1}|}|
% generates the commands for typesetting a caption in a float and a
% caption in a listing.
%    \begin{macrocode}
\newcommand{\newfloatentry}[4][\@empty]{%
%    \end{macrocode}
% \begin{macro}{\c@X}
% \begin{macro}{\theX}
% Create the new counter. An error if it exists.
%    \begin{macrocode}
  \@ifundefined{c@#2}{% 
    \ifx \@empty#1\relax
      \newcounter{#2}
    \else
      \@ifundefined{c@#1}{\PackageWarning{ccaption}%
                          {#1 has no counter for use as a `within'}
        \newcounter{#2}}%
      {\newcounter{#2}[#1]%
       \expandafter\edef\csname the#2\endcsname{%
         \expandafter\noexpand\csname the#1\endcsname.\noexpand\arabic{#2}}}
    \fi
    \setcounter{#2}{0}
  }
  {\PackageError{ccaption}{#2 has been previously defined}{\@eha}}

%    \end{macrocode}
% \end{macro}
% \end{macro}
%
% That finishes off the error checking, rest is defined in any event
% \begin{macro}{\l@X}
% |\l@X{title}{page}| typesets the entry in the listing, but only if
% the |Zdepth| is greater than \meta{level-1}.
%    \begin{macrocode}
  \@namedef{l@#2}##1##2{%
    \ifnum \@nameuse{c@#3depth} > #4\relax
    \vskip \@nameuse{cftbefore#2skip}
    {\leftskip \@nameuse{cft#2indent}\relax
     \rightskip \@tocrmarg
     \parfillskip -\rightskip
     \parindent \@nameuse{cft#2indent}\relax\@afterindenttrue
     \interlinepenalty\@M
     \leavevmode
     \@tempdima \@nameuse{cft#2numwidth}\relax
     \expandafter\let\expandafter\@cftbsnum\csname cft#2presnum\endcsname
     \expandafter\let\expandafter\@cftasnum\csname cft#2aftersnum\endcsname
     \expandafter\let\expandafter\@cftasnumb\csname cft#2aftersnumb\endcsname
     \advance\leftskip\@tempdima \null\nobreak\hskip -\leftskip
     {\@nameuse{cft#2font}##1}\nobreak
     \@nameuse{cft#2fillnum}{##2}}
  \fi
    }  % end of \l@#2

%    \end{macrocode}
% \end{macro}
%
% Now define all the layout commands used by |\l@X|. The default
% values for these correspond to those for figure and table entries.
% \begin{macro}{\cftbeforeXskip}
%    \begin{macrocode}
  \expandafter\newlength\csname cftbefore#2skip\endcsname
    \setlength{\@nameuse{cftbefore#2skip}}{\z@ \@plus .2\p@}
%    \end{macrocode}
% \end{macro}
%
% \begin{macro}{\cftXindent}
% \begin{macro}{\cftXnumwidth}
%    \begin{macrocode}
  \expandafter\newlength\csname cft#2indent\endcsname
  \expandafter\newlength\csname cft#2numwidth\endcsname
%    \end{macrocode}
% Set the default values for the indent and numwidth depending
% on the entry's level. A level of 1 (\meta{level-1} = 0) 
% corresponds to a figure.
%    \begin{macrocode}
  \ifcase #4\relax  % 0
    \setlength{\@nameuse{cft#2indent}}{1.5em}
    \setlength{\@nameuse{cft#2numwidth}}{2.3em}
  \or               % 1
    \setlength{\@nameuse{cft#2indent}}{3.8em}
    \setlength{\@nameuse{cft#2numwidth}}{3.2em}
  \or               % 2
    \setlength{\@nameuse{cft#2indent}}{7.0em}
    \setlength{\@nameuse{cft#2numwidth}}{4.1em}
  \or               % 3
    \setlength{\@nameuse{cft#2indent}}{10.0em}
    \setlength{\@nameuse{cft#2numwidth}}{5.0em}
  \else             % anything else
    \setlength{\@nameuse{cft#2indent}}{1.5em}
    \setlength{\@nameuse{cft#2numwidth}}{2.3em}
  \fi
%    \end{macrocode}
% \end{macro}
% \end{macro}
%
% \begin{macro}{\cftXfont}
% \begin{macro}{\cftXpresnum}
% \begin{macro}{\cftXaftersnum}
% \begin{macro}{\cftXaftersnumb}
% \begin{macro}{\cftXdotsep}
% \begin{macro}{\cftXleader}
% \begin{macro}{\cftXpagefont}
% \begin{macro}{\cftXafterpnum}
%  And the rest of the commands
%    \begin{macrocode}
  \@namedef{cft#2font}{\normalfont}
  \@namedef{cft#2presnum}{}
  \@namedef{cft#2aftersnum}{}
  \@namedef{cft#2aftersnumb}{}
  \@namedef{cft#2dotsep}{\cftdotsep}
  \@namedef{cft#2leader}{\normalfont\cftdotfill{\@nameuse{cft#2dotsep}}}
  \@namedef{cft#2pagefont}{\normalfont}
  \@namedef{cft#2afterpnum}{}
%    \end{macrocode}
% \end{macro}
% \end{macro}
% \end{macro}
% \end{macro}
% \end{macro}
% \end{macro}
% \end{macro}
% \end{macro}
%
% \begin{macro}{\cftXfillnum}
% This typesets the leader and the page number.
%    \begin{macrocode}
  \@namedef{cft#2fillnum}##1{%
    {\@nameuse{cft#2leader}}\nobreak
    \hb@xt@\@pnumwidth{\hfil\@nameuse{cft#2pagefont}##1}%
    \@nameuse{cft#2afterpnum}\par}
%    \end{macrocode}
% \end{macro}
%
%
% \begin{macro}{\toclevel@X}
% This is required for the \Lpack{hyperref} package.
%    \begin{macrocode}
  \@namedef{toclevel@#2}{#4}
%    \end{macrocode}
% \end{macro}
% The end of |\newfloatentry|
%    \begin{macrocode}
} % end \newfloatentry

%    \end{macrocode}
% \end{macro}
%
%
% \begin{macro}{\newfloatlist}
% |\newfloatlist[|\meta{within}|]{|\meta{fenv}|}{|\meta{ext}|}{|\meta{listname}|}{|\meta{capname}|}|
% creates the commands for a new float environment \meta{fenv} (aka |X|)
% and a new List of for \meta{fenv}, using \meta{ext} (aka |Z|) as the
% file extension.
%    \begin{macrocode}
\newcommand{\newfloatlist}[5][\@empty]{%
%    \end{macrocode}
% Call |\newfloatentry[within]{X}{Z}{0}| to set up for typesetting the entry.
%    \begin{macrocode}
  \ifx \@empty#1\relax
    \newfloatentry{#2}{#3}{0}
  \else
    \newfloatentry[#1]{#2}{#3}{0}
  \fi
%    \end{macrocode}
%
% \begin{macro}{\ftype@X}
%  Define the float type, set it to the float counter, and double
% the counter afterwards.
%    \begin{macrocode}
  \@namedef{ftype@#2}{\value{newflo@tctr}}
  \addtocounter{newflo@tctr}{\value{newflo@tctr}}
%    \end{macrocode}
% \end{macro}
%
% \begin{macro}{\ext@X}
% \begin{macro}{Zdepth}
% Define |\ext@X| for the file extension and set the new |Zdepth| 
% depth counter to 1.
%    \begin{macrocode}
  \@namedef{ext@#2}{#3}  % file extension
  \newcounter{#3depth}
  \setcounter{#3depth}{1}

%    \end{macrocode}
% \end{macro}
% \end{macro}
%
% \begin{macro}{\cftmarkZ}
% |\cftmarkZ| specifies the marks for the page headings for the new listing.
%    \begin{macrocode}
  \@namedef{cftmark#3}{%
    \@mkboth{\MakeUppercase{#4}}{\MakeUppercase{#4}}}

%    \end{macrocode}
% \end{macro}
%
% \begin{macro}{\listofX}
% |\listofX| typesets the listing.
%    \begin{macrocode}
 \if@conttitleopt
%    \end{macrocode}
% For the \Lopt{titles} option, basically copy the code from the
% standard |\tableofcontents| command definition.
%    \begin{macrocode}
  \@namedef{listof#2}{%
    \@cfttocstart
    \@ifundefined{chapter}{\section*{#4}}{\chapter*{#4}}
    \@nameuse{cftmark#3}
    \@starttoc{#3}%
    \@cfttocfinish}
 \else
%    \end{macrocode}
% Otherwise, provide a fully parameterised heading style.
%    \begin{macrocode}
  \@namedef{listof#2}{%
    \@cfttocstart
    \par
    \begingroup
      \parindent\z@ \parskip\z@
      \@nameuse{@cftmake#3title}
      \@starttoc{#3}%
    \endgroup
    \@cfttocfinish}
 \fi

%    \end{macrocode}
% \end{macro}
%
% \begin{macro}{\@cftmakeZtitle}
% |\@cftmakeZtitle| typeset the title heading for the liusting.
%    \begin{macrocode}
  \@namedef{@cftmake#3title}{%
    \@ifundefined{chapter}{%
      \vspace{\@nameuse{cftbefore#3titleskip}}}{%
      \vspace*{\@nameuse{cftbefore#3titleskip}}}
    \interlinepenalty\@M
    {\@nameuse{cft#3titlefont}#4}{\@nameuse{cftafter#3title}}
    \@nameuse{cftmark#3}
    \par\nobreak
    \vskip \@nameuse{cftafter#3titleskip}
    \@afterheading}

%    \end{macrocode}
% \end{macro}
%
% \begin{macro}{\cftbeforeZtitleskip}
% \begin{macro}{\cftafterZtitleskip}
% \begin{macro}{\cftZtitlefont}
% \begin{macro}{\cftafterZtitle}
% Define the lengths and commands for controlling the title heading
% layout. The values depend on whether the document is chaptered or not.
%    \begin{macrocode}
   \expandafter\newlength\csname cftbefore#3titleskip\endcsname
   \expandafter\newlength\csname cftafter#3titleskip\endcsname
   \@ifundefined{chapter}{%
      \setlength{\@nameuse{cftbefore#3titleskip}}{3.5ex \@plus 1ex \@minus .2ex}
      \setlength{\@nameuse{cftafter#3titleskip}}{2.3ex \@plus .2ex}
      \@namedef{cft#3titlefont}{\normalfont\Large\bfseries}
    }{%
      \setlength{\@nameuse{cftbefore#3titleskip}}{50pt}
      \setlength{\@nameuse{cftafter#3titleskip}}{40pt}
      \@namedef{cft#3titlefont}{\normalfont\Huge\bfseries}
    }
    \@namedef{cftafter#3title}{}

%    \end{macrocode}
% \end{macro}
% \end{macro}
% \end{macro}
% \end{macro}
%
% \begin{macro}{\fps@X}
% \begin{macro}{\fnum@X}
% \begin{macro}{\flegtoc}
% |\fps@X| is the default float placement specification, |\fnum@X|
% typesets the caption name and number, and |\flegtoc@X| is for
% named legends.
%    \begin{macrocode}
  \@namedef{fps@#2}{tbp}                     % position
  \@namedef{fnum@#2}{#5~\@nameuse{the#2}}    % caption naming
  \@namedef{flegtoc@#2}##1{}                 % named legend

%    \end{macrocode}
% \end{macro}
% \end{macro}
% \end{macro}
%
% \begin{environment}{X}
% \begin{environment}{X*}
% Finally define the new float environment, in both normal and starred
% forms.
%    \begin{macrocode}
  \newenvironment{#2}{\@float{#2}}{\end@float}
  \newenvironment{#2*}{\@dblfloat{#2}}{\end@dblfloat}
%    \end{macrocode}
% \end{environment}
% \end{environment}
%
% This ends the definition of |\newfloatlist|.
%    \begin{macrocode}
} % end \newlistof

%    \end{macrocode}
% \end{macro}
%
% \begin{macro}{\newfloatenv}
% Up to version 2.7 of the package the command 
% |\newfloatenv[|\meta{within}|]{|\meta{fenv}|}{|\meta{ext}|}{|\meta{capname}|}|
% created a new float environment. It was replaced in later versions
% by |\newfloatlist|. Print a warning message if it is used.
%    \begin{macrocode}
\newcommand{\newfloatenv}[4][\@empty]{%
  \PackageError{ccaption}{\protect\newfloatenv\space has been replaced
                          by\MessageBreak
                         \protect\newfloatlist}{\@eha}
}

%    \end{macrocode}
% \end{macro}
%
%
% \begin{macro}{\listfloats}
% Up to version 2.7 the |\listfloats{|\meta{fenv}|}{|\meta{heading}|}| 
% command produced a list of floats for \meta{fenv}. It was replaced
% in later versions by the generated command |\listoffenv|. Print
% an error message.
%    \begin{macrocode}
\newcommand{\listfloats}[2]{%
  \PackageError{ccaption}{\protect\listfloats{#1}{...} has been
                          replaced by\MessageBreak
                          \protect\listof #1}{\@eha}
}

%    \end{macrocode}
% \end{macro}
%
% To define subcaptions for use in a new float environment, say |fenv|, the
% following macros must be defined~\cite{SUBFIGURE}:
% \begin{itemize}
% \item A new counter |subfenv| for subcaption numbering.
% \item A new counter |extdepth|, where |ext| is the file extension
%       for the contents list of |fenv|, for setting the contents depth.
% \item |\thesubfenv| for the formatting of the subcaption number.
% \item |\@thesubfenv| for typesetting the number.
% \item |\@@thesubfenv| for alternative label reference.
% \item |\p@subfenv| for prepending to the subcaption number when it is referenced.
% \item |\ext@subfenv| the file extension for the contents list.
% \item |\l@subfenv| for formatting the contents list entry.
% \item |\@makesubfenvcaption| for typesetting the subcaption.
% \item |\toclevel@subfenv| for hyperref bookmarks
% \end{itemize}
%
% \begin{macro}{\newsubfloat}
% |\newsubfloat{|\meta{fenv}|}| creates the commands for a new
% subfloat for \meta{fenv} (aka X).
%    \begin{macrocode}
\newcommand{\newsubfloat}[1]{%
%    \end{macrocode}
% Call |\newfloatentry[X]{subX}{extX}{1}| to get most of the work done.
%    \begin{macrocode}
  \newfloatentry[#1]{sub#1}{\@nameuse{ext@#1}}{1}
%    \end{macrocode}
%
% \begin{macro}{\ext@subX}
% \begin{macro}{\thesubX}
% \begin{macro}{\@thesubX}
% \begin{macro}{\@@thesubX}
% \begin{macro}{\p@subX}
% \begin{macro}{\@makesubXcaption}
% And now for the rest of the commands for subcaptions.
%    \begin{macrocode}
  \@namedef{ext@sub#1}{\csname ext@#1\endcsname}
  \@namedef{thesub#1}{(\alph{sub#1})}
  \@namedef{@thesub#1}{{\subcaplabelfont\@nameuse{thesub#1}}\space}
  \@namedef{@@thesub#1}{\@nameuse{thesub#1}}
  \@namedef{p@sub#1}{\csname the#1\endcsname}
  \@namedef{@makesub#1caption}{\@makesubfigurecaption}
}

%    \end{macrocode}
% \end{macro}
% \end{macro}
% \end{macro}
% \end{macro}
% \end{macro}
% \end{macro}
% \end{macro}
%
% \begin{macro}{\newfloatpagesoff}
% |\newfloatpagesoff{|\meta{fenv}|}| switches off page numbers in the
% listing for entries of type \meta{fenv}. It does this by redefining
% the |\cftXfillnum| command.
%    \begin{macrocode}
\DeclareRobustCommand{\newfloatpagesoff}[1]{
  \@namedef{cft#1fillnum}##1{%
    \parfillskip=\z@ plus1fil\@nameuse{cft#1afterpnum}\par}}

%    \end{macrocode}
% \end{macro}
%
% \begin{macro}{\newfloatpageson}
% |\newfloatpageson{|\meta{fenv}|}| switches on page numbers in the
% listing for entries of type \meta{fenv}. It does this by redefining
% the |\cftXfillnum| command to its default specification.
%    \begin{macrocode}
\DeclareRobustCommand{\newfloatpageson}[1]{
  \@namedef{cft#1fillnum}##1{%
    {\@nameuse{cft#1leader}}\nobreak
    \hb@xt@\@pnumwidth{\hfil\@nameuse{cft#1pagefont}##1}%
    \@nameuse{cft#1afterpnum}\par}}

%    \end{macrocode}
% \end{macro}
%
% \begin{macro}{\setnewfloatindents}
% |\setnewfloatindents{|\meta{fenv}|}{|\meta{indent}|}{|\meta{width}|}|
% sets the indent and numwidth for the float entry \meta{fenv}.
%    \begin{macrocode}
\newcommand{\setnewfloatindents}[3]{%
  \setlength{\@nameuse{cft#1indent}}{#2}
  \setlength{\@nameuse{cft#1numwidth}}{#3}
}

%    \end{macrocode}
% \end{macro}
%
%
%
%    The end of this package.
%    \begin{macrocode}
%</usc>
%    \end{macrocode}
%
% \appendix
% \section{The perils of empty} \label{sec:peril}
%
%    My original code for the |\@if@contemptyarg| command was as follows:
% \begin{verbatim}
% \newcommand{\@if@contemptyarg}[3]{%
%   \edef\@conttemp{\zap@space#1 \@empty}
%   \ifx\@empty\@conttemp\relax #2\else #3\fi}
% \end{verbatim}
% This uses the |\zap@space| kernel command and I wrote the code after looking
% at various code bits in the kernel and other packages, but I can't now
% remember which ones.
%
%    Donald Arseneau kindly pointed out the error of my ways and provided
% the robust solution which is used in the body of this package. The following
% is a slightly edited version of an email he sent me on the subject.
% \begin{quotation}
% I'm not sure how exactly it is \emph{supposed} to work because there are
% cases for which it will fail spectacularly. [These involved testing an
% an argument that included macros of various forms]
%
%    There are several errors I am sure of though:
% \begin{itemize}
% \item You used |\edef| which is \emph{not} allowed in LaTeX --- this
% creates a moving argument without any protection from |\protect|. Fragile
% commands will produce stack overflows and other errors.
% Even if you use |\protected@edef|, as is correct, you still
% make a moving argument to no purpose.
% \item |\zap@space| is not valid for general arguments. It fails if it
% ever sees an empty macro following a space. [e.g., |\def\none{}| used
% as |\@if@contemptyarg{ \none}{}{}|]
% \item By making |\@if@contemptyarg| skip over one of its parameters
% (|#2|, |#3|) you make it fail for nesting tabular or array environments.
% \item |\@if@contemptyarg| is itself a fragile command, and will require
% |\protect| if it ever appears in a title or other moving argument.
% Since it is possible to do the test by expandable operations alone, it
% should be done that way.
% \end{itemize}
%
%     I suggest you read \file{CTAN:tex-archive/info/aro-bend/answer.002}
% for a past discussion of detecting empty arguments, and then use a definition
% of |\@if@contemptyarg| based on that discussion. You'll find it in
% \file{amsgen.dtx}, or use instead the improved version \ldots.
% \end{quotation}
%
%    The definition of |\@if@contemptyarg| is based on the improved version
% that Donald supplied, only the macro names being changed.
%
%    For checking if an optional argument is present I used code along the
% lines: \\
% \verb?\newcommand{\com}[4][\@empty]{...? \\
% \verb?\ifx \@empty#1\else %argument present? \\
% Unfortunately I was not consistent, as Benjamin Bayart found\footnote{Email
% to me on 2005/03/29.} when he used an optional argument that started with
% a double character, like \verb?\bicaption[ccapt3]{...?, which caused
% nasty things to happen. In these cases I had coded: \\
% \verb?\ifx #1\@empty\else %argument present? \\
% I really should have known better as this results in TRUE with 
% \verb?apt3? being left dangling (and typeset).
%
% \bibliographystyle{alpha}
%
% \begin{thebibliography}{MCCG95}
%
% \bibitem[Coc95]{SUBFIGURE}
% Steven Douglas Cochran.
% \newblock \emph{The subfigure package}.
% \newblock February 2002.
% \newblock (Available from CTAN as file \texttt{subfigure.dtx})
%
% \bibitem[GMS94]{GOOSSENS94}
% Michel Goossens, Frank Mittelbach, and Alexander Samarin.
% \newblock \emph{The LaTeX Companion}.
% \newblock Addison-Wesley Publishing Company, 1994.
%
% \bibitem[Lin95]{LINGNAU95}
% Anselm Lingnau.
% \newblock \emph{{An Improved Environment for Floats}}.
% \newblock March 1995.
% \newblock (Available from CTAN as file \texttt{float.dtx})
%
% \bibitem[McCG95]{ENDFLOAT}
% James Darrell McCauley and Jeff Goldberg.
% \newblock \emph{{The endfloat package}}.
% \newblock October 1995.
% \newblock (Available from CTAN as file \texttt{endfloat.dtx})
%
% \bibitem[NiGa98]{SIDECAP}
% Rolf Niepraschk and Hubert G\"{a}\ss lein.
% \newblock \emph{{The sidecap package}}.
% \newblock June 1998.
% \newblock (Available from CTAN as file \texttt{sidecap.dtx})
%
% \bibitem[Rec97]{EPSLATEX}
% Keith Reckdahl.
% \newblock \emph{{Using Imported Graphics in LaTeX2e}}.
% \newblock December 1997.
% \newblock (Available from CTAN as file \texttt{info/epslatex.ps}
% or \texttt{info/epslatex.pdf})
%
% \bibitem[Som95]{CAPTION2}
% Harald Axel Sommerfeldt.
% \newblock \emph{{The caption package}}.
% \newblock October 1995.
% \newblock (Available from CTAN as file \texttt{caption2.dtx})
%
% \bibitem[Wil96]{PRW96i}
% Peter~R. Wilson.
% \newblock \emph{{LaTeX for standards: The LaTeX package files user manual}}.
% \newblock NIST Report NISTIR, June 1996.
%
% \bibitem[Wil01]{TOCLOFT}
% Peter~R. Wilson.
% \newblock \emph{{The tocloft package}}.
% \newblock March 2001.
% \newblock (Available from CTAN as file \texttt{tocloft.dtx})
%
% \end{thebibliography}
%
% 
% \Finale
% \PrintIndex
%
\endinput

%% \CharacterTable
%%  {Upper-case    \A\B\C\D\E\F\G\H\I\J\K\L\M\N\O\P\Q\R\S\T\U\V\W\X\Y\Z
%%   Lower-case    \a\b\c\d\e\f\g\h\i\j\k\l\m\n\o\p\q\r\s\t\u\v\w\x\y\z
%%   Digits        \0\1\2\3\4\5\6\7\8\9
%%   Exclamation   \!     Double quote  \"     Hash (number) \#
%%   Dollar        \$     Percent       \%     Ampersand     \&
%%   Acute accent  \'     Left paren    \(     Right paren   \)
%%   Asterisk      \*     Plus          \+     Comma         \,
%%   Minus         \-     Point         \.     Solidus       \/
%%   Colon         \:     Semicolon     \;     Less than     \<
%%   Equals        \=     Greater than  \>     Question mark \?
%%   Commercial at \@     Left bracket  \[     Backslash     \\
%%   Right bracket \]     Circumflex    \^     Underscore    \_
%%   Grave accent  \`     Left brace    \{     Vertical bar  \|
%%   Right brace   \}     Tilde         \~}


