% \iffalse meta-comment
% epigraph.dtx
% Author: Peter Wilson (CUA)
% Copyright 1998 --- 2004 Peter R. Wilson
%
% This work may be distributed and/or modified under the
% conditions of the LaTeX Project Public License, either
% version 1.3c of this license or (at your option) any 
% later version.
% The latest version of the license is in
%    http://www.latex-project.org/lppl.txt
% and version 1.3c or later is part of all distributions of
% LaTeX version 2003/06/01 or later.
%
% This work has the LPPL maintenance status "author-maintained".
%
% This work consists of the files listed in the README file.
%
% 
%<*driver>
\documentclass{ltxdoc}
\EnableCrossrefs
\CodelineIndex
\setcounter{StandardModuleDepth}{1}
\begin{document}
  \DocInput{epigraph.dtx}
\end{document}
%</driver>
%
% \fi
%
% \CheckSum{242}
%
% \DoNotIndex{\',\.,\@M,\@@input,\@addtoreset,\@arabic,\@badmath}
% \DoNotIndex{\@centercr,\@cite}
% \DoNotIndex{\@dotsep,\@empty,\@float,\@gobble,\@gobbletwo,\@ignoretrue}
% \DoNotIndex{\@input,\@ixpt,\@m}
% \DoNotIndex{\@minus,\@mkboth,\@ne,\@nil,\@nomath,\@plus,\@set@topoint}
% \DoNotIndex{\@tempboxa,\@tempcnta,\@tempdima,\@tempdimb}
% \DoNotIndex{\@tempswafalse,\@tempswatrue,\@viipt,\@viiipt,\@vipt}
% \DoNotIndex{\@vpt,\@warning,\@xiipt,\@xipt,\@xivpt,\@xpt,\@xviipt}
% \DoNotIndex{\@xxpt,\@xxvpt,\\,\ ,\addpenalty,\addtolength,\addvspace}
% \DoNotIndex{\advance,\Alph,\alph}
% \DoNotIndex{\arabic,\ast,\begin,\begingroup,\bfseries,\bgroup,\box}
% \DoNotIndex{\bullet}
% \DoNotIndex{\cdot,\cite,\CodelineIndex,\cr,\day,\DeclareOption}
% \DoNotIndex{\def,\DisableCrossrefs,\divide,\DocInput,\documentclass}
% \DoNotIndex{\DoNotIndex,\egroup,\ifdim,\else,\fi,\em,\endtrivlist}
% \DoNotIndex{\EnableCrossrefs,\end,\end@dblfloat,\end@float,\endgroup}
% \DoNotIndex{\endlist,\everycr,\everypar,\ExecuteOptions,\expandafter}
% \DoNotIndex{\fbox}
% \DoNotIndex{\filedate,\filename,\fileversion,\fontsize,\framebox,\gdef}
% \DoNotIndex{\global,\halign,\hangindent,\hbox,\hfil,\hfill,\hrule}
% \DoNotIndex{\hsize,\hskip,\hspace,\hss,\if@tempswa,\ifcase,\or,\fi,\fi}
% \DoNotIndex{\ifhmode,\ifvmode,\ifnum,\iftrue,\ifx,\fi,\fi,\fi,\fi,\fi}
% \DoNotIndex{\input}
% \DoNotIndex{\jobname,\kern,\leavevmode,\let,\leftmark}
% \DoNotIndex{\list,\llap,\long,\m@ne,\m@th,\mark,\markboth,\markright}
% \DoNotIndex{\month,\newcommand,\newcounter,\newenvironment}
% \DoNotIndex{\NeedsTeXFormat,\newdimen}
% \DoNotIndex{\newlength,\newpage,\nobreak,\noindent,\null,\number}
% \DoNotIndex{\numberline,\OldMakeindex,\OnlyDescription,\p@}
% \DoNotIndex{\pagestyle,\par,\paragraph,\paragraphmark,\parfillskip}
% \DoNotIndex{\penalty,\PrintChanges,\PrintIndex,\ProcessOptions}
% \DoNotIndex{\protect,\ProvidesClass,\raggedbottom,\raggedright}
% \DoNotIndex{\refstepcounter,\relax,\renewcommand,\reset@font}
% \DoNotIndex{\rightmargin,\rightmark,\rightskip,\rlap,\rmfamily,\roman}
% \DoNotIndex{\roman,\secdef,\selectfont,\setbox,\setcounter,\setlength}
% \DoNotIndex{\settowidth,\sfcode,\skip,\sloppy,\slshape,\space}
% \DoNotIndex{\symbol,\the,\trivlist,\typeout,\tw@,\undefined,\uppercase}
% \DoNotIndex{\usecounter,\usefont,\usepackage,\vfil,\vfill,\viiipt}
% \DoNotIndex{\viipt,\vipt,\vskip,\vspace}
% \DoNotIndex{\wd,\xiipt,\year,\z@}
%
% \changes{v1.0}{1998/11/15}{First public release}
% \changes{v1.1}{1998/11/29}{Added epigraphs before chapter headings}
% \changes{v1.2}{1999/09/12}{Added cleartoevenpage and extended documentation}
% \changes{v1.2a}{1999/12/01}{Extended documentation for epigraphs on Part pages}
% \changes{v1.3}{1999/12/01}{Added \cs{dropchapter} and \cs{undodrop} commands}
% \changes{v1.4}{2000/01/16}{Added flushleftright environment}
% \changes{v1.5}{2000/02/20}{Generalised \cs{cleartoevenpage} command}
% \changes{v1.5a}{2002/10/22}{Fixed class with ccaption package}
% \changes{v1.5b}{2004/04/22}{Updated license and contact details}
%
% \def\dtxfile{epigraph.dtx}
% \def\fileversion{v1.5}
% \def\filedate{2000/02/20}
% \def\fileversion{v1.5a}
% \def\filedate{2002/10/22}
% \def\fileversion{v1.5b}
% \def\filedate{2004/04/22}
% \newcommand*{\Lpack}[1]{\textsf {#1}}           ^^A typeset a package
% \newcommand*{\Lopt}[1]{\textsf {#1}}            ^^A typeset an option
% \newcommand*{\file}[1]{\texttt {#1}}            ^^A typeset a file
% \newcommand*{\Lcount}[1]{\textsl {\small#1}}    ^^A typeset a counter
% \newcommand*{\pstyle}[1]{\textsl {#1}}          ^^A typeset a pagestyle
% \newcommand*{\Lenv}[1]{\texttt {#1}}            ^^A typeset an environment
%  ^^A fake a rhsepigraph for demo purposes
% \newcommand{\tepi}[2]{\vspace{.5\baselineskip}\begin{flushright}\small
%    \begin{minipage}{2.5in}
%    \begin{minipage}{2.5in}\begin{flushleft} #1\\
%       \rule[.5ex]{2.5in}{.4pt}
%    \end{flushleft}\end{minipage} \\
%    \begin{minipage}{2.5in}\begin{flushright} #2
%    \end{flushright}\end{minipage}
%    \end{minipage}\end{flushright}\vspace{.5\baselineskip}}
%
% \title{The \Lpack{epigraph} package\thanks{This
%        file (\texttt{\dtxfile}) has version number \fileversion, last revised
%        \filedate.}}
%
% \author{%
% Peter Wilson\\
% Catholic University of America}
% \date{\filedate}
% \maketitle
% \begin{abstract}
%    The \Lpack{epigraph} package can be used to typeset a relevant
% quotation or saying as an \emph{epigraph}, usually just after a
% sectional heading. Various handles are provided to tweak the appearance.
% \end{abstract}
% \tableofcontents
%
% \StopEventually{}
%
% 
%
% \section{Introduction}
%
% \tepi{My soul, seek not the life of immortals; but enjoy to the full
%       the resources that are within your reach}
%      {\textit{Pythian Odes \\ Pindar}}
%
% Some authors like to add an interesting quotation at either the start
% or end of a section. The \Lpack{epigraph} package provides commands
% to assist in the typesetting of a single epigraph. Other authors like to 
% add many such quotations and the package provides environments to
% cater for these as well.
%
% This manual is typeset according to the conventions of the
% \LaTeX{} \textsc{docstrip} utility which enables the automatic
% extraction of the \LaTeX{} macro source files~\cite{GOOSSENS94}.
%
%    Section~\ref{sec:usc} describes the usage of the package.
% Commented source code for the package is in Section~\ref{sec:code}.
%
% \section{The \Lpack{epigraph} package} \label{sec:usc}
%
% \tepi{The whole is more than the sum of the parts.}
%      {\textit{Metaphysica \\ Aristotle}}
%
%    The \Lpack{epigraph} package provides commands for typesetting
% a single epigraph and environments for typesetting a list of epigraphs.
% Epigraphs can be typeset at either the left, the center or the right of 
% the typeblock. A few example eipgraphs are exhibited here, and
% others can be found in an article by
% Christina Thiele~\cite{TTC199}.
%
% \subsection{The \texttt{epigraph} command}
%
%  The original inspiration for |\epigraph| was Doug Schenck's
% for the epigraphs in our book~\cite{EBOOK}. That was hard wired for
% the purpose at hand. The version here provides much more flexibility.
%
%
%
% \DescribeMacro{\epigraph}
%  The command |\epigraph{|\meta{text}|}{|\meta{source}|}| 
% typesets
% an epigraph using \meta{text} as the main text of the epigraph and
% \meta{source} being the original author (or book, article, etc.)
% of the quoted text. By default the epigraph is placed at the right hand 
% side of the textblock, and the \meta{source} is typeset at the bottom
% right of the \meta{text}.
%
%
% \subsection{The \texttt{epigraphs} environment}
%
% \DescribeEnv{epigraphs}
% The |epigraphs| environment typesets a list of epigraphs, and by default
% places them at the right hand side of the textblock.
%
% \DescribeMacro{\qitem}
%  Each epigraph in the list is specified by a 
% |\qitem{|\meta{text}|}{|\meta{source}|}| command (analagous to the |\item|
% command in ordinary list environments).
% By default, the \meta{source} is typeset at the bottom right of the
% \meta{text}. 
%
% 
%
% \subsection{General}
%
% \tepi{Example is the school of mankind, and they will learn at no other.}
%      {\textit{Letters on a Regicide Peace}\\ \textsc{Edmund Burke}}
%
%   The commands described in this section apply to both the |\epigraph|
% command and the |epigraphs| environment. But first of all, note that an
% epigraph immediately after a heading will cause the first paragraph
% of the following text to be indented. If you want the initial paragraph
% to have no indentation, then start it with the |\noindent| command.
%
% \DescribeMacro{\epigraphwidth}
% \DescribeMacro{\textflush}
%  The epigraphs are typeset in a minipage of width |\epigraphwidth|. The default
% value for this can be changed using the |\setlength| command. Typically,
% epigraphs are typset in a measure much less than the width of the textblock.
% In order to avoid bad line breaks, the \meta{text} is normally typeset 
% raggedright. 
% The |\textflush| command controls the \meta{text} typesetting style,
% and it can be redefined from its default value of
% \texttt{flushleft} (which produces raggedright). The sensible values
% are \texttt{center} for centered text, \texttt{flushright} for raggedleft
% text, and \texttt{flushepinormal} for normal justified text.
%
%    If by any chance you want the \meta{text} to be typeset in some
% other layout style, the easiest
% way to do this is by defining a new environment which sets the paragraphing
% parameters to your desired values. For example, as the \meta{text} is
% typeset in a minipage, there is no paragraph indentation. If you
% want the paragraphs to be indented and justified then define
% a new environment like:\\
% |\newenvironment{myparastyle}{\setlength{\parindent}{1em}}{}| \\
% and use it as: \\
% |\renewcommand{\textflush}{myparastyle}|. 
%
% \DescribeMacro{\epigraphflush}
%  As noted, the default position of epigraphs is at the right hand side
% of the textblock. The positioning is controlled by |\epigraphflush|
% whose default value is \texttt{flushright}. This can be changed
% to \texttt{flushleft} for positioning at the left hand side or to
% \texttt{center} for positioning at the center of the textblock.
%
% \DescribeMacro{\sourceflush}
% The |\sourceflush| command controls the position of the \meta{source}.
% The default value is \texttt{flushright}. It can be changed to
% \texttt{flushleft}, \texttt{center} or \texttt{flushepinormal}.
%
% For example, to have epigraphs centered with the \meta{source} at the left,
% add the following to your
% document (after loading the \Lpack{epigraph} package):
% \begin{verbatim}
% \renewcommand{\epigraphflush}{center}
% \renewcommand{\sourceflush}{flushleft}
% \end{verbatim}
%
% \DescribeMacro{\epigraphsize}
% Epigraphs are often typeset in a smaller font than the main text. The
% |\epigraphsize| command sets the font size to be used.
% If you don't like the default value, change it by redefining the command,
% for example: \\
% |    \renewcommand{\epigraphsize}{\footnotesize}|
%		
% \DescribeMacro{\epigraphrule}
% By default, a rule is drawn between the \meta{text} and \meta{source},
% with the rule thickness being given by the value of |\epigraphrule|.
% The value can be changed by using the \LaTeX{} |\setlength| command.
% A value of \texttt{0pt} will eliminate the rule. Personally, I dislike
% the rule in the list environments.
%
% \DescribeMacro{\beforeepigraphskip}
% \DescribeMacro{\afterepigraphskip}
% The two |...skip| commands specify the amount of vertical space inserted
% before and after typeset epigraphs. Again, these can be changed by
% |\setlength|. It is desireable that the sum of their values should be an 
% integer multiple of the |\baselineskip|.
%
% Note that you can use normal \LaTeX{} commands in the \meta{text} and
% \meta{source} arguments. You may wish to use different fonts for the
% \meta{text} (say roman) and the \meta{source} (say italic).
%
% The epigraph at the start of this section can be specified as:
% \begin{verbatim}
% \epigraph{Example is the school of mankind,
%           and they will learn at no other.}%
%  {\textit{Letters on a Regicide Peace}\\ \textsc{Edmund Burke}}
% \end{verbatim}
%
% \subsection{Epigraphs before chapter headings}
%
%    The |\epigraph| command and the |epigraphs| environment typeset
% an epigraph at the point in the text where they are placed. The
% first thing that a |\chapter| command does is to start off a new page,
% so another mechanism is provided for placing an epigraph just before
% a chapter heading.
%    
% \DescribeMacro{\epigraphhead}
%  The |\epigraphhead[|\meta{distance}|]{|\meta{text}|}| stores \meta{text} 
% for printing at \meta{distance} below the header on a page.
% \meta{text} can be ordinary text or, more likely, can be either an
% |\epigraph| command or an |epigraphs| environment. By default, the 
% epigraph will be typeset at the righthand margin.
% If the command is immediately preceeded by a |\chapter| or |\chapter*| 
% command, the epigraph is typeset on the chapter title page.
%
%    The default value for the optional \meta{distance} argument is set so
% that an |\epigraph| consisting of a single line of quotation and a single
% line denoting the source is aligned with the bottom of the `Chapter X'
% line produced by the |\chapter| command. In other cases you will
% have to experiment with the \meta{distance} value. The value for
% \meta{distance} can be either a integer or a real number. The units
% are in terms of the current value for |\unitlength|. A typical value
% for \meta{distance} for a single line quotation and source for 
% a |\chapter*| might be about 70 (points). A positive value
% of \meta{distance} places the epigraph below the page heading and a negative
% value will raise it above the page heading.
%
%    Here's some example code:
% \begin{verbatim}
% \chapter*{Celestial navigation}
% \epigraphhead[70]{\epigraph{Star crossed lovers.}{\textit{The Bard}}}
% \end{verbatim}
%
% The |\epigraphhead| command changes the page style for the page on
% which it is specified, so there should be no text between the
% |\chapter| and the |\epigraphhead| commands.
%
% The \meta{text} argument is put into a minipage of width |\epigraphwidth|.
% If you use something other than |\epigraph| or |epigraphs| for the
% \meta{text} argument, you may have to so some positioning of the text
% yourself so that it is properly located in the minipage. For example
% \begin{verbatim}
% \chapter{Short}
% \renewcommand{\epigraphflush}{center}
% \epigraphhead{\centerline{Short quote}}
% \end{verbatim}
%
% \changes{v1.3}{1999/12/01}{Added description of dropchapter and undodrop}
% \DescribeMacro{\dropchapter}
% \DescribeMacro{\undodrop}
% If a long epigraph is placed before a chapter title it is possible that the
% bottom of the epigraph may interfere with the chapter title. The command
% |\dropchapter{|\meta{length}|}| will lower any subsequent chapter titles by 
% \meta{length}; a negative \meta{length} will raise the titles.
% The command |\undodrop| restores subsequent chapter titles to their default
% positions. For example:
% \begin{verbatim}
% \dropchapter{2in}
% \chapter{Title}
% \epigraphhead{long epigraph}
% \undodrop
% \end{verbatim}
%
% \changes{v1.2}{1999/09/12}{Added description of cleartoevenpage uses}
% \DescribeMacro{\cleartoevenpage}
% On occasions it may be desirable to put something (e.g., an epigraph, a map,
% a picture) on the page facing the start
% of a chapter, where the something belongs to the chapter that is about to 
% start rather than the chapter that has just ended. In order to do this 
% in a document that is going to be printed
% doublesided, the chapter must start on an odd numbered page and the 
% pre-chapter material put on the immediately preceeding even numbered page.
% The |\cleartoevenpage| command is like the |\cleardoublepage| except
% that the page following the command will be an even numbered page, and the
% command takes an optional argument, i.e., |\cleartoevenpage{|[arg|]|,
% which is applied to the skipped page (if any).
%
%    Here is an example:
% \begin{verbatim}
% ... end previous chapter.
% \cleartoevenpage
% \begin{center}
% \begin{picture}... \end{picture}
% \end{center}
% \chapter{Next chapter}
% \end{verbatim}
% If the style is such that chapter headings are put at the top of the pages,
% then it would be advisable to include |\thispagestyle{empty}| (or |plain|)
% immediately after |\cleartoevenpage| to avoid a heading related to the
% previous chapter from appearing on the page. 
%
% If the something is like a figure with a numbered caption and the numbering
% depends on the chapter numbering, then the numbers have to be hand set (unless
% you define a special chapter command for the purpose). For example:
% \begin{verbatim}
% ... end previous chapter.
% \cleartoevenpage[\thispagestyle{empty}] % skipped page, if any, to be empty
% \thispagestyle{plain}
% \addtocounter{chapter}{1} % increment the chapter number
% \setcounter{figure}{0}    % initialise figure counter
% \begin{figure}
% ...
% \caption{Pre chapter figure}
% \end{figure}
%
% \addtocounter{chapter}{-1} % decrement the chapter number
% \chapter{Next chapter}     % increments chapter number, initialises figure number
% \addtocounter{figure}{1}   % to account for pre-chapter figure
% \end{verbatim}
% 
%
% \subsection{Epigraphs on Part pages}
%
%     The \Lpack{epigraph} package as it stands cannot put an epigraph on the
% same page as a |\part| or |\part*| title page in 
% a \Lpack{book} or \Lpack{report} class. This is because the |\part| command
% internally does some page flipping before and after the title page.
% However, it is easy enough to put epigraphs on part pages.
%
% \begin{itemize}
% \item Create a file called, say, \file{epipart.sty} which looks like this:
% \begin{verbatim}
% % epipart.sty
% \let\@epipart\@endpart
% \renewcommand{\@endpart}{\thispagestyle{epigraph}\@epipart}
% \endinput
% \end{verbatim}
%
% \item Start your document like:
% \begin{verbatim}
% \documentclass[...]{...}
% \usepackage{epigraph}
% \usepackage{epipart}
% \end{verbatim}
%
% \item Immediately \emph{before} each |\part| command put an 
% |\epigraphhead| command. For example:
% \begin{verbatim}
% \epigraphhead[300]{Epigraph text}
% \part{Part title}
% \end{verbatim}
% The value of the optional argument may need changing to vertically adjust
% the position of the epigraph. If there is any |\part| that does not have an
% epigraph then an empty |\epigraphhead| command (i.e., |\epigraphhead{}|)
% must be placed immediately befor the |\part| command.
%
% \end{itemize}
%
%     A similar scheme may be used for epigraphs on other kinds of pages. 
% The essential
% trick is to make sure that the \pstyle{epigraph} pagestyle is used for
% the page.
%
%
% \subsection{Epigraphed bibliographies}
% \changes{v1.2}{1999/09/12}{Added application of an epigraph to a bibliography}
% One author asked how to associate an epigraph with his bibliography. The
% following is one way to do it (the example is based on the book class).
% \begin{enumerate}
% \item Copy the definition of the |thebibliography| environment from 
% \file{book.cls} to your own file called, say \file{epibib.sty}.
%
% \item Edit \file{epibib.sty} to include the definition of a vacuous
% command called, say, |\bibadd|. Edit the definition of the |thebibliography|
% to include |\bibadd| immediately before the |\list| command. The relevant
% portions of \file{epibib.sty} will look like this:
% \begin{verbatim}
% % epibib.sty
% ...
% \newcommand{\bibadd}{}
% \renewenvironment{thebibliography}[1]
%   {\chapter*{\bibname
%      \@mkboth{\MakeUppercase\bibname}{\MakeUppercase\bibname}}%
%    \bibadd
%    \list{\@biblabel .....
% \end{verbatim}
%
% \item In your document, start it off like this:
% \begin{verbatim}
% \documentclass...
% \usepackage{epigraph}
% \usepackage{epibib}
% \end{verbatim}
% At the point where the bibliography is to go, do something like the following:
% \begin{verbatim}
% ...
% \newcommand{\bepi}{<definition of epigraph>}
% \renewcommand{\bibadd}{\bepi}
% \bepi  % seems to be required if using \epigraphhead in \bepi
% \begin{thebibliography}{...} % or \bibliography{...}
% ...
% \end{verbatim}
% \end{enumerate}
%    The same idea can be applied to document elements like an abstract or an
% index. Of course |\bibadd| can be defined to be anything you want to typeset
% between the bibliography heading and the start of the reference list.
%
%
% \section{The package code} \label{sec:code}
%
% \tepi{And now for something completely different.}{\textit{Monty Python}}
%
%    Announce the name and version of the package, which requires
% \LaTeXe.
%    \begin{macrocode}
%<*usc>
\NeedsTeXFormat{LaTeX2e}
\ProvidesPackage{epigraph}[2002/10/22 v1.5a typesetting epigraphs]
%    \end{macrocode}
%
% \begin{macro}{\beforeepigraphskip}
% \begin{macro}{\afterepigraphskip}
% \begin{macro}{\epigraphwidth}
% \begin{macro}{\epigraphrule}
%    The several length commands, which can be changed by the user with
% |\setlength|.
%    \begin{macrocode}
\newlength{\beforeepigraphskip}
  \setlength{\beforeepigraphskip}{.5\baselineskip}
\newlength{\afterepigraphskip}
  \setlength{\afterepigraphskip}{.5\baselineskip}
\newlength{\epigraphwidth}
  \setlength{\epigraphwidth}{.4\textwidth}
\newlength{\epigraphrule}
  \setlength{\epigraphrule}{.4\p@}
%    \end{macrocode}
% \end{macro}
% \end{macro}
% \end{macro}
% \end{macro}
%
% \begin{macro}{\epigraphsize}
%    The size of the font to be used.
%    \begin{macrocode}
\newcommand{\epigraphsize}{\small}
%    \end{macrocode}
% \end{macro}
%
% \begin{macro}{\epigraphflush}
% \begin{macro}{\textflush}
% \begin{macro}{\sourceflush}
%  The three commands to position epigraphs in the textblock and to position
%  the components of the epigraph.
%    \begin{macrocode}
\newcommand{\epigraphflush}{flushright}
\newcommand{\textflush}{flushleft}
\newcommand{\sourceflush}{flushright}

%    \end{macrocode}
% \end{macro}
% \end{macro}
% \end{macro}
%
% \begin{environment}{flushleftright}
%    An environment for |\textflush| to use normal minipage typesetting. It
% is vacuous. Defining this was a mistake as the \Lpack{ccaption}
% package also defines |\flushleftright|.
% \changes{v1.5a}{2002/10/22}{Made flushleftright environment an error}
%    \begin{macrocode}
\AtBeginDocument{%
  \@ifundefined{flushleftright}{%
    \newenvironment{flushleftright}{%
      \PackageError{epigraph}%
      {The flushleftright environment has been removed.\MessageBreak
       Use the flushepinormal environment instead}{\@ehc}}{}}%
    {\PackageWarningNoLine{epigraph}%
     {flushleftright has been previously defined.\MessageBreak
      Use flushepinormal for epigraphs instead}}}

%    \end{macrocode}
% \end{environment}
%
% \begin{environment}{flushepinormal}
%    An environment for |\textflush| to use normal minipage typesetting. It
% is vacuous.
% \changes{v1.5a}{2002/10/22}{Added flushepinormal environment}
%    \begin{macrocode}
\newenvironment{flushepinormal}{}{}

%    \end{macrocode}
% \end{environment}
%
%
% \begin{macro}{\@epirule}
%  The internal command to draw a rule between text and source.
%    \begin{macrocode}
\newcommand{\@epirule}{\rule[.5ex]{\epigraphwidth}{\epigraphrule}}
%    \end{macrocode}
% \end{macro}
%
% \begin{macro}{\@epitext}
% The internal command to typeset the \meta{text}. Put it into a minipage of the
% right size and typeset per |\textflush|.
%    \begin{macrocode}
\newcommand{\@epitext}[1]{%
  \begin{minipage}{\epigraphwidth}\begin{\textflush} #1\\
%    \end{macrocode}
%  Draw a rule if it will be visible, otherwise add some extra vertical space.
%    \begin{macrocode}
    \ifdim\epigraphrule>\z@ \@epirule \else \vspace*{1ex} \fi
  \end{\textflush}\end{minipage}}
%    \end{macrocode}
% \end{macro}
%
% \begin{macro}{\@episource}
%    The internal command for typesetting the \meta{source}, which is put 
% into a minipage and typeset according to |\sourceflush|.
%    \begin{macrocode}
\newcommand{\@episource}[1]{%
  \begin{minipage}{\epigraphwidth}\begin{\sourceflush} #1\end{\sourceflush}
  \end{minipage}}
%    \end{macrocode}
% \end{macro}
%
% \begin{macro}{\epigraph}
%    Having got the preliminaries out of the way, here's the user command
% for a single epigraph. This is set in a minipage to prevent breaking
% across a page. Position it according to |\epigraphflush|.
%    \begin{macrocode}
\newcommand{\epigraph}[2]{\vspace{\beforeepigraphskip}
  {\epigraphsize\begin{\epigraphflush}\begin{minipage}{\epigraphwidth}
    \@epitext{#1}\\ \@episource{#2}
    \end{minipage}\end{\epigraphflush}
    \vspace{\afterepigraphskip}}}
%    \end{macrocode}
% \end{macro}
%
%
% \begin{macro}{\qitem}
% \begin{macro}{\qitemlabel}
%    |\qitem| is the epigraph list version of |\item|. 
% Set everything inside a minipage.
%    \begin{macrocode}
\newcommand{\qitem}[2]{{\raggedright\item \begin{minipage}{\epigraphwidth}
  \@epitext{#1}\\ \@episource{#2}
  \end{minipage}}}
%    \end{macrocode}
%  |\qitemlabel| is needed for a list as well. It is not going to 
% typeset anything.
%    \begin{macrocode}
\newcommand{\qitemlabel}[1]{\hfill}
%    \end{macrocode}
% \end{macro}
% \end{macro}
%
%
% \begin{environment}{epigraphs}
%  Now for the epigraph list. This is defined in terms of a |list|
% environment. 
%    \begin{macrocode}
\newenvironment{epigraphs}{%
%    \end{macrocode}
% Do the vertical space, set the font size, position according to 
% |\epigraphflush|, and put everyting into a minipage.
%    \begin{macrocode}
  \vspace{\beforeepigraphskip}\begin{\epigraphflush}
  \epigraphsize
  \begin{minipage}{\epigraphwidth}
   \list{}%
%    \end{macrocode}
%  Make the list just fit the minipage (i.e., no indents).
%    \begin{macrocode}
    {\itemindent\z@ \labelwidth\z@ \labelsep\z@
     \leftmargin\z@ \rightmargin\z@
     \let\makelabel\qitemlabel}}%
  {\endlist\end{minipage}\end{\epigraphflush}
   \vspace{\afterepigraphskip}}
%    \end{macrocode}
% \end{environment}
%
% \subsection{Epigraphs before a chapter title}
%
% \begin{macro}{\cleartoevenpage}
% Like |\cleardoublepage| except that it skips pages until an even one, and
% its optional argument is applied to the skipped page, if any.
% The code is based on the kernel |\cleardoublepage| in \file{ltoutput.dtx}.
% \changes{v1.2}{1999/09/12}{Added \cs{cleartoevenpage} command}
% \changes{v1.5}{2000/02/20}{Added optional arg to \cs{cleartoevenpage}}
%    \begin{macrocode}
\providecommand{\cleartoevenpage}[1][\@empty]{%
  \clearpage%
  \ifodd\c@page\hbox{}#1\clearpage\fi}
%    \end{macrocode}
% \end{macro}
%
% \begin{macro}{\@epichapapp}
% \begin{macro}{\dropchapter}
% \begin{macro}{\undodrop}
% \changes{v1.3}{1999/12/01}{Defined chapter dropping and support commands}
%    Commands to drop and restore positions of chapter titles. Dropping is
% accomplished by inserting vertical space before the |\@chapapp| command.
%    \begin{macrocode}
\newcommand{\dropchapter}[1]{%
  \let\@epichapapp\@chapapp
  \renewcommand{\@chapapp}{\vspace*{#1}\@epichapapp}}
\newcommand{\undodrop}{\let\@chapapp\@epichapapp}
%    \end{macrocode}
% \end{macro}
% \end{macro}
% \end{macro}
%
% Placing an epigraph before a chapter title uses the scheme outlined
% by Piet van Oostrum~\cite{OOSTRUM96}. This is to put a zero sized
% picture into the page header. 
%
% \begin{macro}{\if@epirhs}
% \begin{macro}{\if@epicenter}
%  Two booleans for testing whether an epigraph is to be at the RH margin,
% centered, or at the LH margin. The default is RH margin.
%    \begin{macrocode}
\newif\if@epirhs     \@epirhstrue
\newif\if@epicenter  \@epicentertrue
%    \end{macrocode}
% \end{macro}
% \end{macro}
%
% \begin{macro}{\@epipos}
%    This routine sets the |\if@epi...| booleans according to the value of
% |\epigraphflush|. If |\epigraphflush| is neither |center| nor |flushleft|
% then it defaults to |flushright|. We have to use this to be upward 
% compatible with |\epigraphflush| being set by the user with |\renewcommand|.
%    \begin{macrocode}
\newcommand{\@epipos}{
  \long\def\@ept{flushleft}
  \ifx\epigraphflush\@ept
    \@epirhsfalse \@epicenterfalse
  \else
    \long\def\@ept{center}
    \ifx\epigraphflush\@ept
      \@epirhsfalse \@epicentertrue
    \else
      \@epirhstrue  \@epicenterfalse
    \fi
  \fi}
%    \end{macrocode}
% \end{macro}
%
%
% \begin{macro}{\epigraphhead}
% |\epigraphhead[|\meta{distance}|]{|\meta{text}|}| puts \meta{text} at
% \meta{distance} (a number, not a length) below the header at the 
% page position specified by |\epigraphflush|.
%    \begin{macrocode}
\newcommand{\epigraphhead}[2][95]{%
%    \end{macrocode}
% We have to use |\def| instead of the normal \LaTeX{} definition commands
% as we will keep on
% (re)defining things. For reasons that are not fully clear to me \LaTeX{}
% doesn't seem to like me using a |\savebox| for storing the epigraph text,
% so I'll use a command instead.
%    \begin{macrocode}
  \def\@epitemp{\begin{minipage}{\epigraphwidth}#2\end{minipage}}
%    \end{macrocode}
% Define an |epigraph| page style.
%    \begin{macrocode}
  \def\ps@epigraph{\let\@mkboth\@gobbletwo
%    \end{macrocode}
% There are three possible definitions for |\@oddhead| depending on the
% value of |\epigraphflush|. We call |\@epipos| to decide which one to do.
%    \begin{macrocode}
    \@epipos
    \if@epirhs
      \def\@oddhead{\hfil\begin{picture}(0,0)
                         \put(0,-#1){\makebox(0,0)[r]{\@epitemp}}
                         \end{picture}}
    \else
      \if@epicenter
        \def\@oddhead{\hfil\begin{picture}(0,0)
                           \put(0,-#1){\makebox(0,0)[b]{\@epitemp}}
                           \end{picture}\hfil}
      \else
        \def\@oddhead{\begin{picture}(0,0)
                           \put(0,-#1){\makebox(0,0)[l]{\@epitemp}}
                           \end{picture}\hfil}
      \fi
    \fi
    \let\@evenhead\@oddhead
    \def\@oddfoot{\reset@font\hfil\thepage\hfil}
    \let\@evenfoot\@oddfoot}
%    \end{macrocode}
% Make |epigraph| be the page style for this page.
%    \begin{macrocode}
  \thispagestyle{epigraph}}
%    \end{macrocode}
% \end{macro}
%
%
%    The end of this package.
%    \begin{macrocode}
%</usc>
%    \end{macrocode}
%
%
% \bibliographystyle{alpha}
%
% \begin{thebibliography}{GMS94}
%
% \bibitem[GMS94]{GOOSSENS94}
% Michel Goossens, Frank Mittelbach, and Alexander Samarin.
% \newblock {\em The LaTeX Companion}.
% \newblock Addison-Wesley Publishing Company, 1994.
%
% \bibitem[vO96]{OOSTRUM96}
% Piet van Oostrum
% \newblock {\em Page layout in \LaTeX}, June 1996.
% \newblock (Available from CTAN as file \texttt{fancyhdr.tex}).
%
% \bibitem[SW94]{EBOOK}
% Douglas Schenck and Peter Wilson.
% \newblock {\em Information Modeling the EXPRESS Way}.
% \newblock Oxford University Press, 1994 (ISBN 0-19-508714-3).
%
% \bibitem[Thi99]{TTC199}
% Christina Thiele.
% \newblock \emph{The Treasure Chest: Package tours from CTAN},
% \newblock TUGboat, vol.~20, no.~1, pp~53--58, March 1999.
%
% \end{thebibliography}
%
%
% \Finale
% \PrintIndex
%
\endinput

%% \CharacterTable
%%  {Upper-case    \A\B\C\D\E\F\G\H\I\J\K\L\M\N\O\P\Q\R\S\T\U\V\W\X\Y\Z
%%   Lower-case    \a\b\c\d\e\f\g\h\i\j\k\l\m\n\o\p\q\r\s\t\u\v\w\x\y\z
%%   Digits        \0\1\2\3\4\5\6\7\8\9
%%   Exclamation   \!     Double quote  \"     Hash (number) \#
%%   Dollar        \$     Percent       \%     Ampersand     \&
%%   Acute accent  \'     Left paren    \(     Right paren   \)
%%   Asterisk      \*     Plus          \+     Comma         \,
%%   Minus         \-     Point         \.     Solidus       \/
%%   Colon         \:     Semicolon     \;     Less than     \<
%%   Equals        \=     Greater than  \>     Question mark \?
%%   Commercial at \@     Left bracket  \[     Backslash     \\
%%   Right bracket \]     Circumflex    \^     Underscore    \_
%%   Grave accent  \`     Left brace    \{     Vertical bar  \|
%%   Right brace   \}     Tilde         \~}


