% \iffalse meta-comment
%
% hanging.dtx
% Author: Peter Wilson (CUA)
% Maintainer: Will Robertson (will dot robertson at latex-project dot org)
% Copyright 1998 --- 2004 Peter R. Wilson
%
% This work may be distributed and/or modified under the
% conditions of the LaTeX Project Public License, either
% version 1.3c of this license or (at your option) any 
% later version.
% The latest version of the license is in
%    http://www.latex-project.org/lppl.txt
% and version 1.3c or later is part of all distributions of
% LaTeX version 2003/06/01 or later.
%
% This work has the LPPL maintenance status "maintained".
% The Current Maintainer of this work is Will Robertson.
%
% This work consists of the files listed in the README file.
%
%
%<*driver>
\documentclass{ltxdoc}
\usepackage{url}
\EnableCrossrefs
\CodelineIndex
\setcounter{StandardModuleDepth}{1}
\begin{document}
  \DocInput{hanging.dtx}
\end{document}
%</driver>
%
% \fi
%
% \CheckSum{308}
%
% \DoNotIndex{\',\.,\@M,\@@input,\@addtoreset,\@arabic,\@badmath}
% \DoNotIndex{\@centercr,\@cite}
% \DoNotIndex{\@dotsep,\@empty,\@float,\@gobble,\@gobbletwo,\@ignoretrue}
% \DoNotIndex{\@input,\@ixpt,\@m}
% \DoNotIndex{\@minus,\@mkboth,\@ne,\@nil,\@nomath,\@plus,\@set@topoint}
% \DoNotIndex{\@tempboxa,\@tempcnta,\@tempdima,\@tempdimb}
% \DoNotIndex{\@tempswafalse,\@tempswatrue,\@viipt,\@viiipt,\@vipt}
% \DoNotIndex{\@vpt,\@warning,\@xiipt,\@xipt,\@xivpt,\@xpt,\@xviipt}
% \DoNotIndex{\@xxpt,\@xxvpt,\\,\ ,\addpenalty,\addtolength,\addvspace}
% \DoNotIndex{\advance,\Alph,\alph}
% \DoNotIndex{\arabic,\ast,\begin,\begingroup,\bfseries,\bgroup,\box}
% \DoNotIndex{\bullet}
% \DoNotIndex{\cdot,\cite,\CodelineIndex,\cr,\day,\DeclareOption}
% \DoNotIndex{\def,\DisableCrossrefs,\divide,\DocInput,\documentclass}
% \DoNotIndex{\DoNotIndex,\egroup,\ifdim,\else,\fi,\em,\endtrivlist}
% \DoNotIndex{\EnableCrossrefs,\end,\end@dblfloat,\end@float,\endgroup}
% \DoNotIndex{\endlist,\everycr,\everypar,\ExecuteOptions,\expandafter}
% \DoNotIndex{\fbox}
% \DoNotIndex{\filedate,\filename,\fileversion,\fontsize,\framebox,\gdef}
% \DoNotIndex{\global,\halign,\hangindent,\hbox,\hfil,\hfill,\hrule}
% \DoNotIndex{\hsize,\hskip,\hspace,\hss,\if@tempswa,\ifcase,\or,\fi,\fi}
% \DoNotIndex{\ifhmode,\ifvmode,\ifnum,\iftrue,\ifx,\fi,\fi,\fi,\fi,\fi}
% \DoNotIndex{\input}
% \DoNotIndex{\jobname,\kern,\leavevmode,\let,\leftmark}
% \DoNotIndex{\list,\llap,\long,\m@ne,\m@th,\mark,\markboth,\markright}
% \DoNotIndex{\month,\newcommand,\newcounter,\newenvironment}
% \DoNotIndex{\NeedsTeXFormat,\newdimen}
% \DoNotIndex{\newlength,\newpage,\nobreak,\noindent,\null,\number}
% \DoNotIndex{\numberline,\OldMakeindex,\OnlyDescription,\p@}
% \DoNotIndex{\pagestyle,\par,\paragraph,\paragraphmark,\parfillskip}
% \DoNotIndex{\penalty,\PrintChanges,\PrintIndex,\ProcessOptions}
% \DoNotIndex{\protect,\ProvidesClass,\raggedbottom,\raggedright}
% \DoNotIndex{\refstepcounter,\relax,\renewcommand,\reset@font}
% \DoNotIndex{\rightmargin,\rightmark,\rightskip,\rlap,\rmfamily,\roman}
% \DoNotIndex{\roman,\secdef,\selectfont,\setbox,\setcounter,\setlength}
% \DoNotIndex{\settowidth,\sfcode,\skip,\sloppy,\slshape,\space}
% \DoNotIndex{\symbol,\the,\trivlist,\typeout,\tw@,\undefined,\uppercase}
% \DoNotIndex{\usecounter,\usefont,\usepackage,\vfil,\vfill,\viiipt}
% \DoNotIndex{\viipt,\vipt,\vskip,\vspace}
% \DoNotIndex{\wd,\xiipt,\year,\z@}
%
% \changes{v1.0}{1998/11/29}{First public release}
% \changes{v1.1}{1999/05/01}{Made compatible with ltugboat.cls}
% \changes{v1.1a}{1999/07/28}{Added text from Stefan Ulrich}
% \changes{v1.2}{2001/03/15}{Added hanging punctuation from Stefan Ulrich}
% \changes{v1.2a}{2004/04/22}{Updated license and contact info}
%
% \def\dtxfile{hanging.dtx}
% ^^A \def\fileversion{v1.1a}
% ^^A \def\filedate{1999/07/28}
% \def\fileversion{v1.2}
% \def\filedate{2001/03/15}
% \def\fileversion{v1.2a}
% \def\filedate{2004/04/22}
% \newcommand*{\Lpack}[1]{\textsf {#1}}           ^^A typeset a package
% \newcommand*{\Lopt}[1]{\textsf {#1}}            ^^A typeset an option
% \newcommand*{\file}[1]{\texttt {#1}}            ^^A typeset a file
% \newcommand*{\Lcount}[1]{\textsl {\small#1}}    ^^A typeset a counter
% \newcommand*{\pstyle}[1]{\textsl {#1}}          ^^A typeset a pagestyle
% \newcommand*{\Lenv}[1]{\texttt {#1}}            ^^A typeset an environment
%  ^^A fake a hangpara for demo purposes
% \newcommand{\hpdemo}[2]{\hangindent#1\hangafter#2\noindent}
% \newcommand{\Han}{H\`{a}n \ensuremath{\mathrm{Th\acute{\hat{e}}}} Th\`{a}nh}
%
% \title{The \Lpack{hanging} package\thanks{This
%        file (\texttt{\dtxfile}) has version number \fileversion, last revised
%        \filedate.}}
%
% \author{
%   Author: Peter Wilson, Herries Press \\
%   Maintainer: Will Robertson \\
%   \texttt{will dot robertson at latex-project dot org}
% }
% \date{\filedate}
% \maketitle
% \begin{abstract}
%    The \Lpack{hanging} package provides facilities for defining hanging
% paragraphs and hanging punctuation.
% \end{abstract}
% \tableofcontents
%
% \StopEventually{}
%
% 
%
% \section{Introduction}
%
% \hpdemo{3em}{2}Some authors may wish to use hanging paragraphs in 
% their documents.
% Normally only the first line of a paragraph is indented. A hanging paragraph
% is a paragraph like this one where lines other than the first have indentation.
% Other authors might wish to use hanging punctuation.
% In this style of typesetting punctuation marks that come at either the
% start or end of a line are typeset outside the normal text block.
%
% The \Lpack{hanging} package provides facilities for both hanging paragraphs
% and hanging punctuation.
%
% This manual is typeset according to the conventions of the
% \LaTeX{} \textsc{docstrip} utility which enables the automatic
% extraction of the \LaTeX{} macro source files~\cite{GOOSSENS94}.
%
%    Section~\ref{sec:usc} describes the usage of the package.
% Commented source code for the package is in Section~\ref{sec:code}.
%
% \section{The \Lpack{hanging} package} \label{sec:usc}
%
% \subsection{Hanging paragraphs}
%
%    The \Lpack{hanging} package provides a command for producing
% a single hanging paragraph and an environment for typesetting
% a series of hanging paragraphs.
%
% \DescribeMacro{\hangpara}
%  The command |\hangpara{|\meta{indent}|}{|\meta{afternum}|}| 
% placed at the start of a paragraph will cause it to be typeset
% as a hanging paragraph. 
%
% The \meta{indent} argument specifies the
% amount of indentation. The value must be a length. A positive
% value will indent the left hand side of the paragraph and a negative
% value will cause the right hand side of the text to be indented.
%
% \hpdemo{3em}{-2}The \meta{afternum} argument controls the number of 
% lines that will
% be indented. The value can be a positive or negative integer, say $N$.
% If $N > 0$ then lines \emph{after} the $N^{th}$ line (i.e., lines $N+1$ 
% and onwards) will be indented. If $N < 0$
% then the first $N$ lines will be indented. This paragraph has been
% typeset according to the specification |\hangpara{3em}{-2}| whereas
% the first paragraph in the Introduction was typeset according
% to |\hangpara{3em}{2}|.
%
% A word of caution: There must be no linebreak between the |\hangpara|
% command and the start of the paragraph text. Either put the command
% at the start of the paragraph text, or on the previous line with a
% following \%.
%
% \DescribeEnv{hangparas}
% The |hangparas| environment can be used for typeseting a sequence of
% hanging paragraphs. The environment takes the same two arguments as the
% |\hangpara| command. That is,\\
% |\begin{hangparas}{|\meta{indent}|}{|\meta{afternum}|}|.
%
% Another word of caution: The |hangparas| environment changes the definition
% of |\everypar| and may have unexpected interactions with any enclosed
% commands or environments that also change |\everypar|. If this does happen,
% then use the |\hangpara| command for the problematic paragraph(s) instead
% of the |hangparas| environment.
%
% \subsection{Hanging punctuation}
%
% \DescribeEnv{hangpunct}
% Text within the |hangpunct| environment will be typeset with hanging 
% punctuation. Personally, I think that this is more of a curiosity than
% anything else. Others disagree and think it is an essential part
% of fine typography.
%
% Use of the environment may have unpleasant side effects as the punctuation
% characters are made active (i.e., they act as commands rather than as
% glyphs). The hanging punctuation characters are left and right single
% and double quotes (|`'|), period (or full stop) (|.|), exclamation
% mark (|!|), question mark (|?|), colon and semicolon (|:;|), and
% the comma (|,|).
%
%    If you really care about hanging punctuation, \Han's
% \texttt{pdf(La)TeX} has built in support via |\pdfprotrudechars|. This
% is implemented in the \texttt{pdfTeX} program itself~\cite{HTHESIS}, 
% so is robust. The \texttt{pdf(La)TeX} system is available from CTAN but
% the latest version, if not on CTAN, can be obtained from
% \url{ftp://ftp.cstug.cz/pub/tex/lo9cal/cstug/thanh/pdftex}.
%
%    The package takes several options which stop punctuation characters from
% being hung. The available options include:
% \begin{itemize}
% \item[\Lopt{notcomma}]  don't hang the comma (|,|)
% \item[\Lopt{notperiod}] don't hang the period (full stop) (|.|)
% \item[\Lopt{notquote}]  don't hang any quote marks (|` `` '' '|)
% \item[\Lopt{notquery}]  don't hang the question mark (|?|)
% \item[\Lopt{notexcl}]   don't hang the exclamation mark (|!|)
% \item[\Lopt{notcolon}]  don't hang the colon (|:|)
% \item[\Lopt{notscolon}] don't hang the semicolon (|;|)
% \end{itemize}
% If none of these options are given then all the available punctuation 
% marks will be hung. 
%
%    There is one other option:
% \begin{itemize}
% \item[\Lopt{fine}] The default settings for the amount of punctuation
% hang correspond to those given by Donald Knuth in the 
% \textit{TeXbook}~\cite{KNUTH84a}, and are independent of the local
% font in use (use \texttt{pdfLaTeX} for font-dependent settings).
% Stefan Ulrich\footnote{Private email on 2001/03/14 
%               (\texttt{ulrich@cis.uni-muenchen.de}).} 
% suggested a less aggresive hang, which is provided by
% the \Lopt{fine} option. This option also stops hanging question and 
% exclamation marks.
% \end{itemize}
%
% \DescribeMacro{\nhpt}
% \DescribeMacro{\nhlq}
% \DescribeMacro{\nhrq}
% If a non-hanging period, or left or right quote character is required 
% in it's character form within the |hangpunct| environment, then 
% respectively the
% |\nhpt|, |\nhlq| and |\nhrq| commands can be used. For example, if a
% length is to be changed you cannot write: \\
% |\setlength{\mylength}{2.76cm}|. Instead you must code this as: \\
% |\setlength{\mylength}{2\nhpt76cm}|.
%
%
% \section{The package code} \label{sec:code}
%
%
%    Announce the name and version of the package, which requires
% \LaTeXe.
%    \begin{macrocode}
%<*usc>
\NeedsTeXFormat{LaTeX2e}
\ProvidesPackage{hanging}[2001/03/15 v1.2 hanging paragraphs and punctuation]
%    \end{macrocode}
%
% \subsection{Hanging paragraphs}
%
% \begin{macro}{\hangpara}
% The command for a single hanging paragraph.
%    Just package up the \TeX{} |\hang...| commands, together with
% |\noindent| to switch off any existing |\parindent| indentation.
%    \begin{macrocode}
\newcommand{\hangpara}[2]{\hangindent#1\hangafter#2\noindent}
%    \end{macrocode}
% \end{macro}
%
% \begin{environment}{hangparas}
%    The environment for typesetting multiple hanging paragraphs. It takes
% the same arguments as the |\hangpara| command. At the start, set the
% paragraph indent to zero, then change |\everypar| to |\hangpara|.
% At the end of the environment make sure that the last paragraph was 
% finished via |\par|, otherwise the user would have to leave a blank line
% or use |\par| explicitly.
%    \begin{macrocode}
\newenvironment{hangparas}[2]{\setlength{\parindent}{\z@}
  \everypar={\hangpara{#1}{#2}}}{\par}
%    \end{macrocode}
% \end{environment}
%
% \subsection{Hanging punctuation}
%
% The code for hanging punctuation is based on the Dirty Tricks example
% on page 395 of the \TeX book~\cite{KNUTH84a}.
%
% I made an experimental attempt to include hanging hyphens, 
% but this broke too many other things all over the place to be of
% practical use. I believe Knuth when he says that
% hanging hyphens require a font with a zero-width hyphen. I think that
% the basic problem is that the `-' character, as input, represents both
% a hyphen and a minus sign; diddling with a hyphen means that the minus
% sign is similarly diddled. This package is meant to be independent of the
% font, and following Knuth, providing a hanging hyphen is font-dependent.
% Fortunately, most of the time \TeX{} manages to avoid hyphens at line ends.
%
% Stefan Ulrich (\texttt{ulrich@cis.uni-muenchen.de}) pointed out that
% the EC fonts include a hyphen with a smaller, but non-zero, width. Using
% this then the last non-hyphen character at the end of a hyphenated line
% will be a little closer to the margin. To use the smaller EC hyphen, 
% Stefan suggested doing:
% \begin{verbatim}
% \usepackage[T1]{fontenc}
% \hyphenchar\font=127
% \end{verbatim}
%
% \begin{macro}{\ifh@ngcomma}
% \begin{macro}{\ifh@ngfstop}
% \begin{macro}{\ifh@ngquote}
% \begin{macro}{\ifh@ngquery}
% \begin{macro}{\ifh@ngexcl}
% \begin{macro}{\ifh@ngcolon}
% \begin{macro}{\ifh@ngscolon}
% \begin{macro}{\ifh@ngfine}
%  We need some booleans for the options.
% \changes{v1.1}{1999/05/01}{Added options to switch off hanging particular characters}
% \changes{v1.2}{2001/03/15}{Added the fine option.}
%    \begin{macrocode}
\newif\ifh@ngcomma  \h@ngcommatrue
\newif\ifh@ngfstop  \h@ngfstoptrue
\newif\ifh@ngquote  \h@ngquotetrue
\newif\ifh@ngquery  \h@ngquerytrue
\newif\ifh@ngexcl   \h@ngexcltrue
\newif\ifh@ngcolon  \h@ngcolontrue
\newif\ifh@ngscolon \h@ngscolontrue
\newif\ifh@ngfine   \h@ngfinefalse
%    \end{macrocode}
% \end{macro}
% \end{macro}
% \end{macro}
% \end{macro}
% \end{macro}
% \end{macro}
% \end{macro}
% \end{macro}
%
%    Now do the options.
% 
%    \begin{macrocode}
\DeclareOption{notcomma}{\h@ngcommafalse}
\DeclareOption{notperiod}{\h@ngfstopfalse}
\DeclareOption{notquote}{\h@ngquotefalse}
\DeclareOption{notquery}{\h@ngqueryfalse}
\DeclareOption{notexcl}{\h@ngexclfalse}
\DeclareOption{notcolon}{\h@ngcolonfalse}
\DeclareOption{notscolon}{\h@ngscolonfalse}
\DeclareOption{fine}{\h@ngfinetrue}
\ProcessOptions\relax
\ifh@ngfine
  \h@ngqueryfalse \h@ngexclfalse
\fi
%    \end{macrocode}
%
% \begin{macro}{\h@ngcommawd}
% \begin{macro}{\h@ngfstopwd}
% \begin{macro}{\h@ngquotewd}
% \begin{macro}{\h@ngdquotewd}
% \begin{macro}{\h@ngquerywd}
% \begin{macro}{\h@ngexclwd}
% \begin{macro}{\h@ngcolonwd}
% \begin{macro}{\h@ngscolonwd}
% We need some lengths for kerning for each punctuation mark that might be 
% hung. These are set to the width of the character.
%    \begin{macrocode}
\newlength{\h@ngcommawd}  \settowidth{\h@ngcommawd}{,}
\newlength{\h@ngfstopwd}  \settowidth{\h@ngfstopwd}{.}
\newlength{\h@ngquotewd}  \settowidth{\h@ngquotewd}{`}
\newlength{\h@ngdquotewd} \settowidth{\h@ngdquotewd}{``}
\newlength{\h@ngquerywd}  \settowidth{\h@ngquerywd}{?}
\newlength{\h@ngexclwd}   \settowidth{\h@ngexclwd}{!}
\newlength{\h@ngcolonwd}  \settowidth{\h@ngcolonwd}{:}
\newlength{\h@ngscolonwd} \settowidth{\h@ngscolonwd}{;}
%    \end{macrocode}
%
% These values are changed if the \Lopt{fine} option is given. The following
% is by courtesy of Stefan Ulrich.
%    \begin{macrocode}
\ifh@ngfine
  \setlength{\h@ngcommawd}{0.35\h@ngcommawd}
  \setlength{\h@ngfstopwd}{0.30\h@ngfstopwd}
  \setlength{\h@ngquotewd}{0.30\h@ngquotewd}
  \setlength{\h@ngdquotewd}{0.35\h@ngdquotewd}
  \setlength{\h@ngcolonwd}{0.20\h@ngcolonwd}
  \setlength{\h@ngscolonwd}{0.20\h@ngscolonwd}
\fi

%    \end{macrocode}
% \end{macro}
% \end{macro}
% \end{macro}
% \end{macro}
% \end{macro}
% \end{macro}
% \end{macro}
% \end{macro}
%
% \begin{macro}{\h@ngallowhyphens}
% An incantation to be used to allow hyphenation in a `word' beginning
% with a quote mark.\footnote{Hubert Gaesslein 
% (\texttt{hubertjg@virtualphotonics.cm})
% pointed out that the definition of \texttt{h@ngallowhyphens} should end
% with \texttt{z@skip} rather than my original \texttt{z@}, in case it was used
% in situations I hadn't thought of.}
% \changes{v1.1}{1999/05/01}{Changed allowhyphens to h@ngallowhyphens for compatibility with ltugboat.cls}
% \changes{v1.1a}{1999/05/01}{Made h@ngallowhyphens more robust}
%    \begin{macrocode}
\newcommand{\h@ngallowhyphens}{\nobreak\hskip\z@skip}
%    \end{macrocode}
% \end{macro}
% 
% \begin{macro}{\nhpt}
% \begin{macro}{\nhlq}
% \begin{macro}{\nhrq}
% As the punctuation is going to be made active, we need commands for 
% some of the characters when they will be used in constants.
%    \begin{macrocode}
\newcommand{\nhpt}{.}
\newcommand{\nhlq}{`}
\newcommand{\nhrq}{'}
%    \end{macrocode}
% \end{macro}
% \end{macro}
% \end{macro}
%
% \begin{macro}{\h@nglqq}
% \begin{macro}{\h@ngrqq}
% \begin{macro}{\h@ngrquote}
% \begin{macro}{\h@nglquote}
% \begin{macro}{\h@ngcomma}
% \begin{macro}{\h@ngfstop}
% \begin{macro}{\h@ngquery}
% \begin{macro}{\h@ngexcl}
% \begin{macro}{\h@ngcolon}
% \begin{macro}{\h@ngscolon}
% Commands for punctuation with kerning.
%    \begin{macrocode}
\newcommand{\h@nglqq}{``}
\newcommand{\h@ngrqq}{''}
\newcommand{\h@ngrquote}{'\kern-\h@ngquotewd\kern\h@ngquotewd}
\newcommand{\h@nglquote}{\ifhmode\kern\h@ngquotewd\vadjust{}\else\leavevmode\fi
                         \kern-\h@ngquotewd`\h@ngallowhyphens}
\newcommand{\h@ngcomma}{,\kern-\h@ngcommawd\kern\h@ngcommawd}
\newcommand{\h@ngfstop}{.\kern-\h@ngfstopwd\kern\h@ngfstopwd}
\newcommand{\h@ngquery}{?\kern-\h@ngquerywd\kern\h@ngquerywd}
\newcommand{\h@ngexcl}{!\kern-\h@ngexclwd\kern\h@ngexclwd}
\newcommand{\h@ngcolon}{:\kern-\h@ngcolonwd\kern\h@ngcolonwd}
\newcommand{\h@ngscolon}{;\kern-\h@ngscolonwd\kern\h@ngscolonwd}
%    \end{macrocode}
% \end{macro}
% \end{macro}
% \end{macro}
% \end{macro}
% \end{macro}
% \end{macro}
% \end{macro}
% \end{macro}
% \end{macro}
% \end{macro}
%
% \begin{macro}{\activatepunct}
% At this point we make the punctuation characters 
% active.\footnote{My thanks to Harald Hanche-Olsen 
% \texttt{hanche@math.ntnu.no}
% for showing me how to do this instead of my initial hack of putting
% all the following code into a file to be input by the environment.}
%
% Activate the punctuation characters, leaving the left quote till last.
%    \begin{macrocode}
\def\activatepunct{%
\ifh@ngcomma  \catcode`\,=\active \fi
\ifh@ngfstop  \catcode`\.=\active \fi
\ifh@ngquote  \catcode`\'=\active \fi
\ifh@ngquery  \catcode`\?=\active \fi
\ifh@ngexcl   \catcode`\!=\active \fi
\ifh@ngcolon  \catcode`\:=\active \fi
\ifh@ngscolon \catcode`\;=\active \fi
\ifh@ngquote  \catcode`\`=\active \fi
}
%    \end{macrocode}
% \end{macro}
% Define the simple punctuation macros in terms of the respective 
% |\h@ng...| commands, using |\gdef| as they are instantiated
% in a group.
%    \begin{macrocode}
\begingroup
\activatepunct
\ifh@ngcomma  \gdef,{\h@ngcomma}  \fi
\ifh@ngfstop  \gdef.{\h@ngfstop}  \fi
\ifh@ngquery  \gdef?{\h@ngquery}  \fi
\ifh@ngexcl   \gdef!{\h@ngexcl}  \fi
\ifh@ngcolon  \gdef:{\h@ngcolon}  \fi
\ifh@ngscolon \gdef;{\h@ngscolon}  \fi
%    \end{macrocode}
% Life is more complicated for quote marks as they have to check for 
% ligatures.
%    \begin{macrocode}
\ifh@ngquote
  \gdef'{\futurelet\next\h@ngrqtest}
  \gdef`{\futurelet\next\h@nglqtest}
%    \end{macrocode}
%
% \begin{macro}{\h@ngrqtest}
% \begin{macro}{\h@nglqtest}
% \begin{macro}{\h@ngrquotes}
% \begin{macro}{\h@nglquotes}
% Commands for left and right quotes ligature tests.
%    \begin{macrocode}
  \gdef\h@ngrqtest{\ifx\next'\let\next=\h@ngrquotes\else\let\next=\h@ngrquote\fi\next}
  \gdef\h@nglqtest{\ifx\next`\let\next=\h@nglquotes\else\let\next=\h@nglquote\fi\next}
  \gdef\h@ngrquotes'{\h@ngrqq\kern-\h@ngdquotewd\kern\h@ngdquotewd}
  \gdef\h@nglquotes`{\ifhmode\kern\h@ngdquotewd\vadjust{}\else\leavevmode\fi
    \kern-\h@ngdquotewd\h@nglqq\h@ngallowhyphens}
\fi
%    \end{macrocode}
% Finally, end the group started earlier.
%    \begin{macrocode}
\endgroup
%    \end{macrocode}
% \end{macro}
% \end{macro}
% \end{macro}
% \end{macro}
%
% 
% \begin{environment}{hangpunct}
% The definition of the hanging punctuation environment is now very simple.
% It just calls |\activatepunct|.
%    \begin{macrocode}
\newenvironment{hangpunct}{\activatepunct}{}
%    \end{macrocode}
% \end{environment}
%
%    The end of this package.
%    \begin{macrocode}
%</usc>
%    \end{macrocode}
%
%
% \bibliographystyle{alpha}
%
% \begin{thebibliography}{GMS94}
%
% \bibitem[GMS94]{GOOSSENS94}
% Michel Goossens, Frank Mittelbach, and Alexander Samarin.
% \newblock {\em The LaTeX Companion}.
% \newblock Addison-Wesley Publishing Company, 1994.
%
% \bibitem[Knu84]{KNUTH84a}
% Donald~E. Knuth.
% \newblock {\em The TeXbook}.
% \newblock Addison-Wesley Publishing Company, 1984.
%
% \bibitem[Tha00]{HTHESIS}
% \Han.
% \newblock \emph{Micro-typographic extensions to the TeX typesetting system}.
% \newblock Dissertation.
% \newblock Masaryk University Brno, Faculty of Informatics, October 2000.
% \newblock (Available as \url{http://www.fi-muni.cz/~thanh/download/thesis.pdf})
%
%
% \end{thebibliography}
%
%
% \Finale
% \PrintIndex
%
\endinput

%% \CharacterTable
%%  {Upper-case    \A\B\C\D\E\F\G\H\I\J\K\L\M\N\O\P\Q\R\S\T\U\V\W\X\Y\Z
%%   Lower-case    \a\b\c\d\e\f\g\h\i\j\k\l\m\n\o\p\q\r\s\t\u\v\w\x\y\z
%%   Digits        \0\1\2\3\4\5\6\7\8\9
%%   Exclamation   \!     Double quote  \"     Hash (number) \#
%%   Dollar        \$     Percent       \%     Ampersand     \&
%%   Acute accent  \'     Left paren    \(     Right paren   \)
%%   Asterisk      \*     Plus          \+     Comma         \,
%%   Minus         \-     Point         \.     Solidus       \/
%%   Colon         \:     Semicolon     \;     Less than     \<
%%   Equals        \=     Greater than  \>     Question mark \?
%%   Commercial at \@     Left bracket  \[     Backslash     \\
%%   Right bracket \]     Circumflex    \^     Underscore    \_
%%   Grave accent  \`     Left brace    \{     Vertical bar  \|
%%   Right brace   \}     Tilde         \~}


