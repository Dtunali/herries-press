% \iffalse meta-comment
% booklet.dtx
% Author: Peter Wilson (Herries Press)
% Maintainer: Will Robertson (will dot robertson at latex-project dot org)
% Copyright 2000--2005 Peter R. Wilson
%
% This work may be distributed and/or modified under the
% conditions of the LaTeX Project Public License, either
% version 1.3c of this license or (at your option) any 
% later version: <http://www.latex-project.org/lppl.txt>
%
% This work has the LPPL maintenance status "maintained".
% The Current Maintainer of this work is Will Robertson.
%
% This work consists of the files listed in the README file.
%
% 
%<*driver>
\documentclass[twoside]{ltxdoc}
\usepackage{url}
\usepackage[draft=false,
            plainpages=false,
            pdfpagelabels,
            bookmarksnumbered,
            hyperindex=false
           ]{hyperref}
\providecommand{\phantomsection}{}
\makeatletter
  \@mparswitchfalse
\makeatother
\EnableCrossrefs
\CodelineIndex
%%\OnlyDescription
\renewcommand{\MakeUppercase}[1]{#1}
\pagestyle{headings}
\setcounter{StandardModuleDepth}{1}
\begin{document}
  \DocInput{booklet.dtx}
\end{document}
%</driver>
%
% \fi
%
% \CheckSum{629}
%
% \DoNotIndex{\',\.,\@M,\@@input,\@addtoreset,\@arabic,\@badmath}
% \DoNotIndex{\@centercr,\@cite}
% \DoNotIndex{\@dotsep,\@empty,\@float,\@gobble,\@gobbletwo,\@ignoretrue}
% \DoNotIndex{\@input,\@ixpt,\@m}
% \DoNotIndex{\@minus,\@mkboth,\@ne,\@nil,\@nomath,\@plus,\@set@topoint}
% \DoNotIndex{\@tempboxa,\@tempcnta,\@tempdima,\@tempdimb}
% \DoNotIndex{\@tempswafalse,\@tempswatrue,\@viipt,\@viiipt,\@vipt}
% \DoNotIndex{\@vpt,\@warning,\@xiipt,\@xipt,\@xivpt,\@xpt,\@xviipt}
% \DoNotIndex{\@xxpt,\@xxvpt,\\,\ ,\addpenalty,\addtolength,\addvspace}
% \DoNotIndex{\advance,\Alph,\alph}
% \DoNotIndex{\arabic,\ast,\begin,\begingroup,\bfseries,\bgroup,\box}
% \DoNotIndex{\bullet}
% \DoNotIndex{\cdot,\cite,\CodelineIndex,\cr,\day,\DeclareOption}
% \DoNotIndex{\def,\DisableCrossrefs,\divide,\DocInput,\documentclass}
% \DoNotIndex{\DoNotIndex,\egroup,\ifdim,\else,\fi,\em,\endtrivlist}
% \DoNotIndex{\EnableCrossrefs,\end,\end@dblfloat,\end@float,\endgroup}
% \DoNotIndex{\endlist,\everycr,\everypar,\ExecuteOptions,\expandafter}
% \DoNotIndex{\fbox}
% \DoNotIndex{\filedate,\filename,\fileversion,\fontsize,\framebox,\gdef}
% \DoNotIndex{\global,\halign,\hangindent,\hbox,\hfil,\hfill,\hrule}
% \DoNotIndex{\hsize,\hskip,\hspace,\hss,\if@tempswa,\ifcase,\or,\fi,\fi}
% \DoNotIndex{\ifhmode,\ifvmode,\ifnum,\iftrue,\ifx,\fi,\fi,\fi,\fi,\fi}
% \DoNotIndex{\input}
% \DoNotIndex{\jobname,\kern,\leavevmode,\let,\leftmark}
% \DoNotIndex{\list,\llap,\long,\m@ne,\m@th,\mark,\markboth,\markright}
% \DoNotIndex{\month,\newcommand,\newcounter,\newenvironment}
% \DoNotIndex{\NeedsTeXFormat,\newdimen}
% \DoNotIndex{\newlength,\newpage,\nobreak,\noindent,\null,\number}
% \DoNotIndex{\numberline,\OldMakeindex,\OnlyDescription,\p@}
% \DoNotIndex{\pagestyle,\par,\paragraph,\paragraphmark,\parfillskip}
% \DoNotIndex{\penalty,\PrintChanges,\PrintIndex,\ProcessOptions}
% \DoNotIndex{\protect,\ProvidesClass,\raggedbottom,\raggedright}
% \DoNotIndex{\refstepcounter,\relax,\renewcommand,\reset@font}
% \DoNotIndex{\rightmargin,\rightmark,\rightskip,\rlap,\rmfamily,\roman}
% \DoNotIndex{\roman,\secdef,\selectfont,\setbox,\setcounter,\setlength}
% \DoNotIndex{\settowidth,\sfcode,\skip,\sloppy,\slshape,\space}
% \DoNotIndex{\symbol,\the,\trivlist,\typeout,\tw@,\undefined,\uppercase}
% \DoNotIndex{\usecounter,\usefont,\usepackage,\vfil,\vfill,\viiipt}
% \DoNotIndex{\viipt,\vipt,\vskip,\vspace}
% \DoNotIndex{\wd,\xiipt,\year,\z@}
%
% \changes{v0.5}{2001/08/11}{First public release}
% \changes{v0.6}{2001/08/26}{Reduced use of boxes}
% \changes{v0.6a}{2002/10/05}{Improved documentation, no technical changes}
% \changes{v0.6b}{2004/04/16}{Changed license and contact details}
% \changes{v0.7}{2005/03/23}{Improved print setup for memoir}
% \changes{v0.7a}{2005/06/01}{Additional documentation}
% \changes{v0.7b}{2009/09/02}{New maintainer}
%
% \def\dtxfile{booklet.dtx}
% \def\fileversion{v0.5} \def\filedate{2001/08/11}
% \def\fileversion{v0.6} \def\filedate{2001/08/26}
% \def\fileversion{v0.6a} \def\filedate{2002/10/05}
% \def\fileversion{v0.6b} \def\filedate{2004/04/16}
% \def\fileversion{v0.7} \def\filedate{2005/03/23}
% \def\fileversion{v0.7a} \def\filedate{2005/06/01}
% \def\fileversion{v0.7b} \def\filedate{2009/09/02}
% \newcommand*{\Lpack}[1]{\textsf {#1}}           ^^A typeset a package
% \newcommand*{\Lopt}[1]{\textsf {#1}}            ^^A typeset an option
% \newcommand*{\file}[1]{\texttt {#1}}            ^^A typeset a file
% \newcommand*{\Lcount}[1]{\textsl {\small#1}}    ^^A typeset a counter
% \newcommand*{\pstyle}[1]{\textsl {#1}}          ^^A typeset a pagestyle
% \newcommand*{\Lenv}[1]{\texttt {#1}}            ^^A typeset an environment
%
% \title{Printing booklets with LaTeX\thanks{This
%        file (\texttt{\dtxfile}) has version number \fileversion, last revised
%        \filedate.}}
%
% \author{%
% Author: Peter Wilson, Herries Press\\
% Maintainer: Will Robertson\\
% \texttt{will dot robertson at latex-project dot org}
% }
% \date{\filedate}
% \maketitle
% \begin{abstract}
%    The \Lpack{booklet} package provides some aid in printing
% simple booklets or signatures for longer books.
%
% \end{abstract}
% \tableofcontents
%
% 
%
% \section{Introduction}
%
%     In normal usage the \Lpack{booklet} package subverts LaTeX into
% putting pairs of pages onto a single page. For example, the contents
% of pairs of A4 
% portrait pages can be rotated and printed as two A5 portrait 
% pages side by side on an A4 landscape page.
%
%     In 1993 Timothy Van Zandt wrote the \Lpack{2up} generic macros 
% for printing more than one page on a physical sheet~\cite{2UP};
% his original code and documentation is in appendices~\ref{sec:2upcode}
% and~\ref{sec:2updoc}.
% The \Lpack{booklet}
% package is a poor man's version of some of the \Lpack{2up} macros
% specifically for (pdf)LaTeX.
%
%    The \Lpack{2up} and \Lpack{booklet} packages cannot be used
% simultaneously in one document.
%
% This manual is typeset according to the conventions of the
% \LaTeX{} \textsc{docstrip} utility which enables the automatic
% extraction of the \LaTeX{} macro source files~\cite{GOOSSENS94}.
%
%    Section~\ref{sec:usc} describes the usage of the \Lpack{booklet} 
% package and 
% commented source code is in Section~\ref{sec:code}. Timothy Van Zandt's
% original macros and documentation are supplied in the appendices.
%
%
%
% \section{The \Lpack{booklet} package} \label{sec:usc}
%
% The code provided by the \Lpack{booklet} package is meant to help
% with some aspects of booklet printing.
%
%    Basically, the pages of documents processed with the \Lpack{booklet}
% package will be reordered and scaled so that they can be printed as
% four pages per physical sheet of paper, two pages per side of the
% sheet. The default expectation is that the original document is in
% a portrait orientation and the pages are printed rotated onto a landscape
% sheet. Folding the sheet in half will produce a half-sized document, again
% with portrait orientation.
%
%    Professionally printed books have many pages printed per sheet of (large)
% paper, which is then folded and cut where necessary to produce a 
% \emph{signature} of several smaller sheets. Folding a sheet in
% half produces a one sheet \emph{folio} signature with four pages. 
% Folding it in half again and cutting
% along the original fold gives a two sheet \emph{quarto} signature 
% with eight pages. Folding in half again, 
% results in a four sheet \emph{octavo} signature with 16 pages, and so on
% through \emph{16mo}, \emph{32mo}, to \emph{64mo} with six folds and 128 pages. 
% \emph{Octavo} is the most popular of these.
%
%  Other folds can produce other signatures. For example a \emph{sexto},
% obtained by folding in half and then folding in thirds, is a three sheet
% signature with 12 pages.
%
%    The main differences between the \Lpack{2up} macros and the 
% \Lpack{booklet} package are:
% \begin{itemize}
% \item \Lpack{2up} is generic but \Lpack{booklet} can only be used
%       with (pdf)LaTeX.
% \item \Lpack{2up} can be used for more than booklets
% \item \Lpack{booklet} adds the notion of signatures and reduces the need for |\box|es.
% \item \Lpack{booklet} adds means of automatically controlling 
%       portrait/landscape printing.
% \end{itemize}
%
% \subsection{Options}
%
% When the \Lopt{noprint} option is used the package is essentially vacuous.
% That is, it behaves as though it had not been used at all, with the
% exception that LaTeX will not hiccup at any of the package commands that
% may be used in the document; these commands will also do nothing.
%
% The \Lopt{print} option must be used if you want the package to work 
% as advertised. The reasons for the \Lopt{print} and \Lopt{noprint}
% options are explained later in Section~\ref{sec:usage}.
%
%    The \Lpack{booklet} package can also take one option from each of
% two groups.
%
%    The first group of options specifies how many pages there should be
% to a hypothetical signature. The options are: \Lopt{four}, \Lopt{eight},
% \Lopt{sixteen}, and \Lopt{thirtytwo}. TeX has to hold all the pages
% for a signature before it outputs any of them, and TeX is memory limited.
%
%     If you have a duplex printer you can print out the document double-sided
% and then just fold the sheets forming each signature ready for binding.
% If you don't have a duplex printer, then the sheets can be photocopied
% (1-to-2 sided) ready for folding.
%
%    The second group of options are for selecting one of the package's
% predefined layouts. These options include:
% \begin{itemize}
% \item[\Lopt{twouparticle}] --- for portrait pages with wide margins
%     (similar to those in the standard LaTeX \Lpack{article}, \Lpack{book}
%      and \Lpack{report} classes)
%     and no marginal notes, to go on landscape sheets.
% \item[\Lopt{largetypeblock}] --- for portrait pages with a larger
%      (wider and/or longer) typeblock 
%   and/or marginal notes, to go on landscape sheets.
% \item[\Lopt{1to1}] --- for portrait pages where the original is 
%      designed to fit on half-sized paper, to go on landscape sheets.
% \item[\Lopt{landscape}] --- for landscape pages to go onto
%      portrait sheets. Pairs of original landscape pages will be
%      printed at the top and bottom of a portrait page.
%      This option is called automatically if the
%      \Lopt{landscape} option is used for the document class.
%
% \end{itemize}
%
%    The default options are \Lopt{noprint}, \Lopt{twouparticle} and 
% \Lopt{thirtytwo}.
%
% \subsection{Commands}
%
% The \emph{source} is the document as it would
% be normally typeset by LaTeX (i.e., without using the \Lpack{booklet}
% package).
%
% The \emph{target} is the desired document as it should 
% be printed
% using the \Lpack{booklet} package.
%
% \DescribeMacro{\source}
% The command |\source{|\meta{mag}|}{|\meta{width}|}{|\meta{height}|}|
% specifies the source sheets, where
% \meta{width} and \meta{height} are the width and height for the
% pages of the original document (the document that LaTeX would process
% without the \Lpack{booklet} package). The \meta{mag} argument is the
% magnification factor. 
%
%    The default definition of |\source| is: \\
% |\source{\magstep0}{\paperwidth}{\paperheight}| \\
% which corresponds to a full size default page.
%
% \DescribeMacro{\target}
% The command |\target{|\meta{mag}|}{|\meta{width}|}{|\meta{height}|}|
% specifies the target sheets, where
% \meta{width} and \meta{height} are the width and height for the
% sheets that will be used for printing with the package. For
% example, |\target{...}{11in}{8.5in}| would be typical for a 
% (portrait letterpaper)
% source being put onto a (landscape) letterpaper target.
%
% \DescribeMacro{\magstepN}
% \DescribeMacro{\magstepminusN}
% The package puts two pages onto a single sheet. It often happens that
% in order to do this the natural size of the text pages has to be changed,
% via the \meta{mag} argument noted above. The macro |\magstepminusN|,
% where $0 \leq N \leq 5$ is a shrinkage factor; similarly, |\magstepN|
% specifies a magnification factor. For example, the default |\target|
% is specified as: \\
% |\target{\magstepminus1}{\paperheight}{\paperwidth}| \\
% whereas the target for the \Lopt{largetypeblock} option is: \\
% |\target{\magstepminus2}{\paperheight}{\paperwidth}| \\
% which shrinks the original pages more than the default |\target| does.
%
%    The \Lpack{booklet} package uses the |\source| and |\target| 
% specifications to decide on how to scale and position the original
% full size source pages onto half of a target sheet.
%
%    Normally there is no need to change the default |\source| but it
% may be useful sometimes to change the |\target|. For example, if the
% text on source letterpages is very wide it could be useful to print
% on legal paper instead. In this case |\target| can be defined as: \\
% |\target{\magstepminus2}{14in}{8.5in}|
%
% \DescribeMacro{\pagespersignature}
% If none of the signature options are suitable, then 
% |\pagespersignature{|\meta{num}|}| may be used to specify \meta{num}
% pages per signature. For example, if you wanted a \emph{sexto} signature
% then use |\pagespersignature{12}|.
%
% \DescribeMacro{\ifprintoption}
% The command |\ifprintoption| is set TRUE if the \Lopt{print} option 
% is used otherwise it is set FALSE.
%
% \DescribeMacro{\setdvipstargetpages}
% If you are using LaTeX to generate a \file{.dvi} file the
% |\setdvipstargetpages| macro sets up the correct target page sizes 
% in the \file{.dvi} file assuming that subsequent processing will be
% via a program that understands \file{dvips} \cs{special} commands.
% 
% \DescribeMacro{\setpdftargetpages}
% If you are using pdfLaTeX to generate a \file{.pdf} file the
% |\setpdftargetpages| macro sets up the correct target page sizes for
% pdfLaTeX.
%
% \DescribeMacro{\pagesepwidth}
% \DescribeMacro{\pageseplength}
% \DescribeMacro{\pagesepoffset}
%  A line may be drawn between the output text pages. The length of
% the line is |\pageseplength| (default 6.5in), and line starts a
% distance |\pagesepoffset| from the bottom (side) of the printed sheet
% (default 1in). The line thickness is specified by the length
% |\pagesepwidth|, which is initially defined to be 0pt (i.e., the
% line is normally invisible). To see a dividing line use |\setlength|
% to change the value of |\pagesepwidth|; in LaTeX ruled lines normally
% have a thickness of 0.4pt.
%
% \DescribeMacro{\twoupclearpage}
% \DescribeMacro{\twoupemptypage}
% The command |\twoupclearpage| outputs the current target page (like
% LaTeX's |\newpage|). The command |\twoupemptypage| inserts a completely
% empty page into the target.
%
% \DescribeMacro{\ifuselandscape}
% \DescribeMacro{\ifprintoption}
% The command |\ifuselandscape| is set TRUE if the \Lopt{landscape} option 
% is used otherwise it is set FALSE.
%
% \DescribeMacro{\checkforlandscape}
% The macro |\checkforlandscape| evaluates the value of |\ifprintoption|
% and the current value of |\ifuselandscape| and sets |\ifuselandscape| 
% to TRUE if the document
% requires landscape printing (when either both the \Lopt{landscape} and 
% \Lopt{noprint} options are used, or when just the \Lopt{print}
% option is called for), 
% or to FALSE if the document should be printed in the normal portrait 
% orientation (when either both the \Lopt{landscape} and \Lopt{print}
% options are used, or when just the \Lopt{noprint} option is used without
% \Lopt{landscape}). 
%
% Note that as |\checkforlandscape| may change the value of
% |\ifuselandscape|, the command is only guaranteed when |\ifuselandscape|
% accurately indicates the presence or absence of the \Lopt{landscape}
% option.
%
% In version 0.7 of the package the macro \cs{setdvipstargetpages} was
% introduced and the macro \cs{setpdftargetpages} was modified with the
% result that \cs{checkforlandscape} was effectively
% no longer needed. It is left to maintain upward compatibility for
% older documents.
%
%
% \subsection{Usage} \label{sec:usage}
%
%     You use the \Lpack{booklet} package just like any other package: \\
% |\usepackage[<options>]{booklet}| \\
% but some other things need taking care of as well.
%
%    If you are making any changes to the values of |\paperwidth| or
% |\paperheight|, say through the \Lpack{geometry} package or the
% \Lpack{memoir} class, then the \Lpack{booklet} package must be called
% \emph{after} those changes have been made. In this case you may have
% to specify a different |\source| than usual, but I think that this
% is probably unlikely.
%
%
% \DescribeMacro{\special}
%    Under normal circumstances printing should be done onto landscape
% oriented paper instead of the usual portrait orientation. If you are
% using the \file{dvips} program for transformation from the \file{.dvi}
% to a \file{.ps} file, then you can use the command |\special{landscape}|
% in the preamble so that the output is organised for landscape. Other
% drivers may support similar commands, or command line options for
% accomplishing this.
%
%    For example, it may happen that if you use \file{ghostview} (or
% \file{gsview}) to look at \file{dvips} PostScript output of the final
% printing that the text appears upside down. Putting: \\
% \verb?\special{!TeXDict begin /landplus90{true}store end}? \\
% just before the \verb?\begin{document}? may cure this.
%
%    I use a LaserJet printer with a duplex printing capability (i.e.,
% printing can be done on both sides of a sheet in one pass). When
% printing in duplex mode I also found that it was very convenient
% to change the binding from the long to the short side of the paper;
% the sheets were then all set for folding. Using \file{dvips} and
% the 5SiMx LaserJet I found that the following commands in the
% preamble let me change the printer settings from the document.
% \begin{verbatim}
% \special{!TeXDict begin <</Duplex true>> setpagedevice end} % duplex
% \special{!TeXDict begin <</Tumble true>> setpagedevice end} % short binding
% \end{verbatim}
% How these would fare with another printer or program, I have no idea.
% The CTAN subdirectory \texttt{/dviware/dvipsconfig} contains a
% set of files for performing the above operations, and more, 
% as command line options for \file{dvips}.
%
%    If you can only do simplex printing (one side only), then when
% photocopying from one-sided to two-sided, alternate printed pages
% need to be rotated 180 degrees before being fed into the copier.
%
%    The following physical analogy of how the package works with
% portrait pages may help in achieving results that you want.
%
% \begin{itemize}
% \item Write the original text on portrait oriented transparent 
%       material (e.g., viewfoils) which is the same size as that 
%       specified by the |\source| macro. Mark the centre --- the 
%       point where the diagonals intersect --- of each original page.
%
% \item Take a sheet of paper the same size as that specified by
%       the |\target| macro and in landscape orientation. Mentally
%       divide this in half by a vertical line and then mark the centre
%       points of the two (left and right) halves.
% \item Take an even numbered original sheet, portrait orientation,
%       and place it on top of the landscape oriented target sheet
%       with its centre point coincident with the left half centre
%       point. Do the same with an odd numbered original sheet except
%       that its centre is coincident with the right half centre point.
% \item Take the assembly to a photocopier and make a copy onto a target
%       sized sheet. This is the final result.
% \end{itemize}
% The analogy for landscape originals is similar with the obvious changes
% in the orientations.
%
%    Not that the analogy does not include any scaling. If the size of 
% the target is relatively small then the texts on the original sheets
% may overlap or extend outside the physical target sheet. If the target
% is relatively large then there may be too much white space around the
% texts. The \meta{mag} arguments to the |\source| and |\target| macros
% are provided to adjust the source/target text sizes. Scaling does
% not affect the positions of the texts.
%
%    If you are planning to produce a booklet in the first place, then
% design the initial page layout as though it would be printed on half-sized
% paper. This will produce a better end result as no scaling will
% be necessary. The \Lopt{1to1} option is provided for just this case.
%
%   If the original pages look too crowded after printing, you can specify
% another |\target| with a larger shrink factor (e.g., |\magstepminus3|
% instead of |\magstepminus2|). Conversely, too much white space can be
% countered by decreasing the shrink factor (e.g., |\magstepminus1|
% instead of |\magstepminus2|).
%
% The following are the default settings of |\source| and |\target| (the
% source and target sheets are the same size) for the different layout 
% options.
% \begin{itemize}
% \item[\Lopt{twouparticle:}] \mbox{} \\
%    |\source{\magstep0}{\paperwidth}{\paperheight}| \\
%    |\target{\magstepminus1}{\paperheight}{\paperwidth}| \\
% \item[\Lopt{largetypeblock:}] \mbox{} \\
%    |\source{\magstep0}{\paperwidth}{\paperheight}| \\
%    |\target{\magstepminus2}{\paperheight}{\paperwidth}| \\
% \item[\Lopt{1to1:}] \mbox{} \\
%    |\source{\magstep0}{\paperwidth}{\paperheight}| \\
%    |\target{\magstep0}{\paperheight}{\paperwidth}| \\
% \end{itemize}
%
%    If the original is not planned to be produced as a booklet, then
% obtaining respectable looking full size and booklet results may
% require some trial and error in determining good values for 
% magnifications and/or target paper sizes. The original page margins
% may also need changing for the booklet output.
%
%    LaTeX is not very happy if it is forced to save some pages instead
% of outputting each one as it is finished. It shows its displeasure
% by messing up any internal references in the document, for example
% a Table of Contents or an Index. To counter this, run LaTeX on the
% document \emph{without} the \Lopt{print} package option
% until all references
% have stabilised. Then put the command |\nofiles| in the preamble, use
% the \Lopt{print} option, 
% and run LaTeX a final time before printing.
% (The |\nofiles| command stops LaTeX from generating new \file{.aux},
% \file{.toc} files, etc., but it will still use any old versions).
%
% The following gives an indication of how the relevant portion of
% a preamble may look.
% \begin{verbatim}
% % possibly change default \paperwidth and/or \paperheight
% % \usepackage[noprint,...]{booklet}         % use this 
% % \usepackage[...]{booklet}                 % or this for initial runs
% % \usepackage[print, ...]{booklet} \nofiles % use this for final run
% % perhaps change \target
% \ifpdf                         % from the ifpdf package
%   \pdfoutput = 1               % generate pdf output
%   \setpdftargetpages           % set output page size
% \else
%   \setdvipstargetpages         % use this for dvi output
% \fi
% ...
% \end{verbatim}
% It is safer to use the package without the \Lopt{print} option for
% the initial runs rather than not using the package at all; doing it
% this way will stop LaTeX from complaining if you happened to have
% used any of the \Lpack{booklet} commands in the document.
%
%    TeX allocates a new box for each page in the first signature and then these
% are reused for each subsequent signature. Large
% signatures may use up all available boxes in which case
% TeX will tell you. It is much more probable, though, that TeX will
% run out of memory before this happens. 
%
%     Here is a real example of one booklet setup for typesetting a 68 page
% book. I used the \Lpack{memoir}
% class but afterwards I'll point out how to do the equivalent with the
% \Lpack{book} class.
% \begin{verbatim}
% \documentclass[letterpaper,11pt]{memoir}
% % typeblock size of 5.5+ by 4 inches
% \settypeblocksize{5.5in}{4in}{*}
% \addtolength{\textheight}{\onelineskip}
% \setlrmargins{2in}{*}{*}
% \setulmargins{2.25in}{*}{*}
% \checkandfixthelayout
% 
% % \usepackage[noprint]{booklet}
% \usepackage[print,1to1]{booklet} \nofiles 
% \pagespersignature{16} % 16 pages per signature
% 
% \ifpdf
%   \setpdftargetpages
% \else
%   \setdvipstargetpages
% \fi
% \ifprintoption  % tweak dvi output only for final printing
%   \special{!TeXDict begin /landplus90{true}store end}
%   \special{!TeXDict begin <</Tumble true>> setpagedevice end}
% \fi
%
% \begin{document}
% \pagestyle{empty}
% % Want a blank sheet before the title page
% \hbox{}\cleardoublepage
% % half-title page here
% \cleardoublepage
% 
% % title page here
% \clearpage
% 
% % copyright page here
% \cleardoublepage
% 
% \pagestyle{plain}
% \pagenumbering{roman}
% \tableofcontents*
% \cleardoublepage
% \pagenumbering{arabic}
% \pagestyle{headings}
% 
% \chapter{First}
% % and so on
% 
% % want some blank endpapers to get enough pages into
% % the last signature for easy binding
% \clearpage
% \hbox{}\clearpage\hbox{}\cleardoublepage
% \end{document}
% \end{verbatim}
%
% Using the \Lpack{book} class just involves changing how the page layout
% is specified:
% \begin{verbatim}
% \documentclass[letterpaper,11pt]{book}
% % typeblock size of 5.5 by 4 inches
% \setlength{\textheight}{419pt} \setlength{\textwidth}{289pt}
% \setlength{\oddsidemargin}{72pt} \setlength{\evensidemargin}{108pt} 
% \setlength{\topmargin}{55.9pt} \setlength{\footskip}{27.5pt} 
% \setlength{\headheight}{14.6pt} \setlength{\headsep}{19.9pt} 
%
% % \usepackage[noprint]{booklet}
% ...
% \end{verbatim}
%
%    If you try and use a non-zero magnification with a document that
% has any `true' lengths, for example the \Lpack{amsbook} class, 
% TeX itself will complain with an error message
% of the form: \\
% |!Incompatible magnification (833);| \\
% |the previous value will be retained|\\
% Try forcing LaTeX to continue and check the output because you might
% be lucky and it looks alright.
%
% \changes{v0.7a}{2005/06/01}{Added Gordo's example}
%    Here is another example from a posting to \texttt{comp.text.tex} by
% Gordo in the thread \textit{booklet problem}, 2005/05/26.
% \begin{verbatim}
% \documentclass[twoside,10pt]{article}
% \setlength{\paperwidth}{5.5in}
% \setlength{\paperheight}{8.5in}
% \setlength{\topmargin}{-0.3in}
% \setlength{\textheight}{6.9in}
% \setlength{\oddsidemargin}{0.0in}
% \setlength{\evensidemargin}{-0.5in}
% \setlength{\textwidth}{4.0in}
% \usepackage{makeidx}
% %\usepackage[noprint,1to1]{booklet} % initial runs for ToC and index
% \usepackage[noprint,1to1]{booklet} \nofiles % final printing run
% \source{\magstep0}{5.5in}{8.5in}
% \target{\magstep0}{11in}{8.5in}
% \setpdftargetpages
% \pagespersignature{120}
% \makeindex
% \begin{document}
% \pagenumbering{roman}
% \begin{titlepage} ... \end{titlepage} % the title
% ...
% \newpage
% \begin{titlepage} \mbox{} \end{titlepage} % back cover
% \end{document}     
% \end{verbatim}
%
%
% \changes{v0.7a}{2005/06/01}{Added example for bookletting the manual}
% As yet another example, you can produce this manual as a booklet. Change
% the first part of this file to read:
% \begin{verbatim}
% \documentclass[twoside]{ltxdoc}
% \usepackage{url}
% \usepackage{ifpdf}
% \addtolength{\evensidemargin}{-0.75in}
% % \usepackage[noprint]{booklet} % initial runs
% \usepackage[print,largetypeblock]{booklet}\nofiles\pagespersignature{48} % final run
% %\usepackage[draft=false,
% %            plainpages=false,
% %            pdfpagelabels,
% %            bookmarksnumbered,
% %            hyperindex=false
% %           ]{hyperref}
% \ifpdf
%   \setpdftargetpages
% \else
%   \setdvipstargetpages
% \fi
% \providecommand{\phantomsection}{}
% \makeatletter
%   \@mparswitchfalse
% \makeatother
% ...
% \end{verbatim}
% The \Lpack{hyperref} package interferes with the \cs{setpdftargetpages}
% macro so it is commented out. The \Lopt{largetypeblock} option is used
% otherwise the marginal notes can fall off the edges of the pages in the
% final printing; the \cs{evensidemargin} is also altered for the same reason.
% The value for \cs{pagespersignature} is to ensure that
% only a single signature is produced (you can decrease this if you are
% willing to bind multiple signatures).
%
%
%    The original source of the \Lpack{2up} package is included at
% the end of this manual. It may provide some more insights into how to
% use the \Lpack{booklet} package. Be careful, though, as the two packages
% differ in some significant details.
%
%
%
% \StopEventually{
% \bibliographystyle{alpha}
% \begin{thebibliography}{GMS94}
% \bibitem[GMS94]{GOOSSENS94}
% Michel Goossens, Frank Mittelbach, and Alexander Samarin.
% \newblock \emph{The LaTeX Companion}.
% \newblock Addison-Wesley Publishing Company, 1994.
% \bibitem[Zan93]{2UP}
% Timothy Van Zandt.
% \newblock \emph{Two-up printing for Generic TeX}.
% \newblock January 1993.
% \newblock (Available from CTAN in \texttt{macros/generic/2up})
% \end{thebibliography}
% }
%
%
% \section{The package code} \label{sec:code}
%
% The package code essentially consists of modifications to the \Lpack{2up}
% macros, which are given in section~\ref{sec:2upcode}. I could not
% have written the \Lpack{booklet} package without Timothy Van Zandt's
% \Lpack{2up} macros.
%
%    As much as possible I have tried to keep to the original
% \Lpack{2up} macro names. However, the \Lpack{2up} and \Lpack{booklet}
% packages cannot be used in the same document.
%
%    The code is actually in two packages, \file{booklet.sty} which 
% provides the commands that 
% the user sees, and \file{bkltprnt.sty} which is used by 
% \file{booklet.sty} to do all the hard work.
%
% \subsection{Preliminaries}
%
%    Announce the name and version of the packages, which require
% \LaTeXe.
%    \begin{macrocode}
%<*usc>
\NeedsTeXFormat{LaTeX2e}
\ProvidesPackage{booklet}[2009/09/02 v0.7b booklet printing]

%</usc>
%<*prnt>
\NeedsTeXFormat{LaTeX2e}
\ProvidesPackage{bkltprnt}[2009/09/02 v0.7b support for booklet printing]

%</prnt>
%    \end{macrocode}
%
% \subsection{The main \Lpack{booklet} code}
%
%    The \Lpack{booklet} package provides the user commands. Then, if
% the \Lopt{print} option is used it calls the \Lpack{bkltprnt}
% package to do the work.
%
%    \begin{macrocode}
%<*usc>
%    \end{macrocode}
%
% \begin{macro}{\twoup@end}
%  For the \Lopt{print} and \Lopt{noprint} options.
%    \begin{macrocode}
\newcommand*{\twoup@end}{\endinput}

%    \end{macrocode}
% \end{macro}
%
% \begin{macro}{\pageseplength}
% \begin{macro}{\pagesepwidth}
% \begin{macro}{\pagesepoffset}
% Lengths for the dividing rule.
%    \begin{macrocode}
\newdimen\pageseplength
\newdimen\pagesepwidth
\newdimen\pagesepoffset

%    \end{macrocode}
% \end{macro}
% \end{macro}
% \end{macro}
%
% \begin{macro}{\ifsidebyside}
% TRUE if pages are to be placed side-by-side on the sheet. I have changed
% the original |\if@sidebyside| to |\ifsidebyside| so that a user can
% call it easily. I don't
% don't implement the |\if@twosided|.
%    \begin{macrocode}
\newif\ifsidebyside
  \sidebysidetrue

%    \end{macrocode}
% \end{macro}
%
% \begin{macro}{\ifuselandscape}
% \begin{macro}{\ifprintoption}
% Booleans for some options and for if the final printing should be landscape.
%    \begin{macrocode}
\newif\ifuselandscape
  \uselandscapefalse
\newif\ifprintoption
  \printoptionfalse

%    \end{macrocode}
% \end{macro}
% \end{macro}
%
% \begin{macro}{\c@signature}
% \begin{macro}{\c@sigcount}
% Counters for signature handling.
%    \begin{macrocode}
\newcounter{signature}
\newcounter{sigcount}
  \setcounter{sigcount}{0}
%    \end{macrocode}
% \end{macro}
% \end{macro}
%
% \begin{macro}{\pagespersignature}
% A user-level command for setting the number of pages wanted
% in a signature. I initialise these to 32.
%    \begin{macrocode}
\newcommand*{\pagespersignature}[1]{\setcounter{signature}{#1}}
  \pagespersignature{32}

%    \end{macrocode}
% \end{macro}
%
%
% \begin{macro}{\magstepminus}
% A useful extension of the |\magstep| macro.
%    \begin{macrocode}
\def\magstepminus#1{%
  \ifcase#1 \@m\or 833\or 694\or 579\or 482\or 401\fi\relax}

%    \end{macrocode}
% \end{macro}
%
% \begin{macro}{\target}
% \begin{macro}{\source}
% \begin{macro}{\setpdftargetpages}
% \begin{macro}{\setdvipstargetpages}
% Vacuous versions of the user-level macros.
% \changes{v0.7}{3005/03/23}{Added \cs{setdvipstargetpages}}
%    \begin{macrocode}
\newcommand*{\target}[3]{}
\newcommand*{\source}[3]{}
\newcommand*{\setpdftargetpages}{}
\newcommand*{\setdvipstargetpages}{}

%    \end{macrocode}
% \end{macro}
% \end{macro}
% \end{macro}
% \end{macro}
%
% \begin{macro}{\targettopbottom}
% Vacuous version of the user command.
%    \begin{macrocode}
\newcommand*{\targettopbottom}{}

%    \end{macrocode}
% \end{macro}
%
% \begin{macro}{\twoupemptypage}
% \begin{macro}{\twoupclearpage}
% vacuous versions of the user commands.
%    \begin{macrocode}
\newcommand*{\twoupemptypage}{}
\newcommand*{\twoupclearpage}{}

%    \end{macrocode}
% \end{macro}
% \end{macro}
%
% \begin{macro}{\checkforlandscape}
% Sets the value of |\ifuselandscape| to 
% (|\ifuselandscape| XOR |\ifprintoption|).
%    \begin{macrocode}
\newcommand*{\checkforlandscape}{%
  \ifx\ifuselandscape\ifprintoption
    \uselandscapefalse\else\uselandscapetrue\fi}

%    \end{macrocode}
% \end{macro}
%
%    Now declare the \Lopt{print}, \Lopt{noprint} and \Lopt{landscape}
% options, 
% prepare to pass any other
% options to \Lpack{bkltprnt}, and execute the options.
%    \begin{macrocode}
\DeclareOption{print}{\printoptiontrue\def\twoup@end{}}
\DeclareOption{noprint}{\printoptionfalse\def\twoup@end{\endinput}}
\DeclareOption{landscape}{\uselandscapetrue}
\DeclareOption*{\PassOptionsToPackage{\CurrentOption}{bkltprnt}}
\ProcessOptions\relax
\ifuselandscape
  \PassOptionsToPackage{landscape}{bkltprnt}
\fi

%    \end{macrocode}
%
%  Now call |\twoup@end|, which is either |\endinput| or vacuous, after
% which the \Lpack{bkltprnt} package is called (if |\twoup@end| was vacuous).
%    \begin{macrocode}
\twoup@end

\RequirePackage{bkltprnt}

%    \end{macrocode}
%
%
%    The end of this package.
%
%    \begin{macrocode}
%</usc>
%    \end{macrocode}
%
% \subsection{The main \Lpack{bkltprnt} code}
%
%    \begin{macrocode}
%<*prnt>
%    \end{macrocode}
%
% \begin{macro}{\@targetwidth}
% \begin{macro}{\@targetheight}
% \begin{macro}{\@sourcewidth}
% \begin{macro}{\@sourceheight}
% Parameter registers.
%    \begin{macrocode}
\newdimen\@targetwidth
\newdimen\@targetheight
\newdimen\@sourcewidth
\newdimen\@sourceheight

%    \end{macrocode}
% \end{macro}
% \end{macro}
% \end{macro}
% \end{macro}
%
% \begin{macro}{\if@leftpage}
% \begin{macro}{\@leftpage}
% \begin{macro}{\@rightpage}
% \begin{macro}{\@physicalpage}
% Registers used by output routine.
%    \begin{macrocode}
\newif\if@leftpage
  \@leftpagetrue
\newbox\@leftpage
\newbox\@rightpage
\newcount\@physicalpage

%    \end{macrocode}
% \end{macro}
% \end{macro}
% \end{macro}
% \end{macro}
%
%
% Since pages are both stored and shipped out half as often:
%    \begin{macrocode}
\multiply\maxdeadcycles by 40\relax

%    \end{macrocode}
%
% \begin{macro}{\bookletpage}
% \begin{macro}{\leftpagenumber}
% \begin{macro}{\rightpagenumber}
% Registers used only for booklet layout. These are specified directly
% as this is the only layout.
%    \begin{macrocode}
\newcount\bookletpage
  \bookletpage=0
\newcount\leftpagenumber
\newcount\rightpagenumber

%    \end{macrocode}
% \end{macro}
% \end{macro}
% \end{macro}
%
% \begin{macro}{\twoupp@geboxes}
% I have added this macro so that only the boxes required for a signature
% are created. The original \Lpack{2up} code creates a new box for each output page.
% \changes{v0.6}{2001/08/26}{Added \cs{twoupp@geboxes}}
%    \begin{macrocode}
\newcommand*{\twoupp@geboxes}{%
  \advance\bookletpage\@ne
  \expandafter\newbox\csname bookletbox\the\bookletpage\endcsname
  \ifnum \bookletpage < \c@signature
    \twoupp@geboxes
  \fi}
\AtBeginDocument{\twoupp@geboxes\bookletpage=\z@}

%    \end{macrocode}
% \end{macro}
%
%
% \begin{macro}{\target}
% \begin{macro}{\inv@targetmag}
% |\@targetwidth| and |\@targetheight| are set to the \emph{unmagnified} 
% dimensions
% of the target page. |\inv@targetmag| is the inverse of the target
% magnification.
%    \begin{macrocode}
{\catcode`\p=12\catcode`\t=12\gdef\@@inv@@mag#1pt#2{\def#2{#1}}}
\def\target#1#2#3{%
  \mag #1\relax
  \@targetwidth=1000pt
  \divide\@targetwidth by #1\relax
  \expandafter\@@inv@@mag\the\@targetwidth\inv@targetmag
  \@targetwidth=#2\relax
  \@targetwidth=\inv@targetmag\@targetwidth
  \@targetheight=#3\relax
  \@targetheight=\inv@targetmag\@targetheight}

%    \end{macrocode}
% \end{macro}
% \end{macro}
%
% \begin{macro}{\source}
% Like |\target|, but for the source:
%    \begin{macrocode}
\def\source#1#2#3{%
  \@sourcewidth=1000pt
  \divide\@sourcewidth by #1\relax
  \expandafter\@@inv@@mag\the\@sourcewidth\inv@sourcemag
  \@sourcewidth=#2\relax
  \@sourcewidth=\inv@sourcemag\@sourcewidth
  \@sourceheight=#3\relax
  \@sourceheight=\inv@sourcemag\@sourceheight}

%    \end{macrocode}
% \end{macro}
%
% \begin{macro}{\setpdftargetpages}
% This macro specifies the size of the target page for pdfLaTeX.
% It ensures that \Lpack{memoir}'s version (\cs{fixpdflayout}) does nothing.
% \changes{v0.7}{2005/03/23}{Modified \cs{setpdftargetpages} to cater for memoir}
%    \begin{macrocode}
\renewcommand*{\setpdftargetpages}{%
  \ifprintoption
    \setlength{\pdfpageheight}{\@targetheight}%
    \setlength{\pdfpagewidth}{\@targetwidth}%
    \let\fixpdflayout\relax
  \fi}
%    \end{macrocode}
% \end{macro}
%
% \begin{macro}{\setdvipstargetpages}
% This macro specifies the size of the target page for when dvips is
% used. It ensures that \Lpack{memoir}'s version (\cs{fixdvipslayout}) 
% does nothing.
% \changes{v0.7}{2005/03/23}{Implemented \cs{setdvipstargetpages}}
%    \begin{macrocode}
\renewcommand*{\setdvipstargetpages}{%
  \ifprintoption
    \AtBeginDvi{\special{papersize=\the\@targetwidth,\the\@targetheight}}%
    \let\fixdvipslayout\relax
  \fi}

%    \end{macrocode}
% \end{macro}
%
% \begin{macro}{\targetBooklet}
% \begin{macro}{\ship@@@leftpage}
% \begin{macro}{\ship@@@rightpage}
% I only provide the original \Lpack{2up} Booklet, not booklet. For
% the \Lpack{booklet} package the Booklet code can be processed
% immediately. The code in this chunk is equivalent to the original
% |\target@Booklet|.
%    \begin{macrocode}
\def\targetBooklet{%
  \def\ship@@@leftpage{\save@booklet\@leftpage}%
  \def\ship@@@rightpage{\save@booklet\@rightpage}%
  \@leftpagefalse}
\targetBooklet

%    \end{macrocode}
% \end{macro}
% \end{macro}
% \end{macro}
%
% \begin{macro}{\targettopbottom}
% \begin{macro}{\make@@halfpage}
% \begin{macro}{\make@fullpage}
% I have renamed the original |\target@topbottom| to |\targettopbottom|
%    \begin{macrocode}
\def\targettopbottom{%
  \def\make@@halfpage{\make@@halftopbottom}%
  \def\make@fullpage{\make@fulltopbottom}%
  \sidebysidefalse}

%    \end{macrocode}
% \end{macro}
% \end{macro}
% \end{macro}
%
% Specify the options. There are new ones for signature sizes, and some
% of the original \Lpack{2up} source/targets are now treated as options.
%    \begin{macrocode}
\DeclareOption{four}{\setcounter{signature}{4}}
\DeclareOption{eight}{\setcounter{signature}{8}}
\DeclareOption{sixteen}{\setcounter{signature}{16}}
\DeclareOption{thirtytwo}{\setcounter{signature}{32}}
\DeclareOption{twouparticle}{\twouparticle}
\DeclareOption{landscape}{\twouplandscape}
\DeclareOption{largetypeblock}{\twoupplain}
\DeclareOption{1to1}{\twouponetoone}

%    \end{macrocode}
%
% \begin{macro}{\shipout}
% TeX's |\shipout| primitive is saved as |\&normal@shipout|, 
% and then |\shipout|
% is defined to save each page to |\@leftpage| or |\@rightpage| 
% and to print out every two. There is no two-sided layout in this version.
%    \begin{macrocode}
\expandafter\let\csname &normal@shipout\endcsname\shipout
\def\shipout{%
  \if@leftpage
    \global\@leftpagefalse
    \def\next{\afterassignment\ship@leftpage\global\setbox\@leftpage=}%
  \else
    \global\@leftpagetrue
    \def\next{\afterassignment\ship@rightpage\global\setbox\@rightpage=}%
  \fi
  \next}

%    \end{macrocode}
% \end{macro}
%
% \begin{macro}{\ship@leftpage}
% \begin{macro}{\ship@rightpage}
% The job of |\ship@leftpage| and |\ship@rightpage| is to invoke 
% |\ship@@leftpage| or |\ship@@rightpage| at the right time. 
% |\shipout| is followed either:
% \begin{enumerate}
% \item by an |\hbox|, |\vbox| or |\vtop|, in which case 
% |\ship@leftpage| is invoked after the opening |{|; 
% |\@leftpage| is void, and |\ship@leftpage| invokes
% |\ship@@leftpage| after the closing |}|: or
%
% \item by a |\box| or |\copy|, in which case |\ship@leftpage| is invoked after
% the full assignment; |\@leftpage| is not void, and |\ship@leftpage| invokes
% |\ship@@leftpage| immediately.
% \end{enumerate}
%
%    \begin{macrocode}
\def\ship@leftpage{%
  \ifvoid\@leftpage\aftergroup\ship@@leftpage\else\ship@@leftpage\fi}
\def\ship@rightpage{%
  \ifvoid\@rightpage\aftergroup\ship@@rightpage\else\ship@@rightpage\fi}

%    \end{macrocode}
% \end{macro}
% \end{macro}
%
% \begin{macro}{\ship@@leftpage}
% \begin{macro}{\ship@@rightpage}
%     |\ship@@leftpage|/|\ship@@rightpage| take the output box, 
% and first make it
% into a fully-size source page (with |\make@halfpage|) and then this is
% centered horizontally and vertically in half of a target page (with
% |\make@@halfpage|). Then they are shipped individually or together.
%    \begin{macrocode}
\def\ship@@leftpage{\make@halfpage\@leftpage\ship@@@leftpage}
\def\ship@@rightpage{\make@halfpage\@rightpage\ship@@@rightpage}

%    \end{macrocode}
% \end{macro}
% \end{macro}
%
% \begin{macro}{\make@halfpage}
%    \begin{macrocode}
\def\make@halfpage#1{%
  \dp#1=\z@
  \setbox#1=\vbox to\@sourceheight{%
    \vskip \inv@sourcemag in
    \vskip \voffset
    \hbox to\@sourcewidth{\hskip\inv@sourcemag in\hskip\hoffset\box#1\hss}%
    \vss}%
  \make@@halfpage#1}

%    \end{macrocode}
% \end{macro}
%
% \begin{macro}{\make@@halfsidebyside}
% \begin{macro}{\make@@halfpage}
% The definition of |\make@@halfpage| depends on the target layout.
%    \begin{macrocode}
\def\make@@halfsidebyside#1{%
  \global\setbox#1=\vbox to\@targetheight{\vss
    \hbox to.5\@targetwidth{\hss\box#1\hss}\vss}}
\def\make@@halfpage{\make@@halfsidebyside}

%    \end{macrocode}
% \end{macro}
% \end{macro}
%
% \begin{macro}{\make@@halftopbottom}
%    \begin{macrocode}
\def\make@@halftopbottom#1{%
  \global\setbox#1=\vbox to.5\@targetheight{\vss
    \hbox to\@targetwidth{\hss\box#1\hss}\vss}}

%    \end{macrocode}
% \end{macro}
%
% \begin{macro}{\ship@twoup}
% The pages are generally shipped in pairs:
%    \begin{macrocode}
\def\ship@twoup{%
  \begingroup
    \voffset=-\inv@targetmag in
    \hoffset=\voffset
    \global\advance\@physicalpage by 1
    \count\z@=\@physicalpage
    \csname &normal@shipout\endcsname\make@fullpage
  \endgroup}

%    \end{macrocode}
% \end{macro}
%
% \begin{macro}{\make@fullsidebyside}
% \begin{macro}{\make@fulltopbottom}
%    \begin{macrocode}
\def\make@fullsidebyside{%
  \hbox{\box\@leftpage\pagesep@sidebyside\box\@rightpage}}
\def\make@fulltopbottom{%
  \vbox{\offinterlineskip\box\@leftpage\pagesep@topbottom\box\@rightpage}}

%    \end{macrocode}
% \end{macro}
% \end{macro}
%
% \begin{macro}{\make@fullpage}
% The definition of |\make@fullpage| depends on the layout:
%    \begin{macrocode}
\def\make@fullpage{\make@fullsidebyside}

%    \end{macrocode}
% \end{macro}
%
% \begin{macro}{\pagesep@sidebyside}
% \begin{macro}{\pagesep@topbottom}
% A vertical or horizontal rule can be inserted. These can be redefined
% for other tricks:
%    \begin{macrocode}
\def\pagesep@sidebyside{%
  \begingroup
    \advance\pageseplength by \pagesepoffset
    \pagesepwidth=\inv@targetmag\pagesepwidth
    \kern -.5\pagesepwidth
    \vrule height \inv@targetmag\pageseplength
           depth -\inv@targetmag\pagesepoffset
           width \pagesepwidth
    \kern -.5\pagesepwidth
  \endgroup}
\def\pagesep@topbottom{%
  \begingroup
    \pagesepwidth=\inv@targetmag\pagesepwidth
    \vskip -.5\pagesepwidth
    \moveright\inv@targetmag\pagesepoffset\hbox{%
      \vrule height\pagesepwidth width\inv@targetmag\pageseplength}%
    \vskip -.5\pagesepwidth
  \endgroup}

%    \end{macrocode}
% \end{macro}
% \end{macro}
%
% \changes{v0.7a}{2005/06/01}{Added sewing example}
%    For example it may be helpful to mark the positions of sewing holes
% along the spine of your booklet. For \Lopt{landscape} this can be done 
% by redefining \cs{pagesep@topbottom} and for the more typical portrait 
% style booklet by changing \cs{pagesep@sidebyside}. The general technique
% is to use a zero-sized picture which \LaTeX{} will think takes up no space.
% For instance, for letterpaper (11in by 8.5in) you can do something
% like this in your preamble:
% \begin{verbatim}
% \makeatletter
% \renewcommand*{\pagesep@sidebyside}{%
%   \begingroup
%   \setlength{\unitlength}{1in}%      measurements in inches
%   \begin{picture}(0,0)(0,-8.5)%      zero-sized picture, origin at page top
%     \put(0,-1){\makebox(0,0){.}}%    mark 1in down from top
%     \put(0,-4.25){\makebox(0,0){.}}% mark at page center
%     \put(0,-7.5){\makebox(0,0){.}}%   mark 1in up from bottom
%   \end{picture}%
%   \endgroup}
% \makeatother
% \end{verbatim}
%
% \begin{macro}{\save@booklet}
% With the Booklet layout, the pages are saved rather than
% shipped. I have added the signature code to the original macro. This
% outputs all the pages making up a signature even if there are still more
% to come. The page boxes used in |\save@booklet| are assigned initially
% by |\twoupp@geboxes|.
% \changes{v0.6}{2001/08/26}{Modified \cs{save@booklet}}
%    \begin{macrocode}
\begingroup
\let\newbox\relax
\gdef\save@booklet#1{%
  \begingroup
    \globaldefs=1
    \ifnum\c@sigcount=\z@\bookletpage=0\fi
    \advance\bookletpage by 1
    \addtocounter{sigcount}{1}
%%%%    \expandafter\newbox\csname bookletbox\the\bookletpage\endcsname
    \expandafter\setbox\csname bookletbox\the\bookletpage\endcsname\box#1%
  \endgroup
  \ifnum\c@sigcount=\c@signature     %%% PW
    \twoup@eject%           
    \setcounter{sigcount}{0}%
  \fi}
\endgroup

%    \end{macrocode}
% \end{macro}
%
% \begin{macro}{\make@bookletpage}
% \begin{macro}{\booklet@loop}
% \begin{macro}{\Booklet@@loop}
% The pages are then printed at the end with the following macros:
%    \begin{macrocode}
\def\make@bookletpage#1{%
  \setbox\ifodd#1\@rightpage\else\@leftpage\fi=%
    \expandafter\box\csname bookletbox\the#1\endcsname}

\def\booklet@loop{%
  \count\z@\rightpagenumber
  \make@bookletpage\leftpagenumber
  \make@bookletpage\rightpagenumber
  \ship@twoup
  \Booklet@@loop
}

\def\Booklet@@loop{%
  \advance\rightpagenumber by 1
  \advance\leftpagenumber by -1
  \ifnum\leftpagenumber<\rightpagenumber\else\expandafter\booklet@loop\fi}

%    \end{macrocode}
% \end{macro}
% \end{macro}
% \end{macro}
%
% \begin{macro}{\twoupemptypage}
% This one is easy:
%    \begin{macrocode}
\def\twoupemptypage{\shipout\hbox{}}

%    \end{macrocode}
% \end{macro}
%
%
% \begin{macro}{\twoup@eject}
% This clears a whole target page if there is a saved left page. Note that
% this does not invoke the output routine; i.e., it is not like |\clearpage|
% or |\supereject|. See |\twoupclearpage| below. (The \Lpack{booklet}
% package doesn't need |\twoupeject| which is for non-LaTeX systems).
%
% This is the definition of |\twoup@eject| with the Booklet layout.
%    \begin{macrocode}
\def\twoup@eject{
  \leftpagenumber\bookletpage
  \advance\leftpagenumber by 3
  \divide\leftpagenumber by 4
  \multiply\leftpagenumber by 4
  \rightpagenumber=1
  \ifnum\leftpagenumber>\bookletpage
    \setbox\@leftpage\hbox{}%
    \make@@halfpage\@leftpage
    \loop
      \setbox\@rightpage\copy\@leftpage
      \save@booklet\@rightpage
    \ifnum\leftpagenumber>\bookletpage
    \repeat
  \fi
  \booklet@loop}

%    \end{macrocode}
% \end{macro}
%
% \begin{macro}{\twoupclearpage}
% This modification is needed for LaTeX in order to get the last page
% printed out if the final page is a left page (the catcode business is
% because |\enddocument| is |\let| to |\bye| in amstex):
%    \begin{macrocode}
\begingroup
\catcode`\>=9\relax
>>\gdef\twoupclearpage{\clearpage\twoup@eject}
>>\expandafter\@temptokena\expandafter{\enddocument}
>>\xdef\enddocument{\noexpand\twoupclearpage\the\@temptokena}
\endgroup

%    \end{macrocode}
% \end{macro}
%
%
% \begin{macro}{\TwoupWrites}
% This is one workaround for the page cross-references problem
%    \begin{macrocode}
\def\TwoupWrites{%
  \let\TwoupSaved@write\write
  \let\TwoupSaved@read\read
  \let\TwoupSaved@openout\openout
  \let\TwoupSaved@closeout\closeout
  \def\write{\TwoupSaved@write-1{}\immediate\TwoupSaved@write}%
  \def\read{\TwoupSaved@write-1{}\immediate\TwoupSaved@read}%
  \def\openout{\TwoupSaved@write-1{}\immediate\TwoupSaved@openout}%
  \def\closeout{\TwoupSaved@write-1{}\immediate\TwoupSaved@closeout}%
  \let\TwoupWrites\relax}

%    \end{macrocode}
% \end{macro}
%
% \begin{macro}{\twouparticle}
% \begin{macro}{\twoupplain}
% \begin{macro}{\twouplandscape}
% \begin{macro}{\twouponetoone}
% The pre-defined layouts. I have taken advantage of LaTeX2e's
% paper size lengths to generalise.
%    \begin{macrocode}
\def\twouparticle{\target{\magstepminus1}{\paperheight}{\paperwidth}}
\def\twoupplain{\target{\magstepminus2}{\paperheight}{\paperwidth}}
\def\twouplandscape{\target{\magstepminus2}{\paperwidth}{\paperheight}%
  \targettopbottom}
\def\twouponetoone{\target{\magstep0}{\paperheight}{\paperwidth}}

%    \end{macrocode}
% \end{macro}
% \end{macro}
% \end{macro}
% \end{macro}
%
% Set the default for the source.
%    \begin{macrocode}
\source{\magstep0}{\paperwidth}{\paperheight}
%    \end{macrocode}
%
% Set the defaults for the rule.
%    \begin{macrocode}
\pagesepwidth 0pt
\pageseplength 6.5in
\pagesepoffset 1in
%    \end{macrocode}
%
% I also found that |\TwoupWrites| was really not optional.
%    \begin{macrocode}
\TwoupWrites

%    \end{macrocode}
%
% The default options are \Lopt{twouparticle} and \Lopt{thirtytwo} pages
% per signature.
%    \begin{macrocode}
\ExecuteOptions{twouparticle,thirtytwo}
\ProcessOptions

%    \end{macrocode}
%
%
%    The end of this package.
%    \begin{macrocode}
%</prnt>
%    \end{macrocode}
%
%
% \appendix
%
% \section{The code for 2up} \label{sec:2upcode}
%
%    Following is a verbatim copy of Timothy Van Zandt's \Lpack{2up} code.
% I hope that by including this I have partly met his COPYING conditions.
%
% \begin{verbatim}
% 
% %% BEGIN 2up.tex/2up.sty
% %%
% \def\fileversion{1.2}
% \def\filedate{93/01/28}
% %%
% %% COPYRIGHT 1992, 1993 by Timothy Van Zandt, tvz@Princeton.EDU
% %%
% %% DESCRIPTION:
% %%   2up.tex/2up.sty provides two-up printing for Generic TeX (e.g.,
% %%   Plain, LaTeX, AmSTeX and AmS-LaTeX). It produces a standard dvi file,
% %%   and does not involve an additional dvi or PostScript filter. It has a
% %%   flexible interface for specifying paper size and layout.
% %%
% %% INSTALLATION:
% %%   Put this file where your TeX looks for inputs, under the name 2up.tex.
% %%   Name a copy 2up.sty to use as a LaTeX style option, or create a file
% %%   2up.sty with the lines:
% %%     \input 2up.tex
% %%     \endinput
% %%
% %% DOCUMENTATION:
% %%   Input 2up.tex, or include 2up as a LaTeX style option. There is a
% %%   good chance you will get the desired layout. (But you will probably
% %%   need to generate new font bitmaps to get high quality output.) See
% %%   2up.doc, which might be appended to this file, for detailed
% %%   documentation.
% %%
% %% COPYING:
% %%   Copying of part or all of this file is allowed under the following
% %%   conditions only:
% %%   (1) You may freely distribute unchanged copies of the file. Please
% %%       include the documentation when you do so.
% %%   (2) You may modify a renamed copy of the file, but only for personal
% %%       use or use within an organization.
% %%   (3) You may copy fragments from the file, for personal use or for use
% %%       in a macro package for distribution, as long as credit is given
% %%       where credit is due.
% %%
% %%   You are NOT ALLOWED to take money for the distribution or use of
% %%   this file or modified versions or fragments thereof, except for
% %%   a nominal charge for copying etc.
% %%
% %% CODE:
% %
% \csname TwoUpLoaded\endcsname
% \let\TwoUpLoaded\endinput
% %
% \edef\TheAtCode{\the\catcode`\@}
% \catcode`\@=11\relax
% \message{\space\space v\fileversion\space\space \filedate\space\space <tvz>}
% %
% % Parameter registers:
% \newdimen\@targetwidth
% \newdimen\@targetheight
% \newdimen\@sourcewidth
% \newdimen\@sourceheight
% \newdimen\pageseplength
% \newdimen\pagesepwidth
% \newdimen\pagesepoffset
% \newif\if@sidebyside
% \@sidebysidetrue
% \newif\if@twosided
% %
% % Registers used by output routine.
% \newif\if@leftpage
% \@leftpagetrue
% \newbox\@leftpage
% \newbox\@rightpage
% \newcount\@physicalpage
% %
% % Since pages are shipped out half as often:
% \multiply\maxdeadcycles by 2
% %
% % Registers used only for booklet layout:
% \begingroup
%   \let\newcount\relax
%   \gdef\booklet@registers{%
%     \newcount\bookletpage
%     \bookletpage=0
%     \newcount\leftpagenumber
%     \newcount\rightpagenumber
%     \multiply\maxdeadcycles by 20}
% \endgroup
% %
% % A useful extension of the \magstep macro.
% \def\magstepminus#1{%
%   \ifcase#1 \@m\or 833\or 694\or 579\or 482\or 401\fi\relax}
% %
% % \@targetwidth and \@targetheight are set to the *unmagnified* dimensions
% % of the target page. \inv@targetmag is the inverse of the target
% % magnification.
% {\catcode`\p=12\catcode`\t=12\gdef\@@inv@@mag#1pt#2{\def#2{#1}}}
% \def\target#1#2#3{%
%   \mag #1\relax
%   \@targetwidth=1000pt
%   \divide\@targetwidth by #1\relax
%   \expandafter\@@inv@@mag\the\@targetwidth\inv@targetmag
%   \@targetwidth=#2\relax
%   \@targetwidth=\inv@targetmag\@targetwidth
%   \@targetheight=#3\relax
%   \@targetheight=\inv@targetmag\@targetheight}
% %
% % Like \target, but for the source:
% \def\source#1#2#3{%
%   \@sourcewidth=1000pt
%   \divide\@sourcewidth by #1\relax
%   \expandafter\@@inv@@mag\the\@sourcewidth\inv@sourcemag
%   \@sourcewidth=#2\relax
%   \@sourcewidth=\inv@sourcemag\@sourcewidth
%   \@sourceheight=#3\relax
%   \@sourceheight=\inv@sourcemag\@sourceheight}
% %
% % \targetlayout does a loop that reads the comma separated arguments.
% % There can be no extraneous spaces.
% \def\targetlayout#1{\process@targetlayout#1,stop,}
% \def\process@targetlayout#1,{%
%   \expandafter\let\expandafter\next\csname target@#1\endcsname
%   \ifx\next\relax
%     \begingroup
%       \errhelp{Valid target layouts are "topbottom", "twosided",
%         "booklet", "Booklet" and "dvidvi".}%
%       \errmessage{`#1' is invalid 2up target layout - ignored.}%
%     \endgroup
%     \expandafter\process@targetlayout
%   \else
%     \next
%   \fi}
% \def\target@stop{}
% \def\target@booklet{%
%   \booklet@registers
%   \def\ship@@@leftpage{\save@booklet\@leftpage}%
%   \def\ship@@@rightpage{\save@booklet\@rightpage}%
%   \@leftpagefalse
%   \def\twoup@eject{\twoup@eject@booklet}%
%   \expandafter\process@targetlayout}
% \def\target@Booklet{%
%   \def\booklet@@loop{\Booklet@@loop}%
%   \target@booklet}
% \def\target@twosided{%
%   \@twosidedtrue
%   \expandafter\process@targetlayout}
% \def\target@topbottom{%
%   \def\make@@halfpage{\make@@halftopbottom}%
%   \def\make@fullpage{\make@fulltopbottom}%
%   \@sidebysidefalse
%   \expandafter\process@targetlayout}
% \def\target@dvidvi{%
%   \def\ship@@@leftpage{\ship@dvidvi\@leftpage}%
%   \def\ship@@@rightpage{\ship@dvidvi\@rightpage}%
%   \expandafter\process@targetlayout}
% %
% % TeX's \shipout primitive is saved as \&normal@shipout, and then \shipout
% % is defined to save each page to \@leftpage or \@rightpage and to print out
% % every two. With the twosided layout, filler pages are added when needed.
% \expandafter\let\csname &normal@shipout\endcsname\shipout
% \def\shipout{%
%   \if@leftpage
%     \global\@leftpagefalse
%     \def\next{\afterassignment\ship@leftpage\global\setbox\@leftpage=}%
%     \if@twosided
%       \ifodd\count\z@
%         \global\setbox\@leftpage=\hbox{}%
%         \make@@halfpage\@leftpage\ship@@@leftpage
%         \def\next{\shipout}%
%       \fi
%     \fi
%   \else
%     \global\@leftpagetrue
%     \def\next{\afterassignment\ship@rightpage\global\setbox\@rightpage=}%
%     \if@twosided
%       \ifodd\count\z@
%       \else
%         \global\setbox\@rightpage=\hbox{}%
%         \make@@halfpage\@rightpage\ship@@@rightpage
%         \def\next{\shipout}%
%       \fi
%     \fi
%   \fi
%   \next}
% %
% % The job of \ship@leftpage and \ship@rightpage is to invoke \ship@@leftpage
% % or \ship@@rightpage at the right time. \shipout is followed either
% % (i) by an \hbox, \vbox or \vtop, in which case \ship@leftpage is invoked
% % after the opening {. \@leftpage is void, and \ship@leftpage invokes
% % \ship@@leftpage after the closing }, or
% % (ii) by a \box or \copy, in which case \ship@leftpage is invoked after
% % the full assignment. \@leftpage is not voide, and \ship@leftpage invokes
% % \ship@@leftpage immediately.
% \def\ship@leftpage{%
%   \ifvoid\@leftpage\aftergroup\ship@@leftpage\else\ship@@leftpage\fi}
% \def\ship@rightpage{%
%   \ifvoid\@rightpage\aftergroup\ship@@rightpage\else\ship@@rightpage\fi}
% %
% % \ship@@leftpage/\ship@@rightpage take the output box, and first make it
% % into a fully-size source page (with \make@halfpage) and then this is
% % centered horizontally and vertically in half of a target page (with
% % \make@@halfpage). Then they are shipped individually or together.
% \def\ship@@leftpage{\make@halfpage\@leftpage\ship@@@leftpage}
% \def\ship@@rightpage{\make@halfpage\@rightpage\ship@@@rightpage}
% \def\make@halfpage#1{%
%   \dp#1=\z@
%   \setbox#1=\vbox to\@sourceheight{%
%     \vskip \inv@sourcemag in
%     \vskip \voffset
%     \hbox to\@sourcewidth{\hskip\inv@sourcemag in\hskip\hoffset\box#1\hss}%
%     \vss}%
%   \make@@halfpage#1}
% %
% % The definition of \make@@halfpage depends on the target layout.
% \def\make@@halfsidebyside#1{%
%   \global\setbox#1=\vbox to\@targetheight{\vss
%     \hbox to.5\@targetwidth{\hss\box#1\hss}\vss}}
% \def\make@@halftopbottom#1{%
%   \global\setbox#1=\vbox to.5\@targetheight{\vss
%     \hbox to\@targetwidth{\hss\box#1\hss}\vss}}
% \def\make@@halfpage{\make@@halfsidebyside}
% %
% % The pages are generaly shipped in pairs:
% \def\ship@twoup{%
%   \begingroup
%     \voffset=-\inv@targetmag in
%     \hoffset=\voffset
%     \global\advance\@physicalpage by 1
%     \count\z@=\@physicalpage
%     \csname &normal@shipout\endcsname\make@fullpage
%   \endgroup}
% \let\ship@@@leftpage\relax
% \def\ship@@@rightpage{\ship@twoup}
% %
% % The definition of \make@fullpage depends on the layout:
% \def\make@fullsidebyside{%
%   \hbox{\box\@leftpage\pagesep@sidebyside\box\@rightpage}}
% \def\make@fulltopbottom{%
%   \vbox{\offinterlineskip\box\@leftpage\pagesep@topbottom\box\@rightpage}}
% \def\make@fullpage{\make@fullsidebyside}
% %
% % A vertical or horizontal rule can be inserted. These can be redefined
% % for other tricks:
% \def\pagesep@sidebyside{%
%   \begingroup
%     \advance\pageseplength by \pagesepoffset
%     \pagesepwidth=\inv@targetmag\pagesepwidth
%     \kern -.5\pagesepwidth
%     \vrule height \inv@targetmag\pageseplength
%            depth -\inv@targetmag\pagesepoffset
%            width \pagesepwidth
%     \kern -.5\pagesepwidth
%   \endgroup}
% \def\pagesep@topbottom{%
%   \begingroup
%     \pagesepwidth=\inv@targetmag\pagesepwidth
%     \vskip -.5\pagesepwidth
%     \moveright\inv@targetmag\pagesepoffset\hbox{%
%       \vrule height\pagesepwidth width\inv@targetmag\pageseplength}%
%     \vskip -.5\pagesepwidth
%   \endgroup}
% %
% % With the dvidvi layout, the pages are shipped individually:
% \def\ship@dvidvi#1{%
%   \begingroup
%     \voffset=-\inv@targetmag in
%     \hoffset=\voffset
%     \csname &normal@shipout\endcsname\box#1%
%   \endgroup}
% %
% % With the booklet or Booklet layout, the pages are saved rather than
% % shipped.
% \begingroup
% \let\newbox\relax
% \gdef\save@booklet#1{%
%   \begingroup
%     \globaldefs=1
%     \advance\bookletpage by 1
%     \expandafter\newbox\csname bookletbox\the\bookletpage\endcsname
%     \expandafter\setbox\csname bookletbox\the\bookletpage\endcsname\box#1%
%   \endgroup}
% \endgroup
% %
% % The pages are then printed at the end with the following macros:
% \def\make@bookletpage#1{%
%   \setbox\ifodd#1\@rightpage\else\@leftpage\fi=%
%     \expandafter\box\csname bookletbox\the#1\endcsname}
% \def\booklet@loop{%
%   \count\z@\rightpagenumber
%   \make@bookletpage\leftpagenumber
%   \make@bookletpage\rightpagenumber
%   \ship@twoup
%   \booklet@@loop}
% \def\booklet@@loop{%
%   \advance\rightpagenumber by 2
%   \advance\leftpagenumber by -2
%   \ifnum\leftpagenumber<1\else\expandafter\booklet@loop\fi}
% \def\Booklet@@loop{%
%   \advance\rightpagenumber by 1
%   \advance\leftpagenumber by -1
%   \ifnum\leftpagenumber<\rightpagenumber\else\expandafter\booklet@loop\fi}
% %
% % This one is easy:
% \def\twoupemptypage{\shipout\hbox{}}
% %
% % This clears a whole target page if there is a saved left page. Note that
% % this does not invoke the output routine; i.e., it is not like \clearpage
% % or \supereject. See \twoupclearpage and \twoupeject below.
% \def\twoup@eject{%
%   \if@leftpage\else
%     \global\setbox\@rightpage\hbox{}%
%     \make@@halfpage\@rightpage\ship@@@rightpage
%     \global\@leftpagetrue
%   \fi}
% %
% % This is the definition of \twoup@eject with the booklet option:
% \def\twoup@eject@booklet{%
%   \leftpagenumber\bookletpage
%   \advance\leftpagenumber by 3
%   \divide\leftpagenumber by 4
%   \multiply\leftpagenumber by 4
%   \rightpagenumber=1
%   \ifnum\leftpagenumber>\bookletpage
%     \setbox\@leftpage\hbox{}%
%     \make@@halfpage\@leftpage
%     \loop
%       \setbox\@rightpage\copy\@leftpage
%       \save@booklet\@rightpage
%     \ifnum\leftpagenumber>\bookletpage
%     \repeat
%   \fi
%   \booklet@loop}
% %
% % This modification is needed for \LaTeX in order to get the last page
% % printed out if the final page is a left page (the catcode business is
% % because \enddocument is \let to \bye in amstex):
% \begingroup
% \expandafter\ifx\csname @latexerr\endcsname\relax
%   \catcode`\>=14\else\catcode`\>=9\fi\relax
% >>\gdef\twoupclearpage{\clearpage\twoup@eject}
% >>\expandafter\@temptokena\expandafter{\enddocument}
% >>\xdef\enddocument{\noexpand\twoupclearpage\the\@temptokena}
% \endgroup
% %
% % For most other macro packages we could just leave be and all pages would
% % always be printed because of the way the \end primitive works (except that
% % TeX will go bonkers with the booklet layout). However,
% % sometimes a blank filler page would be printed *with* headings. We prefer
% % the filler page to be truly blank. To achieve this,  we hack the definition
% % of \end. This may cause problems with some macros.
% \expandafter\ifx\csname @latexerr\endcsname\relax
%   \let\twoup@@@end\end
%   \def\end{\twoup@eject\twoup@@@end}
%   \def\twoupeject{\par\vfil\supereject\twoup@eject}
% \fi
% %
% % This is one workaround for the page cross-references problem
% \def\TwoupWrites{%
%   \let\TwoupSaved@write\write
%   \let\TwoupSaved@read\read
%   \let\TwoupSaved@openout\openout
%   \let\TwoupSaved@closeout\closeout
%   \def\write{\TwoupSaved@write-1{}\immediate\TwoupSaved@write}%
%   \def\read{\TwoupSaved@write-1{}\immediate\TwoupSaved@read}%
%   \def\openout{\TwoupSaved@write-1{}\immediate\TwoupSaved@openout}%
%   \def\closeout{\TwoupSaved@write-1{}\immediate\TwoupSaved@closeout}%
%   \let\TwoupWrites\relax}
% %
% %  Defaults:
% \def\twouparticle{%
%   \target{\magstepminus1}{11in}{8.5in}%
%   \source{\magstep0}{8.5in}{11in}}
% \def\twoupplain{%
%   \target{\magstepminus2}{11in}{8.5in}%
%   \source{\magstep0}{8.5in}{11in}}
% \def\twouplegaltarget{%
%   \target{\magstepminus1}{14in}{8.5in}%
%   \source{\magstep0}{8.5in}{11in}}%
% \def\twouplandscape{%
%   \target{\magstepminus2}{8.5in}{11in}%
%   \source{\magstep0}{11in}{8.5in}%
%   \targetlayout{topbottom}}
% \expandafter\ifx\csname @latexerr\endcsname\relax
%   \twoupplain\else\twouparticle\fi
% \pagesepwidth 0pt
% \pageseplength 6.5in
% \pagesepoffset 1in
% %
% \expandafter\catcode`\@=\TheAtCode\relax
% \endinput
% %% END 2up.tex/2up.sty
% 
% \end{verbatim}
%
%
% \section{The documentation for 2up} \label{sec:2updoc}
%
%    Following is an almost verbatim copy of the documentation for Timothy
% Van Zandt's \Lpack{2up}. I hope that by including this I have 
% now completely met his COPYING conditions. The only change that
% I have made to the original is replacing every occurrence of `|{verbatim}|'
% by `|{verbtm}|', so LaTeX does not get confused when processing the
% current document.
%
% \begin{verbatim}
% 
% %% BEGIN 2up.doc
% %%
% %% Documentation for 2up.tex/2up.sty.
% %% Run through LaTeX, with or without the NFSS.
% %%
% %% See below if you want to try out two-up printing.
% %%
% %% LaTeX version of this documentation courtesy of
% %%    H. David Todd <hdtodd@mockingbird.wesleyan.edu>
% %%
% %%
% \documentstyle[12pt,twoside]{article}
% 
% \def\FileDate{January 28, 1993}
% \def\FileVersion{1.2}
% 
% %% INSTRUCTIONS FOR TWO-UP PRINTING.
% %% 1. Change \iffalse below to \iftrue.
% %% 2. Uncomment \special{landscape} if using Rokicki's dvips, or otherwise
% %%      remember to print the document out in landscape mode.
% %% 3. Uncomment one of the booklet options, if desired.
% %%
% \iffalse
%   \input 2up.tex
%   \targetlayout{twosided}
%   \TwoupWrites
%   % \special{landscape}     %% This works with Rokicki's dvips
%   % \targetlayout{Booklet}  %% Booklet printing with 2-sided printer/copier.
%   % \targetlayout{booklet}  %% Booklet printing with 1-sided printer/copier.
% \fi
% 
% \makeatletter
% 
% %% PAGE NUMBERING:
% %% Adjust page numbering if using booklet layout option.
% \setcounter{page}{0}
% \def\next{\if@leftpage\else\stepcounter{page}\fi}
% \@ifundefined{target@stop}{}{\next}
% 
% %% CONTENTS:
% \def\tableofcontents{%
%   \par\begin{center}
%   \large\bf Contents
%   \end{center}
%   \begin{quote}\@starttoc{toc}\end{quote}}
% 
% %% PAGE PARAMETERS:
% \setlength{\parindent}{0pt}          % paragraph indent
% \setlength{\parskip}{4pt plus 1pt minus 1pt}
% \setlength{\topmargin}{0pt}
% \setlength{\headheight}{12pt}        % height of running head
% \setlength{\headsep}{30pt}           % distance between header and text
% \setlength{\textheight}{8.2in}        % height of text on page
% 
% %% DATES, VERSIONS AND TITLES:
% \def\@maketitle{%
%  \begin{center}
%    {\Large\bf \@title \par}
%    \vskip 1.2em {\lineskip .5em
%    \begin{tabular}[t]{c}\@author\end{tabular}\par}
%    \vskip .8em {\@date}%
%   \end{center}
%   \par
%   \vskip .5em}
% 
% %% Page Style:
% \pagestyle{myheadings}
% \markboth{Two-Up Style Guide}{Version \FileVersion, \FileDate}
% 
% %% Special list:
% \newcommand{\namelistlabel}[1]{\mbox{\bf #1}\hfil}
% \newenvironment{namelist}[1]%
%   {\begin{list}{}%
%     {\let\makelabel\namelistlabel
%     \settowidth{\labelwidth}{#1}
%     \setlength{\leftmargin}{1.1\labelwidth}}}%
%  {\end{list}}
% 
% %% VERBATIM:
% %%
% \begingroup
%   \catcode`\{=12 \catcode`\}=12
%   \catcode`\(=1 \catcode`\)=2
%   \catcode`\+=0 \catcode`\\=12
%   +gdef+|(\)          % \| = character \
%   +gdef+{({)          % \{ = character {
%   +gdef+}(})          % \} = character }
% +endgroup
% \def\Backslash{\protect\|}
% %%
% %% Define some functional font commands:
% %%
% \def\MainFont{\tt}      % For macro definitions.
% \def\UsageFont{\tt}           % For in-line macro names.
% \def\InlineFont{\tt}          % For other in-line snipets of code.
% \def\MetaFont{\rm\it}            % For meta arguments.
% %%
% %% Short-cuts for font commands:
% %%
% \def\s#1{\string#1}
% \def\t#1{{\InlineFont\string#1}}   % For other in-line code.
% \def\m#1{{\MetaFont #1\/}}         % For just putting things in italics
% \def\M#1{{\InlineFont\{\m{#1}\}}}  % Arg is in italic, enclosed in tt braces.
% \let\N\t                           % For in-line macro names.
% \catcode`\<=13 \def<#1>{\m{#1}}    % <meta>
% \catcode`\"=13\def"{\verb"}        % Short verb
% %%
% %% This is for listing macro definitions in a quote-like environment.
% %%
% \begingroup
%   \catcode`\[=1 \catcode`\]=2
%   \catcode`\{=13 \catcode`\}=12
%   \gdef\@MD[%
%     \catcode`\{=13 \catcode`\}=12
%     \def{##1}[\{[\MetaFont ##1]\}]]
% \endgroup
% \def\MD{%
%   \quote
%   \begingroup
%     \@MD
%     \def\end{\endgroup\end}%
%     \def\\{\@centercr\s}%
%     \MainFont
%     \s}
% \def\endMD{\endquote}
% %%
% 
% %% End preamble.
% \catcode`\@=12
% 
% \begin{document}
% 
% \title{Documentation for 2up.tex:\\
%   Two-up printing for Generic TeX}
% \author{Timothy Van Zandt\thanks{The documentation was put into
%   \LaTeX\ format by David Todd.}\\ tvz@Princeton.EDU}
% \date{Version \FileVersion\\[3pt] \FileDate}
% 
% \maketitle
% \thispagestyle{empty}
% 
%   "2up.tex"/"2up.sty" allows one to print a document two-up, with
% considerable flexibility as to paper size and layout. It produces a standard
% dvi file, and does not involve an additional dvi or PostScript filter. It
% should work with most \TeX\ macro packages.
% 
%   {\bf Usage:} Input "2up.tex", or include "2up" as a LaTeX style option.
% There is a good chance you will get the desired layout. (But you will probably
% need to generate new font bitmaps to get high quality output.) See the rest of
% this documentation for detailed information on controlling 2up.tex.
% 
% \begingroup
% \def\baselinestretch{.8}\large\normalsize
% \tableofcontents
% \endgroup
% \clearpage
% 
% \section{Comparison with other methods}
% 
% There are other tools for two-up printing:
% \begin{description}
% \item[PostScript filters]
% These work well as long as you only use PostScript fonts (as opposed to the
% usual Metafont fonts). Otherwise, the PostScript filter scales bitmapped
% fonts, and the quality is lower than with "2up.tex".
% 
% \item[Rokicki's dvidvi]
% "dvidvi" is a versatil tool that does much more than
% 2up printing. However, if only rearranges the pages, and does not adjust the
% magnification to ensure that the pages fit properly in a two-up format. It
% also adds one more step to the \TeX-dvi-printer cycle. "2up.tex", on the other
% hand, combines the changes to \TeX's magnification and page layout with the
% rearranging of the pages, all with an easy-to-use interface and from within
% your \TeX\ document. (See Section \ref{dvidvi} on using both "2up.tex" and
% "dvidvi" together.)
% \end{description}
% 
% \section{A warning about fonts}
% 
% Two-up printing generally involves reducing a document, and thus using fonts
% sizes that are not typically found on your system. If you are using bitmapped
% fonts (standard for \TeX\ users), and you do not have and cannot generate the
% extra fonts bitmaps, you will find 2up.sty (and any other method of two-up
% printing) to give very poor quality output.
% 
% If you are using a program like Rokicki's "dvips" that can generate extra font
% bitmaps as needed, you will find that it takes a long time to print your 2up
% document the first few times, as "dvips" is busy making the extra fonts. This
% problem will go away. Be thankful you have such an easy way to get extra font
% bitmaps.
% 
% \section{Usage}
% 
% Input "2up.tex". \LaTeX\ users can include "2up" as a style option.
% 
% Then set the target and source layout:
% \begin{MD}
%   \source{mag}{width}{height}\\
%   \target{mag}{width}{height}
% \end{MD}
% 
% \begin{itemize}
% \item The ``source'' layout is the layout of your document
% the way it is usually printed.  The ``target'' layout is the layout when it
% is printed two-up.
% 
% \item <width> and <height> are the true width and height of the paper,
% when looking at the output right-side-up.
% 
% \item <mag> refers to the source and target magnification. It should be an
% integer, equal to 1000 times the magnification factor. There are also some
% magic magnification numbers:
% \begin{description}
% \item[$\setminus$magstep<n>] for a magnification of 1.2$^n$ (n=0,...,5), or
% 
% \item[$\setminus$magstepminus<n>] for a magnification of 1.2$^{-n}$
% (n=0,...,5).
% \end{description}
% Using these magnification values will make best use of the available fonts on
% your system.
% \end{itemize}
% 
% "2up.tex" takes each source page, and centers in half a target page. To get
% the pages to fit, the magnification of the document typically has to be
% reduced. If you find that the pages are too crowded, try reducing the target
% magnification. If they are too lonely, try increasing the target
% magnification.
% 
% If you want to adjust the positioning of the scaled pages, try changing the
% values of "\hoffset" or "\voffset" with "\advance" or \LaTeX's "\addtolength".
% 
% Don't change the magnification of your document after using the "\target"
% command, as you will override the settings made by 2up.tex.
% 
% You can also specify some options:
% \begin{MD}
%   \targetlayout{options}
% \end{MD}
% where <options> is a comma separated list with no spaces. Valid
% options are:
% \begin{namelist}{topbottomxxx}
% 
% \item [topbottom] The two source pages are printed one on top of the other.
% \item [twosided]  Odd pages are always printed on the right or the bottom.
% \item [booklet]   Pages are printed like a booklet. See Section \ref{booklet}.
% \item [Booklet]   Like "booklet", but for two-sided printers/copiers.
% \item [dvidvi]    For arranging in two-up layout with "dvidvi". See Section
% \ref{dvidvi}.
% 
% \end{namelist}
% 
% The only other changes you might want to make to your document are the
% following:
% \begin{itemize}
% \item You have to see to it that the target output is printed with the
% correct orientation (landscape or portrait). See the documentation for your
% dvi driver for details.
% 
% \item "2up.tex" will inserts a line between pages if you change the following
% dimensions:
% \begin{namelist}{pagesepwidthxxx}
% \item [$\setminus$pagesepwidth]   Thickness of line. (Default 0pt)
% \item [$\setminus$pageseplength]  Length of line. (Default 6.5in)
% \item [$\setminus$pagesepoffset]  Distance from bottom of page to bottom of
% line, or (with "topbottom" layout) from side of page to beginning of line.
% (Default 1in)
% \end{namelist}
% 
% \item You can, but do not need to, use the following commands:
% \begin{namelist}{twoupemptypagexxx}
% 
% \item [$\setminus$twoupemptypage] inserts a completely blank half page
% in the target.
% 
% \item [$\setminus$twoupclearpage] \LaTeX\ users: ejects a complete target
% page.
% 
% \item [$\setminus$twoupeject] Plain \TeX users: ejects a complete target
% page.
% 
% \end{namelist}
% \end{itemize}
% 
% \section{Defaults}
% 
%   The following commands set the parameters for some common sources and
% targets. The definitions of these commands are also useful examples.
% 
%   For A4 paper, just replace "8.5in" and "11in" in the definitions below by
% the width and height of A4 paper.
% 
% \begin{description}
% 
% \item [$\setminus$twouparticle]
% 
% This is the default for \LaTeX.
% \begin{namelist}{Sourcexx}
% 
% \item [Source:] A portrait document on 8.5x11 inch paper with wide margins as
% used in \LaTeX's article style, and without marginal notes.
% 
% \item [Target:] A landscape document on the same size paper, with the source
% pages printed side by side.
% 
% \item [Definition:]\hspace{1pt}
% \begin{verbtm}
% \target{\magstepminus1}{11in}{8.5in}
% \source{\magstep0}{8.5in}{11in}}
% \end{verbtm}
% \end{namelist}
% 
% If you use narrower side margins, or if you use marginal notes, then you
% will need to use "\twoupplain" or \\ "\twouplegaltarget", defined below.
% 
% \item [$\setminus$twoupplain]
% 
% This is the default for non-\LaTeX.
% 
% It is the same as "\twouparticle", but for source documents with narrower
% margins.
% \begin{namelist}{Sourcexxx}
% 
% \item [Definition:]\hspace{1pt}
% \begin{verbtm}
% \target{\magstepminus2}{11in}{8.5in}
% \source{\magstep0}{8.5in}{11in}
% \end{verbtm}
% \end{namelist}
% 
% \item [$\setminus$twouplegaltarget]
% 
% \begin{namelist}{Sourcexxx}
% \item [Source:] A portrait document on 8.5x11 inch paper with just about any
% margins and perhaps with marginal notes.
% 
% \item [Target:] A landscape document on legal size paper (8.5x14 inches),
% with the source pages printed side by side.
% 
% \item [Definition:]\hspace{1pt}
% \begin{verbtm}
% \target{\magstepminus1}{14in}{8.5in}
% \source{\magstep0}{8.5in}{11in}\
% end{verbtm}
% \end{namelist}
% 
% \item [$\setminus$twouplandscape]\hspace{1pt}
% \begin{namelist}{Sourcexxx}
% 
% \item [Source:] A landscape document on 8.5x11 inch paper with just about any
% margins.
% 
% \item [Target:] A portrait document on the same size paper, with the
% source pages stacked vertically.
% 
% \item [Definition:]\hspace{1pt}
% \begin{verbtm}
% \target{\magstepminus2}{8.5in}{11in}
% \source{\magstep0}{11in}{8.5in}
% \targetlayout{topbottom}
% \end{verbtm}
% \end{namelist}
% 
% \end{description}
% 
% 
% \section{True Dimensions}
% 
% \TeX\ lets you use `true' dimensions, as in
% \begin{verbtm}
%     \topmargin 1truein
%     \setlength{\topmargin}{1truein}
% \end{verbtm}
% This sets the length to 1 inch
% as it appears on the page, no matter what the magnification.
% 
% Once you use `true' dimensions, it is impossible to change the
% magnification of your document, and hence it is almost impossible
% to use 2up.tex. You will get an error like:
% \begin{verbtm}
%     ! Incompatible magnification (833);
%       the previous value will be retained (2000).
% \end{verbtm}%
% 
% The use of `true' dimensions might be hiding in macros or style files
% you are using. For example, Plain \TeX's "\magnification" command uses
% true dimensions; use "\mag" instead, and then set "\hsize" and
% "\vsize" appropriately scaled. If you can't find the culprit `true'
% dimension in your document or input files, try putting this on the
% (very) first line of your document:
% \begin{verbtm}
%     \mag 143 \voffset 1truein \mag 1000
% \end{verbtm}
% 
% You will get the error listed above wherever true dimensions are used
% (and in some other places as well).
% 
% 
% \section{Page Cross-References}
% 
%   Page cross-references (e.g., in an index or table of contents) will not be
% accurate with "2up.tex", because finished pages are not immediately shipped
% out. This can also cause problems with some special \LaTeX\ styles that have
% multiple ".aux" files (e.g., for each chapter).
% 
%   To get better, but not perfect, page cross-references, try putting the
% command
% \begin{verbtm}
%     \TwoupWrites
% \end{verbtm}
% at the beginning of your document, after inputting "2up.tex". This is good for
% drafts, and may fix problems with \LaTeX\ styles that have multiple ".aux"
% files. "\TwoupWrites" makes "\write" and "\read" "\immediate", if you know
% what that means. It might cause problems of its own, but try it and see.
% 
%   To get accurate page cross-references:
% \begin{enumerate}
% \item Run the job without "2up.tex", until all cross-references
% are properly resolved. Run any index utility at this point.
% 
% \item Run the job ONE MORE TIME with "2up.tex"; it will use the
% correct cross-references from the previous runs. (In \LaTeX\ documents,
% it is useful at this point to include the "\nofiles" command, so that
% the correct cross-references are preserved.)
% \end{enumerate}
% 
% \section{Booklets\label{booklet}}
% 
% Here are some special considerations for making booklets:
% 
% \begin{itemize}
% 
% \item  Use the "booklet" target layout if your printer or copier cannot print
% two
% sides directly. "2up.tex" prints out the ``tops'' of the pages, and then the
% ``bottoms'', so that you can copy or print on both sides by doing one side,
% flipping the output over, and doing the other side.
% 
% \item Use "Booklet" if your printer or copier can print directly on two sides.
% 
% \item You must have enough memory and box registers to hold all the pages in
% the document. A Big\TeX\ should be able to handle at least a 32-page booklet.
% 
% \item Sometimes booklet printing is part of the design of a document, rather
% than an afterthought. In this case, you should set up the margins for the
% target half-page rather than using "2up.tex" to scale the document. For
% example, if using \LaTeX, 8.5x11 inch paper, \LaTeX's twoside style option,
% and headings, try the following page parameter changes:
% \begin{verbtm}
%     \setlength{\oddsidemargin}{-0.2in}
%     \setlength{\evensidemargin}{-0.5in}
%     \setlength{\textwidth}{4.2in}
%     \setlength{\textheight}{6.5in}
%     \setlength{\topmargin}{-.4in}
% \end{verbtm}
% If you do this, "2up.tex" isn't adjusting margins or scaling the
% document. You coould therefore get by without "2up.tex" by using "dvidvi".
% \end{itemize}
% 
% \section{Using dvidvi\label{dvidvi}}
% 
%   With the "dvidvi" layout option, "2up.tex" takes care of adjusting the
% margins, and you then use "dvidvi" to arrange the pages in a two-up
% format. The advantage is that cross referencing comes out right.
% The disadvantage is that it involves one more step.
% 
% Start by \TeX ing your document with "2up.tex" and the "dvidvi" layout option.
% Then run the dvi file through "dvidvi". Specify the offsets for "dvidvi" at
% the middle of the page. E.g., for
% a side-by-side layout on 8.5x11in paper, use
% \begin{verbtm}
%     dvidvi '2:0,1(5.5in,0in)' foo bar
% \end{verbtm}
% For other paper sizes, replace "5.5in" by half the target paper width. For the
% "topbottom" layout, reverse the dimensions, e.g.,
% (0in,5.5in). For booklet printing, see the
% the documentation for "dvidvi".
% 
% 
% \section{Compatibility\label{compatibility}}
% 
%   These macros should work with most \TeX\ macro packages, except for
% the following possible but unlikely compatibility problems:
% 
% \begin{itemize}
% 
% \item If you use a macro package other than \LaTeX\ that does unusual things
% to end a document and/or that modifies the definition of "\end", you may
% find that the last page is nor printed or that an extra page with headings
% is inserted at the end of the document. This is fixable on a case-by-case
% basis. See the redefinition of "\end" in "2up.tex".
% 
% \item "2up.tex" might not work if you meddle with \TeX's "\shipout" primitive
% after loading "2up.tex".
% 
% \item If the first page of the source document is not on the first two-up
% target
% page (possible, for example, with two-sided printing and the "booklet"
% option, but generally unlikely), "\special"'s that come at the beginning
% of the document also do not get onto the first page of output. This can mess
% up "\special"'s for landscape printing or
% including a PostScript header file. In this case, you have to specify
% these things as command line options when using your dvi driver.
% 
% \end{itemize}
% 
% 
% \section{Changes}
% \begin{namelist}{V0.9xxx}
% \raggedright
% 
% \item [V0.9a] Everything redone. Too many differences to point out.
% 
% \item [V0.9b] Interface redone again. "booklet" option added.
% 
% \item [V0.9c] Pages ship out a little earlier. A few bugs fixed.
% 
%   Parameters "\pageseplength" and "\pagesepthickness" added.
% 
% \item [V0.9d] Fixed bug with "\twoupclearpage" and "\twoupeject".
% 
% \item [V0.9e] Made pages be counted (from dvi driver's viewpoint) by the
% physical page number.
% 
% \item [V0.9f] Parameter "\pagesepoffset" added, and "\pagesepthickness"
% changed to "\pagesepwidth".
% 
% \item [V1.0]  None. Just seemed seemed like it was time to call it 1.0
% 
% \item [V1.1] Added "Booklet" and "dvidvi" target layouts.
% 
%   "\hoffset" and "\voffset" in source document now have expected effect.
% 
% \item[V1.2] Fixed incompatibility problem with AmS-\TeX/AmS-\LaTeX.
% 
%   Added "\TwoupWrites" command.
% 
% \end{namelist}
% \end{document}
% % END 2up.doc
% 
% \end{verbatim}
%
%    I used a slightly modified version of \file{2up.doc} as one of the
% test documents for the \Lpack{booklet} package. First, to get a baseline,
% I followed the instructions in \file{2up.doc} and edited and printed
% it using \Lpack{2up} to get a booklet.
%
%    I then copied \file{2up.doc} to \file{2updoc.tex} and then edited
% the new file so that the first part of the preamble became:
% \begin{verbatim}
% %% BEGIN 2updoc.tex
% %% ....
% \documentclass[12pt,twoside]{article}
% \usepackage[print,four]{booklet}  % change options to suit
% \usepackage{ifpdf}                % from CTAN
%
% \ifpdf
%   \pdfoutput=1
%   \setpdftargetpages
% \else
%   \checkforlandscape
%   \ifuselandscape
%     \special{dvips}
%     \ifprintoption
%       \special{!TeXDict begin <</Duplex true /Tumble true>> setpagedevice end}
%     \fi
%   \fi
% \fi
% 
% \def\FileDate....
% ...
% \iffalse
%   \input 2up.tex
%   ...
%    % \fi and \makeatletter were here
% %% PAGE NUMBERING:
% ...
% \@ifundefined{target@stop}...
% \fi     % this \fi and \makeatletter moved down from before NUMBERING
% \makeatletter
%
% %% CONTENTS     % no changes after this
% ... 
% \end{verbatim}
%
% Processing, printing and folding this to make a booklet enabled me to
% determine if the package effectively produced the same final result
% as \Lpack{2up}. 
% 
%
% \Finale
% \PrintIndex
%
\endinput

%% \CharacterTable
%%  {Upper-case    \A\B\C\D\E\F\G\H\I\J\K\L\M\N\O\P\Q\R\S\T\U\V\W\X\Y\Z
%%   Lower-case    \a\b\c\d\e\f\g\h\i\j\k\l\m\n\o\p\q\r\s\t\u\v\w\x\y\z
%%   Digits        \0\1\2\3\4\5\6\7\8\9
%%   Exclamation   \!     Double quote  \"     Hash (number) \#
%%   Dollar        \$     Percent       \%     Ampersand     \&
%%   Acute accent  \'     Left paren    \(     Right paren   \)
%%   Asterisk      \*     Plus          \+     Comma         \,
%%   Minus         \-     Point         \.     Solidus       \/
%%   Colon         \:     Semicolon     \;     Less than     \<
%%   Equals        \=     Greater than  \>     Question mark \?
%%   Commercial at \@     Left bracket  \[     Backslash     \\
%%   Right bracket \]     Circumflex    \^     Underscore    \_
%%   Grave accent  \`     Left brace    \{     Vertical bar  \|
%%   Right brace   \}     Tilde         \~}


