% \iffalse meta-comment
%
% fonttable.dtx
%
%  Author: Peter Wilson (Herries Press)
%  Maintainer: Will Robertson (will dot robertson at latex-project dot org)
%  Copyright 2005--2009 Peter R. Wilson
%
%  This work may be distributed and/or modified under the
%  conditions of the Latex Project Public License, either
%  version 1.3c of this license or (at your option) any
%  later version: <http://www.latex-project.org/lppl.txt>
%
%  This work has the LPPL maintenance status "maintained".
%  The Current Maintainer of this work is Will Robertson.
%
%  This work consists of the files listed in the README file.
%
%
%<*driver>
\documentclass[twoside]{ltxdoc}
\usepackage{url}
\usepackage[draft=false,
            plainpages=false,
            pdfpagelabels,
            bookmarksnumbered,
            hyperindex=false,
            colorlinks,linktocpage
           ]{hyperref}
\providecommand{\phantomsection}{}
%%\OnlyDescription %% comment this out for the full glory
\EnableCrossrefs
\CodelineIndex
\setcounter{StandardModuleDepth}{1}
\makeatletter
  \@mparswitchfalse
\makeatother
\renewcommand{\MakeUppercase}[1]{#1}
\pagestyle{headings}
\newenvironment{addtomargins}[1]{%
  \begin{list}{}{%
    \topsep 0pt%
    \addtolength{\leftmargin}{#1}%
    \addtolength{\rightmargin}{#1}%
    \listparindent \parindent
    \itemindent \parindent
    \parsep \parskip}%
  \item[]}{\end{list}}
\begin{document}
  \raggedbottom
  \DocInput{fonttable.dtx}
\end{document}
%</driver>
%
% \fi
%
% \CheckSum{911}
%
% \DoNotIndex{\',\.,\@M,\@@input,\@addtoreset,\@arabic,\@badmath}
% \DoNotIndex{\@centercr,\@cite}
% \DoNotIndex{\@dotsep,\@empty,\@float,\@gobble,\@gobbletwo,\@ignoretrue}
% \DoNotIndex{\@input,\@ixpt,\@m}
% \DoNotIndex{\@minus,\@mkboth,\@ne,\@nil,\@nomath,\@plus,\@set@topoint}
% \DoNotIndex{\@tempboxa,\@tempcnta,\@tempdima,\@tempdimb}
% \DoNotIndex{\@tempswafalse,\@tempswatrue,\@viipt,\@viiipt,\@vipt}
% \DoNotIndex{\@vpt,\@warning,\@xiipt,\@xipt,\@xivpt,\@xpt,\@xviipt}
% \DoNotIndex{\@xxpt,\@xxvpt,\\,\ ,\addpenalty,\addtolength,\addvspace}
% \DoNotIndex{\advance,\Alph,\alph}
% \DoNotIndex{\arabic,\ast,\begin,\begingroup,\bfseries,\bgroup,\box}
% \DoNotIndex{\bullet}
% \DoNotIndex{\cdot,\cite,\CodelineIndex,\cr,\day,\DeclareOption}
% \DoNotIndex{\def,\DisableCrossrefs,\divide,\DocInput,\documentclass}
% \DoNotIndex{\DoNotIndex,\egroup,\ifdim,\else,\fi,\em,\endtrivlist}
% \DoNotIndex{\EnableCrossrefs,\end,\end@dblfloat,\end@float,\endgroup}
% \DoNotIndex{\endlist,\everycr,\everypar,\ExecuteOptions,\expandafter}
% \DoNotIndex{\fbox}
% \DoNotIndex{\filedate,\filename,\fileversion,\fontsize,\framebox,\gdef}
% \DoNotIndex{\global,\halign,\hangindent,\hbox,\hfil,\hfill,\hrule}
% \DoNotIndex{\hsize,\hskip,\hspace,\hss,\if@tempswa,\ifcase,\or,\fi,\fi}
% \DoNotIndex{\ifhmode,\ifvmode,\ifnum,\iftrue,\ifx,\fi,\fi,\fi,\fi,\fi}
% \DoNotIndex{\input}
% \DoNotIndex{\jobname,\kern,\leavevmode,\let,\leftmark}
% \DoNotIndex{\list,\llap,\long,\m@ne,\m@th,\mark,\markboth,\markright}
% \DoNotIndex{\month,\newcommand,\newcounter,\newenvironment}
% \DoNotIndex{\NeedsTeXFormat,\newdimen}
% \DoNotIndex{\newlength,\newpage,\nobreak,\noindent,\null,\number}
% \DoNotIndex{\numberline,\OldMakeindex,\OnlyDescription,\p@}
% \DoNotIndex{\pagestyle,\par,\paragraph,\paragraphmark,\parfillskip}
% \DoNotIndex{\penalty,\PrintChanges,\PrintIndex,\ProcessOptions}
% \DoNotIndex{\protect,\ProvidesClass,\raggedbottom,\raggedright}
% \DoNotIndex{\refstepcounter,\relax,\renewcommand,\reset@font}
% \DoNotIndex{\rightmargin,\rightmark,\rightskip,\rlap,\rmfamily,\roman}
% \DoNotIndex{\roman,\secdef,\selectfont,\setbox,\setcounter,\setlength}
% \DoNotIndex{\settowidth,\sfcode,\skip,\sloppy,\slshape,\space}
% \DoNotIndex{\symbol,\the,\trivlist,\typeout,\tw@,\undefined,\uppercase}
% \DoNotIndex{\usecounter,\usefont,\usepackage,\vfil,\vfill,\viiipt}
% \DoNotIndex{\viipt,\vipt,\vskip,\vspace}
% \DoNotIndex{\wd,\xiipt,\year,\z@}
%
% \changes{v1.0}{2005/11/27}{First public release}
% \changes{v1.0a}{2005/12/06}{Minor bug fix}
% \changes{v1.1}{2006/10/02}{Minor extensions}
% \changes{v1.2}{2006/10/02}{Minor extensions of texts}
% \changes{v1.3}{2009/04/30}{Fix serious bug}
% \changes{v1.4}{2009/05/06}{Added another way to select a font}
% \changes{v1.5}{2009/05/12}{Added code for Knuth's testfont.tex}
% \changes{v1.51}{2009/05/14}{Eliminated a clash with babel package}
% \changes{v1.5b}{2009/09/02}{New maintainer (Will Robertson)}
% \changes{v1.6}{2009/10/15}{New spacing of the decimals from Peter Wilson}
%
% \def\fileversion{v1.0} \def\filedate{2005/11/27}
% \def\fileversion{v1.0a} \def\filedate{2005/12/06}
% \def\fileversion{v1.1} \def\filedate{2006/10/02}
% \def\fileversion{v1.2} \def\filedate{2008/05/08}
% \def\fileversion{v1.3} \def\filedate{2009/04/30}
% \def\fileversion{v1.4} \def\filedate{2009/05/06}
% \def\fileversion{v1.5} \def\filedate{2009/05/12}
% \def\fileversion{v1.51} \def\filedate{2009/05/14}
% \def\fileversion{v1.5b} \def\filedate{2009/09/02}
% \def\fileversion{v1.5c} \def\filedate{2009/09/20}
% \def\fileversion{v1.5d} \def\filedate{2009/09/22}
% \def\fileversion{v1.6} \def\filedate{2009/10/15}
% \newcommand*{\Lpack}[1]{\textsf {#1}}           ^^A typeset a package
% \newcommand*{\Lopt}[1]{\textsf {#1}}            ^^A typeset an option
% \newcommand*{\file}[1]{\texttt {#1}}            ^^A typeset a file
% \newcommand*{\Lcount}[1]{\textsl {\small#1}}    ^^A typeset a counter
% \newcommand*{\pstyle}[1]{\textsl {#1}}          ^^A typeset a pagestyle
% \newcommand*{\Lenv}[1]{\texttt {#1}}            ^^A typeset an environment
% \newcommand*{\AD}{\textsc{ad}}
% \newcommand*{\thisfont}{OandS}
%
% \title{The \Lpack{fonttable} package\thanks{This
%        file has version number \fileversion, last revised
%        \filedate.}}
%
% \author{%
% Author: Peter Wilson, Herries Press\\
% Maintainer: Will Robertson\\
% \texttt{will dot robertson at latex-project dot org}%
% }
% \date{\filedate}
% \maketitle
% \begin{abstract}
%    The package lets you typeset the characters in a font in tabular and/or
% running text forms.
% \end{abstract}
% \tableofcontents
%
% \section{Introduction}
%
%    The \Lpack{fonttable} package lets you typeset a font's character set
% in tabular and/or running text forms.
%
% This manual is typeset according to the conventions of the
% \LaTeX{} \textsc{docstrip} utility which enables the automatic
% extraction of the \LaTeX{} macro source files~\cite{COMPANION}.
%
% \section{The package}
%
%    The package provides commands to typeset a table of all the glyphs in
% a given font and to typeset an example of regular text. For font designers
% it provides commands to typeset a `test' glyph among sets of glyphs
% from the font.
%
% \DescribeMacro{\fnthours}
% As a convenience, \cs{fnthours} prints the time of day when the file was 
% processed; it uses the 24 hour clock notation. (The macro \cs{today} prints
% the date when the file was processed.)
%
% \subsection{Table and texts}
%
%
% \DescribeMacro{\fonttable}
% The command \\
% \cs{fonttable}\marg{testfont} \\
% typesets a table showing all the
% glyphs in the \meta{testfont}, where \meta{testfont} is the name of a 
% font file\footnote{More precisely, the name of a \texttt{.tfm} file.}
% like \texttt{cmr10} (for Computer Modern Roman) or \texttt{pzdr} 
% (for Zapf Dingbats).
%
% NOTE: The \Lpack{mftinc} package~\cite{MFTINC} for pretty-printing 
% METAFONT code also defines a \cs{fonttable} macro that is akin to this 
% one. If you want to use both packages together then you can use the following
% general procedure for when a macro \cs{macro} is defined in both 
% \Lpack{packA} and \Lpack{packB} packages.
% \begin{verbatim}
% \usepackage{packA}
% \let\macroA\macro%   save packA's definition
% \let\macro\relax%    undefine \macro
% \usepackage{packB}%  now it's packB's definition of \macro
% ...
% \macro %   use the packB defintion
% \macroA %  use the packA definition   
% \end{verbatim}
%
% \DescribeMacro{\xfonttable}
% The command \\
% \cs{xfonttable}\marg{encoding}\marg{family}\marg{series}\marg{shape} \\
% typesets a table showing all the
% glyphs in the font with encoding \meta{encoding} (e.g., T1 or OMS),
% family \meta{family} (e.g., \texttt{ppl} for Palatino or \texttt{cmbrs}
% for CM Bright Math (OMS)), font series \meta{series} (e.g., \texttt{sb}
% for semibold of \texttt{m} for medium), and font shape \meta{shape}
% (e.g., \texttt{n} for normal or \texttt{sc} for small caps).
% For example: \\
% \verb?\xfonttable{U}{pzd}{m}{n}? \\
% for Zapf Dingbats.
%
% \DescribeMacro{\pikfont}
% The command\footnote{The name was chosen in an attempt to avoid
% clashes with other macros that might perform similar functions.}   \\
% \cs{pikfont}\marg{encoding}\marg{family}\marg{series}\marg{shape} \\
% selects the font with encoding \meta{encoding} (e.g., T1 or OMS),
% family \meta{family} (e.g., \texttt{ppl} for Palatino or \texttt{cmbrs}
% for CM Bright Math (OMS)), font series \meta{series} (e.g., \texttt{sb}
% for semibold of \texttt{m} for medium), and font shape \meta{shape}
% (e.g., \texttt{n} for normal or \texttt{sc} for small caps).
% For example: \\
% \verb?\pikfont{T1}{ppl}{m}{sc}? \\
% for Palatino small caps.
% The size of the font corresponds to the current setting (e.g.,
% \cs{footnotesize}, \cs{normalsize}, \cs{Large}). It can also be
% changed after being selected by the incantation \\
% \cs{fontsize}\marg{size}\marg{baselineskip}\cs{selectfont} \\
% where \meta{size} is the normal height and \meta{baselineskip} is the
% distance between text lines; the measurement system is \texttt{pts}
% but just use numbers with no units specified. For example: \\
% \verb?\fontsize{12}{15}\selectfont? \\
% for a 12pt font with 15pts between baselines.
%
% If you are unsure about the meaning of the various arguments 
% of \cs{xfonttable} and \cs{pikfont} see
% \textit{The Companion}~\cite[Chapter 7]{COMPANION} or the 
% \textit{LaTeX2e font selection} manual (\texttt{fntguide.tex}; try
% \verb?texdoc fntguide?).
%
% \DescribeMacro{\fontrange}
% The package attempts to populate the table with a maximum of 256 glyphs,
% numbered from 0 to 255.
% The \cs{fontrange}\marg{low}\marg{high} declaration changes this by reducing
% the range so that it extends from \meta{low} to \meta{high}, where \meta{low}
% should be at least 0 and \meta{high} at most 256, and \meta{low} less than
% \meta{high}.
%
%    The table is composed of blocks of sixteen characters. If necessary the
% value of \meta{low} is adjusted lower and \meta{high} is adjusted higher
% to match this block structure. For example, if you wanted a table of the
% lower 128 characters then \verb?\fontrange{0}{127}? would do the job,
% while the upper half of a 256 character font could be tabulated via
% \verb?\fontrange{128}{255}?.
%
% \DescribeMacro{\decimals}
% \DescribeMacro{\nodecimals}
% Normally each cell in the table includes the decimal number of the position
% in the (256) character set. \cs{nodecimals} turns off this numbering and
% \cs{decimals} turns it on. The default is \cs{decimals}.
%
% \DescribeMacro{\hexoct}
% \DescribeMacro{\nohexoct}
% Normally the columns and rows in the table are numbered using hexadecimal 
% and octal numbers. These can be turned off by \cs{nohexoct} and turned 
% on again with \cs{hexoct}, which is the default.
%
% \DescribeMacro{\ftablewidth}
% The font table's width is the length \cs{ftablewidth}, which by default
% is set to the normal textwidth (or more exactly, to \cs{hsize}). The table
% itself is left aligned. However, if \cs{nohexoct} is in effect the width of
% the table is its natural width.
%
% \DescribeMacro{\fntcolwidth}
% When \cs{nohexoct} is in effect the minimum width of a table column
% is \cs{fntcolwidth}. This is initially declared as \\
% \verb?\setwidth{\fntcolwidth}{0.08\ftablewidth}?
%
% \DescribeMacro{\fonttext}
% The command \cs{fonttext}\marg{testfont} typesets an example text using
% the \meta{testfont} (e.g. \texttt{cmr10}).
%
% \DescribeMacro{\simpletext}
% \DescribeMacro{\fulltext}
% The example text can be just a paragraph and a line of capitals,
%  or include more complex accented
% words as well. Following the declaration \cs{fulltext} the complex words are
% included as well as the example paragraph. The default is \cs{simpletext}
% for just the paragraph.
%
% \DescribeMacro{\regulartext}
% The command \cs{regulartext}\marg{fontspec} typesets the example text
% using \meta{fontspec}, for example \verb?\rmfamily\itshape?
% or \verb?\pikfont{T1}{pnc}{m}{it}?.
%
% \DescribeMacro{\fonttexts}
% \DescribeMacro{\regulartexts}
% The macro \cs{fonttexts}\marg{testfont}\marg{text} typesets \meta{text} using
% the \meta{testfont} (e.g \texttt{cmr10}). Similarly the macro
% \cs{regulartexts}\marg{fontspec}\marg{text} typesets \meta{text}
% using \meta{fontspec} (e.g., \verb?\rmfamily\itshape? or 
% \verb?\pikfont{T1}{ppl}{m}{it}?).
%
% \DescribeMacro{\germanparatext}
% \DescribeMacro{\latinparatext}
% \cs{germanparatext} expands to a German language paragraph, borrowed from the 
% \Lpack{blindtext} package~\cite{BLINDTEXT}. 
% \cs{latinparatext} expands to one version of
% a paragraph of the traditional \textit{lorem ipsum} dummy Latin text.
% Either, or both, of these could be used as the \meta{text} argument
% to \cs{fonttexts} or \cs{regulartexts}. \\
% NOTE: These were originally called \cs{germantext} and \cs{latintext}
%  but on 2009/05/14 I was told that the \Lpack{babel} package defines \cs{latintext}, 
%  which causes unexpected results if it is used in the same document as this
%  package. To try and be on the safe side I renamed \cs{germantext}
% as well as \cs{latintext}.
%
% \DescribeMacro{\aztext}
% \DescribeMacro{\AZtext}
% \DescribeMacro{\digitstext}
% \DescribeMacro{\punctext}
% \cs{aztext} expands to the lowercase Latin alphabet a to z, and \cs{AZtext}
% is the corresponding command for the uppercase A to Z. 
% The macros \cs{digitstext} and \cs{punctext} expand respectively to the
% digits 0 to 9, and to the typical punctuation marks. In all cases there 
% is a space between each character. 
%
% \subsection{Testing a glyph}
%
%    The macros here are a reimplementation of Donald Knuth's 
% \texttt{testfont.tex}, which is available from CTAN. 
%
%    In the following, the value of a glyph argument can be specified as
% its location in the font (i.e., as a decimal number). With a few exceptions,
% if the glyph is within the visible ASCII range (33--126) it may instead be 
% specified by the ASCII character prefixed with a single open quote 
% mark\footnote{Sometimes called a
% `backquote'.} (`). The exceptions are nos: 37 (\verb?%?), 92 (\verb?\?)
% 123 (\verb?{?) and 125 (\verb?}?) (but there may be others). 
% In any case, the glyph representing the
% character \texttt{p} can be specified either as \texttt{`p} or as 
% \texttt{112}. 
%
%    The glyphs are taken from the current font. If the font does not
% have Latin alphabet glyphs in the ASCII locations then 
% in the descriptions below phrases like `lowercase alphabet' or `uppercase
% alphabet' or `digits', should be taken to mean (the glyphs in) those 
% locations.
%
% \DescribeMacro{\glyphmixture}
% \cs{glyphmixture}\marg{T}\marg{S}\marg{E} typesets the \meta{T} (test) glyph 
% between the glyphs in the range from \meta{S} (start) to \meta{E} (end).
% For example \\
% \verb?\glyphmixture{`e}{`f}{`g}? will produce \\
% \texttt{efeeffeeefffef} \\
% \texttt{egeeggeeegggeg}
%
% \DescribeMacro{\glyphalternation}
% \cs{glyphalternation}\marg{T}\marg{S}\marg{E} typesets the \meta{T} 
% glyph alternately between each glyph in the range from \meta{S} 
% to \meta{E}.
% For example \\
% \verb?\glyphalternation{`e}{`f}{`g}? will produce \\
% \texttt{efefefefefefefefe} \\
% \texttt{egegegegegegegege}
%
% \DescribeMacro{\glyphseries}
% \cs{glyphseries}\marg{T}\marg{S}\marg{E} typesets the \meta{T}
% glyph between the glyphs in the range from \meta{S}
% to \meta{E}.
% For example \\
% \verb?\glyphseries{`e}{`f}{`h}? will produce \\
% \texttt{efegehe}
%
% \DescribeMacro{\glyphalphabet}
% \DescribeMacro{\GLYPHALPHABET}
% \cs{glyphalphabet}\marg{T} typesets the \meta{T} glyph between
% each letter of the lowercase Latin alphabet plus a few others. 
% \cs{GLYPHALPHABET}\marg{T}
% does the same but using the uppercase Latin alphabet.
% For example, the output of \\
% \verb?\glyphalphabet}{`3}? is like \\
% \texttt{3a3b3c3d3e3f3g\ldots3z3\char31 3}\verb?~?\texttt{3\char33 3\char34 3}
%
% \DescribeMacro{\glyphlowers}
% \DescribeMacro{\glyphlowers}
% \DescribeMacro{\glyphdigits}
% \cs{glyphlowers} takes each character of the lowercase alphabet in turn as
% a test glyph and sets it interpersed among the other lowercase characters.
% \cs{glyphuppers} and \cs{glyphdigits} are similar except that they use
% the uppercase alphabet and the ten digits instead. For example, 
% \cs{glyphdigits} produces output like \\
% \texttt{000102030405060708090} \\
% \texttt{101112131415161718191} \\
% \texttt{202122232425262728292} \\
% \ldots \\
% \texttt{909192939495969798999}
%
% \DescribeMacro{\glyphpunct}
% \cs{glyphpunct} sets a collection of words with an assortment of
% punctuation marks.
%
%
% \StopEventually{
% \bibliographystyle{alpha}
% \renewcommand{\refname}{Bibliography}
% \begin{thebibliography}{GMS94}
% \addcontentsline{toc}{section}{\refname}
%
% \bibitem[Lik05]{BLINDTEXT}
% Knut Lickert.
% \newblock \emph{Blindtext.sty: Creating text for testing / Texterzeugung
%                 zum testen}.
% \newblock October 2005.
% \newblock (Available from CTAN in \url{macros/latex/contrib/blindtext})
%
% \bibitem[MG04]{COMPANION}
% Frank Mittelbach and Michel Goossens.
% \newblock \emph{The LaTeX Companion}.
% \newblock Second edition.
% \newblock Addison-Wesley Publishing Company, 2004.
%
% \bibitem[Pak05]{MFTINC}
% Scott Pakin
% \newblock `The \Lpack{mftinc} package',
% \newblock January 2005.
% \newblock (Available from CTAN in \url{macros/latex/contrib/mftinc})
%
%
% \end{thebibliography}
%
% \PrintIndex
%
% }
%
%
%
% \section{The code} \label{sec:mf}
%
%    \begin{macrocode}
%<*pack>
\NeedsTeXFormat{LaTeX2e}
\ProvidesPackage{fonttable}[2009/10/15 v1.6 displays a font]

%    \end{macrocode}
%
% \subsection{Table and texts}
%
% Most of the code below is an edited version of code used in 
% \texttt{nfssfont.tex} for displaying aspects of the set of glyphs in a font.
%
% \begin{macro}{\sevenrm}
% A small fixed size roman font.
%    \begin{macrocode}
\providecommand*{\sevenrm}{\fontsize{7}{9pt}\rmfamily}
%    \end{macrocode}
% \end{macro}
%
% \begin{macro}{\f@tm}
% \begin{macro}{\f@tn}
% \begin{macro}{\f@tp}
% \begin{macro}{\f@tdim}
% Counts and a dimen.
%    \begin{macrocode}
\newcount\f@tm \newcount\f@tn \newcount\f@tp \newdimen\f@tdim

%    \end{macrocode}
% \end{macro}
% \end{macro}
% \end{macro}
% \end{macro}
%
% \begin{macro}{\fonttable}
% \cs{fonttable}\marg{font} typesets a table of all the glyphs in the 
% \meta{font} (e.g., auncl10).
%    \begin{macrocode}
\newcommand*{\fonttable}[1]{%
  \def\f@tfontname{#1}%
  \bgroup
  \f@tstartfont 
  \ftable
  \egroup}

%    \end{macrocode}
% \end{macro}
%
% \begin{macro}{\pikfont}
% \cs{pikfont}\marg{encoding}\marg{family}\marg{series}\marg{shape} 
% selects the font with \meta{encoding},
% \meta{family}, \meta{series} and \meta{shape}.
% \changes{v1.4}{2009/05/06}{Added \cs{pikfont}}
%    \begin{macrocode}
\DeclareRobustCommand{\pikfont}[4]{%
  \fontencoding{#1}\fontfamily{#2}\fontseries{#3}\fontshape{#4}\selectfont}

%    \end{macrocode}
% \end{macro}
%
% \begin{macro}{\xfonttable}
% \cs{xfonttable}\marg{encoding}\marg{family}\marg{series}\marg{shape} 
% typesets a table of all the glyphs in the font with \meta{encoding},
% \meta{family}, \meta{series} and \meta{shape}
% (e.g., \verb?\xfonttable{T1}{pnc}{m}{it}? for New Century Schoolbook italic).
% The original code for the macro was supplied by Enrico Gregorio.
% \changes{v1.4}{2009/05/06}{Added \cs{xfonttable}}
%    \begin{macrocode}
\newcommand*{\xfonttable}[4]{\bgroup
  \pikfont{#1}{#2}{#3}{#4}%
  \edef\f@tfontname{\fontname\font}\normalfont
  \f@tstartfont 
  \ftable
  \egroup}

%    \end{macrocode}
% \end{macro}
%
% \begin{macro}{\f@tstartfont}
% Sets up for a font table.
%    \begin{macrocode}
\newcommand*{\f@tstartfont}{\font\f@ttestfont=\f@tfontname
  \f@ttestfont \f@tsetbaselineskip
  \ifdim\fontdimen6\f@ttestfont<10pt \rightskip=0pt plus 20pt
  \else\rightskip=0pt plus 2em \fi
  \spaceskip=\fontdimen2\f@ttestfont % space between words (\raggedright)
  \xspaceskip=\fontdimen2\f@ttestfont \advance\xspaceskip
  by\fontdimen7\f@ttestfont}

%    \end{macrocode}
% \end{macro}
%
% \begin{macro}{\f@tsetbaselineskip}
%    \begin{macrocode}
\newcommand*{\f@tsetbaselineskip}{\setbox0=\hbox{\f@tn=0
  \loop\char\f@tn \ifnum \f@tn<255 \advance\f@tn 1 \repeat}
  \baselineskip=6pt \advance\baselineskip\ht0 \advance\baselineskip\dp0 }

%    \end{macrocode}
% \end{macro}
%
% \begin{macro}{\f@toct}
% \cs{f@toct}\marg{onum} typesets the octal constant \meta{onum}.
%    \begin{macrocode}
\newcommand*{\f@toct}[1]{\hbox{\rmfamily\'{}\kern-.2em\itshape
           #1\/\kern.05em}} % octal constant
%    \end{macrocode}
% \end{macro}
%
% \begin{macro}{\f@thex}
% \cs{f@thex}\marg{hnum} typesets the hexadecimal constant \meta{hnum}.
%    \begin{macrocode}
\newcommand*{\f@thex}[1]{\hbox{\rmfamily\H{}\ttfamily#1}} % hexadecimal constant
%    \end{macrocode}
% \end{macro}
%
% \begin{macro}{\f@tsetdigs}
% \cs{f@tsetdigs}
%    \begin{macrocode}
\def\f@tsetdigs#1"#2{\gdef\h{#2}% \h=hex prefix; \0\1=corresponding octal
 \f@tm=\f@tn \divide\f@tm by 64 \xdef\0{\the\f@tm}%
 \multiply\f@tm by-64 \advance\f@tm by\f@tn \divide\f@tm by 8 \xdef\1{\the\f@tm}}
%    \end{macrocode}
% \end{macro}
%
% \begin{macro}{\f@ttestrow}
% \cs{f@ttestrow} checks if there are any characters in the next block of 
% 16 slots.
%    \begin{macrocode}
\newcommand*{\f@ttestrow}{\setbox0=\hbox{\penalty 1\def\\{\char"\h}%
 \\0\\1\\2\\3\\4\\5\\6\\7\\8\\9\\A\\B\\C\\D\\E\\F%
 \global\f@tp=\lastpenalty}} % \f@tp=1 if none of the characters exist

%    \end{macrocode}
% \end{macro}
%
% \begin{macro}{\ifhexoct}
% \begin{macro}{\hexoct}
% \begin{macro}{\nohexoct}
% Flag for (not) setting hex and octal numbers.
%    \begin{macrocode}
\newif\ifhexoct
\newcommand*{\hexoct}{\hexocttrue}
\newcommand*{\nohexoct}{\hexoctfalse}
\hexoct

%    \end{macrocode}
% \end{macro}
% \end{macro}
% \end{macro}
% 
%
% \begin{macro}{\f@toddlinenum}
% \cs{f@toddline}
%    \begin{macrocode}
\newcommand*{\f@toddline}{\cr
  \noalign{\nointerlineskip}
  \multispan{19}\hrulefill&
  \setbox0=\hbox{\lower 2.3pt\hbox{\f@thex{\h x}}}\smash{\box0}
  \cr
  \noalign{\nointerlineskip}}

%    \end{macrocode}
% \end{macro}
%
% \begin{macro}{\iff@tskipping}
% \begin{macro}{\f@tskippingtrue}
% \begin{macro}{\f@tskippingfalse}
%    \begin{macrocode}
\newif\iff@tskipping

%    \end{macrocode}
% \end{macro}
% \end{macro}
% \end{macro}
%
% \begin{macro}{\fontrange}
% \cs{fontrange}\marg{low}\marg{high} sets the character range to be 
% output.
%    \begin{macrocode}
\newcommand*{\fontrange}[2]{%
  \ifnum#1<#2\relax
%    \end{macrocode}
% Set \cs{f@tlow} to the nearest multiple of 16 that is at or below \meta{low},
% but first make sure that it will be at least 0.
%    \begin{macrocode}
  \ifnum#1<\z@
    \f@tm=\z@
  \else
    \f@tm=#1
    \divide \f@tm \sixt@@n
    \multiply \f@tm \sixt@@n
  \fi
  \edef\f@tlow{\the\f@tm}
%    \end{macrocode}
% Set \cs{f@thigh} to the nearest multiple of 16 at or above \meta{high},
% finally making sure that its maximum is 256.
%    \begin{macrocode}
  \f@tm=#2
  \divide \f@tm \sixt@@n
  \advance \f@tm \@ne
  \multiply \f@tm \sixt@@n
  \ifnum \f@tm > \@cclvi  \f@tm=\@cclvi \fi
  \edef\f@thigh{\the\f@tm}
 \else
   \PackageError{fonttable}{%
      Improper values for fontrange. Default values substituted}{\@ehc}
   \def\f@tlow{0} \def\f@thigh{256}
  \fi}
\fontrange{0}{256}

%    \end{macrocode}
% \end{macro}
%
% \begin{macro}{\f@tloopforsixteen}
% \cs{f@tloopforsixteen} sets up a block of sixteen character slots.
%    \begin{macrocode}
\newcommand*{\f@tloopforsixteen}{%
  \ifnum\f@tn<\f@tlow \global\f@tn=\f@tlow\fi
  \loop\f@tskippingfalse
  \ifnum\f@tn<\f@thigh \f@tm=\f@tn \divide\f@tm \sixt@@n \chardef\next=\f@tm
  \expandafter\f@tsetdigs\meaning\next \f@ttestrow
  \ifnum\f@tp=\@ne \f@tskippingtrue \fi\fi
  \iff@tskipping \global\advance\f@tn \sixt@@n \repeat}

%    \end{macrocode}
% \end{macro}
%
% \begin{macro}{\f@tevenline}
% \begin{macro}{\f@tevenlinenonum}
% \cs{f@tevenline} gets next non-empty set of a block of 16 characters.
% It either calls \cs{f@tmorechart} to print them, or \cs{f@tendchart} to
% finish off the table if all 256 potential characters have been processed.
%
% \cs{f@tevenlinenonum} does something similar when no external numbers
% are printed.
%    \begin{macrocode}
\newcommand*{\f@tevenline}{%
  \f@tloopforsixteen
  \ifnum\f@tn=\f@thigh \let\next=\f@tendchart\else\let\next=\f@tmorechart\fi
  \next}
\newcommand*{\f@tevenlinenonum}{%
  \f@tloopforsixteen
  \ifnum\f@tn=\f@thigh
    \\\hline
  \else
    \\\hline
    \f@tmorechartnonum
  \fi}

%    \end{macrocode}
% \end{macro}
% \end{macro}
%
% \begin{macro}{\f@tmorechart}
% \begin{macro}{\f@tmorechartnonum}
% \cs{f@tmorechart} sets two lines of the table, and \cs{f@tmorechartnonum}
% does the same when there are no external numbers.
%    \begin{macrocode}
\newcommand*{\f@tmorechart}{\cr\noalign{\hrule\penalty5000}
 \f@tchartline \f@toddline \f@tm=\1 \advance\f@tm 1 \xdef\1{\the\f@tm}
 \f@tchartline \f@tevenline}
\newcommand*{\f@tmorechartnonum}{%
  \f@tsimpleline \\ \hline
  \f@tsimpleline \f@tevenlinenonum}

%    \end{macrocode}
% \end{macro}
% \end{macro}
%
% \begin{macro}{\f@tchartline}
% \begin{macro}{\f@tsimpleline}
% \cs{f@tchartline} does a line of the table, including external numbers,
%  and \cs{f@tsimpleline} does an unnumbered line.
%    \begin{macrocode}
\newcommand*{\f@tchartline}{%
  &\f@toct{\0\1x}&&\f@tpsg{}&&\f@tpsg{}&&\f@tpsg{}&&\f@tpsg{}&&\f@tpsg{}&&\f@tpsg{}&&\f@tpsg{}&&\f@tpsg{}&&}
\newcommand*{\f@tsimpleline}{%
  \f@tpsg{}\f@tchartstrut& \f@tpsg{} & \f@tpsg{} & \f@tpsg{} & \f@tpsg{} & \f@tpsg{} & \f@tpsg{} & \f@tpsg{}}

%    \end{macrocode}
% \end{macro}
% \end{macro}
%
% \begin{macro}{\f@tchartstrut}
% \begin{macro}{\ftablewidth}
% \begin{macro}{\fntcolwidth}
% \cs{f@tchartstrut} is a strut used in each table line. \cs{ftablewidth} is 
% width of an externally numbered table. \cs{fntcolwidth} is the minimum
% width of a column in an unnumbered table.
%    \begin{macrocode}
\newcommand*{\f@tchartstrut}{\lower4.5pt\vbox to14pt{}}
\newdimen\ftablewidth
  \ftablewidth=\hsize
\newdimen\fntcolwidth
  \setlength{\fntcolwidth}{0.08\ftablewidth}
%    \end{macrocode}
% \end{macro}
% \end{macro}
% \end{macro}
%
% \begin{macro}{\f@tcol}
% \begin{macro}{\f@tstartchartnonum}
% \cs{f@tstartchartnonum} is a table line of spaces, with no verticals.
%    \begin{macrocode}
\newcommand*{\f@tcol}{%
  \multicolumn{1}{c}{\hspace*{\fntcolwidth}}}
\newcommand*{\f@tstartchartnonum}{%
  \f@tcol &\f@tcol &\f@tcol &\f@tcol &\f@tcol &\f@tcol &\f@tcol &\f@tcol}

%    \end{macrocode}
% \end{macro}
% \end{macro}
%
% \begin{macro}{\ftable}
% \begin{macro}{\f@tftablenum}
% \begin{macro}{\f@tftablenonum}
% \cs{ftable} sets a complete character table. The actual code is in either
% \cs{f@tftablenum} or \cs{f@tftablenonum} for externally numbered or 
% plain tables, respectively.
% \changes{v1.0a}{2005/12/06}{Added missing zeroing of \cs{f@tn} to \cs{f@tftablenonum}}
%    \begin{macrocode}
\newcommand*{\f@tftablenum}{$$\global\f@tn=\z@
  \halign to\ftablewidth\bgroup
    \f@tchartstrut##\tabskip0pt plus10pt&
    &\hfil##\hfil&\vrule##\cr
    \lower6.5pt\null
    &&&\f@toct0&&\f@toct1&&\f@toct2&&\f@toct3&&\f@toct4&&\f@toct5&&\f@toct6&&\f@toct7&%
    \f@tevenline}
\newcommand*{\f@tftablenonum}{%
  \global\f@tn=\z@
  \begin{tabular}{|c|c|c|c|c|c|c|c|}
    \f@tstartchartnonum
    \f@tevenlinenonum
  \end{tabular}}
\newcommand*{\ftable}{\ifhexoct\f@tftablenum\else\f@tftablenonum\fi}

%    \end{macrocode}
% \end{macro}
% \end{macro}
% \end{macro}
%
% \begin{macro}{\f@tendchart}
% \cs{f@tendchart} sets the last line of an externally numbered table with
% the relevant hex digits.
%    \begin{macrocode}
\newcommand*{\f@tendchart}{\cr\noalign{\hrule}
  \raise11.5pt\null&&&\f@thex 8&&\f@thex 9&&\f@thex A&&\f@thex B&
  &\f@thex C&&\f@thex D&&\f@thex E&&\f@thex F&\cr
  \egroup$$\par}

%    \end{macrocode}
% \end{macro}
%
% \begin{macro}{\f@tpsg}
% \changes{v1.3}{2009/04/30}{Replaced redefinition of \cs{:} by \cs{f@tpsg}}
% \changes{v1.5c}{2009/09/20}{\cs{renewcommand}\cs{f@tpsg} instead (props Michael Niedermair)}
% \changes{v1.5d}{2009/09/22}{Fix the bug for real this time}
% \begin{macro}{\f@placechar}
% \begin{macro}{\f@placedecimal}
% \cs{f@tpsg} typesets a single glyph, possibly with its decimal slot number.
% \cs{f@placechar} is the function to typeset the glyph with its number that is
% internally defined as \cs{f@placedecimal} if decimals are to be shown.
%    \begin{macrocode}
\newcommand*{\f@tpsg}{%
  \setbox\z@=\hbox{\f@placechar{\char\f@tn}{\the\f@tn}}%
  \ifdim\ht\z@>7.5pt\relax
    \f@treposition
  \else
    \ifdim\dp\z@>2.5pt\relax
      \f@treposition
    \fi
  \fi
  \box\z@
  \global\advance\f@tn\@ne
}
%    \end{macrocode}
% Change this definition to adjust the typesetting of the decimal numbers:
%    \begin{macrocode}
\newcommand*\f@placedecimal[2]{#1\ {\tiny #2}}
%    \end{macrocode}
% \end{macro}
% \end{macro}
% \end{macro}
%
% \begin{macro}{\decimals}
% \begin{macro}{\nodecimals}
% Following \cs{decimals}, which is the default, decimal numbers are
% printed in the table. Following \cs{nodecimals} they are not printed.
%    \begin{macrocode}
\newcommand*{\nodecimals}{%
  \renewcommand*\f@placechar{\@firstoftwo}%
}
%    \end{macrocode}
% 
%    \begin{macrocode}
\newcommand{\decimals}{%
  \renewcommand*\f@placechar{\f@placedecimal}%
}
\newcommand*\f@placechar{}
\decimals
%    \end{macrocode}
% \end{macro}
% \end{macro}
%
% \begin{macro}{\f@treposition}
% \cs{f@treposition}
%    \begin{macrocode}
\newcommand*{\f@treposition}{\setbox0=\vbox{\kern2pt\box0}\f@tdim=\dp0
  \advance\f@tdim 2pt \dp0=\f@tdim}

%    \end{macrocode}
% \end{macro}
%
% \begin{macro}{\fonttext}
% \cs{fonttext}\marg{font} typesets \cs{knutext} using \meta{font} (e.g.
% auncl10).
%    \begin{macrocode}
\def\fonttext#1{%
  \def\f@tfontname{#1}%
  \bgroup
  \f@tstartfont
  \knutext
  \egroup}

%    \end{macrocode}
% \end{macro}
%
% \begin{macro}{\regulartext}
% \cs{regulartext}\marg{fontspec} typesets \cs{knutext} using \meta{fontspec} 
% (e.g., \cs{aunclfamily}).
%    \begin{macrocode}
\def\regulartext#1{%
  \bgroup
  #1
  \knutext
  \egroup}

%    \end{macrocode}
% \end{macro}
%
% \begin{macro}{\knutext}
% Deathless prose from Knuth for testing a font. It includes
% \cs{moreknutext}, \cs{capknutext}, and \cs{knunames}.
%    \begin{macrocode}
\def\knutext{{
On November 14, 1885, Senator \& Mrs.~Leland Stanford called together
at their San Francisco mansion the 24~prominent men who had been
chosen as the first trustees of The Leland Stanford Junior University.
They handed to the board the Founding Grant of the University, which
they had executed three days before. This document---with various
amendments, legislative acts, and court decrees---remains as the
University's charter.  In bold, sweeping language it stipulates that
the objectives of the University are ``to qualify students for
personal success and direct usefulness in life; and to promote the
publick welfare by exercising an influence in behalf of humanity and
civilization, teaching the blessings of liberty regulated by law, and
inculcating love and reverence for the great principles of government
as derived from the inalienable rights of man to life, liberty, and
the pursuit of happiness.'' 

\moreknutext

\capknutext

\knunames
\par}}

%    \end{macrocode}
% \end{macro}
%
% \begin{macro}{\@moreknutext}
% Some more text with a variety of ligatures and accents.
%    \begin{macrocode}
\def\@moreknutext{?`But aren't Kafka's Schlo{\ss} and {\AE}sop's
{\OE}uvres often na{\"\i}ve vis-\`a-vis the d{\ae}monic ph{\oe}nix's
official r\^ole in fluffy souffl\'es? }

%    \end{macrocode}
% \end{macro}
%
% \begin{macro}{\@capknutext}
% \begin{macro}{\capknutext}
% Text using only capital letters and some punctutation.
%    \begin{macrocode}
\newcommand{\@capknutext}{%
(!`THE DAZED BROWN FOX QUICKLY GAVE 12345--67890 JUMPS!)}
\let\capknutext\@capknutext

%    \end{macrocode}
% \end{macro}
% \end{macro}
%
% \begin{macro}{\@knunames}
% Lots of accents masquerading in personal names.
%    \begin{macrocode}
\def\@knunames{ {\AA}ngel\aa\ Beatrice Claire
  Diana \'Erica Fran\c{c}oise Ginette H\'el\`ene Iris
  Jackie K\=aren {\L}au\.ra Mar{\'\i}a N\H{a}ta{\l}{\u\i}e {\O}ctave
  Pauline Qu\^eneau Roxanne Sabine T\~a{\'\j}a Ur\v{s}ula
  Vivian Wendy Xanthippe Yv{\o}nne Z\"azilie\par}

%    \end{macrocode}
% \end{macro}
%
% \begin{macro}{\guillemotleft}
% \begin{macro}{\guillemotright}
% \begin{macro}{\flqq}
% \begin{macro}{\frqq}
% Just in case the french quotes are not defined, as they are called for in
% the subsequent \cs{germantext}.
% \changes{v1.2}{2008/05/08}{Replaced \cs{providecommand} for guillemots
%                           by \cs{DeclareTextSymbol}}
%    \begin{macrocode}
\DeclareTextSymbol{\guillemotleft}{OT1}{`\'}
\DeclareTextSymbol{\guillemotright}{OT1}{`\`}
\providecommand{\flqq}{\guillemotleft}
\providecommand{\frqq}{\guillemotright}

%    \end{macrocode}
% \end{macro}
% \end{macro}
% \end{macro}
% \end{macro}
%
% \begin{macro}{\germantext}
% \begin{macro}{\germanparatext}
% Text from the \Lpack{Blindtext} package.
% \changes{v1.1}{2006/10/02}{Added \cs{germantext}}
% \changes{v1.51}{2009/05/14}{Changed \cs{germantext} to \cs{germanparatext}}
%    \begin{macrocode}
\providecommand*{\germantext}{%
  \PackageWarning{fonttable}{\protect\germantext\space is deprecated, 
                 \MessageBreak use \protect\germanparatext\space instead}}
\newcommand*{\germanparatext}{%
Dies hier ist ein Blindtext zum Testen von Textausgaben. Wer
diesen Text liest, ist selbst schuld. Der Text gibt lediglich den
Grauwert der Schrift an. Ist das wirklich so? Ist es
gleich\-g\"ul\-tig ob ich schreibe: \frqq Dies ist ein
Blindtext\flqq\ oder \frqq Huardest gefburn\flqq? Kjift --
mitnichten! Ein Blindtext bietet mir wichtige Informationen. An
ihm messe ich die Lesbarkeit einer Schrift, ihre Anmutung, wie
harmonisch die Figuren zueinander stehen und pr\"u\-fe, wie breit
oder schmal sie l\"auft. Ein Blindtext sollte m\"og\-lichst viele
verschiedene Buchstaben enthalten und in der Originalsprache
gesetzt sein. Er mu\ss\ keinen Sinn ergeben, sollte aber lesbar
sein. Fremdsprachige Texte wie \frqq Lorem ipsum\flqq\ dienen
nicht dem eigentlichen Zweck, da sie eine
falsche Anmutung vermitteln.\par}

%    \end{macrocode}
% \end{macro}
% \end{macro}
%
% \begin{macro}{\latintext}
% \begin{macro}{\latinparatext}
% The traditional printers' text.
% \changes{v1.1}{2006/10/02}{Added \cs{latintext}}
% \changes{v1.51}{2009/05/14}{Changed \cs{latintext} to \cs{latinparatext} 
% because of a clash with the babel package}
%    \begin{macrocode}
\providecommand*{\latintext}{%
  \PackageWarning{fonttable}{\protect\latintext\space may be overriden by the 
   babel package \MessageBreak use
                 \protect\latinparatext\space instead}}
\newcommand*{\latinparatext}{%
Lorem ipsum dolor sit amet, consectetuer adipiscing elit. Etiam
lobortis facilisis sem. Nullam nec mi et neque pharetra
sollicitudin. Praesent imperdiet mi nec ante. Donec ullamcorper,
felis non sodales commodo, lectus velit ultrices augue, a
dignissim nibh lectus placerat pede. Vivamus nunc nunc, molestie
ut, ultricies vel, semper in, velit. Ut porttitor. Praesent in
sapien. Lorem ipsum dolor sit amet, consectetuer adipiscing elit.
Duis fringilla tristique neque. Sed interdum libero ut metus.
Pellentesque placerat. Nam rutrum augue a leo. Morbi sed elit sit
amet ante lobortis sollicitudin. Praesent blandit blandit mauris.
Praesent lectus tellus, aliquet aliquam, luctus a, egestas a,
turpis. Mauris lacinia lorem sit amet ipsum. Nunc quis urna dictum
turpis accumsan semper.\par}

%    \end{macrocode}
% \end{macro}
% \end{macro}
%
% \begin{macro}{\simpletext}
% \begin{macro}{\fulltext}
% \begin{macro}{\moreknutext}
% \begin{macro}{\knunames}
% \cs{simpletext} kills off \cs{moreknutext} and \cs{knunames}.
% \cs{fulltext} restores \cs{moreknutext} and \cs{knunames}.
% Make \cs{fulltext} the default.
%    \begin{macrocode}
\newcommand*{\simpletext}{\let\moreknutext\relax \let\knunames\relax}
\newcommand*{\fulltext}{\let\moreknutext\@moreknutext \let\knunames\@knunames}
\fulltext

%    \end{macrocode}
% \end{macro}
% \end{macro}
% \end{macro}
% \end{macro}
%
% \begin{macro}{fonttexts}
% \cs{fonttexts}\marg{font}\marg{text} typesets \meta{text} using \meta{font} (e.g.
% auncl10).
% \changes{v1.1}{2006/10/02}{Added \cs{fonttexts}}
%    \begin{macrocode}
\def\fonttexts#1#2{%
  \def\f@tfontname{#1}%
  \bgroup
  \f@tstartfont
  #2
  \egroup}

%    \end{macrocode}
% \end{macro}
%
% \begin{macro}{\regulartexts}
% \cs{regulartext}\marg{fontspec}\marg{text} typesets \meta{text} using \meta{fontspec} 
% (e.g., \cs{aunclfamily}).
% \changes{v1.1}{2006/10/02}{Added \cs{regulartexts}}
%    \begin{macrocode}
\def\regulartexts#1#2{%
  \bgroup
  #1 #2
  \egroup}

%    \end{macrocode}
% \end{macro}
%
% \begin{macro}{\aztext}
% \begin{macro}{\AZtext}
% \begin{macro}{\digitstext}
% \begin{macro}{\punctext}
%  The various characters used for Latin texts.
% \changes{v1.2}{2008/05/08}{Added \cs{aztext}, \cs{AZtext}, \cs{digitstext}
%                            and \cs{punctext}}
%    \begin{macrocode}
\newcommand*{\aztext}{a b c d e f g h i j k l m n o p q r s t u v w x y z}
\newcommand*{\AZtext}{A B C D E F G H I J K L M N O P Q R S T U V W X Y Z}
\newcommand*{\digitstext}{0 1 2 3 4 5 6 7 8 9}
\newcommand*{\punctext}{` ! @ \$ \& * ( ) \_ - + =  [ ] < > \{ \} : ; ' , . ? /}

%    \end{macrocode}
% \end{macro}
% \end{macro}
% \end{macro}
% \end{macro}
%
% \subsection{Testing a glyph}
%
% This is a reimplementation of Donald Knuth's \texttt{testfont.tex} which
% is available from CTAN and there is also a commented version in Appendix H
% of \textit{The METAFONT Book}.
%
% \begin{macro}{\fnthours}
% \begin{macro}{\f@ttwodigits}
% The time of day on a 24 hour clock.
%    \begin{macrocode}
%%%%%%%% using \@tempcnta for Knuth's \m and \@tempcntb for his \n
\newcommand*{\fnthours}{\@tempcntb=\time \divide\@tempcntb 60
  \@tempcnta=-\@tempcntb \multiply\@tempcnta 60 \advance\@tempcnta \time
  \f@ttwodigits\@tempcntb:\f@ttwodigits\@tempcnta}
\newcommand*{\f@ttwodigits}[1]{\ifnum #1<10 0\fi \number#1}

%    \end{macrocode}
% \end{macro}
% \end{macro}
%
% \begin{macro}{\f@tgettsechars}
% \begin{macro}{\f@ttchar}
% \begin{macro}{\f@tschar}
% \begin{macro}{\f@techar}
% \cs{f@tgettsechars}\marg{T}\marg{S}\marg{E} gets three characters
% and \cs{chardef}s \cs{f@ttchar} to \meta{T} (the test character),
% \cs{f@tschar} to \meta{S} (start character) and \cs{f@techar}
% to \meta{E} (the end character).
%    \begin{macrocode}
\newcommand*{\f@tgettsechars}[3]{%
  \chardef\f@ttchar=#1 \chardef\f@tschar=#2 \chardef\f@techar=#3}

%    \end{macrocode}
% \end{macro}
% \end{macro}
% \end{macro}
% \end{macro}
%
% \begin{macro}{\glyphmixture}
% \begin{macro}{\f@tmixpattern}
% \begin{macro}{\f@tdomix}
%  \cs{glyphmixture}\marg{T}\marg{S}\marg{E} sets a mix of \meta{T} within
% the glyph range from \meta{S} to \meta{E} according to the pattern
% \cs{f@tmixpattern}. The work is done by \cs{f@tdomix}.
%    \begin{macrocode}
\newcommand*{\glyphmixture}[3]{\f@tgettsechars{#1}{#2}{#3}%
                               \f@tdomix\f@tmixpattern}
\newcommand*{\f@tmixpattern}{\0\1\0\0\1\1\0\0\0\1\1\1\0\1}
\newcommand*{\f@tdomix}[1]{\par\chardef\0=\f@ttchar \@tempcntb=\f@tschar
   \loop \chardef\1=\@tempcntb #1\endgraf
   \ifnum \@tempcntb<\f@techar \advance\@tempcntb \@ne \repeat}

%    \end{macrocode}
% \end{macro}
% \end{macro}
% \end{macro}
%
% \begin{macro}{\glyphalternation}
% \begin{macro}{\f@taltpattern}
% These are similar to \cs{glyphmixture} and \cs{f@tmixpattern} except
% that the glyphs are alternated.
%    \begin{macrocode}
\newcommand*{\glyphalternation}[3]{\f@tgettsechars{#1}{#2}{#3}%
                                   \f@tdomix\f@taltpattern}
\newcommand*{\f@taltpattern}{\0\1\0\1\0\1\0\1\0\1\0\1\0\1\0\1\0}

%    \end{macrocode}
% \end{macro}
% \end{macro}
%
% \begin{macro}{\f@tdisc}
% For breaking long lines so that the test character will be at the end
% of one line and repeated at the start of the next one.
%    \begin{macrocode}
\newcommand*{\f@tdisc}{\discretionary{\f@ttchar}{\f@ttchar}{\f@ttchar}}

%    \end{macrocode}
% \end{macro}
%
% \begin{macro}{\glyphseries}
% \begin{macro}{\f@tdoseries}
% \cs{glyphseries}\marg{T}\marg{S}\marg{E} puts the test character \meta{T}
% between all the others in the range \meta{S} to \meta{E}. The work is
% done by \cs{f@tdoseries}.
%    \begin{macrocode}
\newcommand*{\glyphseries}[3]{\f@tgettsechars{#1}{#2}{#3}%
  \f@tdisc\f@tdoseries\f@tschar\f@techar\par}
\newcommand*{\f@tdoseries}[2]{\@tempcntb=#1\relax
  \loop\char\@tempcntb\f@tdisc
    \ifnum\@tempcntb<#2\advance\@tempcntb \@ne \repeat}

%    \end{macrocode}
% \end{macro}
% \end{macro}
%
% \begin{macro}{\glyphalphabet}
% \begin{macro}{\GLYPHALPHABET}
% \begin{macro}{\f@tcomplower}
% \begin{macro}{\f@tcompupper}
% \cs{glyphalphabet}\marg{T} inserts the test glyph \meta{T} between the lowercase
% alphabetic characters. Similarly \cs{GLYPHALPHABET}\marg{T} does
% the same with the uppercase characters. The work is done by, respectively,
% \cs{f@tcomplower} and \cs{f@tcompupper}.
%    \begin{macrocode}
\newcommand*{\glyphalphabet}{\f@tcomplower}
\newcommand*{\GLYPHALPHABET}{\f@tcompupper}
\newcommand*{\f@tcomplower}[1]{\chardef\f@ttchar=#1 
  \f@tdisc\f@tdoseries{`a}{`z}\f@tdoseries{31}{34}\par}
\newcommand*{\f@tcompupper}[1]{\chardef\f@ttchar=#1 
  \f@tdisc\f@tdoseries{`A}{`Z}\f@tdoseries{35}{37}\par}

%    \end{macrocode}
% \end{macro}
% \end{macro}
% \end{macro}
% \end{macro}
%
% \begin{macro}{\glyphlowers}
% \begin{macro}{\glyphuppers}
% \begin{macro}{\glyphdigits}
% \begin{macro}{\f@tclc}
% \begin{macro}{\f@tcuc}
% \begin{macro}{\f@tdgs}
% \begin{macro}{\f@tdocomprehensive}
% These macros generate an extended mix of characters of a particular kind.
% The work is done by \cs{f@tdocomprensive} wih \cs{f@tclc}, \cs{f@tcuc}, and
% \cs{f@tdgs} setting up the glyph sets.
%    \begin{macrocode}
\newcommand*{\glyphlowers}{\f@tdocomprehensive\f@tclc{`a}{`z}{31}{34}}
\newcommand*{\glyphuppers}{\f@tdocomprehensive\f@tcuc{`A}{`Z}{35}{37}}
\newcommand*{\glyphdigits}{\f@tdocomprehensive\f@tdgs{`0}{`4}{`5}{`9}}
\newcommand*{\f@tdocomprehensive}[5]{\par\chardef\f@ttchar=#2
  \loop{#1} \ifnum\f@ttchar<#3\@tempcnta=\f@ttchar\advance\@tempcnta \@ne
  \chardef\f@ttchar=\@tempcnta \repeat
  \chardef\f@ttchar=#4
  \loop{#1} \ifnum\f@ttchar<#5\@tempcnta=\f@ttchar\advance\@tempcnta \@ne
  \chardef\f@ttchar=\@tempcnta \repeat}
\newcommand*{\f@tclc}{\f@tdisc\f@tdoseries{`a}{`z}\f@tdoseries{31}{34}\par}
\newcommand*{\f@tcuc}{\f@tdisc\f@tdoseries{`A}{`Z}\f@tdoseries{35}{37}\par}
\newcommand*{\f@tdgs}{\f@tdisc\f@tdoseries{`0}{`9}\par}

%    \end{macrocode}
% \end{macro}
% \end{macro}
% \end{macro}
% \end{macro}
% \end{macro}
% \end{macro}
% \end{macro}
%
% \begin{macro}{\glyphpunct}
% \begin{macro}{\f@tdopunct}
% \cs{glyphpunct} sets punctuation marks in combination with different
% sorts of letters. The work is done by \cs{f@tdopunct}.
%    \begin{macrocode}
\newcommand*{\glyphpunct}{\par\f@tdopunct{min}\f@tdopunct{pig}\f@tdopunct{hid}
                     \f@tdopunct{HIE}\f@tdopunct{TIP}\f@tdopunct{fluff}
  \$1,234.56 + 7/8 = 9\% @ \#0\par}
\newcommand*{\f@tdopunct}[1]{#1,\ #1:\ #1;\ `#1'\
  ?`#1?\ !`#1!\ (#1)\ [#1]\ #1*\ #1.\par}

%    \end{macrocode}
% \end{macro}
% \end{macro}
%
% The end of the package.
%    \begin{macrocode}
%</pack>
%    \end{macrocode}
%
%
%
%
% \Finale
\endinput

%% \CharacterTable
%%  {Upper-case    \A\B\C\D\E\F\G\H\I\J\K\L\M\N\O\P\Q\R\S\T\U\V\W\X\Y\Z
%%   Lower-case    \a\b\c\d\e\f\g\h\i\j\k\l\m\n\o\p\q\r\s\t\u\v\w\x\y\z
%%   Digits        \0\1\2\3\4\5\6\7\8\9
%%   Exclamation   \!     Double quote  \"     Hash (number) \#
%%   Dollar        \$     Percent       \%     Ampersand     \&
%%   Acute accent  \'     Left paren    \(     Right paren   \)
%%   Asterisk      \*     Plus          \+     Comma         \,
%%   Minus         \-     Point         \.     Solidus       \/
%%   Colon         \:     Semicolon     \;     Less than     \<
%%   Equals        \=     Greater than  \>     Question mark \?
%%   Commercial at \@     Left bracket  \[     Backslash     \\
%%   Right bracket \]     Circumflex    \^     Underscore    \_
%%   Grave accent  \`     Left brace    \{     Vertical bar  \|
%%   Right brace   \}     Tilde         \~}


