% \iffalse meta-comment
% newfile.dtx
% Author: Peter Wilson, Herries Press
% Maintainer: Will Robertson (will dot robertson at latex-project dot org)
% Copyright 2002--2004 Peter R. Wilson
%
% This work may be distributed and/or modified under the
% conditions of the LaTeX Project Public License, either
% version 1.3c of this license or (at your option) any 
% later version: <http://www.latex-project.org/lppl.txt>
%
% This work has the LPPL maintenance status "maintained".
% The Current Maintainer of this work is Will Robertson.
%
% This work consists of the files listed in the README file.
%
% 
%<*driver>
\documentclass{ltxdoc}
\EnableCrossrefs
\CodelineIndex
\setcounter{StandardModuleDepth}{1}
\begin{document}
  \DocInput{newfile.dtx}
\end{document}
%</driver>
%
% \fi
%
% \CheckSum{329}
%
% \DoNotIndex{\',\.,\@M,\@@input,\@addtoreset,\@arabic,\@badmath}
% \DoNotIndex{\@centercr,\@cite}
% \DoNotIndex{\@dotsep,\@empty,\@float,\@gobble,\@gobbletwo,\@ignoretrue}
% \DoNotIndex{\@input,\@ixpt,\@m}
% \DoNotIndex{\@minus,\@mkboth,\@ne,\@nil,\@nomath,\@plus,\@set@topoint}
% \DoNotIndex{\@tempboxa,\@tempcnta,\@tempdima,\@tempdimb}
% \DoNotIndex{\@tempswafalse,\@tempswatrue,\@viipt,\@viiipt,\@vipt}
% \DoNotIndex{\@vpt,\@warning,\@xiipt,\@xipt,\@xivpt,\@xpt,\@xviipt}
% \DoNotIndex{\@xxpt,\@xxvpt,\\,\ ,\addpenalty,\addtolength,\addvspace}
% \DoNotIndex{\advance,\Alph,\alph}
% \DoNotIndex{\arabic,\ast,\begin,\begingroup,\bfseries,\bgroup,\box}
% \DoNotIndex{\bullet}
% \DoNotIndex{\cdot,\cite,\CodelineIndex,\cr,\day,\DeclareOption}
% \DoNotIndex{\def,\DisableCrossrefs,\divide,\DocInput,\documentclass}
% \DoNotIndex{\DoNotIndex,\egroup,\ifdim,\else,\fi,\em,\endtrivlist}
% \DoNotIndex{\EnableCrossrefs,\end,\end@dblfloat,\end@float,\endgroup}
% \DoNotIndex{\endlist,\everycr,\everypar,\ExecuteOptions,\expandafter}
% \DoNotIndex{\fbox}
% \DoNotIndex{\filedate,\filename,\fileversion,\fontsize,\framebox,\gdef}
% \DoNotIndex{\global,\halign,\hangindent,\hbox,\hfil,\hfill,\hrule}
% \DoNotIndex{\hsize,\hskip,\hspace,\hss,\if@tempswa,\ifcase,\or,\fi,\fi}
% \DoNotIndex{\ifhmode,\ifvmode,\ifnum,\iftrue,\ifx,\fi,\fi,\fi,\fi,\fi}
% \DoNotIndex{\input}
% \DoNotIndex{\jobname,\kern,\leavevmode,\let,\leftmark}
% \DoNotIndex{\list,\llap,\long,\m@ne,\m@th,\mark,\markboth,\markright}
% \DoNotIndex{\month,\newcommand,\newcounter,\newenvironment}
% \DoNotIndex{\NeedsTeXFormat,\newdimen}
% \DoNotIndex{\newlength,\newpage,\nobreak,\noindent,\null,\number}
% \DoNotIndex{\numberline,\OldMakeindex,\OnlyDescription,\p@}
% \DoNotIndex{\pagestyle,\par,\paragraph,\paragraphmark,\parfillskip}
% \DoNotIndex{\penalty,\PrintChanges,\PrintIndex,\ProcessOptions}
% \DoNotIndex{\protect,\ProvidesClass,\raggedbottom,\raggedright}
% \DoNotIndex{\refstepcounter,\relax,\renewcommand,\reset@font}
% \DoNotIndex{\rightmargin,\rightmark,\rightskip,\rlap,\rmfamily,\roman}
% \DoNotIndex{\roman,\secdef,\selectfont,\setbox,\setcounter,\setlength}
% \DoNotIndex{\settowidth,\sfcode,\skip,\sloppy,\slshape,\space}
% \DoNotIndex{\symbol,\the,\trivlist,\typeout,\tw@,\undefined,\uppercase}
% \DoNotIndex{\usecounter,\usefont,\usepackage,\vfil,\vfill,\viiipt}
% \DoNotIndex{\viipt,\vipt,\vskip,\vspace}
% \DoNotIndex{\wd,\xiipt,\year,\z@}
%
% \changes{v1.0}{2002/06/22}{First public release}
% \changes{v1.0a}{2002/07/07}{Corrected documentation}
% \changes{v1.0b}{2004/05/10}{Changed license from LPPL v1.0 to v1.3}
% \changes{v1.0c}{2009/09/03}{New maintainer (Will Robertson)}
%
% \def\dtxfile{newfile.dtx}
% \def\fileversion{v1.0}
% \def\filedate{2002/06/22}
% \def\fileversion{v1.0a}
% \def\filedate{2002/07/07}
% \def\fileversion{v1.0b}
% \def\filedate{2004/05/10}
% \newcommand*{\Lpack}[1]{\textsf {#1}}           ^^A typeset a package
% \newcommand*{\Lopt}[1]{\textsf {#1}}            ^^A typeset an option
% \newcommand*{\file}[1]{\texttt {#1}}            ^^A typeset a file
% \newcommand*{\Lcount}[1]{\textsl {\small#1}}    ^^A typeset a counter
% \newcommand*{\pstyle}[1]{\textsl {#1}}          ^^A typeset a pagestyle
% \newcommand*{\Lenv}[1]{\texttt {#1}}            ^^A typeset an environment
%
% \title{The \Lpack{newfile} package\thanks{This
%        file (\texttt{\dtxfile}) has version number \fileversion, last revised
%        \filedate.}}
%
% \author{
%   Author: Peter Wilson, Herries Press \\
%   Maintainer: Will Robertson \\
%   \texttt{will dot robertson at latex-project dot org}
% }
% \date{\filedate}
% \maketitle
% \begin{abstract}
%    The \Lpack{newfile} package provides convenient user level commands 
% for reading and writing new files during a LaTeX run.
% \end{abstract}
% \tableofcontents
%
%
% \section{Introduction}
%
% TeX has a maximum of 16 input and 16 output \emph{streams} 
% for reading and writing
% files. The \Lpack{newfile} package provides means of associating
% several different files with a particular stream.
%
% The \Lpack{newfile} package requires the \Lpack{verbatim} 
% package~\cite{VERBATIM}, which is one of the required packages in
% a LaTeX distribution.
%
% This manual is typeset according to the conventions of the
% \LaTeX{} \textsc{docstrip} utility which enables the automatic
% extraction of the \LaTeX{} macro source files~\cite{GOOSSENS94}.
%
%    Section~\ref{sec:usc} describes the usage of the package.
% Commented source code for the package is in Section~\ref{sec:code}.
%
% \section{The \Lpack{newfile} package} \label{sec:usc}
%
%
% \DescribeMacro{\newoutputstream}
% \DescribeMacro{\newinputstream}
%    The command |\newoutputstream|\marg{stream} creates a new stream
% called \meta{stream} for writing out text and commands. The \meta{stream}
% should be just alphabetic characters with no spaces; for example |myout|.
% The command |\newinputstream|\marg{stream} creates a new stream for
% reading from a file. The \meta{stream} names must be unique --- you cannot
% use the same name for both an input and an output stream.
%
% If you try and create too many streams, TeX will tell you via an error 
% message.
%
% These two commands also attempt to create two other new commands, 
% called respectively
% |\atstreamopen<stream>| and |\atstreamclose<stream>|. If these commands
% already exist then they are left untouched, otherwise they are
% created (like using |\providecommand{\atstreamopen<stream}{}|). 
% For example if
% if you have used |mystr| as the name of a stream (either input or output),
% then the macros |\atstreamopenmystr| and |\atstreamclosemystr| are
% defined; by default they do nothing, but you can |\renewcommand| them
% to do something.
%
% \DescribeMacro{\openoutputfile}
% The macro |\openoutputfile|\marg{filename}\marg{stream} opens the
% file called \meta{filename} and the output stream \meta{stream}. It
% then attaches the file to the stream for writing and calls the macro
% |\atstreamopen<stream>|. Any pre-existing contents of \meta{filename}
% are deleted.
%
% \DescribeMacro{\closeoutputstream}
% The macro |\closeoutputstream|\marg{stream} calls the macro
% |\atstreamclose<stream>|, closes the output stream
% \meta{stream} and closes whatever file
% is currently attached to \meta{stream}. It then detaches the file from
% the stream.
%
% \DescribeEnv{writeverbatim}
% The |writeverbatim| environment takes one argument, the name of an output
% stream, which must be open. The contents of the environment are written
% out verbatim to the file currently attached to the stream.
%
% \DescribeMacro{\addtostream}
% The command |\addtostream|\marg{stream}\marg{text} writes \meta{text}
% to the file currently attached to the output stream \meta{stream}, which
% must be open. Any commands within \meta{text} will be processed before
% being written. To prevent this, put |\protect| before any command that
% you do not want expanding.
%
% \DescribeMacro{\openinputfile}
% The macro |\openinputfile|\marg{filename}\marg{stream} opens the
% file called \meta{filename} and the input stream \meta{stream}. It
% then attaches the file to the stream for reading and calls the macro
% |\atstreamopen<stream>|. It is an error if \meta{filename} can not be found.
%
% \DescribeMacro{\closeinputstream}
% The macro |\closeinputstream|\marg{stream} calls the macro
% |\atstreamclose<stream>|, closes the input stream
% \meta{stream} and closes whatever file
% is currently attached to \meta{stream}. It then detaches the file from
% the stream.
%
% \DescribeMacro{\readstream}
% The macro |\readstream|\marg{stream} reads the contents of the file that
% is currently associated with the input stream \meta{stream}. This provides
% the same functionality as |\input|\marg{filename} does.
%
% \DescribeMacro{\readaline}
% The macro |\readaline|\marg{stream} reads what TeX considers to be
% one line from the file that
% is currently associated with the input stream \meta{stream}. Multiple lines
% can be read by calling |\readaline| multiple times. A warning is issued
% if there are no more lines to be read (i.e., the end of the file has
% been reached).
%
%
% \DescribeMacro{\readverbatim}
% The macro |\readverbatim|\marg{stream} reads the contents of the file that
% is currently associated with the input stream \meta{stream} as |verbatim|
% text. This provides
% the same functionality as the \Lpack{verbatim} package's
% |\verbatiminput|\marg{filename} command does.
%
% \DescribeMacro{\streamvfont}
% Text read in verbatim is typeset using the font specified by
% |\streamvfont|\marg{font}. The default is 
% |\streamvfont{\normalfont\ttfamily}|. To typeset in a smaller font, try
% for example,
% |\streamvfont{\small\ttfamily}|. 
%
%
% \DescribeMacro{\numbervstream}
% \DescribeMacro{\marginnumbervstream}
% \DescribeMacro{\streamvnumfont}
% The declaration |\numbervstream| causes |\readverbatim| to number each 
% line it reads.
% The declaration |\marginnumbervstream| is similar to |\numbervstream|
% except it puts the numbers in the margin.
% The |streamvline| counter is used for the line numbering. Both
% the numbering declarations (re)set it to zero. The numbers are
% typeset in the font specified by |\streamvnumfont|\marg{font}. 
% The default is |\streamvnumfont{\footnotesize}|.
%
% \DescribeMacro{\plainvstream}
% The declaration |\plainvstream| stops any numbering. The default
% is |\plainvstream|.
%
% \DescribeEnv{comment}
% The |comment| environment, which is part of the \Lpack{verbatim}
% package, throws away everything inside the environment.
%
% The \Lpack{verbatim} package~\cite{VERBATIM} provides some other 
% facilities, apart from
% the |comment| environment, and its documentation explains them.
%
% \section{Examples}
%
% \subsection{Output stream}
%
% This example is inspired by the \Lpack{endfloat} package~\cite{ENDFLOAT}
% which provides a more sophisticated approach than shown below. The
% example demonstrates the use of an output stream.
%
% Suppose that in an \Lpack{article} class document you want all the
% figures to be collected at the end, but to specify them at the appropriate
% places in the body of the text.
%
%    \begin{macrocode}
%<*ex>
%    \end{macrocode}
%    \begin{macrocode}
\documentclass{article}
\usepackage{newfile}
...
%    \end{macrocode}
% Create a new output stream called |figs| and open it to write to file
% |figures.tex|.
%    \begin{macrocode}
\newoutputstream{figs}
\openoutputfile{figures.out}{figs}
...
%    \end{macrocode}
% Write out verbatim the first |figure| environment to the |figs| stream.
%    \begin{macrocode}
\begin{writeverbatim}{figs}
\begin{figure}
...
\end{figure}
\end{writeverbatim}
...
%    \end{macrocode}
% After the last |figure| is written to |figs|, close the |figs| output
% stream, which also closes the |figures.out| file. 
%    \begin{macrocode}
\begin{writeverbatim}{figs}
\begin{figure}
...
\end{figure}
\end{writeverbatim}
\closeoutputstream{figs}
...
%    \end{macrocode}
% At the end |\input| |figures.out| to typeset the figures.
%    \begin{macrocode}
\input{figures.out}
\end{document}
%    \end{macrocode}
%
% \subsection{An output and input stream}
%
% This example is the kind of thing that the \Lpack{answers} 
% package~\cite{ANSWERS} does rather better. The example illustrates
% the use of an output and an input stream.
%
% Suppose you are writing a document that includes questions and answers
% and you want all the answers at the end. It is most convenient if you
% can write the answer to each question as it is posed, and then only
% print them at the end.
%
% \setcounter{CodelineNo}{0}
%    \begin{macrocode}
\documentclass{article}
\usepackage{newfile}
...
%    \end{macrocode}
% Create new output and input streams called |ans| and |ansin|. Just to 
% demonstrate the use of the |\atstreamopen...| command, redefine
% |\atstreamopenansin| so that it will start a section called 
% `Answers to all the questions'. Then open the |ans| stream to write to
% a file called |answers.all|.
%    \begin{macrocode}
\newoutputstream{ans}
\newinputstream{ansin}
\renewcommand{\atstreamopenansin}{\section{Answers to all the questions}}
\openoutputfile{answers.all}{ans}
...
%    \end{macrocode}
% For each question, write out the answer verbatim to the |ans| stream.
%    \begin{macrocode}
This is a question.
\begin{writeverbatim}{ans}
This is the answer to a question.
\end{writeverbatim}
...
%    \end{macrocode}
% At the end, close the |ans| stream and open the |ansin| stream to read from
% the file |answers.all|. Opening the stream will also start the new
% `Answers to all the questions' section (from |\atstreamopenansin|). Then
% read from the |ansin| stream.
%    \begin{macrocode}
\closeoutputstream{ans}
\openinputfile{answers.all}{ansin}
\readstream{ansin}
%    \end{macrocode}
% Finally, close the |ansin| stream, and we are done.
%    \begin{macrocode}
\closeinputstream{ansin}
\end{document}
%    \end{macrocode}
%
% If desired, instead of inserting the answers at the end of the question
% document, |answers.all| could have been |\input| into a seperate
% answer document.
%
% Along the same lines as above, perhaps the original document is
% a \Lpack{book} class, with questions (and answers) at the end of each
% chapter. A seperate answer file could be produced for each chapter, like:
% \setcounter{CodelineNo}{0}
%    \begin{macrocode}
...
\newoutputstream{ansout}
\newinputstream{ansin}
...
\chapter{First chapter} % chapter 1
\openoutputfile{\jobname\thechapter.ans}{ansout}
%    \end{macrocode}
% The |\jobname| is the name of the main LaTeX document file, without
% the |.tex| extension, so if the name of the LaTeX source file is 
% |mybook.tex| the above line creates a file called |mybook1.ans|.
%    \begin{macrocode}
...
\begin{writeverbatim}{ansout}
An answer
\end{writeverbatim}
...
\closeoutputstream{ansout}
\chapter{Another chapter} % chapter N
\openoutputfile{\jobname\thechapter.ans}{ansout}
...
\closeoutputstream{ansout}

\chapter{Answers}

\section{Chapter 1}
\openinputstream{\jobname1.ans}{ansin}
\readstream{ansin}
\closestream{ansin}
...
\section{Chapter N}
\openinputstream{\jobnameN.ans}{ansin}
\readstream{ansin}
\closestream{ansin}
...
%    \end{macrocode}
%
% The above example shows how you can associate different files with a single
% stream.
%
% \subsection{Multiple streams}
%
% This is a more complex example, again inspired by the \Lpack{endfloat}
% package. In a \Lpack{book} class document you want 
% all the
% figures and tables to be collected at the end, 
% but to specify them at the appropriate
% places in the body of the text.
%
% \setcounter{CodelineNo}{0}
%     \begin{macrocode}
\documentclass{book}
\usepackage{newfile}
...
%    \end{macrocode}
% Create and open two new output streams, one for figures (|figs|) and the 
% other (|tabs|) for tables.
% For demonstration purposes, also create a new input stream called figtab.
%    \begin{macrocode}
\newoutputstream{figs}
\openoutputfile{figures.out}{figs}
\newoutputstream{tabs}
\openoutputfile{tables.out}{tabs}
\newinputstream{figtab}
...
%    \end{macrocode}
% There is a slight difficulty with this example, as in the \Lpack{book}
% class, figures and table numbers start anew with each chapter and the
% chapter number is prepended to the sequence number. We can make a
% start on solving this by creating a new pseudo chapter number and changing
% the default definitions of the figure and table numbers (which are
% |\renewcommand{\thefigure}{\thechapter.\arabic{figure}}| and similarly
% for tables).
%    \begin{macrocode}
\newcounter{pseudochapter}
\renewcommand{\thepseudochapter}{\arabic{pseudochapter}}
\renewcommand{\thefigure}{\thepseudochapter.\arabic{figure}}
\renewcommand{\thetable}{\thepseudochapter.\arabic{table}}
...
%    \end{macrocode}
% Do the usual things at the start of the document.
%    \begin{macrocode}
\begin{document}
\maketitle
\tableofcontents
\listoffigures
..
%    \end{macrocode}
% At the start of each chapter we need to set the |pseudochapter| counter
% to the chapter number, and write this to the output streams so
% that there is a record of which chapters the figures and tables came from
% Normally, each chapter resets the figure and table numbers, but as
% these will now be at the end, we have to fake the start of chapters
% in the output files. We can do all of this by the |\addtostream| macro.
%    \begin{macrocode}
\chapter{First}
%    \end{macrocode}
% The next bit of code will result in \\
% |\setcounter{pseudochapter}{N}| \\
% where |N| is the number of this chapter, appearing in the output files.
%    \begin{macrocode}
\addtostream{figs}{\protect\setcounter{pseudochapter}{\thechapter}}
\addtostream{tabs}{\protect\setcounter{pseudochapter}{\thechapter}}
%    \end{macrocode}
% The next bit of code results in the following two lines in the output files:\\
% |\refstepcounter{chapter}| \\
% |\addtocounter{chapter}{-1}| \\
% The first of these has the effect of resetting the figure and table 
% counters by increasing the chapter counter by one. The second line resets
% the chapter counter back to its original value.
%    \begin{macrocode}
\addtostream{figs}{\protect\refstepcounter{chapter}}
\addtostream{figs}{\protect\addtocounter{chapter}{-1}}
\addtostream{tabs}{\protect\refstepcounter{chapter}}
\addtostream{tabs}{\protect\addtocounter{chapter}{-1}}
%    \end{macrocode}
% Write out verbatim each figure to the |figs| stream.
%    \begin{macrocode}
\begin{writeverbatim}{figs}
\begin{figure}
...
\end{figure}
\end{writeverbatim}
...
%    \end{macrocode}
% and write each table verbatim to the |tabs| stream.
%    \begin{macrocode}
\begin{writeverbatim}{tabs}
\begin{table}
...
\end{table}
\end{writeverbatim}
...
%    \end{macrocode}
% When we have done all the chapters, figures and tables, 
% close the two output streams.
%    \begin{macrocode}
\closeoutputstream{figs}
\closeoutputstream{tabs}
%    \end{macrocode}
% Then if we want the figures to be typeset before the tables, open
% the |figtab| stream to read from |figures.out|, read the stream and 
% close it.
%    \begin{macrocode}
\openinputfile{figures.out}{figtab}
\readstream{figtab}
\closestream{figtab}
%    \end{macrocode}
% We can use the same |figtab| stream for reading the tables.
%    \begin{macrocode}
\openinputfile{tables.out}{figtab}
\readstream{figtab}
\closestream{figtab}
\end{document}
%    \end{macrocode}
%    \begin{macrocode}
%</ex>
%    \end{macrocode}
%
% Note that the \Lpack{endfloat} package~\cite{ENDFLOAT} will produce the
% same effect as the previous example, but much simply.
%
%
%    \StopEventually{}
%
%
% \section{The package code} \label{sec:code}
%
%    This package can only be used with LaTeX2e, and requires the
% \Lpack{verbatim} package~\cite{VERBATIM}, which is one of the required 
% packages for a LaTeX distribution.
% \setcounter{CodelineNo}{0}
%    \begin{macrocode}
%<*outin>
\NeedsTeXFormat{LaTeX2e}[1996/06/01]
\ProvidesPackage{newfile}[2009/09/03 v1.0c Output and input files]
\RequirePackage{verbatim}
%    \end{macrocode}
%
% To try and avoid name clashes with other packages, each internal macro
% in this package includes the character string `|stre@m|'.
%
% \begin{macro}{\newoutputstream}
% |\newoutputstream|\marg{stream} creates a new output stream called \meta{stream}.
% Different files may be associated with the \meta{stream}. Note that
% TeX permits no more than 16 output streams.
%    \begin{macrocode}
\newcommand{\newoutputstream}[1]{%
  \@ifundefined{#1outstre@m}%
    {\expandafter\newwrite\csname #1outstre@m\endcsname
     \csname newif\expandafter\endcsname
       \csname ifstre@m#1open\endcsname
     \global\csname stre@m#1openfalse\endcsname
     \expandafter\ifx\csname atstreamopen#1\endcsname\relax
       \global\@namedef{atstreamopen#1}{}%
     \fi
     \expandafter\ifx\csname atstreamclose#1\endcsname\relax
       \global\@namedef{atstreamclose#1}{}%
     \fi
    }%
    {\PackageError{newfile}{Output stream #1 is already defined}{\@ehc}}}

%    \end{macrocode}
% \end{macro}
%
% \begin{macro}{\newinputstream}
% |\newinputstream|\marg{stream} creates a new input stream called \meta{stream}.
% Different files may be associated with the \meta{stream}. Note that
% TeX permits no more than 16 input streams.
%    \begin{macrocode}
\newcommand{\newinputstream}[1]{%
  \@ifundefined{#1instre@m}%
    {\expandafter\newread\csname #1instre@m\endcsname
     \csname newif\expandafter\endcsname
       \csname ifstre@m#1open\endcsname
     \global\csname stre@m#1openfalse\endcsname
     \expandafter\ifx\csname atstreamopen#1\endcsname\relax
       \global\@namedef{atstreamopen#1}{}%
     \fi
     \expandafter\ifx\csname atstreamclose#1\endcsname\relax
       \global\@namedef{atstreamclose#1}{}%
     \fi
    }%
    {\PackageError{newfile}{Input stream #1 is already defined}{\@ehc}}}

%    \end{macrocode}
% \end{macro}
%
%    Some checking macros will be useful as some of the checks occur in
% multiple places.
%
% \begin{macro}{\@ifstre@mopen}
% |\@ifstre@mopen|\marg{stream}\marg{TRUE code}\marg{FALSE code}
% checks if stream \meta{stream} is currently open.
%    \begin{macrocode}
\newcommand{\@ifstre@mopen}[3]{%
  \csname ifstre@m#1open\endcsname#2\else#3\fi}
%    \end{macrocode}
% \end{macro}
%
% \begin{macro}{\instre@mandopen}
% |\instre@mandopen|\marg{stream}\marg{TRUE code} checks if \meta{stream}
% is an input stream and is open. If so, it executes \meta{TRUE code}.
%    \begin{macrocode}
\newcommand{\instre@mandopen}[2]{%
  \@ifundefined{#1instre@m}{%
    \PackageError{newfile}{#1\space is not an input stream}{\@ehc}}%
  {\@ifstre@mopen{#1}{#2}{%
    \PackageError{newfile}{Input stream #1\space is not open}{\@ehc}}}}

%    \end{macrocode}
% \end{macro}
%
% \begin{macro}{\instre@mandclosed}
% |\instre@mandclosed|\marg{stream}\marg{TRUE code} checks if \meta{stream}
% is an input stream and is closed (not open). 
% If so, it executes \meta{TRUE code}.
%    \begin{macrocode}
\newcommand{\instre@mandclosed}[2]{%
  \@ifundefined{#1instre@m}{%
    \PackageError{newfile}{#1\space is not an input stream}{\@ehc}}%
  {\@ifstre@mopen{#1}{%
    \PackageError{newfile}{Input stream #1\space is open}{\@ehc}}{#2}}}

%    \end{macrocode}
% \end{macro}
%
% \begin{macro}{\outstre@mandopen}
% |\outstre@mandopen|\marg{stream}\marg{TRUE code} checks if \meta{stream}
% is an output stream and is open. If so, it executes \meta{TRUE code}.
%    \begin{macrocode}
\newcommand{\outstre@mandopen}[2]{%
  \@ifundefined{#1outstre@m}{%
    \PackageError{newfile}{#1\space is not an output stream}{\@ehc}}%
  {\@ifstre@mopen{#1}{#2}{%
    \PackageError{newfile}{Output stream #1\space is not open}{\@ehc}}}}

%    \end{macrocode}
% \end{macro}
%
% \begin{macro}{\outstre@mandclosed}
% |\outtre@mandclosed|\marg{stream}\marg{TRUE code} checks if \meta{stream}
% is an output stream and is closed (not open). 
% If so, it executes \meta{TRUE code}.
%    \begin{macrocode}
\newcommand{\outstre@mandclosed}[2]{%
  \@ifundefined{#1outstre@m}{%
    \PackageError{newfile}{#1\space is not an output stream}{\@ehc}}%
  {\@ifstre@mopen{#1}{%
    \PackageError{newfile}{Output stream #1\space is open}{\@ehc}}{#2}}}

%    \end{macrocode}
% \end{macro}
%
% \begin{macro}{\openoutputfile}
% |\openoutputfile|\marg{filename}\marg{stream} opens the file called 
% \meta{filename} and attaches it to the stream \meta{stream} for
% writing. However, if the |\nofiles| command has been given the
% file is \emph{not} attached to the stream. No more than one file can be 
% attached to a stream at any given time.
%    \begin{macrocode}
\newcommand{\openoutputfile}[2]{%
  \outstre@mandclosed{#2}{%
    \global\@namedef{#1@filename}{#1}%
    \if@filesw 
      \immediate\openout\@nameuse{#2outstre@m}=\@nameuse{#1@filename}%
    \fi
    \global\csname stre@m#2opentrue\endcsname%
    \@nameuse{atstreamopen#2}%
  }%
}

%    \end{macrocode}
% \end{macro}
%
% \begin{macro}{\closeoutputstream}
% |\closeoutputstream|\marg{stream} closes the stream \meta{stream}.
%    \begin{macrocode}
\newcommand{\closeoutputstream}[1]{%
  \outstre@mandopen{#1}{%
    \@nameuse{atstreamclose#1}%
    \immediate\closeout\@nameuse{#1outstre@m}%
    \global\csname stre@m#1openfalse\endcsname}%
}

%    \end{macrocode}
% \end{macro}
%
% \begin{macro}{\openinputfile}
% |\openinputfile|\marg{filename}\marg{stream} opens the file called 
% \meta{filename} and attaches it to the stream \meta{stream} for
% reading. The file is added to the list of files. 
% No more than one file can be attached to a stream at
% any given time.
%    \begin{macrocode}
\newcommand{\openinputfile}[2]{%
  \IfFileExists{#1}{%                   file exists
    \instre@mandclosed{#2}{% 
      \@addtofilelist{#1}%
      \global\@namedef{#1@filename}{#1}%
      \immediate\openin\@nameuse{#2instre@m}=\@nameuse{#1@filename}%
      \global\csname stre@m#2opentrue\endcsname%
      \@nameuse{atstreamopen#2}}}%
  {%                                    file not found
    \PackageError{newfile}{Can't find file #1}{\@ehc}%
  }%
}

%    \end{macrocode}
% \end{macro}
%
% \begin{macro}{\closeinputstream}
% |\closeinputstream|\marg{stream} closes the stream \meta{stream}.
%    \begin{macrocode}
\newcommand{\closeinputstream}[1]{%
  \instre@mandopen{#1}{%
     \@nameuse{atstreamclose#1}%
     \immediate\closein\@nameuse{#1instre@m}%
     \global\csname stre@m#1openfalse\endcsname}%
}

%    \end{macrocode}
% \end{macro}
%
% \begin{environment}{writeverbatim}
% |\begin{writeverbatim}|\marg{stream} writes the contents of
% the environment as verbatim text to the given \meta{stream}.
%    \begin{macrocode}
\def\writeverbatim#1{%
  \@bsphack
  \let\do\@makeother\dospecials
  \catcode`\^^M\active
  \verbatim@startline
  \verbatim@addtoline
  \verbatim@finish
  \def\verbatim@processline{%
    \immediate\write\@nameuse{#1outstre@m}{\the\verbatim@line}}%
  \verbatim@start}
\def\endwriteverbatim{\@esphack}

%    \end{macrocode}
% \end{environment}
%
% \begin{macro}{\addtostream}
% |\addtostream|\marg{stream}\marg{text} writes \meta{text} to the
% given \meta{stream}.
%    \begin{macrocode}
\newcommand{\addtostream}[2]{%
  \@bsphack
  \outstre@mandopen{#1}{%
    {\let\protect\string
     \immediate\write\@nameuse{#1outstre@m}{#2}%
    }}%
  \@esphack
}

%    \end{macrocode}
% \end{macro}
%
% \begin{macro}{\ifstre@mnoteof}
% \begin{macro}{\checkstre@mnoteof}
% |\checkstre@mnoteof|\marg{stream} sets |\ifstre@mnoteof| to TRUE if
% \meta{stream} is not at the end of the file (i.e., it is the opposite
% of |\ifeof|).
%    \begin{macrocode}
\newif\ifstre@mnoteof
\newcommand{\checkstre@meof}[1]{%
  \stre@mnoteoftrue\ifeof\@nameuse{#1instre@m}\stre@mnoteoffalse\fi}

%    \end{macrocode}
% \end{macro}
% \end{macro}
%
% \begin{macro}{\readstream}
% |\readstream|\marg{stream} reads the contents of
% the given \meta{stream} as |\input| text.
%    \begin{macrocode}
\def\readstream#1{
  \instre@mandopen{#1}{%
    \loop \checkstre@meof{#1} \ifstre@mnoteof
      \read\@nameuse{#1instre@m} to\temptokstre@m
     \temptokstre@m
    \repeat
    }%
}
  
%    \end{macrocode}
% \end{macro}
%
% \begin{macro}{\readaline}
% |\readaline|\marg{stream} reads what TeX considers to be one line
% from the given \meta{stream} as |\input| text.
%    \begin{macrocode}
\def\readaline#1{
  \instre@mandopen{#1}{%
    \ifeof\@nameuse{#1instre@m}
      \PackageWarning{newfile}{No more to read from stream #1}
    \else
      \read\@nameuse{#1instre@m} to\temptokstre@m
      \temptokstre@m
    \fi
    }%
}
  
%    \end{macrocode}
% \end{macro}
%
%
% \begin{macro}{\readverbatim}
% \begin{macro}{\stre@mverbatim@input}
% \begin{macro}{\verbatim@readstre@m}
% |\readverbatim|\marg{stream} reads the contents of
% the given \meta{stream} as verbatim text.
%
% The read verbatim code is a slight variation on code from the
% \Lpack{verbatim} package. Most of the setup is done by the macros
% |\stre@mverbatim@input|\marg{setup}\marg{stream} and
% |\verbatim@readstre@m|\marg{stream}. Finally, |\verbatim@read@file|
% is a \Lpack{verbatim} package macro.
%    \begin{macrocode}
\def\readverbatim{\begingroup
  \@ifstar{\stre@mverbatim@input\relax}%
          {\stre@mverbatim@input{\frenchspacing\@vobeyspaces}}}

\def\stre@mverbatim@input#1#2{%
  \@ifstre@mopen{#2}%
    {\@verbatim #1\relax
     \def\verbatim@in@stream{\@nameuse{#2instre@m}}
     \verbatim@readstre@m{#2}\endtrivlist\endgroup\@doendpe}%
    {\PackageError{newfile}{Stream #2 is not open}{\@ehc}\endgroup}%
}

\def\verbatim@readstre@m#1{%
  \verbatim@startline
  \expandafter\endlinechar\expandafter\m@ne
  \expandafter\verbatim@read@file
  \expandafter\endlinechar\the\endlinechar\relax
  \verbatim@finish
}

%    \end{macrocode}
% \end{macro}
% \end{macro}
% \end{macro}
%
% \begin{macro}{\plainvstream}
% A macro that sets |\verbatim@processline| to its default definition.
%    \begin{macrocode}
\newcommand{\plainvstream}{%
  \def\verbatim@processline{\the\verbatim@line\par}%
}

%    \end{macrocode}
% \end{macro}
%
% \begin{macro}{\streamvnumfont}
% \begin{macro}{\stre@mvnfont}
% We need a counter for numbering lines read in verbatim.
%
% |\streamvnumfont|\marg{font} defines |\stre@mvnfont| to be \meta{font}.
%    \begin{macrocode}
\newcounter{streamvline}
\newcommand{\streamvnumfont}[1]{\def\stre@mvnfont{#1}}
\streamvnumfont{\footnotesize}

%    \end{macrocode}
% \end{macro}
% \end{macro}
%
% \begin{macro}{\streamvfont}
% \begin{macro}{\verbatim@font}
% |\streamvfont|\marg{font} defines |\verbatim@font| to use \meta{font}.
%    \begin{macrocode}
\newcommand{\verbatimfont}[1]{%
  \def\verbatim@font{#1%
    \hyphenchar\font\m@ne
    \let\do\do@noligs
    \verbatim@nolig@list}}
\verbatimfont{\normalfont\ttfamily}

%    \end{macrocode}
% \end{macro}
% \end{macro}
%
% \begin{macro}{\numbervstream}
% |\numbervstream| puts numbers at the start of each line read verbatim.
%    \begin{macrocode}
\newcommand{\numbervstream}{%
  \setcounter{streamvline}{0}%
  \def\verbatim@processline{%
    \addtocounter{streamvline}{1}%
    \leavevmode
    {\stre@mvnfont \thestreamvline}\space
    \the\verbatim@line\par}%
}

%    \end{macrocode}
% \end{macro}
%
% \begin{macro}{\marginnumbervstream}
% |\marginnumbervstream| puts numbers in the margin at the start of 
% each line read verbatim.
%    \begin{macrocode}
\newcommand{\marginnumbervstream}{%
  \setcounter{streamvline}{0}%
  \def\verbatim@processline{%
    \addtocounter{streamvline}{1}%
    \leavevmode
    \llap{{\stre@mvnfont \thestreamvline} \hskip\@totalleftmargin}
    \the\verbatim@line\par}%
}

%    \end{macrocode}
% \end{macro}
%
%    The end of the \Lpack{newfile} package.
%    \begin{macrocode}
%</outin>
%    \end{macrocode}
%
%
%
% \bibliographystyle{alpha}
%
% \begin{thebibliography}{GMS94}
%
% \bibitem[GMS94]{GOOSSENS94}
% Michel Goossens, Frank Mittelbach, and Alexander Samarin.
% \newblock \emph{The LaTeX Companion}.
% \newblock Addison-Wesley Publishing Company, 1994.
%
% \bibitem[MG95]{ENDFLOAT}
% James Darrell McCauley and Jeff Goldberg.
% \newblock `The \texttt{endfloat} package'.
% \newblock October, 1995.
%
% \bibitem[Pif96]{ANSWERS}
% Mike Piff.
% \newblock `Production of solution sheets in LaTeX2e'.
% \newblock July, 1996.
%
% \bibitem[SRR99]{VERBATIM}
% Rainer Sch\"{o}pf, Bernd Raichle, and Chris Rowley.
% \newblock `A New Implementation of LaTeX's \texttt{verbatim} and 
%            \texttt{verbatim*} Environments'.
% \newblock December, 1999.
%
% \end{thebibliography}
%
%
% \Finale
% \PrintIndex
%
\endinput

%% \CharacterTable
%%  {Upper-case    \A\B\C\D\E\F\G\H\I\J\K\L\M\N\O\P\Q\R\S\T\U\V\W\X\Y\Z
%%   Lower-case    \a\b\c\d\e\f\g\h\i\j\k\l\m\n\o\p\q\r\s\t\u\v\w\x\y\z
%%   Digits        \0\1\2\3\4\5\6\7\8\9
%%   Exclamation   \!     Double quote  \"     Hash (number) \#
%%   Dollar        \$     Percent       \%     Ampersand     \&
%%   Acute accent  \'     Left paren    \(     Right paren   \)
%%   Asterisk      \*     Plus          \+     Comma         \,
%%   Minus         \-     Point         \.     Solidus       \/
%%   Colon         \:     Semicolon     \;     Less than     \<
%%   Equals        \=     Greater than  \>     Question mark \?
%%   Commercial at \@     Left bracket  \[     Backslash     \\
%%   Right bracket \]     Circumflex    \^     Underscore    \_
%%   Grave accent  \`     Left brace    \{     Vertical bar  \|
%%   Right brace   \}     Tilde         \~}

